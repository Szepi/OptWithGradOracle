%!TEX root =  bgo.tex
% please do not delete or change the first line (needed by Csaba's editor)
%%%%%%%%%%%%%%%%%%%%%%%%%%%%%%%%%%%%%%%%%%%%%%%%%%%%%%%%%%%%%%%%%%%%%%%%%%%%%%% 
\subsubsection{Proof in one dimension}
For brevity, let $\Delta_n^{(1)*}$ denote the minimax error $\Delta_n^*(\F, c_1,c_2)$ for the case of $1$-dimensional function family $\F$.
The proof uses only $(c_1,c_2)$ Type-I oracles $\gamma$ that map $\cK$ to $\R$ (i.e., the oracles do not have memory).
Fix a  $(c_1,c_2)$ Type-I oracle $\gamma$ and an algorithm $\A$.
We restrict the class of oracles to those that return a random gradient estimate $G = m(x) + \xi$ with some map $m: \cK \to \R$,
where $\xi$ is standard normal with variance $\sigma^2 = C_2 \delta^{-q}$, satisfying the variance requirement. 
The map $m$, which, by slightly abusing notation, we will also denote by $\gamma$ in what follows, will be chosen based on $f$ to satisfy the requirement on the bias. The $Y$ value returned by the oracles is made equal to $x$.

Let $\delta$ denote the tolerance parameter chosen by $\A$.
Define the probability space $(\Omega, \B, P_{\A,\gamma})$, 
where $\Omega = \R^n\times \{-1,1\}$, $\B$ is the associated Borel sigma algebra. 
Further, the probability measure $P_{\A,\gamma} := p_{\A,\gamma} d(\lambda \times m)$, 
	$\lambda$ is the Lebesgue measure on $\R^n$, 
	$m$ is the counting measure on $\{-1,1\}$ and 
	$p_{\A,\gamma}$ is the density function defined as
\begin{align*}
p_{\A,\gamma}(g_{1:n}, v) = &\dfrac{1}{2} \bigg(p_{\A,\gamma}(g_n \mid g_{1:n-1})\times \dots \times p_{\A,\gamma }(g_{n-1} \mid g_{1:n-2}) \ldots p_{\A,\gamma}(g_1)\bigg)\\
 = & \dfrac{1}{2} \bigg( p_{\N}(g_n - \gamma(\A_n(g_{1:n-1}))) \cdot
%  &\times p_{\normal(0,\sigma^2)}(g_{n-1} - \gamma(\A_n(g_{1:n-2}))) \times \ldots \\
 \ldots \cdot  p_{\N}(g_1 - \gamma(\A_1))\bigg),
\end{align*}
where $v\in\{0,1\}$, $p_{\N}$ is the density of a $\normal(0,\sigma^2)$ random variable,
and $\cA_t$ denotes the map from the algorithm's past observations
that picks the point that is sent to the oracle in round $t$.
\todoc{Enough to consider deterministic algorithms}

%%%%%%%%%%%%%%%%%%%%%%%%%%%%%%%%%%%%%%%%%%%%%%%%%%%%%%%%%%%%%%%%%%%%%%%%%%%%%%% 

By assumption,  $\{f_+, f_-\}\subset \F$, with 
\begin{align*}
  f_+(x) := \dfrac{\epsilon}{2} (x - 1)^2 \text{ and } f_-(x) := \dfrac{\epsilon}{2} (x + 1)^2, \,\, x \in \cK\,;
\end{align*}
we will slightly abuse notation by using $f_+$ ($f_-$) in place of $f_{+1}$ (resp., $f_{-1}$).
%%%%%%%%%%%%%%%%%%%%%%%%%%%%%%%%%%%%%%%%%%%%%%%%%%%%%%%%%%%%%%%%%%%%%%%%%%%%%%% 

Clearly, $f_+$ (resp, $f_-$) is minimized at $x^*_+ = 1$ (resp. $x^*_- = -1$) with the minimum value being zero.
Hence,
%Using the fact that $f_+$ and $f_-$ are strongly convex with associated constant $\left(\dfrac{\epsilon}{2}\right)$, we obtain
\begin{align}
  f_+(x) - f_+(x^*_+)
  = &  \dfrac{\epsilon}{2} (x - 1)^2 \ge  \dfrac{\epsilon}{2}  \indic{x  < 0}. \label{eq:fv-lb}
\end{align}
Similarly,   $f_-(x) - f_-(x^*_-) \ge  \dfrac{\epsilon}{2}  \indic{x  >0}$.
We will consider the oracles $\gamma_v$ defined as 
\begin{align}
 \gamma_v(x) = \epsilon(x-v) - v\, \min(\epsilon,C_1 \delta^p) + \xi, \label{eq:oracle-1d}
\end{align}
where $\xi \sim \normal(0,\frac{C_2}{\delta^q})$; as with $f_v$, we will also use $\gamma_{+}$ ($\gamma_-$) 
to denote $\gamma_{+1}$ (resp., $\gamma_{-1}$).
The oracle is indeed a $(c_1,c_2)$ Type-I oracle, with $c_1(\delta)=C_1\delta^p$ and $c_2(\delta)=\frac{C_2}{\delta^q}$.
%%%%%%%%%%%%%%%%%%%%%%%%%%%%%%%%%%%%%%%%%%%%%%%%%%%%%%%%%%%%%%%%%%%%%%%%%%%%%%% 

The minimax error \eqref{eq:minimaxerrdef} is lower bounded by
%\footnote{$f_{+1} \equiv f_+$ and $f_{-1}\equiv f_-$.}:
\begin{align}
\MoveEqLeft 
\Delta_n^{(1)*} %\nonumber\\
  \ge  \inf_{\A} \,  \E[f_V(\hat X_n) - \inf_{x \in X}
  f_V(x)],\label{eq:avg-bd}
  \end{align}
where the expectation is w.r.t. the distribution $\P:= \dfrac{1}{2} \left(P_{\A, \gamma_+} \indic{v=+1} + P_{\A, \gamma_-}\indic{v=-1}\right)$ and $V: \Omega \to \{\pm 1 \}$ is defined by $V(g_{1:n},v) = v$.%
\footnote{Here, we are slightly abusing the notation as $\P$ depends on $\A$, but the dependence is suppressed.
In what follows, we will define several other distributions derived from $\P$, which will all depend on $\A$, but
for brevity this dependence will also be suppressed.}
%Here $\gamma_v$  is the oracle for $f_v$, for $v=+1,-1$ and is defined by $\gamma_v(x) = \epsilon(x-v) + v C_1 \delta^p$. 
Using \eqref{eq:fv-lb}, we obtain
\begin{align}
\Delta_n^{(1)*}  \ge & \inf_{\A} \dfrac{\epsilon}{2}\,  \P(\hat X_n V < 0), \label{eq:strong-convex-bd}\\
  = & \inf_{\A} \dfrac{\epsilon }{2} \, \left(\P_{+}(\hat X_n < 0) + \P_{-}(\hat X_n > 0)\right), \label{eq:Pplus}\\
  \ge &\inf_{\A} \dfrac{\epsilon }{2} \,\left(1 - \tvnorm{\P_{+}- \P_{-}}\right), \label{eq:lecam}\\
  \ge &\inf_{\A} \dfrac{\epsilon }{2}  \,\left( 1 - \left(\frac12\dkl{P_{+}}{P_{-}}\right)^{\frac{1}{2}}\right), \label{eq:pinsker}
\end{align}
where 
the equality in \eqref{eq:Pplus} uses the definitions $\P_{+}(\cdot)
:= \P(\cdot\mid V=1)$, $\P_{-}(\cdot) := \P(\cdot\mid
V=-1)$\todoa{This notation is somewhat confusing given the previous conventions.}. Further, the inequality in \eqref{eq:lecam} follows from the definition of total variation distance, while \eqref{eq:pinsker} follows from Pinsker's inequality. %\todoc{You lost a factor of two here!}


%%%%%%%%%%%%%%%%%%%%%%%%%%%%%%%%%%%%%%%%%%%%%%%%%%%%%%%%%%%%%%%%%%%%%%%%%%%%%%% 
Define $G_t$ to be the $t$th observation of $\A$: $G_t:\Omega \to \R$, with $G_t( g_{1:n}, v) = g_t$.
Let $P_+^t(g_1,\dots,g_t)$ denote the joint distribution of $G_1,\dots,G_t$ conditioned on $V=+1$.
Let $P_{+}^t(\cdot\mid g_1,\ldots,g_{t-1})$ denote the distribution of $G_t$ conditional on $V=+1$ and $G_1=g_1,\ldots,G_{t-1}=g_{t-1}$. Define  $P_{-j}^t(\cdot\mid g_1,\ldots,g_{t-1})$ in a similar fashion.
Then, by the chain rule for KL-divergences, we have
\begin{align}
\label{eq:dklchain}
&\dkl{P_{+}}{P_{-}}= \sum_{t=1}^n \int_{\R^{t-1}} \dkl{P_{+}^t(\cdot\mid g_{1:t-1})}{P_{-}^t(\cdot\mid g_{1:t-1})} d P_{+}^t( g_{1:t-1}).\nonumber
\end{align}
By the oracle's definition, $G_t \sim  \normal(\gamma_{+}(\cA_t(G_{1:t-1})),\sigma^2)$. Hence, 
$P_{+}^t(\cdot\mid g_{1:t-1})$ is the normal distribution with the mean $\cA_t(g_{1:t-1})$ and variance $\sigma^2$.
%, where $A(g_{1:t-1})$ denotes the point chosen by the algorithm given observations $g_1,\ldots, g_{t-1}$ and $\gamma_{+}$ is the gradient oracle defined earlier. 
Observe that, for any $x\in \R$, $f_+'(x) - f_-'(x) = 2\epsilon$ and hence
\begin{align}
 |\gamma_+(x) - \gamma_-(x)| 
& = | f'_+(x) - \min(\epsilon,C_1 \delta^p) - (f'_-(x)+\min(\epsilon,C_1 \delta^p)) | \nonumber \\
& = 2 (\epsilon - C_1 \delta^p)_+\,,
 \label{eq:gdiff-ub}
\end{align}
where $(\epsilon - C_1 \delta^p)_+ = \max(\epsilon - C_1 \delta^p,0)$. 
From the foregoing, 
\begin{align}
\dkl{P_{+}^t(\cdot\mid g_{1:t-1})}{P_{-}^t(\cdot\mid g_{1:t-1})}\le&\dfrac{(\gamma_{+}(A(g_{1:t-1})) - \gamma_{-}(A(g_{1:t-1})))^2}{2 C_2\delta^{-q}}\label{eq:dkgauss1}\\
 =& \dfrac{2(\epsilon-C_1\delta^p)_+^2\delta^q}{C_2}.\label{eq:dkgauss}
\end{align}
The equality \eqref{eq:dkgauss1} follows from the fact that the KL-divergence between normal distributions $\normal(\mu_1,\sigma^2)$ and $\normal(\mu_2,\sigma^2)$ is equal to $\dfrac{(\mu_1 - \mu_2)^2}{2 \sigma^2}$, while the equality \eqref{eq:dkgauss} follows from \eqref{eq:gdiff-ub}.

Plugging \eqref{eq:dkgauss} into \eqref{eq:dklchain}, we obtain
\begin{align}
\dkl{P_{+}}{P_{-}} \le \dfrac{2n(\epsilon-C_1\delta^p)_+^2 \delta^q}{C_2}.\label{eq:dkbd}
\end{align}
Note that the above bound holds uniformly over all algorithms $\A$. 
Substituting the above bound into \eqref{eq:pinsker}, we obtain 
\begin{align}
 \Delta_n^{(1)*}
  \ge & \dfrac{\epsilon}{2} \left(1 - \sqrt{
    n}  \dfrac{(\epsilon-C_1\delta^p)_+\delta^{q/2}}{C_2}
  \right)\label{eq:final-lower-bd}
\end{align}

%%%%%%%%%%%%%%%%%%%%%%%%%%%%%%%%%%%%%%%%%%%%%%%%%%%%%%%%%%%%%%%%%%%%%%%%%%%%%%%
% \paragraph{Derivations of rates:}
%  Optimizing over $\epsilon$ in \eqref{eq:final-lower-bd}, we get
% $ \left(\dfrac{\delta^{-q/2}C_2}{4\sqrt{n}} + \dfrac{C_1\delta^p}{2}\right)$. We choose
% \[
% \epsilon^* = \left(\dfrac{\delta^{-q/2}C_2}{4\sqrt{n}} + C_1\delta^p\right)
% \]
% to guarantee that $\epsilon^*>C_1 \delta^p$.
% 
%  Plugging in $\epsilon^*$, we obtain
%  \[
%  \Delta_n^{(1)*}
%  \ge \frac18 \left(\dfrac{\delta^{-q/2}C_2}{2\sqrt{n}} + C_1\delta^p \right).
%  \]
% 
% Substituting $p=1$ and $q=2$, we obtain
% \[
% \delta^*= \sqrt{\dfrac{C_2}{2C_1}}\dfrac{1}{n^{\frac{1}{4}}} \text{ and } 
% \Delta_n^{(1)*} \ge \left(\dfrac{\sqrt{2C_1C_2}}{4}\right)\dfrac{1}{n^{\frac{1}{4}}}.
% \]
% 
% On the other hand, substituting $p=q=2$, we obtain
% \[
% \delta^*= \left(\dfrac{C_2}{2C_1}\right)^{\frac{1}{3}}\dfrac{1}{n^{\frac{1}{6}}} \text{ and } \Delta_n^{(1)*} \ge \left(\dfrac{C_1^{\frac{1}{3}}C_2^{\frac{2}{3}}}{4\, 2^{\frac{1}{3}}}\right)\dfrac{1}{n^{\frac{1}{3}}}.
% \]

\paragraph{Derivation of the rates uniformly for all $\delta$:}
Observe that $\max_{\delta\ge 0} (\epsilon - C_1 \delta^p)_+ \delta = \max_{\delta\ge 0} (\epsilon - C_1 \delta^p) \delta$. 
In other words, minimizer $\delta^*$ of \eqref{eq:final-lower-bd} can never be in the zero arm of the positive part function, i.e., $(\epsilon - C_1 \delta^p)_+ $, and hence we minimize \eqref{eq:final-lower-bd} by replacing the positive part with $(\epsilon - C_1 \delta^p)$.

Assuming $q \ge 2$, we obtain the following 
\[
\delta^*=\left(\frac{\epsilon q}{2C_1(p+q/2)}\right)^{1/p}
\]
and for any choice of $\delta$, the following lower bound for the minimax error:
\[
\Delta_n^{(1)*} \ge \dfrac{\epsilon}{2} \left(1 - \sqrt{n}  K_1 \epsilon^{\frac{p+\tfrac{q}{2}}{p}}\right), \text{ where } K_1 = \dfrac{p}{C_2(p+\tfrac{q}{2})} \left(\dfrac{q}{2C_1(p+\tfrac{q}{2})}\right)^{\frac{q}{2p}}.
\]
 
Plugging in $\epsilon = \left(\dfrac{1}{2\sqrt{n} K_1} \right)^{\frac{p}{p+\frac{q}{2}}}$, we obtain
\begin{align}
\Delta_n^{(1)*} \ge \dfrac{1}{4}\left(\dfrac{1}{2 K_1 \sqrt n}\right)^{\frac{p}{p+\frac{q}{2}}}.
\label{eq:lb-pq}
\end{align}

Now, when $q=2$ and $p=1$, this simplifies to
\[
\Delta_n^{(1)*} \ge \dfrac{\sqrt{C_1 C_2}}{2\sqrt{2}} n^{-1/4}, \text{ for } \epsilon=\sqrt{2C_1C_2} n^{-1/4}.
\]
On the other hand, for $p=q=2$, we get
\[
\Delta_n^{(1)*} \ge  \frac{3}{8}\left(\frac{C_1 C_2^2}{16}\right)^{1/3} n^{-1/3} \text{ for } \epsilon=\frac{3}{2}\left(\frac{C_1 C_2^2}{16}\right)^{1/3}n^{-1/3}.
\]



%%%%%%%%%%%%%%%%%%%%%%%%%%%%%%%%%%%%%%%%%%%%%%%%%%%%%%%%%%%%%%%%%%%%%%%%%%%%%%%
%%%%%%%%%%%%%%%%%%%%%%%%%%%%%%%%%%%%%%%%%%%%%%%%%%%%%%%%%%%%%%%%%%%%%%%%%%%%%%%
%%%%%%%%%%%%%%%%%%%%%%%%%%%%%%%%%%%%%%%%%%%%%%%%%%%%%%%%%%%%%%%%%%%%%%%%%%%%%%%
\subsubsection{Generalization to $d$ dimensions}

Let $\cK \subset \R^d$.
Define $f_v$ for any $v\in \{-1,+1\}^d$ as follows: For any $x \in \cK$, 
\begin{align*}
  f_v(x) :=& \sum_{i=1}^d f^{(i)}_{v_i}(x_i), \,\,  \text{ where }
  f^{(i)}_{v_i}(x_i) := \dfrac{\epsilon}{2} (x_i - v_i)^2, \,\, \text{ for } i=1,\ldots,d.
\end{align*}
By definition, $f_v$ is additively separable and minimized at $x^*_v=v$ with the minimum value being zero.
We consider oracles $\gamma_v$ defined by
$\gamma_v(x) = \sum_{i=1}^d \gamma_{v_i}^{(i)}e_i$, where 
$e_i$ is the $i$th unit vector in the standard Euclidean base,
$\gamma_{v_i}^{(i)} = \epsilon(x_i-v_i) - v_i\, \frac{C_1 \delta^p}{\sqrt{d}} + \xi_i$, where $\xi_i \sim \normal(0,\frac{C_2}{d\delta^q})$. 
It is evident that the $\gamma$ is a $(c_1,c_2)$ Type-I oracle, with $c_1(\delta)=C_1\delta^p$ and $c_2(\delta)=\frac{C_2}{\delta^q}$.

Let $\Delta_n^{(d)*}$ denote the minimax error $\Delta_n^*\left(\F_d, C_1\delta^p,\frac{C_2}{\delta^q}\right)$ for the $d$-dimensional family of functions $\F_d$, which is composed of $f_v$s, as defined above. We have
\begin{align*}
 \Delta_n^{(d)*} \ge& \inf_\A  \sup_{v\in \{\pm 1 \}^d}\sum_{i=1}^d L_i(v), \text{ where, for } i=1,\ldots,d,\\
 L_i(v) := & \int f^{(i)}_{v_i}(\hat X_{ni}(g_{1:n}) )\, dP_{\A,\gamma_v,\sigma^2}(g_{1:n}),  
\end{align*}
where $\hat{X}_{ni}$ denotes the $i$th coordinate of the vector $\hat{X}_n \in \R^d$ chosen by $\A$, for $i=1,\ldots,d$
(again, to simplify the notation the dependence of $\hat{X}_n$ on $\A$ is suppressed).


% Fix an $i$ and consider the loss $L_i(v)$.
% We now argue that, since $f_v$ is separable, any algorithm $\A$ that uses information in dimensions other than $i$ in the gradient samples of the oracle, can be replaced by $d$ algorithms $\A_1^*$, $\dots$, $\A_d^*$ such that $\A_i^*$ is only using information for dimension $i$ and is only predicting the $i$th coordinate at the end, and the sum of losses they incur is utmost that of $\A$ in the worst-case.

\begin{lemma}
For any algorithm $\A$, there exist  $d$ ``$1$-dimensional'' algorithms, $\A_i^*$, $1\le i \le d$,
the $i$th algorithm interacting with only a one dimensional oracle, supplying gradient estimates about $f_{v_i}^{(i)}$ 
such that if $L^{\A}(v)$ is the loss of $\A$ on environment with index $v$ and the following holds
the 
\begin{align}
\label{eq:onedimlb}
% \begin{split}
% \MoveEqLeft 
&\max_{v\in \{\pm 1\}^d} L^{\A}(v) 
\ge   \max_{v_1\in \{\pm 1\}} L_1^{\A_1^*}(v_1) + \dots + \max_{v_d\in \{\pm 1 \}} L_d^{\A^*_d}(v_d)\,.
% \end{split}
\end{align}
\end{lemma}
\begin{proof}
Let $v_{-i}=(v_1,\ldots,v_{i-1},v_{i+1},\ldots,v_d)$ denote a $d-1$ dimensional vector that removes the $i$th coordinate from $v$. By an abuse of notation, we let $v=(v_i, v_{-i})$. 
We now describe the construction of $\A^*=(\A_1^*, \ldots, \A_d^*)$.
We describe $\A_1^*$ only, other algorithm components are formed in a similar manner.

To define $\A_1^*$, consider the solution of the following max-min problem:
\[
\max_{v_1} \min_{v_{-1}} L_1(v_1, v_{-1})).
\]
Let the optimal solution of this problem be denoted by $(\hat{v}_1^*,v_{-1}^*)$.
The algorithm will make use of $v_{-1}^*$ (which it does have access to) in the following manner:
Recall that $L$ and $L_1$ are defined as a function of $(\gamma_v)_{v\in \{\pm\}^d}$.
Let $v\in \{\pm 1\}^d$ be fixed, unknown to $\A$ and $\A_1^*$. 
The algorithm construction will make use of $\gamma_v$ and $\gamma_{(1,v_{-1}^*)}$ in the following manner:
At time $t=1$, $\A_1^*$ consults $\A$ to get $\delta$ and the point $X_1\in \cK$.
Next, it asks the oracle $\gamma_v$ for a $d$ dimensional gradient estimate of $f_v$ at $X_1$ with tolerance $\delta$,
obtaining $G_1'$.
At the same time, it also asks $\gamma_{(1,v_{-1}^*)}$ for another $d$ dimensional gradient estimate (this has nothing to do with $f_v$), obtaining $G_1''$.
Then, $G_1 = (G_{1,1}',G_{1,-1}'')$ is fed to $\A$, which returns $X_2\in \cK$. From this point up to the end the process is similar: Both oracles are queried and the gradient to be fed to $\A$ is synthetized by putting together the first component of the gradient estimate from $\gamma_v$ and the last $d-1$ components obtained from $\gamma_{(1,v_{-1}^*)}$.
Finally, at the end, $\A$ produces $\hat{X}_n$, and $\A_1^*$ returns $\hat{X}_{n,1}$.
Clearly, by construction and the definition of $L_1$,
the expected loss $L^*_1(v_1)$ of $\A_1^*$ on $f_{v_1}^{(1)}$ is $L_1(v_1,v_{-1}^*)$, while $\A_1^*$ only uses the first coordinates of the gradients that it gets from the oracle $\gamma_v$.
With similar definitions, $L^*_i(v_i)$, the expected loss of $\A_i^*$ on $f_{v_i}^{(i)}$ is $L_i(v_i,v_{-i}^*)$, $2\le i \le d$.
Now, notice that $L_i(\hat{v}_i^*,v_{-i}^*) = \min_{v_{-i} } L_i(\hat{v}_i^*,v_{-i}) \le L_i(\hat{v}_i^*, \hat{v}_{-i}^*)$,
where $\hat{v}_{-i}^*$ is the vector obtained by discarding the $i$th component of $\hat{v}^* = (\hat{v}_1^*,\dots,\hat{v}_d^*)$.
Thus,
\begin{align*}
&\max_{v\in \{\pm 1 \}^d }
L^*_1(v_1) + \dots + L^*_d(v_d) \\
&   = \max_{v\in \{\pm 1 \}^d } L_1(v_1,v_{-1}^*) + \dots + L_d(v_d,v_{-d}^*)\\
& = \max_{v_1\in \{\pm 1\}} L_1(v_1,v_{-1}^*) + \dots +  \max_{v_d\in \{\pm 1\}} L_d(v_d,v_{-d}^*)\\
& = L_1(\hat{v}^*_1,v_{-1}^*) + \dots +   L_d(\hat{v}^*_d,v_{-d}^*)\\
& \le L_1(\hat{v}^*_1,\hat{v}_{-1}^*) + \dots +  L_d(\hat{v}^*_d,\hat{v}_{-d}^*)\\
& \le L(\hat{v}^*) \le \max_{v\in \{\pm 1\}^d } L(v)\,,
\end{align*}
which was the claim to be proven.
\end{proof}

\paragraph{Main proof:}
The minimax error $\Delta_n^{(d)*}$ can be lower bounded as follows:
\begin{align}
\Delta_n^{(d)*}  &\ge  \inf_{\A} \max_{v\in {\pm 1}^d} L^\A(v) \nonumber\\
 & \ge \inf_{\A_1} \max_{v_1\in \{\pm 1\}} L_1^{\A_1}(v_1) + \dots + 
 			\inf_{A_d} \max_{v_d\in \{\pm 1 \}} L_d^{\A_d}(v_d) \label{eq:splitA}\\
%             \ge & \inf_{\A_1^*} L_1(v^*) + \ldots + \inf_{\A_d^*} L_d(v^*) \label{eq:splitA}\\
               & \ge d \Delta_n^{*}\left(\F_1,\frac{C_1\delta^p}{\sqrt d}, \frac{C_2}{d\delta^q}\right)\,, \label{eq:dlb}
\end{align}
where in inequality \eqref{eq:splitA}, algorithms $\A_i$ are interacting with the respective one-dimensional oracles $\gamma^{(i)}_{v_i}$ only 
and the inequality follows by~\eqref{eq:onedimlb},
while the inequality \eqref{eq:dlb} follows from the fact that each infimum term in \eqref{eq:splitA} is solving a $1$-dimensional problem and hence the bound in \eqref{eq:final-lower-bd} applies with $C_1$ replaced by $C_1/\sqrt{d}$ and $C_2$ replaced by $C_2/d$. The resulting bound is denoted by $\Delta_n^{*}\left(\F_1,\frac{C_1\delta^p}{\sqrt d}, \frac{C_2}{d\delta^q}\right)$. 
% is the one-dimensional minimax error, which is lower-bounded in \eqref{eq:final-lower-bd}. \todoc{However, I would have written just the $1$d final bound there, or $\Delta_n^*( \dots )$ with some notation, because we don't want to redo this whole optimization business..}

% Using the one-dimensional minimax error $\Delta_n^{(1)*}$, we obtain
% \[\Delta_n^* \ge d \Delta_n^{(1)*}.\]
\paragraph{Rates:}\ \\
General $p$ and $q\ge 2$:
\begin{align}
\Delta_n^{(d)*} \ge \dfrac{1}{4} d^{\frac{q}{(4p+2q)}} \left(\dfrac{1}{2 K_1 \sqrt n}\right)^{\frac{p}{p+\frac{q}{2}}} \text{ for } 
\epsilon = d^{\frac{-(4p+q)}{(4p+2q)}}\left(\dfrac{1}{2\sqrt{n} K_1} \right)^{\frac{p}{p+\frac{q}{2}}}.
\label{eq:lb-pq-d}
\end{align}

Case $p=1$ and $q=2$:
\begin{align*}
% \Delta_n^{*}\left(\F_1,\frac{C_1\delta^p}{\sqrt d}, \frac{C_2}{d\delta^q}\right) \ge &\frac{1}{2}\sqrt{\frac{C_1 C_2}{2 d^{\frac{3}{2}}}} n^{-1/4} \text{ and hence,} \\
\Delta_n^{(d)*}  \ge& \frac{\sqrt{C_1 C_2}}{2\sqrt 2}\left(\frac{d}{n}\right)^{\frac{1}{4}}.
\end{align*}

Case $p=q=2$:
\begin{align*}
% \Delta_n^{*}\left(\F_1,\frac{C_1\delta^p}{\sqrt d}, \frac{C_2}{d\delta^q}\right) \ge &\dfrac{3}{8}\left(\frac{C_1 C_2^2}{16 d^{\frac{5}{2}}}\right)^{1/3} n^{-1/3} \text{ and hence,} \\
\Delta_n^{(d)*}  \ge& \dfrac{3}{8}\left(\frac{  C_1 C_2^2}{16 }\right)^{1/3} \left(\frac{ \sqrt{d} }{ n}\right)^{1/3}.
\end{align*}

%%% Local Variables: 
%%% mode: latex
%%% TeX-master: "bgo"
%%% End: 

\subsubsection{Bounds for strongly convex functions}
We follow the notational convention used earlier for convex functions in one dimension. 

We consider functions $f_v$, for any $v \in \{-\nu,+\nu\}$, defined as
By assumption,  $\{f_+, f_-\}\subset \F$, with 
\begin{align*}
  f_v(x) := \dfrac{1}{2} x^2 - v x \,\, x \in \cK\,;
\end{align*}
we will slightly abuse notation by using $f_+$ ($f_-$) in place of $f_{+\nu}$ (resp., $f_{-\nu}$).
%%%%%%%%%%%%%%%%%%%%%%%%%%%%%%%%%%%%%%%%%%%%%%%%%%%%%%%%%%%%%%%%%%%%%%%%%%%%%%% 

Clearly, $f_v$ is minimized at $x^*_v = v$.
By the definition of $f_v$, we have
%Using the fact that $f_+$ and $f_-$ are strongly convex with associated constant $\left(\dfrac{\epsilon}{2}\right)$, we obtain
\begin{align}
  f_v(x) - f_v(x^*_v)
\ge  \dfrac{\nu^2}{2}  \indic{x v  < 0}. \label{eq:fv-lb-sc}
\end{align}
We will consider the oracles $\gamma_v$ defined as 
\begin{align}
 \gamma_v(x) = \epsilon(x-v) - \sign(v)\, \min(v,C_1 \delta^p) + \xi, \label{eq:oracle-1d}
\end{align}
where $\xi \sim \normal(0,\frac{C_2}{\delta^q})$; as with $f_v$, we will also use $\gamma_{+}$ ($\gamma_-$) 
to denote $\gamma_{+1}$ (resp., $\gamma_{-1}$).
The oracle is indeed a $(c_1,c_2)$ Type-I oracle, with $c_1(\delta)=C_1\delta^p$ and $c_2(\delta)=\frac{C_2}{\delta^q}$.
%%%%%%%%%%%%%%%%%%%%%%%%%%%%%%%%%%%%%%%%%%%%%%%%%%%%%%%%%%%%%%%%%%%%%%%%%%%%%%% 

Using arguments similar to those in the proof of lower bound for convex functions, we obtain
\begin{align}
\Delta_n^{(1)*} %\nonumber\\
  \ge  &\inf_{\A} \dfrac{\epsilon }{2}  \,\left( 1 - \left(\frac12\dkl{P_{+}}{P_{-}}\right)^{\frac{1}{2}}\right), \label{eq:pinskersc}
\end{align}
Note that $P_+$ (resp. $P_-$) is $\P$ conditioned on the event $V=+\nu$ (resp. $V=-\nu$), since $v$ takes values in $\{+\nu,-\nu\}$. The conditioning in the earlier proof for convex function was on $V=+1$ for $P_+$ as $v$ there was taking values $\{+1,-1\}$.
%%%%%%%%%%%%%%%%%%%%%%%%%%%%%%%%%%%%%%%%%%%%%%%%%%%%%%%%%%%%%%%%%%%%%%%%%%%%%%% 

Observe that, for any $x\in \R$, $f_+'(x) - f_-'(x) = 2v$ and hence
\begin{align}
 |\gamma_+(x) - \gamma_-(x)| 
& = | f'_+(x) - \min(v,C_1 \delta^p) - (f'_-(x)+\min(v,C_1 \delta^p)) | \nonumber \\
& = 2 (\nu - C_1 \delta^p)_+.
 \label{eq:gdiff-ub-sc}
\end{align}
From the foregoing, 
\begin{align}
 \MoveEqLeft \dkl{P_{+}^t(\cdot\mid g_{1:t-1})}{P_{-}^t(\cdot\mid g_{1:t-1})}\nonumber\\
 \le & \dfrac{2(\nu-C_1\delta^p)_+^2\delta^q}{C_2},\label{eq:dkgausssc}
\end{align}
where the inequality \eqref{eq:dkgausssc} follows from \eqref{eq:gdiff-ub-sc}.
Thus, we obtain
\begin{align}
\dkl{P_{+}}{P_{-}} \le \dfrac{2n(\nu-C_1\delta^p)_+^2 \delta^q}{C_2}.\label{eq:dkbd}
\end{align}
Substituting the above bound into \eqref{eq:pinskersc}, we obtain 
\begin{align}
 \Delta_n^{(1)*}
  \ge & \dfrac{\nu^2}{2} \left(1 - \sqrt{
    n}  \dfrac{(\nu-C_1\delta^p)_+\delta^{q/2}}{C_2}
  \right)\label{eq:final-lower-bd-sc}
\end{align}


\paragraph{Derivation of the rates uniformly for all $\delta$:}
Replacing the positive part function in \eqref{eq:final-lower-bd-sc} and optimizing $\delta$, we obtain the following for $q \ge 2$:
\[
\delta^*=\left(\frac{\nu q}{2C_1(p+q/2)}\right)^{1/p}
\]
and for any choice of $\delta$, the minimax error $\Delta_n^{(1)*}$ satisfies
\[
\Delta_n^{(1)*} \ge \dfrac{\nu^2}{2} \left(1 - \sqrt{n}  K_1 \nu^{\frac{p+\tfrac{q}{2}}{p}}\right), \text{ where } K_1 = \dfrac{p}{C_2(p+\tfrac{q}{2})} \left(\dfrac{q}{2C_1(p+\tfrac{q}{2})}\right)^{\frac{q}{2p}}.
\]

Plugging in $\nu = \left(\dfrac{1}{2\sqrt{n} K_1} \right)^{\frac{2p}{p+\frac{q}{2}}}$, we obtain
\begin{align}
\Delta_n^{(1)*} \ge \dfrac{1}{4}\left(\dfrac{1}{2 K_1 \sqrt n}\right)^{\frac{2p}{p+\frac{q}{2}}}.
\label{eq:lb-pq}
\end{align}

Now, when $q=2$ and $p=1$, this simplifies to
\[
\Delta_n^{(1)*} \ge \dfrac{C_1 C_2}{2} n^{-1/2}, \text{ for } \nu = 2C_1C_2 n^{-1/2}.
\]
On the other hand, for $p=q=2$, we get
\[
\Delta_n^{(1)*} \ge  \frac{9}{4}\left(\frac{C_1 C_2^2}{16}\right)^{1/3} n^{-2/3} \text{ for } \nu=\frac{9}{2}\left(\frac{C_1 C_2^2}{16}\right)^{1/3}n^{-2/3}.
\]

%%%%%%%%%%%%%%%%%%%%%%%%%%%%%%%%%%%%%%%%%%%%%%%%%%%%%%%%%%%%%%%%%%%%%%%%%%%%%%%
%%%%%%%%%%%%%%%%%%%%%%%%%%%%%%%%%%%%%%%%%%%%%%%%%%%%%%%%%%%%%%%%%%%%%%%%%%%%%%%

\paragraph{Generalization to $d$ dimensions:}
The proof in $d$ dimensions for strongly convex functions follows in similar fashion as that for the case of convex functions and we obtain the following rates:
General $p$ and $q\ge 2$:
\begin{align}
\Delta_n^{(d)*} \ge \dfrac{1}{4} d^{\frac{-2p}{(2p+q)}} \left(\dfrac{1}{2 K_1 \sqrt n}\right)^{\frac{2p}{p+\frac{q}{2}}} \text{ for } 
\nu = d^{\frac{-(4p+q)}{(2p+q)}}\left(\dfrac{1}{2\sqrt{n} K_1} \right)^{\frac{p}{p+\frac{q}{2}}}.
\label{eq:lb-pq-d}
\end{align}

Case $p=1$ and $q=2$:
\begin{align*}
% \Delta_n^{*}\left(\F_1,\frac{C_1\delta^p}{\sqrt d}, \frac{C_2}{d\delta^q}\right) \ge &\frac{1}{2}\sqrt{\frac{C_1 C_2}{2 d^{\frac{3}{2}}}} n^{-1/4} \text{ and hence,} \\
\Delta_n^{(d)*}  \ge& \dfrac{C_1 C_2}{2} n^{-1/2} d^{-1/2}.
\end{align*}

Case $p=q=2$:
\begin{align*}
% \Delta_n^{*}\left(\F_1,\frac{C_1\delta^p}{\sqrt d}, \frac{C_2}{d\delta^q}\right) \ge &\dfrac{3}{8}\left(\frac{C_1 C_2^2}{16 d^{\frac{5}{2}}}\right)^{1/3} n^{-1/3} \text{ and hence,} \\
\Delta_n^{(d)*}  \ge& \frac{9}{4}\left(\frac{C_1 C_2^2}{16}\right)^{1/3} n^{-2/3} d^{-2s/3}.
\end{align*}
