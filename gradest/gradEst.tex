\documentclass[11pt,letterpaper,english]{article}

%!TEX root =  bgo-cam-ready.tex
% please do not delete or change the first line (needed by Csaba's editor)

%%%%%%%%%%%%%%%%%%%% Added for arxiv report %%%%%%%%%%%%%%%%%%%%%
\usepackage[T1]{fontenc}
\usepackage[utf8]{inputenc}
\usepackage[english]{babel}
\usepackage{authblk}
\usepackage{times}
\usepackage[margin=1in]{geometry}
%%%%%%%%%%%%%%%%%%%%%%%%%%%%%%%%%%%%%%%%%%%%%%%%%%%%%%%%%%%%%%%%%


\usepackage{etex}
\usepackage[utf8]{inputenc}
\usepackage{amsmath}
\usepackage{amsfonts}
\usepackage{amssymb}
%\usepackage{amsthm}
\usepackage{mathtools}
\usepackage{graphicx}
\usepackage{algorithm}
\usepackage[numbers]{natbib}
\usepackage[noend]{algpseudocode}
\usepackage{url}
\usepackage{bm}
% \usepackage[margin=2.5cm]{geometry}
\setlength{\marginparwidth}{20mm}
\usepackage{booktabs}
\usepackage{multirow}
\usepackage{colortbl}
\usepackage{hhline}
\usepackage{paralist}
\usepackage{xspace}
\usepackage{sidecap}

\usepackage{caption}
\usepackage{subcaption}
\usepackage{tikz}
\usepackage{pgfplots}
\usepackage{makecell}
\usetikzlibrary{positioning,arrows.meta,shapes,calc,patterns,fit,decorations}
\usepgfplotslibrary{fillbetween}
\pgfplotsset{compat=1.10}

%%% Added by Prashanth: the following two lines are used to make compilation faster with tikz. Each tikzfigure will become a seprate tex job and the figures get stored in a sub-folder named tikz. 
%%%% Dont forget to add --shell-escape(or -enable-write18 if you are using MikTeX) to the pdflatex commandline
\usetikzlibrary{external}
\tikzexternalize[prefix=tikz/]
%%%%%%%%%%

\usepackage[disable]{todonotes}
%\usepackage{todonotes}
%%% Added by Prashanth: The tikz externalize business to speed up compilation is incompatible with todonotes, unless we add the following workaround that disables tikz-externalize in todo definition
\makeatletter
\renewcommand{\todo}[2][]{\tikzexternaldisable\@todo[#1]{#2}\tikzexternalenable}
\makeatother
%%%%%%%%%%%

\newcommand{\todoc}[2][]{\xspace\todo[color=red!20!white,size=\tiny,#1]{Cs: #2}}
\newcommand{\todox}[2][]{\xspace\todo[color=orange!20!white,size=\tiny,#1]{X: #2}}
\newcommand{\todoa}[2][]{\xspace\todo[color=blue!20!white,size=\tiny,#1]{A: #2}}
\newcommand{\todop}[2][]{\xspace\todo[color=green!20!white,size=\tiny,#1]{P: #2}}

\makeatletter
\def\BState{\State\hskip-\ALG@thistlm}
\makeatother


%\usepackage{amsthm}
%\newtheorem{remark}{Remark}
%\newtheorem{proposition}{Proposition}
%\newtheorem{definition}{Definition}
% \newtheorem{theorem}{Theorem}
\usepackage[amsmath,standard,thmmarks]{ntheorem} % ntheorem makes cleveref work properly
% \newtheorem{corollary}[theorem]{Corollary}
% \newtheorem{lemma}[theorem]{Lemma}

% Hyperlinks make it very easy to navigate an electronic document.
% In addition, this is where you should specify the thesis title
% and author as they appear in the properties of the PDF document.
% Use the "hyperref" package
% N.B. HYPERREF MUST BE THE SECOND TO LAST PACKAGE LOADED; ADD ADDITIONAL PKGS ABOVE
\usepackage[pdftex,letterpaper=true,pagebackref=false]{hyperref} % with basic options
		% N.B. pagebackref=true provides links back from the References to the body text. This can cause trouble for printing.
\hypersetup{
    plainpages=false,       % needed if Roman numbers in frontpages
    pdfpagelabels=true,     % adds page number as label in Acrobat's page count
    bookmarks=true,         % show bookmarks bar?
    unicode=false,          % non-Latin characters in Acrobat’s bookmarks
    pdftoolbar=true,        % show Acrobat’s toolbar?
    pdfmenubar=true,        % show Acrobat’s menu?
    pdffitwindow=false,     % window fit to page when opened
    pdfstartview={FitH},    % fits the width of the page to the window
    pdftitle={(Bandit) Convex Optimization with Biased Noisy Gradient Oracles},    % set
    pdfauthor={},    % set
%     pdfauthor={D\'avid Szepesv\'ari},    % set
%    pdfsubject={Subject},  % subject: CHANGE THIS TEXT! and uncomment this line
%    pdfkeywords={keyword1} {key2} {key3}, % list of keywords, and uncomment this line if desired
    pdfnewwindow=true,      % links in new window
    colorlinks=true,        % false: boxed links; true: colored links
    linkcolor=blue,         % color of internal links
    citecolor=blue,        % color of links to bibliography
    filecolor=magenta,      % color of file links
    urlcolor=cyan           % color of external links
}
% N.B. CLEVEREF MUST BE LOADED; AFTER HYPERREF
\usepackage[capitalize]{cleveref}
\newtheorem{ass}{Assumption}
\crefname{ass}{Assumption}{Assumptions}

%%%%%%%%%%%%%%%%%%%%%%%%%%%%%%%%%%%%%%%%%%%%%%%%%%%%%%%%
\newcommand{\cE}{\mathcal{E}}
\newcommand{\cF}{\mathcal{F}}
\newcommand{\cA}{\mathcal{A}}
\newcommand{\cK}{\mathcal{K}}
\newcommand{\cB}{\mathcal{B}}
\newcommand{\cY}{\mathcal{Y}}
\newcommand{\R}{\mathbb{R}}
\newcommand{\EE}[1]{\mathbb{E}\left[#1\right]}
\newcommand{\ip}[1]{\langle #1 \rangle}
\newcommand{\norm}[1]{\|#1\|}
\newcommand{\scnorm}[1]{\left\|#1\right\|}
\DeclareMathOperator{\fspan}{span}
\DeclareMathOperator\erf{erf}
\DeclareMathOperator{\argmin}{argmin}
\DeclareMathOperator{\argmax}{argmax}
\DeclareMathOperator{\interior}{int}
\DeclareMathOperator{\dom}{dom}
\DeclareMathOperator{\esssup}{ess\,sup}
\DeclareMathOperator{\essup}{ess\,sup}
\newcommand\numberthis{\addtocounter{equation}{1}\tag{\theequation}}
\newcommand{\DR}{D_{\mathcal{R}}}
\newcommand{\g}{\gamma}
\newcommand{\og}{\overline{\gamma}}


%%% Prashanth: some of my macros..need to remove duplicates
\newcommand{\C}{\mathcal{C}}
\newcommand{\A}{\mathcal{A}}
\newcommand{\I}{\mathcal{I}}
\newcommand{\B}{\mathcal{B}}
\newcommand{\V}{\mathcal{V}}
\newcommand{\F}{\mathcal{F}}
\newcommand{\N}{\mathcal{N}}
\newcommand{\E}{\mathbb{E}}
\renewcommand{\P}{\mathbb{P}}
\newcommand{\G}{\mathcal{G}}
\renewcommand{\O}{\mathcal{O}}
\newcommand{\D}{\mathcal{D}}
\newcommand{\cD}{\mathcal{D}}
\newcommand{\K}{\mathcal{K}}
\newcommand{\cR}{\mathcal{R}}
\newcommand{\tf}{\tilde{f}}

\newcommand{\noise}{\psi}
\newcommand{\noiseCap}{\Psi}

\newcommand{\dnorm}[1]{\norm{#1}_*} % Dual norm
\def\<{\left\langle} % Angle brackets
\def\>{\right\rangle}




\newcommand{\sign}{\mathop{\rm sign}}
% \newcommand{\norm}[1]{\left\|{#1}\right\|} % A norm with 1 argument
\newcommand{\tvnorm}[1]{\norm{#1}_{\rm TV}}
\newcommand{\dkl}[2]{D_{\rm kl}\left({#1} |\!| {#2} \right)}
\newcommand{\normal}{\mathsf{N}}  % Normal distribution
\newcommand{\openleft}[2]{\left({#1},{#2}\right]} % Interval open on left
\def\mbf#1{\mathbf{#1}}
\def\indic#1{\mbf{1}\left\{#1\right\}} % Indicator function
\newcommand{\tr}{^\mathsf{\scriptscriptstyle T}} % transpose



\begin{document}
Given a vector space $\mathcal{X} \subset \R^d$, $\mu$ is a measure on it, $f: \R^d \to \R$ is a measurable function. The convolution of 
$f$ and $\mu$ at $x\in \mathcal{X}$ is defined as
\begin{align*}
\left( f*\mu \right) (x) := \int f(x-y)\mu (d y)
= \int f(y)\mu(x-dy) = \int f(y)\mu_x(dy)\,,
\end{align*}
where $\mu_x(A) = \mu(x-A)$ for any $A\in Borel(\mathcal{X})$. $Borel(\mathcal{X})$ is the Borel $\sigma$-field of $\mathcal{X}$.

Now we can see $\left\lbrace \mu_x: x\in \mathcal{X} \right\rbrace$ is a family of measures with the parameter $x$. The weak differentiability of $\mu_x$ can be defined as follows.
\begin{definition}
Fix $x_0 \in \mathcal{X}$, we say $\mu_x$ is weakly differentiable at $x_0$, if there exists a finite measure $\mu'_{x_0}: \mathcal{X}\to \mathcal{X}^*$ such that
\begin{align*}
\forall f \in \mathcal{C}_B(\mathcal{X}), \quad
\int f(y)\mu_{x_0+\xi}(dy) - \int f(y)\mu_{x_0}(dy)
-\ip{\int f(y)\mu'_{x_0}(dy), \xi} = \sigma(\norm{\xi}) \,,
\end{align*}
which goes to $0$ as $\norm{\xi} \to 0$. Note that $\norm{\mu'_{x_0}(A)}_*< \infty$ for any $A\in Borel(\mathcal{X})$, $\mathcal{C}_B(\mathcal{X})$ denotes the continuous, bounded, real-valued functions defined on $\mathcal{X}$.
\end{definition}
Consequently, we call $\mu'_x$ the weak derivative of $\mu_x$. And $\mu'_x$ is unique.

\begin{proposition}
\label{prop:diffconvolution}
For any $f \in \mathcal{C}_B(\mathcal{X})$, the differentiability of $f*\mu$ is equivalent to the differentiability of $\mu$ in the weak sense.
Moreover, $\forall x_0\in \mathcal{X}$, 
$\dfrac{d}{dx} (f*\mu) (x_0) = \int f(y)\mu'_{x_0}(dy)$.
\end{proposition}
\begin{proof}
Denote $\hat{f}(x) := (f*\mu)(x)$. Given any $x_0\in \mathcal{X}$, if $\hat{f}$ is differentiable at $x_0$, by the definition of differentiability, there exists $g \in \mathcal{X}^*$ such that
\begin{align}
\label{eq:diffhatf}
\hat{f}(x_0+\xi) - \hat{f}(x_0) 
= \int f(y)\mu_{x_0+\xi}(dy) - \int f(y)\mu_{x_0}(dy)
= \ip{g, \xi}+\sigma(\norm{\xi}) \,.
\end{align}
Choose some $\mathcal{X}^*$-valued measure $\lambda$ such that 
$g = \int f(y)\lambda(dy)$, then plug this into \eqref{eq:diffhatf}. Recall the definition of weak differentiability, combined with arbitrary $f$, we can see that $\lambda$ is the weak derivative of $\mu$. Hence, the differentiability of $(f*\mu)$ is a sufficient condition of the differentiability of $\mu$. 

Next, to prove its necessity, let $g=\int f(y)\mu'_{x_0}(dy)$. It is obvious that \eqref{eq:diffhatf} will also hold. Therefore, $(f*\mu)$ is differentiable. Its gradient $g$ at $x_0$ is $\int f(y)\mu'_{x_0}(dy)$.

Now we want to give a probability based representation of $\dfrac{d}{dx} (f*\mu) (x_0)$.

By \cref{prop:diffconvolution}, we have for any $x\in \mathcal{X}$,
\begin{align*}
\dfrac{d}{dx} (f*\mu) (x) = \int f(y)\mu'_{x}(dy)
= \int f(x-y)\mu'(dy) \,,
\end{align*}
where the second equality results from the definition $\mu'_{x}(A)=\mu'(x-A)$. Since $\mu':\mathcal{X}\to \mathcal{X}^*$ is a finite measure, we can define 
\begin{align*}
\forall A \in Borel(\mathcal{X}), \quad
\nu(A) = \dfrac{\norm{\mu'(A)}_*}{\norm{\mu'(\mathcal{X})}_*}\,,
\end{align*}
which is a probability measure. Hence 
$\dfrac{d}{dx} (f*\mu) (x) = \norm{\mu'(\mathcal{X})}_* \int f(x-y) \dfrac{d\mu'}{d\nu}(y) \nu(dy)$, where $\dfrac{d\mu'}{d\nu}$ is the Radon-Nikodym derivative of $\mu'$ w.r.t $\nu$.

Further assume $\mu$ is a probability measure with the distribution function $p(\cdot)$, i.e., $\mu(dx) = p(x)dx$. If $p(x)$ is differentiable, then
$\nu(dy) = \dfrac{\norm{p'(y)}_*}{\int \norm{p'(x)}_*dx}dy$.
So we have
\begin{align*}
\dfrac{d}{dx} (f*\mu) (x) = U \int f(x-y) q(y) \nu (dy)\,,
\end{align*}
where $U = \int \norm{p'(x)}_*dx$, $q(y) = \dfrac{p'(y)}{\norm{p'(y)}_*}$.



\end{proof}
\end{document}
