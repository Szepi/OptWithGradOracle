\documentclass[11pt,letterpaper,english]{article}

%!TEX root =  bgo.tex
% please do not delete or change the first line (needed by Csaba's editor)

\usepackage{etex}
\usepackage[utf8]{inputenc}
\usepackage{amsmath}
\usepackage{amsfonts}
\usepackage{amssymb}
%\usepackage{amsthm}
\usepackage{mathtools}
\usepackage{graphicx}
\usepackage{algorithm}
\usepackage{natbib}
\usepackage[noend]{algpseudocode}
\usepackage{url}
\usepackage{bm}
\usepackage[margin=2.5cm]{geometry}
\setlength{\marginparwidth}{20mm}

\usepackage[textsize=tiny]{todonotes}
%\usepackage[disable]{todonotes}
\newcommand{\todoc}[2][]{\todo[color=red!20!white,size=\tiny,#1]{Cs: #2}}
\newcommand{\todox}[2][]{\todo[color=orange!20!white,size=\tiny,#1]{X: #2}}
\newcommand{\todoa}[2][]{\todo[color=blue!20!white,size=\tiny,#1]{A: #2}}
\newcommand{\todop}[2][]{\todo[color=green!20!white,size=\tiny,#1]{P: #2}}
\usepackage{hyperref}

\makeatletter
\def\BState{\State\hskip-\ALG@thistlm}
\makeatother

%\newtheorem{theorem}{Theorem}
%\newtheorem{corollary}{Corollary}[theorem]
%\newtheorem{lemma}[theorem]{Lemma}
\usepackage[capitalize]{cleveref}
\usepackage[amsmath,standard,thmmarks]{ntheorem} % ntheorem makes cleveref work properly 

\newcommand{\cE}{\mathcal{E}}
\newcommand{\cF}{\mathcal{F}}
\newcommand{\cA}{\mathcal{A}}
\newcommand{\cK}{\mathcal{K}}
\newcommand{\cB}{\mathcal{B}}
\newcommand{\cY}{\mathcal{Y}}
\newcommand{\R}{\mathbb{R}}
\newcommand{\EE}[1]{\mathbb{E}\left[#1\right]}
\newcommand{\ip}[1]{\langle #1 \rangle}
\newcommand{\norm}[1]{\|#1\|}
\newcommand{\scnorm}[1]{\left\|#1\right\|}
\DeclareMathOperator\erf{erf}
\DeclareMathOperator{\argmin}{argmin}
\DeclareMathOperator{\argmax}{argmax}
\DeclareMathOperator{\interior}{int}
\DeclareMathOperator{\dom}{dom}
\newcommand\numberthis{\addtocounter{equation}{1}\tag{\theequation}}

%%% Prashanth: some of my macros..need to remove duplicates
\newcommand{\A}{\mathcal{A}}
\newcommand{\I}{\mathcal{I}}
\newcommand{\B}{\mathcal{B}}
\newcommand{\V}{\mathcal{V}}
\newcommand{\F}{\mathcal{F}}
\newcommand{\N}{\mathcal{N}}
\newcommand{\E}{\mathbb{E}}
\renewcommand{\P}{\mathbb{P}}
\newcommand{\G}{\mathcal{G}}
\renewcommand{\O}{\mathcal{O}}



\newcommand{\sign}{\mathop{\rm sign}}
% \newcommand{\norm}[1]{\left\|{#1}\right\|} % A norm with 1 argument
\newcommand{\tvnorm}[1]{\norm{#1}_{\rm TV}}
\newcommand{\dkl}[2]{D_{\rm kl}\left({#1} |\!| {#2} \right)}
\newcommand{\normal}{\mathsf{N}}  % Normal distribution
\newcommand{\openleft}[2]{\left({#1},{#2}\right]} % Interval open on left
\def\mbf#1{\mathbf{#1}}
\def\indic#1{\mbf{1}\left\{#1\right\}} % Indicator function
\newcommand{\Borel}{\fB}
\newcommand{\Cb}{\cC_b}
\begin{document}
Let $\cX$ be a Banach space over the reals $\R$, 
$\cX^*$ its topological dual (i.e., the set of continuous linear, $\R$-valued functionals on $\cX$). 
We shall assume that $\cX^*$ is separable. \todoc{needed?}
We will denote the value $\phi(x)$ of a functional $\phi$ from $\cX^*$ applied to a vector $x\in \cX$ by $\ip{\phi,x}$.
We will treat $\cX$ as a measurable space with the Borel $\sigma$-algebra $\Borel(\cX)$ on it. 

Let $\cY$ be a Banach space.
An $\cY$-valued vector measure $\nu$ is called finite if for any $A\in \Borel(\cX)$, $\norm{\nu(A)}<\infty$.
The convolution of a measurable function  $f: \cX \to \R$  and a measure $\mu$ on $\cX$ at $x\in \cX$ is defined as
\todoc{Changed to $y-x$..}
\begin{align*}
\left( f*\mu \right) (x) := \int f(y-x)\mu (d y)
 %= \int f(y)\mu(x+dy)
 = \int f(y)\mu_x(dy)\,,
\end{align*}
where $\mu_x$ is defined by $\mu_x(A) = \mu(x+A)$ ($A\in \Borel(\cX)$). 
Let $\Cb(\cX)$ denote the set of continuous, bounded, real-valued functions defined on $\cX$.
%Now we can see $\left\lbrace \mu_x: x\in \cX \right\rbrace$ is a family of measures with the parameter $x$. 
\begin{definition}
We say that $\mu_x$ is \emph{weakly differentiable} 
at $x_0\in \cX$ with weak derivative $\mu'_{(x_0)}$, if 
 $\mu'_{x_0}: \Borel(\cX)\to \cX^*$ is a finite measure such that for any $f \in \Cb(\cX)$,
\begin{align*}
\int f(y)\mu_{x_0+\xi}(dy) - \int f(y)\mu_{(x_0)}(dy)
-\ip{\int f(y)\mu'_{x_0}(dy), \xi} = o(\norm{\xi}) \,\text{ as } \norm{\xi}\to 0\,.
\end{align*}
\end{definition}
\todoc{How is integration defined? Why is $f$ $\mu_{(x_0)}'$ integrable?
If we used instead $\int f(y) \ip{ \mu'_{(x_0)}(dy),\xi}$, the first question would not arise:
$\ip{\mu'_{(x_0)}(\cdot),\xi}$ is an ordinary signed measure.
}
Note that $\mu'_{(x)}$ is uniquely defined.
Further, $\mu'_{(x)} = \mu'_{(x')}$ for any $x,x'\in \cX$.

Recall that the function $f: \cX \to \R$ is differentiable at $x_0\in \cX$ (in the Fr\'echet sense) if 
for some $g\in X^*$,
\begin{align*}
f(x_0+\xi) - f(x_0) - \ip{g,\xi} = o(\norm{\xi}) \text{ as } \norm{\xi} \to 0\,.
\end{align*}
Then, $g$ is called the derivative of $f$ at $x_0$ and is denoted by $\frac{d}{dx}f(x_0)$.
\begin{proposition}
\label{prop:diffconvolution}
The following are equivalent:
\begin{enumerate}
\item \label{prop:diffc:1}
	 For any $f \in \Cb(\cX)$, $f*\mu$ is differentiable at all points $x_0\in \cX$; 
\item \label{prop:diffc:2}
	 For any $f \in \Cb(\cX)$, $f*\mu$ is differentiable at some point $x_0\in \cX$; 
\item \label{prop:diffc:3}
	 $\mu$ is weakly differentiable. 
\end{enumerate}
Moreover, when  for some $f\in \Cb(\cX)$, $f*\mu$ is differentiable at $x_0\in \cX$,
$\frac{d}{dx} (f*\mu) (x_0) = \int f(y)\mu'_{x_0}(dy)$.
\end{proposition}
\begin{proof}
 \eqref{prop:diffc:2} $\Rightarrow$ \eqref{prop:diffc:3}:
 Fix $x_0\in \cX$ and pick any $f\in\Cb(\cX)$.
Denote $\hat{f}(x) := (f*\mu)(x)$. 
Since $\hat{f}$ is differentiable at $x_0$, there exists $g_f \in \cX^*$ such that
\begin{align}
\label{eq:diffhatf}
\hat{f}(x_0+\xi) - \hat{f}(x_0) 
= \int f(y)\mu_{x_0+\xi}(dy) - \int f(y)\mu_{x_0}(dy)
= \ip{g_f, \xi}+o(\norm{\xi}) \,.
\end{align}
Then, $\Lambda:\Cb(\cX) \to X^*$, defined by $\Lambda(f)= g_f$ is a bounded, linear operator (here, $\Cb(\cX)$ is taken with the $\norm{\cdot}_\infty$ topology). By the Riesz representation theorem,  \todoc{Which one?} 
$\Lambda(f) = \int f(y) \lambda(y)$ for some $\cX^*$-valued measure $\lambda$ on $(\cX,\Borel(\cX))$.
\todoc{Things are a little fishy here.}
\todoc[inline]{
Choose some $\cX^*$-valued measure $\lambda$ such that 
$g_f = \int f(y)\lambda(dy)$, 
then plug this into \eqref{eq:diffhatf}.}
Recalling the definition of weak differentiability, we can see that $\lambda$ is the weak derivative of $\mu$. 

 \eqref{prop:diffc:3} $\Rightarrow$ \eqref{prop:diffc:2}:
Next, to prove its necessity, let $g=\int f(y)\mu'_{x_0}(dy)$. It is obvious that \eqref{eq:diffhatf} will also hold. Therefore, $(f*\mu)$ is differentiable. Its gradient $g$ at $x_0$ is $\int f(y)\mu'_{x_0}(dy)$.
\end{proof}

Now we want to give a probability based representation of $\dfrac{d}{dx} (f*\mu) (x_0)$.

By \cref{prop:diffconvolution}, we have for any $x\in \cX$,
\begin{align*}
\dfrac{d}{dx} (f*\mu) (x) = \int f(y)\mu'_{x}(dy)
= \int f(x-y)\mu'(dy) \,,
\end{align*}
where the second equality results from the definition $\mu'_{x}(A)=\mu'(x-A)$. Since $\mu':\cX\to \cX^*$ is a finite measure, we can define 
\begin{align*}
\forall A \in \Borel(\cX), \quad
\nu(A) = \dfrac{\norm{\mu'(A)}_*}{\norm{\mu'(\cX)}_*}\,,
\end{align*}
which is a probability measure. Hence 
$\dfrac{d}{dx} (f*\mu) (x) = \norm{\mu'(\cX)}_* \int f(x-y) \dfrac{d\mu'}{d\nu}(y) \nu(dy)$, where $\dfrac{d\mu'}{d\nu}$ is the Radon-Nikodym derivative of $\mu'$ w.r.t $\nu$.

Further assume $\mu$ is a probability measure with the distribution function $p(\cdot)$, i.e., $\mu(dx) = p(x)dx$. If $p(x)$ is differentiable, then
$\nu(dy) = \dfrac{\norm{p'(y)}_*}{\int \norm{p'(x)}_*dx}dy$.
So we have
\begin{align*}
\dfrac{d}{dx} (f*\mu) (x) = U \int f(x-y) q(y) \nu (dy)\,,
\end{align*}
where $U = \int \norm{p'(x)}_*dx$, $q(y) = \dfrac{p'(y)}{\norm{p'(y)}_*}$.
\end{document}
