%!TEX root =  bgo-cam-ready.tex
% please do not delete or change the first line (needed by Csaba's editor)

%%%%%%%%%%%%%%%%%%%% Added for arxiv report %%%%%%%%%%%%%%%%%%%%%
\usepackage[T1]{fontenc}
\usepackage[utf8]{inputenc}
\usepackage[english]{babel}
\usepackage{authblk}
\usepackage{times}
\usepackage[margin=1in]{geometry}
%%%%%%%%%%%%%%%%%%%%%%%%%%%%%%%%%%%%%%%%%%%%%%%%%%%%%%%%%%%%%%%%%


\usepackage{etex}
\usepackage[utf8]{inputenc}
\usepackage{amsmath}
\usepackage{amsfonts}
\usepackage{amssymb}
%\usepackage{amsthm}
\usepackage{mathtools}
\usepackage{graphicx}
\usepackage{algorithm}
\usepackage[numbers]{natbib}
\usepackage[noend]{algpseudocode}
\usepackage{url}
\usepackage{bm}
% \usepackage[margin=2.5cm]{geometry}
\setlength{\marginparwidth}{20mm}
\usepackage{booktabs}
\usepackage{multirow}
\usepackage{colortbl}
\usepackage{hhline}
\usepackage{paralist}
\usepackage{xspace}
\usepackage{sidecap}

\usepackage[disable]{todonotes}
%\usepackage{todonotes}
\newcommand{\todoc}[2][]{\xspace\todo[color=red!20!white,size=\tiny,#1]{Cs: #2}}
\newcommand{\todox}[2][]{\xspace\todo[color=orange!20!white,size=\tiny,#1]{X: #2}}
\newcommand{\todoa}[2][]{\xspace\todo[color=blue!20!white,size=\tiny,#1]{A: #2}}
\newcommand{\todop}[2][]{\xspace\todo[color=green!20!white,size=\tiny,#1]{P: #2}}

\makeatletter
\def\BState{\State\hskip-\ALG@thistlm}
\makeatother


%\usepackage{amsthm}
%\newtheorem{remark}{Remark}
%\newtheorem{proposition}{Proposition}
%\newtheorem{definition}{Definition}
% \newtheorem{theorem}{Theorem}
\usepackage[amsmath,standard,thmmarks]{ntheorem} % ntheorem makes cleveref work properly
% \newtheorem{corollary}[theorem]{Corollary}
% \newtheorem{lemma}[theorem]{Lemma}

% Hyperlinks make it very easy to navigate an electronic document.
% In addition, this is where you should specify the thesis title
% and author as they appear in the properties of the PDF document.
% Use the "hyperref" package
% N.B. HYPERREF MUST BE THE SECOND TO LAST PACKAGE LOADED; ADD ADDITIONAL PKGS ABOVE
\usepackage[pdftex,letterpaper=true,pagebackref=false]{hyperref} % with basic options
		% N.B. pagebackref=true provides links back from the References to the body text. This can cause trouble for printing.
\hypersetup{
    plainpages=false,       % needed if Roman numbers in frontpages
    pdfpagelabels=true,     % adds page number as label in Acrobat's page count
    bookmarks=true,         % show bookmarks bar?
    unicode=false,          % non-Latin characters in Acrobat’s bookmarks
    pdftoolbar=true,        % show Acrobat’s toolbar?
    pdfmenubar=true,        % show Acrobat’s menu?
    pdffitwindow=false,     % window fit to page when opened
    pdfstartview={FitH},    % fits the width of the page to the window
    pdftitle={(Bandit) Convex Optimization with Biased Noisy Gradient Oracles},    % set
    pdfauthor={},    % set
%     pdfauthor={D\'avid Szepesv\'ari},    % set
%    pdfsubject={Subject},  % subject: CHANGE THIS TEXT! and uncomment this line
%    pdfkeywords={keyword1} {key2} {key3}, % list of keywords, and uncomment this line if desired
    pdfnewwindow=true,      % links in new window
    colorlinks=true,        % false: boxed links; true: colored links
    linkcolor=blue,         % color of internal links
    citecolor=blue,        % color of links to bibliography
    filecolor=magenta,      % color of file links
    urlcolor=cyan           % color of external links
}
% N.B. CLEVEREF MUST BE LOADED; AFTER HYPERREF
\usepackage[capitalize]{cleveref}
\newtheorem{ass}{Assumption}
\crefname{ass}{Assumption}{Assumptions}

%%%%%%%%%%%%%%%%%%%%%%%%%%%%%%%%%%%%%%%%%%%%%%%%%%%%%%%%
\newcommand{\cE}{\mathcal{E}}
\newcommand{\cF}{\mathcal{F}}
\newcommand{\cA}{\mathcal{A}}
\newcommand{\cK}{\mathcal{K}}
\newcommand{\cB}{\mathcal{B}}
\newcommand{\cY}{\mathcal{Y}}
\newcommand{\R}{\mathbb{R}}
\newcommand{\EE}[1]{\mathbb{E}\left[#1\right]}
\newcommand{\ip}[1]{\langle #1 \rangle}
\newcommand{\norm}[1]{\|#1\|}
\newcommand{\scnorm}[1]{\left\|#1\right\|}
\DeclareMathOperator{\fspan}{span}
\DeclareMathOperator\erf{erf}
\DeclareMathOperator{\argmin}{argmin}
\DeclareMathOperator{\argmax}{argmax}
\DeclareMathOperator{\interior}{int}
\DeclareMathOperator{\dom}{dom}
\newcommand\numberthis{\addtocounter{equation}{1}\tag{\theequation}}
\newcommand{\DR}{D_{\mathcal{R}}}

%%% Prashanth: some of my macros..need to remove duplicates
\newcommand{\C}{\mathcal{C}}
\newcommand{\A}{\mathcal{A}}
\newcommand{\I}{\mathcal{I}}
\newcommand{\B}{\mathcal{B}}
\newcommand{\V}{\mathcal{V}}
\newcommand{\F}{\mathcal{F}}
\newcommand{\N}{\mathcal{N}}
\newcommand{\E}{\mathbb{E}}
\renewcommand{\P}{\mathbb{P}}
\newcommand{\G}{\mathcal{G}}
\renewcommand{\O}{\mathcal{O}}
\newcommand{\D}{\mathcal{D}}
\newcommand{\cD}{\mathcal{D}}
\newcommand{\K}{\mathcal{K}}
\newcommand{\cR}{\mathcal{R}}

\newcommand{\noise}{\psi}
\newcommand{\noiseCap}{\Psi}

\newcommand{\dnorm}[1]{\norm{#1}_*} % Dual norm
\def\<{\left\langle} % Angle brackets
\def\>{\right\rangle}




\newcommand{\sign}{\mathop{\rm sign}}
% \newcommand{\norm}[1]{\left\|{#1}\right\|} % A norm with 1 argument
\newcommand{\tvnorm}[1]{\norm{#1}_{\rm TV}}
\newcommand{\dkl}[2]{D_{\rm kl}\left({#1} |\!| {#2} \right)}
\newcommand{\normal}{\mathsf{N}}  % Normal distribution
\newcommand{\openleft}[2]{\left({#1},{#2}\right]} % Interval open on left
\def\mbf#1{\mathbf{#1}}
\def\indic#1{\mbf{1}\left\{#1\right\}} % Indicator function
\newcommand{\tr}{^\mathsf{\scriptscriptstyle T}} % transpose
