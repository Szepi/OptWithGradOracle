\documentclass[twoside]{article}
\usepackage[accepted]{aistats2016}
%!TEX root =  bgo-cam-ready.tex
% please do not delete or change the first line (needed by Csaba's editor)

%%%%%%%%%%%%%%%%%%%% Added for arxiv report %%%%%%%%%%%%%%%%%%%%%
\usepackage[T1]{fontenc}
\usepackage[utf8]{inputenc}
\usepackage[english]{babel}
\usepackage{authblk}
\usepackage{times}
\usepackage[margin=1in]{geometry}
%%%%%%%%%%%%%%%%%%%%%%%%%%%%%%%%%%%%%%%%%%%%%%%%%%%%%%%%%%%%%%%%%


\usepackage{etex}
\usepackage[utf8]{inputenc}
\usepackage{amsmath}
\usepackage{amsfonts}
\usepackage{amssymb}
%\usepackage{amsthm}
\usepackage{mathtools}
\usepackage{graphicx}
\usepackage{algorithm}
\usepackage[numbers]{natbib}
\usepackage[noend]{algpseudocode}
\usepackage{url}
\usepackage{bm}
% \usepackage[margin=2.5cm]{geometry}
\setlength{\marginparwidth}{20mm}
\usepackage{booktabs}
\usepackage{multirow}
\usepackage{colortbl}
\usepackage{hhline}
\usepackage{paralist}
\usepackage{xspace}
\usepackage{sidecap}

\usepackage{caption}
\usepackage{subcaption}
\usepackage{tikz}
\usepackage{pgfplots}
\usepackage{makecell}
\usetikzlibrary{positioning,arrows.meta,shapes,calc,patterns,fit,decorations}
\usepgfplotslibrary{fillbetween}
\pgfplotsset{compat=1.10}

%%% Added by Prashanth: the following two lines are used to make compilation faster with tikz. Each tikzfigure will become a seprate tex job and the figures get stored in a sub-folder named tikz. 
%%%% Dont forget to add --shell-escape(or -enable-write18 if you are using MikTeX) to the pdflatex commandline
\usetikzlibrary{external}
\tikzexternalize[prefix=tikz/]
%%%%%%%%%%

\usepackage[disable]{todonotes}
%\usepackage{todonotes}
%%% Added by Prashanth: The tikz externalize business to speed up compilation is incompatible with todonotes, unless we add the following workaround that disables tikz-externalize in todo definition
\makeatletter
\renewcommand{\todo}[2][]{\tikzexternaldisable\@todo[#1]{#2}\tikzexternalenable}
\makeatother
%%%%%%%%%%%

\newcommand{\todoc}[2][]{\xspace\todo[color=red!20!white,size=\tiny,#1]{Cs: #2}}
\newcommand{\todox}[2][]{\xspace\todo[color=orange!20!white,size=\tiny,#1]{X: #2}}
\newcommand{\todoa}[2][]{\xspace\todo[color=blue!20!white,size=\tiny,#1]{A: #2}}
\newcommand{\todop}[2][]{\xspace\todo[color=green!20!white,size=\tiny,#1]{P: #2}}

\makeatletter
\def\BState{\State\hskip-\ALG@thistlm}
\makeatother


%\usepackage{amsthm}
%\newtheorem{remark}{Remark}
%\newtheorem{proposition}{Proposition}
%\newtheorem{definition}{Definition}
% \newtheorem{theorem}{Theorem}
\usepackage[amsmath,standard,thmmarks]{ntheorem} % ntheorem makes cleveref work properly
% \newtheorem{corollary}[theorem]{Corollary}
% \newtheorem{lemma}[theorem]{Lemma}

% Hyperlinks make it very easy to navigate an electronic document.
% In addition, this is where you should specify the thesis title
% and author as they appear in the properties of the PDF document.
% Use the "hyperref" package
% N.B. HYPERREF MUST BE THE SECOND TO LAST PACKAGE LOADED; ADD ADDITIONAL PKGS ABOVE
\usepackage[pdftex,letterpaper=true,pagebackref=false]{hyperref} % with basic options
		% N.B. pagebackref=true provides links back from the References to the body text. This can cause trouble for printing.
\hypersetup{
    plainpages=false,       % needed if Roman numbers in frontpages
    pdfpagelabels=true,     % adds page number as label in Acrobat's page count
    bookmarks=true,         % show bookmarks bar?
    unicode=false,          % non-Latin characters in Acrobat’s bookmarks
    pdftoolbar=true,        % show Acrobat’s toolbar?
    pdfmenubar=true,        % show Acrobat’s menu?
    pdffitwindow=false,     % window fit to page when opened
    pdfstartview={FitH},    % fits the width of the page to the window
    pdftitle={(Bandit) Convex Optimization with Biased Noisy Gradient Oracles},    % set
    pdfauthor={},    % set
%     pdfauthor={D\'avid Szepesv\'ari},    % set
%    pdfsubject={Subject},  % subject: CHANGE THIS TEXT! and uncomment this line
%    pdfkeywords={keyword1} {key2} {key3}, % list of keywords, and uncomment this line if desired
    pdfnewwindow=true,      % links in new window
    colorlinks=true,        % false: boxed links; true: colored links
    linkcolor=blue,         % color of internal links
    citecolor=blue,        % color of links to bibliography
    filecolor=magenta,      % color of file links
    urlcolor=cyan           % color of external links
}
% N.B. CLEVEREF MUST BE LOADED; AFTER HYPERREF
\usepackage[capitalize]{cleveref}
\newtheorem{ass}{Assumption}
\crefname{ass}{Assumption}{Assumptions}

%%%%%%%%%%%%%%%%%%%%%%%%%%%%%%%%%%%%%%%%%%%%%%%%%%%%%%%%
\newcommand{\cE}{\mathcal{E}}
\newcommand{\cF}{\mathcal{F}}
\newcommand{\cA}{\mathcal{A}}
\newcommand{\cK}{\mathcal{K}}
\newcommand{\cB}{\mathcal{B}}
\newcommand{\cY}{\mathcal{Y}}
\newcommand{\R}{\mathbb{R}}
\newcommand{\EE}[1]{\mathbb{E}\left[#1\right]}
\newcommand{\ip}[1]{\langle #1 \rangle}
\newcommand{\norm}[1]{\|#1\|}
\newcommand{\scnorm}[1]{\left\|#1\right\|}
\DeclareMathOperator{\fspan}{span}
\DeclareMathOperator\erf{erf}
\DeclareMathOperator{\argmin}{argmin}
\DeclareMathOperator{\argmax}{argmax}
\DeclareMathOperator{\interior}{int}
\DeclareMathOperator{\dom}{dom}
\DeclareMathOperator{\esssup}{ess\,sup}
\DeclareMathOperator{\essup}{ess\,sup}
\newcommand\numberthis{\addtocounter{equation}{1}\tag{\theequation}}
\newcommand{\DR}{D_{\mathcal{R}}}
\newcommand{\g}{\gamma}
\newcommand{\og}{\overline{\gamma}}


%%% Prashanth: some of my macros..need to remove duplicates
\newcommand{\C}{\mathcal{C}}
\newcommand{\A}{\mathcal{A}}
\newcommand{\I}{\mathcal{I}}
\newcommand{\B}{\mathcal{B}}
\newcommand{\V}{\mathcal{V}}
\newcommand{\F}{\mathcal{F}}
\newcommand{\N}{\mathcal{N}}
\newcommand{\E}{\mathbb{E}}
\renewcommand{\P}{\mathbb{P}}
\newcommand{\G}{\mathcal{G}}
\renewcommand{\O}{\mathcal{O}}
\newcommand{\D}{\mathcal{D}}
\newcommand{\cD}{\mathcal{D}}
\newcommand{\K}{\mathcal{K}}
\newcommand{\cR}{\mathcal{R}}
\newcommand{\tf}{\tilde{f}}

\newcommand{\noise}{\psi}
\newcommand{\noiseCap}{\Psi}

\newcommand{\dnorm}[1]{\norm{#1}_*} % Dual norm
\def\<{\left\langle} % Angle brackets
\def\>{\right\rangle}




\newcommand{\sign}{\mathop{\rm sign}}
% \newcommand{\norm}[1]{\left\|{#1}\right\|} % A norm with 1 argument
\newcommand{\tvnorm}[1]{\norm{#1}_{\rm TV}}
\newcommand{\dkl}[2]{D_{\rm kl}\left({#1} |\!| {#2} \right)}
\newcommand{\normal}{\mathsf{N}}  % Normal distribution
\newcommand{\openleft}[2]{\left({#1},{#2}\right]} % Interval open on left
\def\mbf#1{\mathbf{#1}}
\def\indic#1{\mbf{1}\left\{#1\right\}} % Indicator function
\newcommand{\tr}{^\mathsf{\scriptscriptstyle T}} % transpose


% If your paper is accepted, change the options for the package
% aistats2016 as follows:
%
%\usepackage[accepted]{aistats2016}
%
% This option will print headings for the title of your paper and
% headings for the authors names, plus a copyright note at the end of
% the first column of the first page.


\begin{document}

% If your paper is accepted and the title of your paper is very long,
% the style will print as headings an error message. Use the following
% command to supply a shorter title of your paper so that it can be
% used as headings.
%
%\runningtitle{I use this title instead because the last one was very long}

% If your paper is accepted and the number of authors is large, the
% style will print as headings an error message. Use the following
% command to supply a shorter version of the authors names so that
% they can be used as headings (for example, use only the surnames)
%
\runningauthor{Xiaowei Hu, Prashanth L.A., Andr\'as Gy\"orgy, and Csaba Szepesv\'ari}

\twocolumn[

\aistatstitle{(Bandit) Convex Optimization with Biased Noisy Gradient Oracles}


\aistatsauthor{Xiaowei  Hu$^1$ \And Prashanth L.A.$^2$ \And Andr\'as Gy\"orgy$^3$ \And Csaba Szepesv\'ari$^1$ \vspace{1mm}}

\aistatsaddress{$^1$ Department of Computing Science, University of Alberta \\
$^2$ Institute for Systems Research, University of Maryland \\
$^3$ Department of Electrical and Electronic Engineering, Imperial College London} ]


\begin{abstract}
\vspace{-0.2cm}
A popular class of algorithms for convex optimization and online learning with bandit 
feedback rely on constructing noisy gradient estimates, which are then used in place
of the actual gradients in appropriately adjusted first-order algorithms.
Depending on the properties of the function to be optimized and the nature of ``noise'' in the bandit feedback,
the bias and variance of gradient estimates exhibit various tradeoffs.
% and, correspondingly,
%the algorithms were proven to achieve various convergence rates.
 In this paper we propose a novel framework that replaces the specific gradient estimation
 methods with an abstract oracle model.
 % capturing the controlleable bias-variance tradeoff
 %of existing gradient estimators. 
 With the help of the new framework we unify previous works,
 reproducing  their results in a clean and concise fashion, 
 while, perhaps more importantly, the framework also allows us to formally show that to achieve the optimal 
  root-$n$ rate %in online/stochastic bandit convex optimization
  either the algorithms that use existing gradient estimators,
  or the proofs have to go beyond what exists today.
\if0
We present a novel noisy gradient oracle model for convex optimization. The model allows to explicitly address the bias-variance tradeoff of gradient estimation methods typical in the literature, as well as to prove upper and lower bounds for the minimax error. As such the oracle model allows a clean way of addressing the limits of achievable performance
when biased, noisy gradient estimators with controllable bias-variance tradeoffs are employed.
When considering the model for online/stochastic convex optimization,
one consequence of our results is that the currently used gradient estimates and the proof techniques used to analyze them cannot 
achieve the optimal square-root regret rate for online/stochastic bandit convex optimization.
\fi
% when the function cannot be queried outside the domain.
\end{abstract}

\vspace{-0.3cm}

\section{Introduction}
\label{sec:intro}
%!TEX root =  bgo-opt-ml-nips.tex
% please do not delete or change the first line (needed by Csaba's editor)

In stochastic bandit convex optimization (also known as  convex optimization with stochastic zeroth order oracles)
an algorithm submits queries to an oracle in a sequential manner in $n$ rounds.
The oracle returns noisy values of the convex objective function at the submitted points.
At the end, the algorithm also produces a guess of the objective's minimizer; the algorithm's performance
is measured in terms of the suboptimality of this guess, measured using the objective function.
In their seminal work \citet{NeYu83} consider two approaches to this problem: Methods that try to construct
gradient information and methods that avoid gradient estimation and rather use 
geometric principles (the ellipsoid method, essentially).
While methods in the second class make the error decay at the $O(1/\sqrt{n})$ rate, 
the error scales extremely poorly with the number of optimization variables $d$.
For example, \citet{liang2014zeroth} proves a bound of the form $\sqrt{d^{14}/n}$, improving the $\sqrt{d^{33}/n}$ bound
of \citet{AgFoHsuKaRa13:SIAM}.
A lower bound due to \citep{shamir2012complexity} however scales only with $\sqrt{d^2/n}$.
Methods that first construct gradient estimates, which are fed to algorithms that expect gradient-information
have a long history, though in early years the focus was either asymptotic convergence,
or asymptotic rates. \todoc{Prashanth: Can you please add some citations}
The story with these methods is that they get the optimal rate and dimension dependence for ``nice'' problems, such
as when the objective function is strongly convex and smooth 
\citep{HaLe14:SOC}, but their performance, mostly in terms of the rate degrades as one removes constraints
from the objective function. For example, for smooth convex problems, the best published results with this technique 
is $O(n^{1/3})$ \citep{saha2011improved}.%
%\footnote{
%A day before the submission to the workshop we became aware of the recent NIPS submission of
%\citet{DeKo15:BSCO}, which claims to improve this result. 
%}
 \todoc{This \citet{jamieson2012query} paper is weird. Also, it does not construct gradient estimates, so I took out from here.}
Our motivation in this paper is to formally study the possible limitations of the gradient-based approach.
We do this by precisely defining a new oracle model. The new oracles can be consulted to obtain gradient estimates.
They have a tuneable parameter, which controls the bias-variance tradeoff that 
exists for all known gradient construction methods. We then prove lower bounds and (sometimes) matching upper bounds
for algorithms that use these oracles. Our lower bound for stochastic smooth bandit convex optimization gives the rate 
$\Omega(n^{1/3})$, assuming that the currently available gradient-estimation methods are unimprovable.


%The authors provided some algorithms and upper bounds, but as they themselves emphasize (cf. pg. 359), the attainable complexity is far from clear. Quite recently, Jamieson et al. (2012) provided an �(??d/T) lower bound for strongly-convex functions, which demonstrates that the �fast� O(1/T ) rate in terms of T , that one enjoys with gradient information, is not possible here. In contrast, the current best-known upper bounds are O(??4 d2/T),O(??3 d2/T),O(??d2/T) for convex, strongly-convex, and strongly-convex-and-smooth functions respectively (Flaxman et al. (2005); Agarwal et al. (2010)); And a O(??d32/T ) bound for convex functions (Agarwal et al. (2011)), which is better in terms of dependence on T but very bad in terms of the dimension d.






\section{Problem Setup}
\label{sec:problem}
%!TEX root =  bgo.tex
% please do not delete or change the first line (needed by Csaba's editor)

$\cK \subset \R^d$: domain, convex, closed, non-emtpy.

Interaction with the oracle + diagrams

Definitions of oracles

\begin{definition}
Input is $x\in \cK,\delta>0$,
$f\in \cF$ ($f$ coming from the environment, others from the algorithm).
Output is $Y\in \cK$,$G\in \R^d$, $G$ is random.

$\norm{x-Y}\le \delta$, 
$\norm{ \EE{G}  - \nabla f(x)  }_* \le C_1 \delta^p$.
$\EE{\norm{ G -  \EE{G} }_*^2} \le C_2 \delta^{-q}$,

Meaning of $Y$: the function is evaluated at $Y$. This will matter only for the online case, when we consider the cumulative regret.
\end{definition}

Next one:

\begin{definition}
Input is $x\in \cK,\delta>0$,
$f\in \cF$ ($f$ coming from the environment, others from the algorithm).
Output is $Y\in \cK$,$G\in \R^d$, $G$ is random.

$\norm{x-Y}\le \delta$, 

Also: there exists $\tilde{f} \in \cF$ such that 
$\norm{\tilde{f}- f}_\infty \le C_1 \delta^p$ (bias)
$\EE{G}  = \nabla \tilde{f}(x)$,
$\EE{\norm{ G -  \EE{G} }_*^2} \le C_2 \delta^{-q}$,

\end{definition}

alternative to bias condition for second definition:
\begin{align}
\norm{\nabla \tilde{f}- \nabla f}_\infty \le C_1 \delta^p
\end{align}


\section{Main Results}
\label{sec:results}
%!TEX root =  bgo.tex
% please do not delete or change the first line (needed by Csaba's editor)

\begin{table*}
\centering
 \caption{Summary of upper and lower bounds for various smooth function classes and different gradient oracles}
\label{tab:mse-1}
\begin{tabular}{|c|c|c|c|c|}
\toprule
\rowcolor{gray!20}
% \multicolumn{3}{|c|}{\multirow{2}{*}{\textbf{Lower bounds}}}\\[1.7em]
% \midrule
  \multirow{2}{*}{\textbf{Oracle type}} & \multicolumn{2}{|c|}{\multirow{2}{*}{\textbf{Convex + Smooth}}} & \multicolumn{2}{|c|}{\multirow{2}{*}{\textbf{Strongly Convex + Smooth}}} \\[1.7em]
 \midrule
\multirow{2}{*}{ $C_1=0$ and $q=0$} & \multirow{2}{*}{$\O\left(\dfrac{d}{\sqrt{n}}\right)$}  & \multirow{2}{*}{$\Omega\left(\dfrac{d}{\sqrt{n}}\right)$} & \multirow{2}{*}{$\O\left(\dfrac{d}{\sqrt{n}}\right)$}  & \multirow{2}{*}{$\Omega\left(\dfrac{d}{\sqrt{n}}\right)$}  \\[1.5em]\midrule
%%%%%%%%%%%%%%%%
\multirow{2}{*}{ $p=1$ and $q=2$} & \multirow{2}{*}{$\O\left(\dfrac{d^{\frac{1}{4}}}{n^{\frac{1}{4}}}\right)$}  & \multirow{2}{*}{$\Omega\left(\dfrac{\sqrt{C_1 C_2} d^\frac{1}{4}}{n^\frac{1}{4}}\right)$}& \multirow{2}{*}{$\O\left(\dfrac{d}{\sqrt{n}}\right)$}  & \multirow{2}{*}{$\Omega\left(\dfrac{C_1 C_2}{\sqrt{d n}}\right)$} \\[1.5em]\midrule
%%%%%%%%%%%%%%%%%%%
\multirow{2}{*}{ $p=2$ and $q=2$} & \multirow{2}{*}{$\O\left(\dfrac{d^{\frac{1}{6}}}{n^{\frac{1}{3}}}\right)$}  & \multirow{2}{*}{$\Omega\left(\dfrac{(C_1 C_2^2)^\frac{1}{3} d^\frac{1}{6}}{n^\frac{1}{3}}\right)$} & \multirow{2}{*}{$\O\left(\dfrac{d}{\sqrt{n}}\right)$}  & \multirow{2}{*}{$\Omega\left(\dfrac{(C_1^2 C_2^4)^\frac{1}{3}}{d^\frac{2}{3} n^\frac{2}{3}}\right)$}\\[1.5em]
 \bottomrule 
\end{tabular}
\end{table*}

Unconstrained case:

Upper bound: 

\begin{algorithm}
	\caption{Mirror Descent with Type-I Oracle}\label{alg}
	\textbf{Input}: Closed convex set $\cK$, regularization function $\mathcal{R}:\mathbb{R}^d\to \mathbb{R}$, tolerance parameter $\delta$, learning rates $\{\eta_t\}_{t=1}^{n-1}$. \\
	Initialize $X_1\in \cK$ arbitrarily.\\
	In round $t=1, 2, \cdots, n-1$:
	\begin{itemize}
	\item Query the oracle at $X_t$.
	\item Receive $G_t$, $Y_t$.
	\item Update
	\[
	X_{t+1}=\argmin_{x\in \mathcal{K}}\left[ \eta_{t} \ip{G_t,x}+D_{\mathcal{R}}(x,X_t) \right] \,.
	\]
	\end{itemize}
	\textbf{Output}: the optimizer
	\[
	\hat{X}_n = \dfrac{1}{n}\sum_{t=1}^n X_t \,.
	\]
	
\end{algorithm}

\begin{theorem}
\label{thm:ub}
Given a bounded, convex set $\cK \subset \R$,
let $\cF$ be a class of convex functions on $\cK$ such that $f \in \cF$ is $L$-smooth.
Assume further that the regularization function $\mathcal{R}$ is $\alpha$-strongly convex w.r.t. some norm $\norm{\cdot}$, and $\interior(\mathcal{K})\subseteq \dom(\mathcal{R})$.
For any $(c_1,c_2)$ Type-I oracle 
 with $c_1(\delta) = C_1 \delta^p$, $c_2(\delta) = C_2 \delta^{-q}$, $p,q>0$, 
 if \cref{alg} is run with 
 \[
 \eta_t = \dfrac{\alpha}{a_t+L} \,,
 \]
 \[
 \delta = \left( \dfrac{C_2}{2aC_1}\sqrt{\dfrac{\alpha}{2D}} \right)^{1/(p+q)}n^{-1/(2p+q)} \,,
 \] 
 \todox{the $n$ should actually be $(n-1)$, the expression will be more natural if we start from $0$.}
 where
 $a = \left( \sqrt{\dfrac{2^{3q}\alpha}{D}}C_1^q C_2^p \right)^{1/(2p+q)}$,
 $a_t = a t^{(p+q)/(2p+q)}$, for $t=1, 2, \cdots, n-1$,
  the minimax error \eqref{eq:minimaxerrdef} can be bounded as
 \begin{align*}
\Delta_n^*(\cF,c_1,c_2)= O(n^{-p/(2p+q)})\,,
 \end{align*}
 where $D=\sup_{x,y\in \cK} \DR(x,y)$. $\DR$ denotes the Bregman divergence w.r.t. $\mathcal{R}$.
 
If the class of functions is also $\mu$-strongly convex w.r.t. $\mathcal{R}$. Choose
\[
 \eta_t = \dfrac{2}{\mu t} \,,
\]
\[
\delta^{p+q} =  \sqrt{\dfrac{\alpha}{2D}}\dfrac{C_2 \log n}{\alpha \mu C_1 n} \,,
\]
the upper bound of minimax error is then
 \begin{align*}
\Delta_n^*(\cF,c_1,c_2)= O(n^{-p/(p+q)})\,.
 \end{align*}
\end{theorem}
\begin{proof}
See Section \ref{sec:ub-proof}.
\end{proof}


Lower bound:
\begin{theorem}
\label{thm:lb-convex}
For any $v \in \{+1,-1\}^d$ and $x \in \cK\subset \R^d$, define $f_v := \sum_{i=1}^d f^{(i)}_{v_i}(x_i)$, where
$f^{(i)}_{v_i}(x_i) := \dfrac{\epsilon}{2} (x_i - v_i)^2$, for $i=1,\ldots,d$.
Consider the space of functions $\F:= \{f_v \mid v  \in \{+1,-1\}^d\}$, with $\epsilon = d^{\frac{-(4p+q)}{(4p+2q)}}\left(\frac{1}{2\sqrt{n} K_1} \right)^{\frac{p}{p+\frac{q}{2}}}$, where \\$K_1 = \frac{p}{C_2(p+\tfrac{q}{2})} \left(\frac{q}{2C_1(p+\tfrac{q}{2})}\right)^{\frac{q}{2p}}$. 
Then, for any algorithm that observes $n$ random elements from a $(c_1,c_2)$ Type-I oracle 
 with $c_1(\delta) = C_1 \delta^p$, $c_2(\delta) = C_2 \delta^{-q}$, $p,q>0$,
 the minimax error \eqref{eq:minimaxerrdef} satisfies
\[
\Delta_n^{*}(\F, c_1,c_2) \ge \dfrac{1}{4}d^{\frac{q}{(4p+2q)}}\left(\dfrac{1}{2 K_1 \sqrt n}\right)^{\frac{p}{p+\frac{q}{2}}}.
\]
\end{theorem}
\begin{proof}
 See Section \ref{sec:lb-proof} for a proof sketch and Appendix \ref{sec:appendix-lbconvex} for a detailed proof.
\end{proof}

Lower bound: Strongly convex case
\begin{theorem}
\label{thm:lb-strongly-convex}
For any $v \in \{+\nu,-\nu\}^d$ and $x \in \cK\subset \R^d$, define $f_v := \sum_{i=1}^d f^{(i)}_{v_i}(x_i)$, where
$f^{(i)}_{v_i}(x_i) := \dfrac{1}{2} x^2 - v x$, for $i=1,\ldots,d$.
Consider the space of functions $\F:= \{f_v \mid v  \in \{+\nu,-\nu\}^d\}$, with $\nu = d^{\frac{-(4p+q)}{(2p+q)}}\left(\frac{1}{2\sqrt{n} K_1} \right)^{\frac{p}{p+\frac{q}{2}}}$, where $K_1$ is as defined in Theorem \ref{thm:lb-convex}. 
Then, for any algorithm that observes $n$ random elements from a $(c_1,c_2)$ Type-I oracle 
 with $c_1(\delta) = C_1 \delta^p$, $c_2(\delta) = C_2 \delta^{-q}$, $p,q>0$,
 the minimax error \eqref{eq:minimaxerrdef} satisfies
\[
\Delta_n^{*}(\F, c_1,c_2) \ge \dfrac{1}{4}  \left(\dfrac{1}{2 K_1 \sqrt {d n}}\right)^{\frac{2p}{p+\frac{q}{2}}}.
\]
\end{theorem}
\begin{proof}
 See Section \ref{sec:lb-proof} for a proof sketch and Appendix \ref{sec:appendix-lbscconvex} for a detailed proof.
\end{proof}


\todop{Extend the result to the case when the algorithm can choose $\delta$ in every step - will update the proof for adaptive $\delta$ soon}



Todo: Online learning is at least as hard as optimization. Hence, we get a regret lower bound from this, too.

Discussion: The functions are smooth and strongly convex.
However, their strong convexity constant converges to zero as $n\to\infty$.

Todo: What can we show for strongly convex functions?

Of course, the same result holds in $d$ dimensions (add details: why?). However, one expects the lower bound to scale linearly with the dimension and we were not able to derive this so far.

Constrained case:

Upper bound: 

Reductions:
Between oracles;
Optimization and cumulative regret minimization;




\section{Applications to Stochastic BCO}
\label{sec:sbco}
%!TEX root =  bgo-cam-ready.tex
% please do not delete or change the first line (needed by Csaba's editor)
The main application of the biased noisy gradient oracle based convex optimization of the previous section
is bandit convex optimization. We introduce here briefly the stochastic version of the problem, while online BCO will be considered in Section~\ref{sec:obco}. Readers familiar with these problems and the associated
gradient estimation techniques, may skip this description to jump directly to \cref{thm:aaa},
and come back to it only to clarify our notation and terminology in case some confusion arises later.

\if0
For example, \cite{AgDeXi10} used an algorithm that queries at two points per round, and defined the incurred loss as the average of losses at the two observation points. In our oracle, it can be stated as in round $t$, the same oracle $\gamma_t$ responds to the same inputs $(X_t, \delta_t, f_t)$ with two different pairs $(G_{t,1}, Y_{t,1})$ and $(G_{t,2}, Y_{t,2})$. The accumulated regret can be written as
$% \[
R_n = \sum_{t=1}^n \dfrac{1}{2}\left( f_t(Y_{t,1})+f_t(Y_{t,2}) \right) -\inf_{x \in \cK}\sum_{t=1}^n f_t(x) \,.
$ %\]
Recalling that $Y_{t,1}$, $Y_{t,2}$ are in the $\delta$-vicinity of $X_t$, the relationship between $f_t(X_t)$ and $\dfrac{1}{2}\left( f_t(Y_{t,1})+f_t(Y_{t,2})\right)$ is then determined by the environment (i.e. the property of $f_t$). It is straightforward to bound $| \dfrac{1}{2}\left( f_t(Y_{t,1})+f_t(Y_{t,2})\right)- f_t(X_t)|$ as a function of $\delta$. \todoc{How? When $f_t$ is Lipschitz, smooth, etc? So you mean, when $f_t$ is a smooth function?}
The common assumption for this setting is: The oracle is a stochastic mapping from $(X, \delta, f)$ to $(G, Y)$; The algorithm selects the point $X_t$ depending on $\left( X_1, G_{1,1}, G_{1,2}, \cdots, X_{t-1},G_{t-1,1}, G_{t-1,2}  \right)$. This two-point feedback can be easily extended to multi-point feedback, too.
\fi

In the \emph{stochastic BCO} setting,
the algorithm sequentially chooses the points $X_1,\dots,X_n\in \cK$ while observing the loss function at these points in noise.
In particular, in round $t$, the algorithm chooses $X_t$ based on the earlier observations $Z_1,\dots,Z_{t-1}\in \R$ and $X_1,\dots,X_{t-1}$, after which it observes $Z_t$, where $Z_t$ is the value of $f(X_t)$ corrupted by ``noise''.


Previous research considered several possible constraints connecting $Z_t$ and $f(X_t)$.
One simple assumption is that $Z_t-f(X_t)$ is an $\cF_t = \sigma(X_{1:t},Z_{1:t-1})$-adapted martingale difference sequence (with favorable tail properties).
% \todoc{Some readers might be put off by martingales..}
A specific case is when $Z_t - f(X_t) = \xi_t$, where $(\xi_t)$ is a sequence of independent and identically distributed (i.i.d.) variables.
A stronger assumption, which is most appropriate in stochastic programming,
is that 
\begin{equation}
\label{eq:controlled}
Z_t = F(X_t,\noiseCap_t),
\end{equation} where $\noiseCap_t\in \R$ is chosen by the algorithm and $\EE{F(x,\noiseCap)}=f(X)$ for some distribution $P_{\noiseCap}$ over the reals. As in \cite{duchi2015optimal}, we assume that the function $F(\cdot, \psi)$ is $L(\psi)$-smooth and the quantity $L(\Psi) = \sqrt{\E [L(\Psi)^2]}$ is finite.  
%That is the algorithm can call the oracle $F(x,\noise)$ by selecting both $x$ and $\noise$. 
Note that the algorithm is aware of $P_{\noiseCap}$, but does not know how different values of $\noise$ affect the noise $\xi(x,\noise)=F(x,\noise)-f(x)$. Nevertheless, the algorithm can control the noise by selecting $\noiseCap_t$, and hence it can obtain unbiased estimates of the function values by selecting $\noiseCap_t$ from $P_{\noiseCap}$; we will refer to this case as the \emph{controlled noise} setting and to other situations as the case of \emph{uncontrolled noise}. 
%The advantage of the controlled noise setting is that the algorithm may obtain samples of $F$ at $X^+$ and $X^-$, with the same noise levels:
In a controlled noise setting, the algorithm may obtain samples of $F$ at $X^+$ and $X^-$, with the same noise levels, leading to improved accuracy in the gradient estimates. This is common in the field of \emph{simulation optimization}
\citep{KlSpNa99,duchi2015optimal} and is especially useful when the gradient estimate is created from multiple observations, see \eqref{eq:twosp}. Note that creating an estimate from $K$ points (which is equivalent to a multi-point feedback setup where $K$ points are queried in each round) changes the number of rounds from $n$ to $n/K$, and  the reduction in the number of rounds, being a fixed constant factor, is negligible, as far as the convergence rates are considered.


%
% as in the online case.
%
%However, what differs in the optimization variant is that the value of the loss at the points sent to the oracle does not matter.
%Hence, in this setting the distinction between one-point and multi-point feedback (as long as the number observations is fixed, independently of the dimension) is irrelevant:


%A major tool in bandit convex optimization is to design gradient estimators, which are then used in conjunction with variants of gradient descent. \todoc{zillions of references.}
\subsection{Estimating the Gradient}

A common popular idea in bandit convex optimization is to use the bandit feedback to construct noisy (and biased) estimates of the gradient.
In the following, we provide a few examples for oracles that construct gradient estimates for function classes that are increasingly general: from smooth, convex to non-differentiable functions.

\paragraph{One-point feedback}
Given $x\in \cK$, $0<\delta\le 1$, common gradient estimates that are
based on a single query to the function evaluation oracle (the so-called
``one-point feedback'') take the form
\begin{align}
  \label{eq:one-point}
G = \frac{Z}{\delta}V, \textrm{ where } Z = f(x+\delta U) + \xi\,,
\end{align}
where $(U,V)\in \R^d\times \R^d$ are jointly distributed random variables
and $\xi$ is the function evaluation noise (the distribution of $\xi$ may depend on $x+\delta U$) and $G$ is the estimate of $\nabla f(x)$ ($f:\cK\to \R$).

In all oracle constructions we will use the following assumption:
\begin{ass}
  \label{ass:gradbasic}
  Let $\K \subset \D^\circ \subset \R^d$, where $f:\D \to \R$.\footnote{Here, $\D^\circ$ denotes the interior of $\D$.}
  For any $x\in \K$, $x+\delta U \in \D$ a.s.,
  and $\EE{\norm{V}_*^2}$, $\EE{ \norm{U}^3 }<+\infty$.
\end{ass}
Note that here the function domain $\D$ can be larger than or equal to the set $\K$, where the algorithm chooses $x$. This is to ensure that the oracle will not receive invalid inputs, that is, queries where $f$ is not defined.
When the functions are defined over $\K$ only and $\K$ is bounded, the above constructions only work for $\delta$ small enough.
In this case, the best approach perhaps is to use Dikin ellipsoids to construct the oracles, as done by \citet{HaLe14:SOC}.

The next proposition, whose proof is based on ideas from \citet{spall1992multivariate}\todoa{add citation for the convex case} shows that the above one-point gradient estimator leads to a type-I (and, hence, also type-II) oracle under slightly stronger assumptions.

\begin{proposition}
\label{prop:grad-onepoint}
Let \cref{ass:gradbasic} hold and let $\gamma$ be the one-point feedback oracle defined in \eqref{eq:one-point}.
Assume further that
  $U$ is symmetrically distributed,
  $V = h(U)$, where $h:\R^d \to \R^d$ is an odd function,
  $\EE{V}=0$, and $\E[V U\tr] = I$.
Then $\gamma$ is a type-I oracle with $c_1(\delta)$ and $c_2(\delta)$ defined for the uncontrolled noise case in \cref{tab:oracles}
when $f$ is either convex and $L$-smooth or three times continuously differentiable (in notation: $f \in \C^3$).
% \todoc{Add the convex+smooth case}
\end{proposition}
%\begin{proof}
%See Section \ref{sec:appendix-grad}.
%\end{proof}

\paragraph{One-point feedback with smoothing}
Another possibility is to use the so-called smoothing technique
\citep{PoTsy90,flaxman2005online,HaLe14:SOC}\todoa{ Cs: Is this correct?}
to obtain type-II oracles:


Type-II oracles can be obtained using the same construction as above, but with a different analysis:
\begin{proposition}
\label{prop:flaxman} Let \cref{ass:gradbasic} hold and let $\gamma$ be the one-point feedback oracle defined in \eqref{eq:one-point}.
Define
%\begin{align*}
$
V = n_W(U)\dfrac{\lvert \partial W\rvert}{\lvert W \rvert}\,,
$
%\end{align*}
where $W \subset \R^n$ is a convex body with boundary $\partial W$, $U$ is uniformly distributed in $\partial W$, $n_W(U)$ denotes the normal vector of $\partial W$ at $U$, and $\lvert \cdot \rvert$ denotes the volume.
Then, if $f$ is Lipschitz, $\gamma$ is a type-II oracle with $c_1(\delta)=C_1 \delta$, $c_2(\delta) = C_2/\delta^2$ for some appropriate constants $C_1,C_2>0$ depending on $W$ and the Lipschitz constant of $f$.
If $f$ is smooth, further assuming $W$ is symmetric w.r.t. the origin, $\gamma$ is a type-I (and type-II oracle) with $c_1(\delta) = C'_1\delta^2$, $c_2(\delta) = C'_2/\delta^2$ for some appropriate constants $C_1,C_2>0$ depending on $W$ and the smoothness parameter of $f$.\todoa{Do we need convexity here?}
\end{proposition}
Note that by the above proposition, the smoothing technique provides an improved gradient oracle for the smooth and convex case than \cref{prop:grad-onepoint}.
%\begin{proof}
%See Section \ref{sec:appendix-grad}.
%\end{proof}



\paragraph{Two-point feedback}
While the one-point estimators are intriguing, as discussed beforehand,
in the optimization setting one can also always group two consecutive observations and obtain similar smoothing-type estimates 
at the price of reducing the number of rounds by a factor of two only, which has a negligible effect on the rate of convergence.
Next we present an oracle that uses two function evaluations to obtain a gradient estimate.
As will be discussed later, this oracle encompasses several simultaneous perturbation methods (see \citealp{bhatnagar-book}):
Given the inputs $x\in \K$,  $0<\delta\le 1$,
% \todoc{We need an upper bound I believe. Add it earlier.}
the gradient estimate is
\begin{align}
G &=  \dfrac{Z^+ - Z^-}{2\delta}\, V \,, 
 \label{eq:twosp}
\end{align}
where $Z^{\pm} = f(X^{\pm}) + \xi^{\pm}$, $X^{\pm} = x \pm \delta U$, $U,V\in \R^d$, $\xi^{\pm}\in \R$ are random, jointly distributed random variables, $U,V$ chosen by the oracle in the uncontrolled case and chosen by the algorithm in the controlled case
from some fixed distribution characterizing the oracle, and $\xi^{\pm}$ being the noise of the returned feedback $Z^{\pm}$ at points $X^{\pm}$.
For the following proposition we consider $4=2\times 2$ cases.
First, the function is either assumed to be $L$-smooth and convex (i.e., the derivative of $f$ is $L$-Lipschitz w.r.t. $\norm{\cdot}_*$), or it is assumed to be three times continuously differentiable ($f\in C^3$).
The other two options are either the controlled noise  setting of \eqref{eq:controlled}, or, in the uncontrolled case, we make the alternate assumptions
\begin{align}
\E[\xi^+-\xi^- |\, U,V] = 0 \text{~~ and ~~}\nonumber\\
\E [ (\xi^{+} - \xi^-)^{2} |\, V] \le \sigma_\xi^2 <\infty\,.
\label{eq:noiseass}
\end{align}

%%%%%%%%%%%%%%%%%%%%%%%%%%%%%%%%%%%%%%%%%%%%%%%%%%%%%%%%%%%%%%%%%%%%%%%%%%%%%%%%%%%%%%%%%%%%%%%%%%%%%%%%%%%%%%%%%%%%%%%%%%%%%%%%%%%%%%%%%%%%%%%
\begin{table}
\small
\centering
\begin{tabular}{|c|c|c|}
\toprule
\textbf{Noise }$\bm{ \rightarrow}$ & \textbf{Controlled } & \textbf{Uncontrolled } \\
\textbf{Function } &(see~\eqref{eq:controlled})&(see~\eqref{eq:noiseass})\\
$\bm{\downarrow}$ &&\\\midrule
\multirow{3}{*}{\textbf{Convex + Smooth}} & \multirow{3}{*}{$(C_1 \delta, C_2)$} & %\multirow{2}{*}
{Prop~\ref{prop:grad-onepoint}: $(C_1\delta, \frac{C_2}{\delta^2}) $}\\
&& \\
 &&{Prop~\ref{prop:flaxman}: $(C'_1\delta^2, \frac{C_2}{\delta^2}) $}\\\midrule
\multirow{2}{*}{$\bm{f \in \C^3}$} & \multirow{2}{*}{$(C_1 \delta^2, \frac{C_2}{\delta^2})$} & \multirow{2}{*}{$(C_1 \delta^2, \frac{C_2}{\delta^2})$} \\
 &&\\\bottomrule
\end{tabular}
\caption{Gradient oracles for different function classes and noise categories. Each table entry specifies the pair $(c_1(\delta), c_2(\delta))$.
For the first row:  
$C_1 = \frac{L(\Psi)}{2} \E[ \dnorm{V} \norm{U}^2]$ and 
and
$C_2 =  2 B_1^2  + \frac{ L(\Psi)^2}{2}\E\left[ \dnorm{V}^2 \norm{U}^4 \right]$, with $B_1 = \sup_{x\in \K} \scnorm{\nabla f(x)}_*$, for the controlled noise;
$C_1 =
\frac{L}{2} \E[ \dnorm{V} \norm{U}^2]$, $C'_1=L \E[\|V\|^2]$,  and
 $C_2 =  C_{2}^{(u)} \doteq 4 \EE{\norm{V}_*^2}\left( \sigma_\xi^2+\fspan(f)\right)$ for the uncontrolled noise.
For the second row, $C_1 = \frac{B_3 \EE{ \norm{V}_* \norm{U}^3 }}{6}$ and $C_2 =  C_{2}^{(u)}$,
with $B_3 = \sup_{x\in \K} \norm{\nabla^3 f(x)}$, where $\norm{\cdot}$ is the implied norm for rank-3 tensors.
}
\label{tab:oracles}
\end{table}
%%%%%%%%%%%%%%%%%%%%%%%%%%%%%%%%%%%%%%%%%%%%%%%%%%%%%%%%%%%%%%%%%%%%%%%%%%%%%%%%%%%%%%%%%%%%%%%%%%%%%%%%%%%%%%%%%%%%%%%%%%%%%%%%%%%%%%%%%%%%%%%%
The following proposition, whose proof is based on \citep[Lemma~1]{spall1992multivariate} and \citep[Lemma~1]{duchi2015optimal}, provides conditions under which the bias-variance parameters $(c_1,c_2)$ can be bounded as shown in \cref{tab:oracles}:
\begin{proposition}
\label{prop:grad-spsa}
Consider a function $f:\D \to \R$, with $\K \subset \D^\circ \subset \R^d$. %\footnote{Here, $\D^\circ$ denotes the interior of $\D$.}
For any $x \in \cK$, and $0< \delta \le 1$ let the oracle $\gamma$ return $G$ as specified in~\eqref{eq:twosp},
where $(U,V)$ are such that $x+\delta U \in \D$ a.s.,
$\E[V U\tr] = I$, while
$\EE{\norm{V}_*^2}$ and $\EE{ \norm{U}^3 }<+\infty$.
Then $\gamma$ is a type-I oracle with $c_1(\delta)$ and $c_2(\delta)$ given by \cref{tab:oracles}.
\end{proposition}
%\begin{proof}
%See Section \ref{sec:appendix-grad}.
%\end{proof}
\todoc[inline]{I am pretty sure that the bias could be brought down
for two-point feedback, too, using the technique of \cref{prop:flaxman}.
}

\paragraph{Popular choices for $U$ and $V$:}\ \\
\begin{inparaenum}[$\bullet$]
 \item If we set $U_i$ to be independent, symmetric $\pm 1$-valued random variables and $V_i = 1/U_i$, then we recover the popular SPSA scheme proposed by \cite{spall1992multivariate}.
It is easy to see that $\EE{  V U\tr } = I$ holds in this case.
 When the norm $\norm{\cdot}$ is the $2$-norm, $C_1 = O(d^2)$ and $C_2 = O(d)$. If we set $\norm{\cdot}$ to be the max-norm, $C_1 = O(\sqrt{d})$ and $C_2 = O(d)$.\\
 \item If we set $V=U$ with $U$ chosen uniform at random on the surface of a sphere with radius $\sqrt{d}$,
 then we recover the RDSA scheme proposed by  \citet[pp.~58--60]{kushcla}.
 In particular, the $(U_i)$ are identically distributed with $\EE{ U_i U_j } = 0$ if $i\ne j$ and $\EE{ U\tr U } = d$, hence $\EE{U_i^2} = 1$. Thus, if we choose $\norm{\cdot}$ to be the $2$-norm, $C_1 = O( d^2 )$ and $C_2 = O(d)$.\\
 \item If we set $V=U$ with $U$ the standard $d$-dimensional Gaussian with unit covariance matrix, we recover the smoothed functional (SF) scheme proposed by \cite{katkul}.
Indeed, in this case, by definition, $\EE{VU\tr} = \EE{U U\tr } = I$.
When $\norm{\cdot}$ is the $2$-norm, $C_1 = O(d^2)$
%$\norm{U}^4 = (\sum_{i=1}^d U_i^2)^{2} = \sum_i U_i^4 + 2 \sum_{i<j} U_i^2 U_j^2$,
%hence $\EE{ \norm{U}^4} = O(d^2)$
 and $C_2 = O( d)$.
 This scheme can also be interpreted as a smoothing operation that  convolves the gradient of the function $f$ with a Gaussian density.
%  , followed by an integration by parts and the resulting integral can be estimated using samples (without access to the gradient of $f$).
\end{inparaenum}


% If the function $f$ is assumed to be convex and smooth, then gradient estimates similar to \eqref{eq:twosp} can be constructed and this does not require higher order smoothness conditions as in Proposition \ref{prop:grad-spsa}.
% \begin{proposition}
% \label{prop:grad-convex}
% Consider a function $f:\D \to \R$, with $\K \subset \D \subset \R^d$,
% that is convex and has a $L$-Lipschitz continuous gradient. .
% For any $x \in \cK$, and $\delta >0$ such that $\B(x,\delta) \in \D$, let the oracle $\gamma$ return
% \begin{align}
% % Y = x+\delta U \,, \quad
% G =  V \left(\dfrac{f(x+\delta U) + \xi^+ - (f(x-\delta U) + \xi^-)}{2\delta}\right),
%  \label{eq:twosp}
% \end{align}
% where $\xi^+, \xi^-$, $V$, $U$ are as in Proposition \ref{prop:grad-spsa}.
% Then, we have that $\gamma$ is a type-I oracle with $c_1(\delta) = C_1 \delta$ and $c_2(\delta) = C_2 d/\delta^2$.
% \end{proposition}
% \begin{proof}
% See Appendix \ref{sec:appendix-grad}.
% \end{proof}


% A popular idea for estimating gradient using one-point feedback is the smoothed function approach, which was originally proposed in \citep{katkul}. The idea is to convolve the gradient of the objective function with a suitable density function and then, via an integration by parts arguments show that the resulting integral is an estimate of the gradient of the smoothed objective function.
% This approach has been adopted in a stochastic convex optimization setup in \cite{duchi2015optimal}.

%% \paragraph{Simultaneous perturbation methods:}
%The idea of simultaneous perturbation can work even for functions that are not differentiable, as shown in \cite{flaxman2005online}.
%% follow this approach in the context of bandit convex optimization.
%Formally,
%% a $(c_1,c_2)$ Type-II oracle when $\cF$ is a general class of functions (this does not even require differentiability).
%given any $f \in \cF$, $x \in \cK$, and $\delta >0$, the oracle returns\footnote{See also \cite[pp.~58-60]{kushcla} for an old reference that proposed a gradient estimate similar to \eqref{eq:flaxman}.}
%\begin{align}
% Y = x+\delta u \in \cK', \quad
% G = \dfrac{d}{\delta}f(x+\delta u)u \in \R^d, \label{eq:flaxman}
%\end{align}
%where $u\in \R^d$ is a random unit vector, so the first condition of \cref{def:oracle2} immediately follows.
%The variance of $G$ is bounded by $d^2C^2 \delta^{-2}$ for some constant $C = \sup_{y\in \cK'}f(y)$.
%\todox[inline]{Fix this proposition statement to say Flaxman scheme is a type II oracle}
%\begin{proposition}
%Let $\tilde{f}(x) = \EE{f(x+\delta v)}$ denote the smoothed version of $f$. Then, we have
%$\EE{G} = \nabla \tilde{f}(x)$.
%\end{proposition}
%\begin{proof}
% See Lemma 1 in \citep{flaxman2005online}.
%\end{proof}
% As to the bias condition, it was proved that $\EE{G} = \nabla \tilde{f}(x)$, where $\tilde{f}$ is a smoothed version of $f$, i.e.,
%$\tilde{f}(x) = \EE{f(x+\delta v)}$,
%$v$ is a random vector in a unit ball. There are different ways to bound the bias depending on the property of $f$.
%If $f$ is $L_{lip}$-Lipschitz over $\cK'$, then we have
%\begin{align*}
%\MoveEqLeft
%\norm{\tilde{f}(x)-f(x)}_\infty \\
%=&\norm{\EE{f(x+\delta v)-f(x)}}_\infty
%\le L_{lip} \delta \,.
%\end{align*}
%If $f$ is convex, and $L_{smo}$-smooth,
%\begin{align*}
%\MoveEqLeft
%\norm{\tilde{f}(x)-f(x)}_\infty \\
%\le& \norm{\EE{\ip{\nabla f(x), \delta v}+\dfrac{L_{smo}}{2}\delta^2\norm{v}^2}}_\infty\\
%\le &\dfrac{L_{smo}}{2}\delta^2 \,.
%\end{align*}
%Therefore, the estimator \eqref{eq:flaxman} can always fit the oracle setting by choosing $c_1(\delta) = C_1 \delta$ (or $C_1\delta^2$), $c_2(\delta) = C_2 \delta^{-2}$, for some constant $C_1$, $C_2$.

% \begin{remark}
%  Instead of picking $u$ randomly on the surface of a unit sphere, one can employ an random variable $u$ that satisfies $E[u u\tr] = I_d$, where $I_d$ is the $d$-dimensional identity matrix. A popular choice for $u$ that satisfies the aforementioned constraint is the $d$-dimensional standard Gaussian - a choice that has been explored in the context of zeroth order optimization in \citep{duchi2015optimal}. See \citep{bhatnagar-book} for an overview of gradient and Hessian estimation techniques using random perturbations.
% \end{remark}


\subsection{Achievable Results for Stochastic BCO}
We now consider stochastic BCO with $L$-smooth functions over a convex, closed non-empty domain $\K$. \todoc{Check that $K$ is always assumed to be closed. Otherwise the optimum may not belong to $K$.}\todoa{Duchi et al (2015) assumes closed functions.}
Let $\cF$ denote the set of these functions.
\cite{duchi2015optimal} proves that the minimax expected optimization error
for the functions $\cF$ with uncontrolled noise is lower bounded by $\Omega(n^{-1/2})$. \todoc{We should sort out dimension dependences later.}
They also give an algorithm which uses two-point gradient estimates which matches this lower bound for the case of \emph{controlled noise}.
For controlled noise, the constructions in the previous section give that for two-point estimators $c_1(\delta) = C_1 \delta^p$ and $c_2(\delta) = C_2\delta^{-q}$ with $p=1$ and $q=0$. Plugging this into
\cref{thm:ub} we get the rate $O(n^{-1/2})$ (which is unsurprising
given that the algorithms and the upper bound proof techniques are essentially the same as that of \citealp{duchi2015optimal}).
However, when the noise is uncontrolled, the best that we get is $p=2$ and $q=2$.
From \cref{thm:lb-convex} we get that with such oracles, no algorithm can get better rate than $\Omega(n^{-1/3})$, while from
\cref{thm:ub} we get that these rates are matched by mirror descent.
We can summarize these findings as follows:
\todoc{Can we show that no better oracles exist?}
\begin{theorem}\label{thm:aaab}
Consider $\F_{L,0}$, the space of convex, $L$-smooth functions over a convex, closed non-empty domain $\K$.
Then, we have the following:\\
\textit{\textbf{Uncontrolled noise}}:
Take any $(\delta^2,\delta^{-2})$ type-I oracle $\gamma$.
There exists an algorithm that uses $\gamma$
and achieves the rate $O(n^{-1/3})$.
Furthermore, no algorithm using $\gamma$
 can achieve better error than $\Omega(n^{-1/3})$ for every $(\delta^2,\delta^{-2})$ type-I oracle $\gamma$.\\
\textit{\textbf{Controlled noise}}:
Take any $(\delta,1)$ type-I oracle $\gamma$.
There exists an algorithm that uses $\gamma$ an
achieves the rate $O(n^{-1/2})$.
Furthermore, no algorithm using $\gamma$
 can achieve better error than $\Omega(n^{-1/2})$ for every $(\delta,1)$ type-I oracle $\gamma$.
\end{theorem}

For stochastic BCO with uncontrolled noise, \cite{AgFoHsuKaRa13:SIAM} analyze a variant of the well-known ellipsoid method and provide regret bounds for the case of convex, $1$-Lipschitz and bounded functions on $[0,1]$. Their regret bound implies a minimax error \eqref{eq:minimaxerrdef} bound of order  $O\left(\sqrt{d^{32}/n}\right)$.
%\todoc{Not for this setting! This paper should be mentioned in the previous section actually. For this setting, we  do not have \emph{any} algorithms that would achieve $O(n^{1/2})$.}
\cite{liang2014zeroth} provide an algorithm based on random walks (and not using gradient estimates) for the setting of convex, bounded functions whose domain is contained in the unit cube and their algorithm results in a bound of the order $\O\left((d^{14}/n)^{1/2}\right)$ for the minimax error.
%\todoc{What are the conditions for their theorem? Boundedness? Lipschitzness? Or nothing?}
%This algorithm achieves an upper bound of $\O\left((d^{14}/n)^{1/2}\right)$,
These bounds decrease faster in $n$ than the bound available in Theorem \ref{thm:aaab}, while showing a much worse dependence on the dimension.\todoc{We should show the dimension dependence in the result to make this point. This should be shown for the same setting as the one considered by \cite{liang2014zeroth}.}
However, what is more interesting is that our results also shows that an $O(n^{-1/2})$ upper bound \emph{cannot} be achieved solely based on the oracle properties of the gradient estimates considered. Since the analysis of all gradient algorithms for stochastic BCO does this, it is no wonder that the best known upper bound for convex+smooth functions is $O(n^{-1/3})$ \citep{saha2011improved}.

The above result also shows that the gradient oracle based algorithms are optimal for smooth problems, under a controlled noise setting.
While \cite{duchi2015optimal} suggests that it is the power of two-point gradient estimators that help to achieve this, we need to add that a critical condition to achieve the optimal rate is that the noise must be controlled. 

Finally, let us make some remarks on the early literature on this problem.
A finite time lower bound for stochastic, smooth BCO is presented by  \citet{Chen88:LB-AoS} for
convex functions on the real line. When applied to our setting in the uncontrolled noise case, his results imply that $\EE{ \norm{\hat{X}_n - x^* }}$, that is, the distance of the estimate to the optimum, is at least $\Omega(n^{-1/3})$, which matches the bound in \cref{thm:aaab}.
Note that this is larger than the error achieved by the algorithms of \cite{liang2014zeroth,BubeckDKP15,BuEl15}, but the apparent contradiction is easily resolved by noticing the difference in their error measure: distance to the optimum vs. distance n the function value.
Similar results are obtained by \citet{PoTsy90}, who also considered distance to the optimum and proved that mirror descent with gradient estimation achieves asymptotically optimal rates for these settings.

%While the lower bound is stated for $r$-smooth functions with $r$ odd (these are functions $f$ with $\norm{f^{(r)}}_{\infty}\le L$), and $r$ greater than one, careful checking of the results show that nor $r>1$, neither that $r$ must be odd is ever used in the proof.
%Hence, the result holds for $r\ge 2$.
%The lower bound presented for a given value of $r$ takes the form $\Omega( n^{ -(r-1)/(2r)} )$. For $r=2$ (which approximately corresponds to the smooth case), we get $\Omega(n^{-1/4})$, while for $r=3$ (which is ``almost'' the same as $f\in \C^3$), one gets $\Omega(n^{-1/3})$.
%While the second bound matches the bound in \cref{thm:aaab}, the first one is larger, and both bounds are larger than the error achieved by
%the algorithm of \cite{liang2014zeroth,BubeckDKP15,BuEl15}. The resolution of the apparent contradiction is that the lower bounds of \citet{Chen88:LB-AoS} concern distance to the optimum (i.e., $\EE{ \norm{\hat{X}_n - x^* }}$), while the optimization error is defined in terms of the objective function gap $\EE{ f(\hat{X}_n) -f( x^*)}$.
%Similar results are obtained by \citet{PoTsy90}, who also considered distance to the optimum and proved that mirror descent with gradient estimation achieves asymptotically optimal rates for these settings.

\todoc{Cite all the other papers, summarizing what they achieve. }

%%% Local Variables:
%%% mode: latex
%%% TeX-master: "bgo"
%%% End:


\section{Applications to Online BCO}
\label{sec:obco}
%!TEX root =  bgo-cam-ready.tex
% please do not delete or change the first line (needed by Csaba's editor)
In the \emph{online BCO} setting a learner sequentially chooses the points $X_1,\dots,X_n\in \cK$ while observing the losses $f_1(X_1),\dots,f_n(X_n)$. More specifically, in round $t$, having observed $f_1(X_1),\dots,f_{t-1}(X_{t-1})$ of the previous rounds, the learner chooses $X_t\in \cK$, after which it observes $f_t(X_t)$. The learner's goal is to minimize its expected regret $\EE{ \sum_{t=1}^n f_t(X_t) - \inf_{x\in \cK} \sum_{t=1}^n f_t(x) }$. 
This problem is also called online convex optimization with one-point feedback.
A slightly different problem is obtained if we allow the learner to choose multiple points in every round, at which points the function $f_t$ is observed. The loss is suffered at $X_t$. The points where the function is observed (``observation points'' for short) may or may not be tied to $X_t$. One possibility is that $X_t$ is one of the observation points.  
Another possibility is that $X_t$ is the average of the observation points (e.g., \citet{AgDeXi10}). Yet another possibility is that there is no relationship between them. 

The oracle constructions from the previous section also apply to the online BCO setting
where the algorithm is evaluated at $Y_t$, though in this case 
one cannot employ two-point feedback as the functions change between rounds. 
This also rules out the controlled noise case. 
Thus, for the online BCO setting, one should consider type-I (and II) oracles with $c_1(\delta) = C_1 \delta^p$ and $c_2(\delta) = C_2\delta^{-q}$ with $p=q=2$.
For these type of oracles, the results from Theorem~\ref{thm:lb-convex} give the following result: 
\begin{theorem}\label{thm:aaa}
Let $\cF_{L,0}$ be the space of convex, $L$-smooth functions over a convex non-empty domain $\K$.
No algorithm that relies on 
 $(\delta^2,\delta^{-2})$ type-I oracles
 can achieve better regret than $\Omega(n^{2/3})$.
\end{theorem}
%\vspace{-0.2cm}
With a noisy gradient oracle of Proposition~\ref{prop:flaxman}, Theorem~\ref{thm:aaa} implies that this regret rate is achievable, essentially recovering, and in some sense proving optimality of the result of \citet{saha2011improved}:
\begin{theorem}
For zeroth order noisy optimization with smooth convex functions, the gradient estimator of Proposition~\ref{prop:flaxman} together with mirror descent (see Algorithm~\ref{alg}) achieve $\O(n^{2/3})$ regret.
\end{theorem}
%\vspace{-0.2cm}
This optimality result shows that with the usual analysis of the current gradient estimation techniques, no gradient method can achieve the optimal regret $O(n^{1/2})$ for online bandit convex optimization, established by \citet{BubeckDKP15,BuEl15}. Note that this shows a contradiction to the recent result of \citet{DeElKo15}, who claimed to achieve $\tilde{O}(n^{5/8})$ regret with the same $(\delta^2,\delta^{-2})$ type-II gradient oracle as \citet{saha2011improved}, but their proof only used the $(\delta^2,\delta^{-2})$ tradeoff in the bias and variance properties of the oracle.\todoa{Actually, all these proofs use Type-IIb oracles....}





%%% Local Variables:
%%% mode: latex
%%% TeX-master: "bgo"
%%% End:


\section{Related Work}
\label{sec:related}
%!TEX root =  bgo-cam-ready.tex
% please do not delete or change the first line (needed by Csaba's editor)
Gradient oracle models have been studied in a number of previous papers 
\citep{dAsp08,Baes09,SchRoBa11,DeGliNe14}.
%assume that set $\K$ is bounded and that the oracle provides at each point $x\in \K$ an approximate gradient $g$ satisfying condition
%$|\ip{g-\nabla f(x), v-w}| \le \delta$ for all $v,w,x\in \K$.
A full comparison between these oracle models is given by \citet{DeGliNe14}.
For illustration, here we only review the model of this latter paper as a typical example of these previous works.
The model of \citet{DeGliNe14} assumes a first-order approximation to the function
with parameters $(\delta,L)$. In particular, 
given $(x,\delta,L)$ and the convex function $f$, 
the oracle gives a pair $(t,g)\in \R \times \R^d$
such that $t + \ip{g,\cdot-x}$ is a linear lower approximation to $f(\cdot)$ in the sense that 
$0\le f(y) - \left\{ t+ \ip{g,y-x}\right\} \le \frac{L}{2} \norm{y-x}^2 + \delta$.
\citet{DeGliNe14} argue that this notion appears naturally in several optimization problems and study whether the so-called accelerated gradient techniques are still superior to their non-accelerated counterparts (and find a negative answer).
The authors study both lower and upper rates of convergence, similarly to our paper.
A major difference between the previous and our settings is that we allow stochastic noise (and bias), which the algorithms can control, while the oracle in these previous paper must guarantee that the accuracy requirements hold in each time step
with probability one.
This is a much stronger requirement, which may be impossible to satisfy in some problems, such as when 
the only information available about the functions is noise contaminated.

Some works, such as \citet{SchRoBa11} allow arbitrary sequences of errors and show error bounds as a function
of the accumulated errors. 
Our proof technique is actually essentially the same (as can be expected).
However, the noisy case requires special care. For example, Proposition~3 of
\citet{SchRoBa11}  bounds the optimization error for the smooth, convex case by 
$O(1/n^2 ( \norm{x_1-x^*}^2 + A_n^2 ))$ where $A_n = O( \sum_{t=1}^n t \norm{e_t})$, $e_t$ being the error of the approximate gradient. This expression becomes $\Theta(\frac{1}{n^2} \sum_{t=1}^n t^2)  \approx n$
%is upper and lower bounded, up to a constant factor by,
assuming that errors' noise level is a positive constant (in all our result, this holds).
This clearly shows that the noisy case requires (somewhat) special treatment.

Similar, but simpler noisy oracle models were introduced \citep{JN11a,Hon12,DvoGa15}, but these models lack the bias-variance tradeoff central to this paper (i.e., they assume the variance and bias can be controlled independently of each other). The results in these papers are upper bounds on the error of certain gradient methods (also to some very specific problem for \cite{Hon12}), and they correspond to the bounds we obtained with $q=0$.
\todoc[inline]{The paper of \citet{SchRoBa11} has 100 citations on google scholar.
The paper of \cite{DeGliNe14} has similarly many.
There are probably more relevant works. Someone could take a quick look and perhaps add a few more relevant ones.
}





%\section{PROOFS}
%\label{sec:proofs}
%%%!TEX root =  bgo.tex
% please do not delete or change the first line (needed by Csaba's editor)
\subsection{Proof of Theorem \ref{thm:ub}}
\label{sec:ub-proof}

Before giving the proof, we introduce a useful lemma, which is essentially Theorem~C.4 of \cite{MahdaviPhd:2014}.
\begin{lemma}
\label{lem:ub}
Let $({\tilde{\cF}})_{t}$ be a filtration such that $X_t$ is ${\tilde{\cF}}_t$-measurable.
Let $\overline G_t = \EE{G_t|{\tilde{\cF}}_t}$ 
and assume that the nonnegative real-valued deterministic sequence $(\beta_t)_{1\le t\le n}$ is such that 
$\norm{\overline G_t - \nabla \tilde{f}(X_t)}_* \le \beta_t$ holds almost surely. 
Further, assume that $D=\sup_{x,y\in \cK} D_{\mathcal{R}}(x,y)$ and let $\eta_t = \frac{\alpha}{a_t+L}$ for some increasing 
sequence $(a_t)_{t=1}^{n-1}$ of numbers. Then, if $\tilde{f}$ is $L$-smooth,
\begin{align*}
& \EE{ \sum_{t=1}^n \tilde{f}(X_t) - \tilde{f}(x) }  \le 	 \EE{\tilde{f}(X_1)-\tilde{f}(x)} \\
 &\quad\qquad +\sqrt{\tfrac{2D}{\alpha}} \sum_{t=1}^{n-1} \beta_t 
 +\frac{D(a_{n-1}+L)}{\alpha} +
	  \sum_{t=1}^{n-1}\frac{\sigma_t^2}{2a_t}\,,
\end{align*}
where $\sigma_t^2 = \EE{ \norm{G_t-\overline G_t}_*^2}$ is the ``variance'' of $G_t$.

Suppose ${\tilde{f}}$ is also $\mu$-strongly convex w.r.t. $\mathcal{R}$, which is $\alpha$-strongly convex, $\alpha > \dfrac{2L}{\mu}$. Letting $\eta_t = \dfrac{2}{\mu t}$, $a_t = \dfrac{\alpha \mu}{2}t-L > 0$, we have 
\begin{align*}
\MoveEqLeft \EE{ \sum_{t=1}^n \tilde{f}(X_t) - \tilde{f}(x) }  \\
\le& 	 \EE{\tilde{f}(X_1)-\tilde{f}(x)}+ 
 \sqrt{\tfrac{2D}{\alpha}} \sum_{t=1}^{n-1} \beta_t 
 +\sum_{t=1}^{n-1}\frac{\sigma_t^2}{2a_t}\,.
\end{align*}
\end{lemma}
This is also identical to Theorem~6.3 of \cite{Bu:Convex14}, who cites \cite{Dekel:minibatch12} as the source. These results assume that $\beta_t=0$ above.
The proof of this lemma can be found in \cref{sec:appendix-md}.

Proof of Theorem \ref{thm:ub}:\footnote{See Appendix \ref{sec:appendix-md} for a detailed proof.}

Since $f$ is convex, by Jensen's inequality and the bias condition in \cref{def:oracle2}, we get
\begin{align*}
\MoveEqLeft
 \E[f(\hat X_n) - \inf_{x \in \cK} f(x)] \\
 \le&  
 \dfrac{1}{n}\EE{ \sum_{t=1}^n f(X_t) - \inf_{x \in \cK}f(x) } \\
 \le &\dfrac{1}{n}\EE{ \sum_{t=1}^n \tilde{f}(X_t) - \inf_{x \in \cK}\tilde{f}(x) } +2C_1 \delta^p
 \,.
\end{align*}
Given $\overline{G}_t=\EE{G_t} = \nabla \tilde{f}(X_t)$,
the result immediately follows from Lemma \ref{lem:ub}. 
When $f$ is smooth,
\todox{$\tilde{f}$ should also be smooth, but this cannot be ensured with current oracle assumption...}
substituting
%\todox{This may be confusing... since we are using type-II oracle, $\beta_t$ is actually 0.}
 $\beta_t = 0$, $\sigma^2_t = C_2 \delta^{-q}$, $a_t=a t^r$ for some $0<r<1$, respectively, we obtain
 \begin{align}
 \MoveEqLeft
 \E[f(\hat X_n) - \inf_{x \in \cK} f(x)]  \nonumber\\
&\le \dfrac{1}{n}\left(\EE{f(X_1)-\inf_{x \in \cK}f(x)}+\dfrac{DL}{\alpha}  \right)\nonumber\\
%& + \sqrt{\frac{2D}{\alpha}} C_1 \delta^p + \frac{D a}{\alpha} n^{r-1} + \frac{C_2 \delta^{-q}}{a(1-r)} n^{-r} \,.
&+\dfrac{Da}{\alpha} n^{r-1}+\dfrac{C_2 \delta^{-q}}{2a(1-r)}n^{-r}+ (2+\dfrac{2}{n})C_1\delta^p \,.
\label{eq:ubToBeOpt}
 \end{align}
 Choosing $r = \dfrac{p+q}{2p+q}$, and $a$, $\delta$ as stated in \cref{thm:ub}, the last $3$ terms in \eqref{eq:ubToBeOpt} are optimized to
 \[
 K_1 C_1^{q/(2p+q)} C_2^{p/(2p+q)} n ^{-p/(2p+q)} \,,
 \]
 where 
% $K_1 = 2^{\dfrac{3q}{2(p+q)}} \left( \sqrt{\dfrac{D}{\alpha}} \right)^{\dfrac{p+q^2+2pq}{(p+q)(2p+q)}}+2^{\dfrac{3q}{2(2p+q)}} \left( \sqrt{\dfrac{D}{\alpha}} \right)^{2-\dfrac{1}{2p+q}}$.
 $(K_1/3)^{2p+q} \le 2^{2q-p}\left(2+q/p \right)^p \left(D/\alpha\right)^{p}$.
 
 When $f$ is also $\mu$-strongly convex,
 \begin{align*}
& \E[f(\hat X_n) - \inf_{x \in \cK} f(x)] -\dfrac{1}{n}\left(\EE{f(X_1)-\inf_{x \in \cK}f(x)}\right)\\
&\le (2+\dfrac{2}{n})C_1\delta^p+\dfrac{C_2 \delta^{-q}}{\alpha \mu n} \sum_{t=1}^{n-1}\dfrac{1}{t-\dfrac{2L}{\alpha \mu}}\\
&\le (2+\dfrac{2}{n})C_1\delta^p+\dfrac{C_2 }{\alpha \mu}\delta^{-q} \dfrac{\log n+1+\alpha \mu/(\alpha \mu-2L)}{n}\\
&\le K_2C_1^{\dfrac{q}{p+q}}C_2^{\dfrac{p}{p+q}} \left( \dfrac{\log n+1+\dfrac{\alpha \mu}{\alpha \mu -2L}}{n} \right)^{\dfrac{p}{p+q}}
 \end{align*}
The last step optimizes the bound via letting 
%$\delta^{p+q} =  \sqrt{\dfrac{\alpha}{2D}}\dfrac{C_2 \log n}{\alpha \mu C_1 n}$, 
$\delta^{p+q} =  \dfrac{C_2\left( \log n+1+\dfrac{\alpha \mu}{\alpha \mu -2L}\right)}{2\alpha \mu C_1 (n+1)}$, 
where
%$K_2^{p+q}=\sqrt{2}^{2p+3q}D^{q/2}\alpha^{-p-q/2}\mu^{-p}$.
$K_2^{p+q}=2^{q}\alpha^{-p}\mu^{-p}$.
%%% Local Variables:
%%% mode: latex
%%% TeX-master: "bgo"
%%% End:

%%!TEX root =  bgo-cam-ready.tex
% please do not delete or change the first line (needed by Csaba's editor)
%\subsection{Proof of Theorem  \ref{thm:lb-convex}}
\label{sec:lb-proof}
We provide a sketch of the proof below for the case of convex and smooth function class $\F_{L,0}(\K)$. The detailed proof for $\F_{L,0}(\K)$ as well as $\F_{L,1}(\K)$ is available in Appendix \ref{sec:appendix-lb-proof}.
We first lower bound  the minimax error in one dimension and then generalize it using a separability argument. For the proof in one dimension, we follow the technique from \cite{shamir2012complexity,duchi2015optimal},  

%%%%%%%%%%%%%%%%%%%%%%%%%%%%%%%%%%%%%%%%%%%%%%%%%%%%%%%%%%%%%%%%%%%%%%%%%%%%%%% 
\paragraph{Proof in one dimension}
% For brevity, let $\Delta_n^{(1)*}$ denote the minimax error $\Delta_n^*(\F, c_1,c_2)$ for the case of $1$-dimensional function family $\F$.
% The proof uses only $(c_1,c_2)$ Type-I oracles $\gamma$ that map $\cK$ to $\R$ (i.e., the oracles do not have memory).
% Fix a  $(c_1,c_2)$ Type-I oracle $\gamma$ and an algorithm $\A$.
% We restrict the class of oracles to those that return a random gradient estimate $G = m(x) + \xi$ with some map $m: \cK \to \R$,
% where $\xi$ is standard normal with variance $\sigma^2 = C_2 \delta^{-q}$, satisfying the variance requirement. 
% The map $m$, which, by slightly abusing notation, we will also denote by $\gamma$ in what follows, will be chosen based on $f$ to satisfy the requirement on the bias. The $Y$ value returned by the oracles is made equal to $x$.

% Let $\delta$ denote the tolerance parameter chosen by $\A$.
% Define the probability space $(\Omega, \B, P_{\A,\gamma})$, 
% where $\Omega = \R^n\times \{-1,1\}$, $\B$ is the associated Borel sigma algebra. 
% Further, the probability measure $P_{\A,\gamma} := p_{\A,\gamma} d(\lambda \times m)$, 
% 	$\lambda$ is the Lebesgue measure on $\R^n$, 
% 	$m$ is the counting measure on $\{-1,1\}$ and 
% 	$p_{\A,\gamma}$ is the density function defined as
% \begin{align*}
% &p_{\A,\gamma}(g_{1:n}, v) = \dfrac{1}{2} \bigg(p_{\A,\gamma}(g_n \mid g_{1:n-1})\times \dots \\
% &\times p_{\A,\gamma }(g_{n-1} \mid g_{1:n-2}) \ldots p_{\A,\gamma}(g_1)\bigg)\\
%  = & \dfrac{1}{2} \bigg( p_{\N}(g_n - \gamma(\A_n(g_{1:n-1}))) \cdot
% %  &\times p_{\normal(0,\sigma^2)}(g_{n-1} - \gamma(\A_n(g_{1:n-2}))) \times \ldots \\
%  \ldots \cdot  p_{\N}(g_1 - \gamma(\A_1))\bigg),
% \end{align*}
% where $v\in\{0,1\}$, $p_{\N}$ is the density of a $\normal(0,\sigma^2)$ random variable,
% and $\cA_t$ denotes the map from the algorithm's past observations
% that picks the point that is sent to the oracle in round $t$.
% \todoc{Enough to consider deterministic algorithms}

%%%%%%%%%%%%%%%%%%%%%%%%%%%%%%%%%%%%%%%%%%%%%%%%%%%%%%%%%%%%%%%%%%%%%%%%%%%%%%% 
% We first prove the theorem in a one-dimensional setting and generalize it later.
Define $\F_{L,0}$ to be comprised of $f_v$ for $v \in \{-1,+1\}$, where
% By assumption,  $\{f_+, f_-\}\subset \F$, with 
%\begin{align*}
%  f_v(x) := \dfrac{\epsilon}{2} (x - v)^2, \,\, x \in \cK \subset \R\,.
%\end{align*}
\begin{align*}
f_v(x) :=\epsilon\left( x-v\right)+2\epsilon^2 \ln\left(1+e^{-\frac{x-v}{\epsilon}}  \right), \,\, x \in \cK \subset \R\,.
\end{align*}
The first and second order derivatives of $f_v$ are
\begin{align*}
f'_v(x) &=\epsilon \dfrac{1-e^{-\frac{x-v}{\epsilon}}}{1+e^{-\frac{x-v}{\epsilon}}} ,\,\,
f''_v(x) = \dfrac{2e^{-\frac{x-v}{\epsilon}} }{\left(  1+e^{-\frac{x-v}{\epsilon}}\right)^2}  .
\end{align*}
% we will slightly abuse notation by using $f_+$ ($f_-$) in place of $f_{+1}$ (resp., $f_{-1}$).
%%%%%%%%%%%%%%%%%%%%%%%%%%%%%%%%%%%%%%%%%%%%%%%%%%%%%%%%%%%%%%%%%%%%%%%%%%%%%%% 
Clearly, $f_v$ is $\frac{1}{2}$-smooth and minimized at $x^*_v = v$.
When $xv<0$, $\epsilon<\dfrac{1}{4\ln 2}$, due to the monotony, we have
\begin{align*}
f_v(x)-f_v(x_v^*) >\epsilon\left( -1 +2\epsilon \ln\dfrac{1+e^{\frac{1}{\epsilon}}}{2}  \right)> \dfrac{\epsilon}{2}.
\end{align*}
Hence,
%Using the fact that $f_+$ and $f_-$ are strongly convex with associated constant $\left(\dfrac{\epsilon}{2}\right)$, we obtain
\begin{align}
  f_v(x) - f_v(x^*_v)
  \ge \dfrac{\epsilon}{2}  \indic{x v  < 0}. \label{eq:main:fv-lb-main}
\end{align}
% Similarly,   $f_-(x) - f_-(x^*_-) \ge  \dfrac{\epsilon}{2}  \indic{x  >0}$.
We will consider the oracles $\gamma_v$ defined as 
\begin{align}
 \gamma_v(x) = \epsilon \dfrac{1-e^{-\frac{x-v}{\epsilon}}}{1+e^{-\frac{x-v}{\epsilon}}} - v\, \min(\epsilon,C_1 \delta^p) + \xi, \label{eq:main:oracle-1d}
\end{align}
where $\xi \sim \normal(0,\frac{C_2}{\delta^q})$.
% ; as with $f_v$, we will also use $\gamma_{+}$ ($\gamma_-$)  to denote $\gamma_{+1}$ (resp., $\gamma_{-1}$).
The oracle is indeed a $(c_1,c_2)$ Type-I oracle, with $c_1(\delta)=C_1\delta^p$ and $c_2(\delta)=\frac{C_2}{\delta^q}$.
%%%%%%%%%%%%%%%%%%%%%%%%%%%%%%%%%%%%%%%%%%%%%%%%%%%%%%%%%%%%%%%%%%%%%%%%%%%%%%% 

Using \eqref{eq:main:fv-lb-main}, the minimax error \eqref{eq:minimaxerrdef} can be lower bounded as follows
%\footnote{$f_{+1} \equiv f_+$ and $f_{-1}\equiv f_-$.}:
\begin{align*}
\MoveEqLeft 
\Delta_n^*(\F_{L,0}, C_1 \delta^p ,C_2 \delta^{-q}) %\nonumber\\
  \ge  \inf_{\A} \,  \E[f_V(\hat X_n) - \inf_{x \in X}
  f_V(x)],
  \end{align*}
where the expectation is w.r.t. $\P:= \dfrac{1}{2} \left(P_{\A, \gamma_+} \indic{v=+1} + P_{\A, \gamma_-}\indic{v=-1}\right)$, with $P_{\A,\gamma_v}$ denoting the joint distribution of the $n$ observations $g_{1:n}$ of the algorithm $\A$, for $v \in \{-1,+1\}$
%Here $\gamma_v$  is the oracle for $f_v$, for $v=+1,-1$ and is defined by $\gamma_v(x) = \epsilon(x-v) + v C_1 \delta^p$. 
Using \eqref{eq:main:fv-lb-main}, we obtain
\begin{align}
&\Delta_n^*(\F_{L,0}, C_1 \delta^p ,C_2 \delta^{-q})  \ge  \inf_{\A} \dfrac{\epsilon}{2}\,  \P(\hat X_n V < 0), \label{eq:main:strong-convex-bd}\\
  = & \inf_{\A} \dfrac{\epsilon }{2} \, \left(\P_{+}(\hat X_n < 0) + \P_{-}(\hat X_n > 0)\right), \label{eq:main:Pplus}\\
  \ge &\inf_{\A} \dfrac{\epsilon }{2} \,\left(1 - \tvnorm{\P_{+}- \P_{-}}\right), \label{eq:main:lecam}\\
  \ge &\inf_{\A} \dfrac{\epsilon }{2}  \,\left( 1 - \left(\frac12\dkl{P_{+}}{P_{-}}\right)^{\frac{1}{2}}\right), \label{eq:main:pinsker-main}
\end{align}
where $\P_{+}$ (resp. $P_-$) is the distribution $\P$ conditioned on $v=+1$ (resp. $v=-1$).

Observe that, for any $x\in \R$, 
%$f_+'(x) - f_-'(x) = -2\epsilon$ and hence
\begin{align*}
0 \le f_-'(x) - f_+'(x) \le 2\epsilon \dfrac{e^{1/\epsilon}-1}{e^{1/\epsilon}+1} < 2\epsilon .
\end{align*}
Therefore,
\begin{align}
& |\gamma_+(x) - \gamma_-(x)| \nonumber\\
& = | f'_+(x) - \min(\epsilon,C_1 \delta^p) - (f'_-(x)+\min(\epsilon,C_1 \delta^p)) | \nonumber \\
& < 2 (\epsilon - C_1 \delta^p)_+\,,
 \label{eq:main:gdiff-ub}
\end{align}
where $(\epsilon - C_1 \delta^p)_+ = \max(\epsilon - C_1 \delta^p,0)$. 
From the foregoing, it can be shown that 
\begin{align}
\dkl{P_{+}}{P_{-}} \le \dfrac{2n(\epsilon-C_1\delta^p)_+^2 \delta^q}{C_2}.\label{eq:main:dkbd-main}
\end{align}
Note that the above bound holds uniformly over all algorithms $\A$. 
Substituting the above bound into \eqref{eq:main:pinsker-main}, we obtain 
\begin{align*}
 \Delta_n^*(\F_{L,0}, C_1 \delta^p ,\frac{C_2}{\delta^{q}})
  \ge & \frac{\epsilon}{2} \left(1 - \sqrt{
    n}  \frac{(\epsilon-C_1\delta^p)_+\delta^{q/2}}{C_2}
  \right).\label{eq:main:final-lower-bd}
\end{align*}

The generalization to $d$ dimensions works by exploiting separability.
%%%%%%%%%%%%%%%%%%%%%%%%%%%%%%%%%%%%%%%%%%%%%%%%%%%%%%%%%%%%%%%%%%%%%%%%%%%%%%%
%%%%%%%%%%%%%%%%%%%%%%%%%%%%%%%%%%%%%%%%%%%%%%%%%%%%%%%%%%%%%%%%%%%%%%%%%%%%%%%
%%%%%%%%%%%%%%%%%%%%%%%%%%%%%%%%%%%%%%%%%%%%%%%%%%%%%%%%%%%%%%%%%%%%%%%%%%%%%%%
\if0
\paragraph{Generalization to $d$ dimensions:}
Let $\F_{L,0}$ be comprised of $f_v$ defined by: For any $v\in \{-1,+1\}^d$
\begin{align*}
  f_v(x) :=& \sum_{i=1}^d f^{(i)}_{v_i}(x_i), \text{ where } f^{(i)}_{v_i}(x_i) := \dfrac{\epsilon}{2} (x_i - v_i)^2,
\end{align*}
for $i=1,\ldots,d$.
Define
$\gamma_v(x) = \sum_{i=1}^d \gamma_{v_i}^{(i)}e_i$, where
$e_i$ is the $i$th unit vector in the standard Euclidean base,
$\gamma_{v_i}^{(i)} = \epsilon(x_i-v_i) - v_i\, \frac{C_1 \delta^p}{\sqrt{d}} + \xi_i$, where $\xi_i \sim \normal(0,\frac{C_2}{d\delta^q})$.
% It is evident that the $\gamma$ is a $(c_1,c_2)$ Type-I oracle, with $c_1(\delta)=C_1\delta^p$ and $c_2(\delta)=\frac{C_2}{\delta^q}$.

Since $f_v$ is additively separable, we bound the $d$-dimensional minimax error using the $1$-dimensional ones as follows
% Let $\Delta_n^{(d)*}$ denote the minimax error $\Delta_n^*\left(\F_d, C_1\delta^p,\frac{C_2}{\delta^q}\right)$ for the $d$-dimensional family of functions $\F_d$, which is composed of $f_v$s, as defined above. We have
% \begin{align*}
%  \Delta_n^{(d)*} \ge& \inf_\A  \sup_{v\in \{\pm 1 \}^d}\sum_{i=1}^d L_i(v), \text{ where, for } i=1,\ldots,d,\\
%  L_i(v) := & \int f^{(i)}_{v_i}(\hat X_{ni}(g_{1:n}) )\, dP_{\A,\gamma_v,\sigma^2}(g_{1:n}),
% \end{align*}
% where $\hat{X}_{ni}$ denotes the $i$th coordinate of the vector $\hat{X}_n \in \R^d$ chosen by $\A$, for $i=1,\ldots,d$
% (again, to simplify the notation the dependence of $\hat{X}_n$ on $\A$ is suppressed).
%
%
% % Fix an $i$ and consider the loss $L_i(v)$.
% % We now argue that, since $f_v$ is separable, any algorithm $\A$ that uses information in dimensions other than $i$ in the gradient samples of the oracle, can be replaced by $d$ algorithms $\A_1^*$, $\dots$, $\A_d^*$ such that $\A_i^*$ is only using information for dimension $i$ and is only predicting the $i$th coordinate at the end, and the sum of losses they incur is utmost that of $\A$ in the worst-case.
%
% \begin{lemma}
% For any algorithm $\A$, there exist  $d$ ``$1$-dimensional'' algorithms, $\A_i^*$, $1\le i \le d$,
% the $i$th algorithm interacting with only a one dimensional oracle, supplying gradient estimates about $f_{v_i}^{(i)}$
% such that if $L^{\A}(v)$ is the loss of $\A$ on environment with index $v$ and the following holds
% the
% \begin{align}
% \label{eq:main:onedimlb}
% % \begin{split}
% % \MoveEqLeft
% &\hspace{-1em}\max_{v\in \{\pm 1\}^d} L^{\A}(v)
% \ge \nonumber\\
% &  \max_{v_1\in \{\pm 1\}} L_1^{\A_1^*}(v_1) + \dots + \max_{v_d\in \{\pm 1 \}} L_d^{\A^*_d}(v_d)\,.
% % \end{split}
% \end{align}
% \end{lemma}
% Using the above lemma, the minimax error $\Delta_n^{(d)*}$ can be lower bounded as follows:
\begin{align*}
\Delta_n^*(\F_{L,0}, C_1 \delta^p ,C_2 \delta^{-q}) \ge d \Delta_n^{*}\left(\F_1,\frac{C_1\delta^p}{\sqrt d}, \frac{C_2}{d\delta^q}\right)\,.
\end{align*}
See Lemma \ref{lemma:sep} in Appendix \ref{sec:appendix-lb-proof} for a rigorous justification of the above inequality.
The final claim follows by plugging the minimax error bounds in one dimension and the reader is referred to Appendix \ref{sec:appendix-lb-proof} for details.
\fi
% where in inequality \eqref{eq:main:splitA}, algorithms $\A_i$ are interacting with the respective one-dimensional oracles $\gamma^{(i)}_{v_i}$ only
% and the inequality follows by~\eqref{eq:main:onedimlb},
% while the inequality \eqref{eq:main:dlb} follows from the fact that each infimum term in \eqref{eq:main:splitA} is solving a $1$-dimensional problem and hence the bound in \eqref{eq:main:final-lower-bd} applies with $C_1$ replaced by $C_1/\sqrt{d}$ and $C_2$ replaced by $C_2/d$. The resulting bound is denoted by $\Delta_n^{*}\left(\F_1,\frac{C_1\delta^p}{\sqrt d}, \frac{C_2}{d\delta^q}\right)$.
% is the one-dimensional minimax error, which is lower-bounded in \eqref{eq:main:final-lower-bd}. \todoc{However, I would have written just the $1$d final bound there, or $\Delta_n^*( \dots )$ with some notation, because we don't want to redo this whole optimization business..}

% Using the one-dimensional minimax error $\Delta_n^{(1)*}$, we obtain
% \[\Delta_n^* \ge d \Delta_n^{(1)*}.\]

%%% Local Variables:
%%% mode: latex
%%% TeX-master: "bgo"
%%% End:

% \paragraph{Proof of Theorem \ref{thm:lb-strongly-convex}:}
%
% \todoc{This proof sketch can be removed if we are short of space}
% % This follows in the same manner as the proof of Theorem \ref{thm:lb-convex}.
% First, lower bound \\$f_v:=\dfrac{1}{2} x^2 - v x,$ as follows: \\$f_v(x) - f_v(x^*_v)
% \ge  \dfrac{\nu^2}{2}  \indic{x v  < 0}.$\\
% Next, use the following oracles:
% \begin{align}
%  \gamma_v(x) = \epsilon(x-v) - \sign(v)\, \min(v,C_1 \delta^p) + \xi, \label{eq:main:oracle-1d}
% \end{align}
% where $\xi \sim \normal(0,\frac{C_2}{\delta^q})$.
% %%%%%%%%%%%%%%%%%%%%%%%%%%%%%%%%%%%%%%%%%%%%%%%%%%%%%%%%%%%%%%%%%%%%%%%%%%%%%%%
% The KL-divergence bound (corresponding to \eqref{eq:main:dkbd-main}) turns out to be
% \begin{align}
% \dkl{P_{+}}{P_{-}} \le \dfrac{2n(\nu-C_1\delta^p)_+^2 \delta^q}{C_2},\label{eq:main:dkbd-sc-main}
% \end{align}
% which implies the following bound:
% \begin{align}
%  \Delta_n^{(1)*}
%   \ge & \dfrac{\nu^2}{2} \left(1 - \sqrt{
%     n}  \dfrac{(\nu-C_1\delta^p)_+\delta^{q/2}}{C_2}
%   \right)\label{eq:main:final-lower-bd-sc}
% \end{align}
% The generalization to $d$-dimensions follows using a separability argument as before.


%%%%%%%%%%%%%%%%%%%%%%%%%%%%%%%%%%%%%%%%%%%%%%%%%%%%%%%%%%%%%%%%%%%%%%%%%%%%%%%
%%%%%%%%%%%%%%%%%%%%%%%%%%%%%%%%%%%%%%%%%%%%%%%%%%%%%%%%%%%%%%%%%%%%%%%%%%%%%%%

%%%!TEX root =  bgo.tex
% please do not delete or change the first line (needed by Csaba's editor)
% \subsubsection{Reductions and Proofs}
% \label{sec:orrel}

\begin{theorem}\label{thm:typered}
\cref{def:oracle1} is a sufficient condition of \cref{def:oracle2}, given a bounded $\cK$. That is, for any $(c_1,c_2)$ Type-I oracle $\gamma$, there exist some constant $R$ such that  $\gamma$ is a $(c'_1,c_2)$ Type-II oracle, with $c'_1(\delta)=Rc_1(\delta)$.

Similarly, \cref{def:oracle2} is sufficient for \cref{def:oracle1} if the alternative condition \eqref{eq:oracle2alt} holds.
\end{theorem}
\begin{proof}
Given $\norm{ \EE{G}  - \nabla f(x)  }_* \le c_1(\delta) $ in \cref{def:oracle1}, we can always construct a function $\tilde{f}:\cK \to \mathbb{R}$, 
\[
\tilde{f}(y) =\EE{ f(y)+ \left( G-\nabla f(x) \right)^\top y} \,, 
\]
where the expection is over the randomness of $G$, satisfying
\begin{align*}
\MoveEqLeft
\norm{\tilde{f}(y)-f(y)}_{\infty} \\
= &
 \norm{ \EE{\left( G-\nabla f(x) \right)^\top y}}_{\infty}\\
 \le &\norm{\EE{G}-\nabla f(x)}_* \norm{y}
 \le  Rc_1(\delta)\,,
\end{align*} 
by choosing $R = \sup_{y \in \cK}\norm{y}$.

Also notice that
\[
 \nabla \tilde{f}(x) = \EE{\nabla f(x)+ \left( G-\nabla f(x)\right)}
 =\EE{G}\,.
\]
Therefore, $\gamma$ is also a $(c'_1,c_2)$ Type-II oracle.

Next, given \eqref{eq:oracle2alt} and $\EE{G}=\nabla \tilde{f}(x)$ in \cref{def:oracle2}, we can immediately get the bias condition of \cref{def:oracle1}. 
\end{proof}



\section{Conclusions}
\label{sec:conc}
%!TEX root =  bgo-opt-ml-nips.tex
% please do not delete or change the first line (needed by Csaba's editor)

%We presented a general noisy  gradient oracle for model for convex optimization. The oracle model covers all gradient estimation methods in the literature designed for algorithms that can observe only noisy function values, while allowing to handle explicitly the bias-variance tradeoff of these estimators. In the smooth convex setting, the framework allows to derive sharp upper and lower bounds on the minimax optimization error, reducing the optimization problem to study properties of gradient estimators.

%\section{Conclusions}
In this paper, motivated by the quest for improving the rates in smooth stochastic bandit convex optimization,
we proposed a new convex optimization framework where the optimization algorithm interacts with the objective function
through a gradient evaluation oracle with a controllable bias-variance tradeoff. Our main results establish matching lower and upper bounds for the optimization error in this model. The application of this result suggests that already for the smooth-convex case, the algorithms that fit our model (which is essentially \emph{all} algorithms considered in the literature so far) cannot achieve the optimal rate for this problem, unless either our understanding of existing gradient estimation techniques or the techniques themselves improve substantially, which remains an interesting open question.

\subsubsection*{Acknowledgements}
This work was supported by the Alberta Innovates Technology Futures through the Alberta Ingenuity Centre for Machine Learning, NSERC, the National Science Foundation (NSF) under Grants CMMI-1434419, CNS-1446665, and CMMI-1362303, and by the Air Force Office of Scientific Research (AFOSR) under Grant FA9550-15-10050.

%\clearpage\newpage
%\begin{small}
\bibliographystyle{apalike}
\bibliography{main}
%\end{small}

%\end{document}

%\subsubsection*{References}
\clearpage\newpage
\onecolumn
\appendix
%!TEX root =  bgo.tex
% please do not delete or change the first line (needed by Csaba's editor)

\section{Upper Bound Proof}
\label{sec:appendix-md}
%!TEX root =  bgo-cam-ready.tex
% please do not delete or change the first line (needed by Csaba's editor)
%In the proof, we use \cref{lem:ub} to deal with noisy and biased gradient, which is essentially Theorem~C.4 of \cite{MahdaviPhd:2014}, and also identical to Theorem~6.3 of \cite{Bu:Convex14}, who cites \citet{Dekel:minibatch12} as the source. The lemma and its proof can be found below.

In this section we prove \cref{thm:ub}. First we derive the bounds for the optimization settings and then for the regret. 

\subsection{Stochastic optimization}

The proof for the stochastic optimization scenario is based on \cref{lem:ub} stated below.
This is essentially Theorem~C.4 of \citet{MahdaviPhd:2014}, and also identical to Theorem~6.3 of \citet{Bu:Convex14}, who cites \citet{Dekel:minibatch12} as the source. For completeness, the proof of the lemma is given in \cref{sec:lemub-proof}.
The lemma is somewhat more general than what we need (we will only need it for the case when $\beta_t=0$); 
the general form is presented because its proof is not significantly different than the simpler form and it may find
other applications in the future.
\begin{lemma}
\label{lem:ub}
Let $({\cS}_t)_{t}$ be a filtration such that $X_t$ is ${{\cS}}_t$-measurable.
Let $\overline G_t = \EE{G_t|{{\cS}}_t}$
and assume that the nonnegative real-valued deterministic sequence $(\beta_t)_{1\le t\le n}$ is such that
$\norm{\overline G_t - \nabla {f}(X_t)}_* \le \beta_t$ holds almost surely.
Further, assume that $\mathcal{R}$ is $\alpha$-strongly convex with respect to $\norm{\cdot}$, $D=\sup_{x,y\in \cK} D_{\mathcal{R}}(x,y) < \infty$,  and let $\eta_t = \frac{\alpha}{a_t+L}$ for some increasing
sequence $(a_t)_{t=1}^{n-1}$ of numbers. Then, the cumulative loss of \cref{alg} for a fixed convex and $L$-smooth  function $f$ can be bounded as
\begin{align*}
\EE{ \sum_{t=1}^n {f}(X_t) - {f}(x) }
\le 	 \EE{{f}(X_1)-{f}(x)}+
  \sqrt{\tfrac{2D}{\alpha}} \sum_{t=1}^{n-1} \beta_t
 +\frac{D(a_{n-1}+L)}{\alpha} +
	  \sum_{t=1}^{n-1}\frac{\sigma_t^2}{2a_t}\,,
\end{align*}
where $\sigma_t^2 = \EE{ \norm{G_t-\overline G_t}_*^2}$ is the ``variance'' of $G_t$.

If ${{f}}$ is also $\mu$-strongly convex with respect to $\mathcal{R}$ with $\mu > 2L/\alpha$, then letting $\eta_t = \dfrac{2}{\mu t}$ and $a_t = \alpha \mu t/2-L > 0$, the cumulative loss of  \cref{alg} can be bounded as
\begin{align*}
 \EE{ \sum_{t=1}^n {f}(X_t) - {f}(x) }
\le 	 \EE{{f}(X_1)-{f}(x)}+
 \sqrt{\tfrac{2D}{\alpha}} \sum_{t=1}^{n-1} \beta_t
 +\sum_{t=1}^{n-1}\frac{\sigma_t^2}{2a_t}\,.
\end{align*}
\end{lemma}

Now we can easily prove the theorem.
%By the discussion in \cref{sec:problem}, we have $\Delta^*_{\F,n}(c_1,c_2) \le R^*_{\F,n}(c_1,c_2)/n$ by Jensen's inequality, so it is enough to bound the regret for the online case.
For type-I oracle, the result immediately follows by substituting $\beta_t = C_1\delta^p$, $\sigma^2_t = C_2 \delta^{-q}$. Choosing $$\eta_t = \alpha/(a_t+L)$$ as in the lemma with 
$a_t=a t^r$ for some $0<r<1$, we get
\begin{align}
\MoveEqLeft
\frac{1}{n} \EE{ \sum_{t=1}^n f( X_t) - \inf_{x \in \cK} \sum_{t=1}^n f(x)} \nonumber \\
&\le \frac{1}{n}\left(\EE{f(X_1)-\inf_{x \in \cK}f(x)}+\frac{DL}{\alpha}  \right) +\sqrt{\dfrac{2D}{\alpha}} C_1\delta^p
+\frac{Da}{\alpha} n^{r-1}+\dfrac{C_2 \delta^{-q}}{2a(1-r)}n^{-r} \,.
\label{eq:ubToBeOptTypeI}
 \end{align}
  Choosing $r = \frac{p+q}{2p+q}$,  
 $a =2^{\frac{q}{2(2p+q)}} \left(\frac{2p+q}{2p}\right)^{\frac{p}{2p+q}} D^{-\frac{1}{2}} C_1^{\frac{q}{2p+q}} C_2^{\frac{p}{2p+q}} $
 and
 $\delta = \alpha^{\frac{1}{2(p+q)}}2^{-\frac{1}{2p+q}}\left(\frac{2p+q}{2p}\right)^{\frac{1}{2p+q}} C_1^{-\frac{2}{2p+q}} C_2^{\frac{1}{2p+q}}n^{-\frac{1}{2p+q}}$,
the last $3$ terms in \eqref{eq:ubToBeOptTypeI} are optimized to
 \[
 K_1 D^{1/2} C_1^{q/(2p+q)} C_2^{p/(2p+q)} n ^{-p/(2p+q)} \,,
 \]
 where
 $K_1 \le 2^{\frac{q}{2(2p+q)}} \left( \alpha^{-1}+2\alpha^{-\frac{q}{2(p+q)}} \right) \left( \frac{2p+q}{2p} \right)^{\frac{p}{2p+q}}$. This implies \eqref{eq:MDbound1TypeI}.

For type-II oracle, from the bias condition in \cref{def:oracle2} and using that the oracle is memoryless and uniform, we get
\begin{align*}
 \frac{1}{n} \EE{ \sum_{t=1}^n f(X_t) - \inf_{x \in \cK}\sum_{t=1}^n f(x) }
 \le \frac{1}{n}\EE{ \sum_{t=1}^n \tilde{f}(X_t) - \inf_{x \in \cK}\sum_{t=1}^n \tilde{f}(x) } +2C_1 \delta^p
 \,.
\end{align*}

Given $\overline{G}_t=\EE{G_t} = \nabla \tilde{f}(X_t)$, where $\tilde{f}\in \cF_{L,0}$ is convex and smooth,
the result immediately follows by applying \cref{lem:ub} to $\tilde{f}$.
Substituting
 $\beta_t = 0$ (since we have a type-II oracle), $\sigma^2_t = C_2 \delta^{-q}$, respectively, we obtain
 \begin{align}
\MoveEqLeft
\frac{1}{n} \EE{ \sum_{t=1}^n f( X_t) - \inf_{x \in \cK} \sum_{t=1}^n f(x)} \nonumber \\
&\le \frac{1}{n}\left(\EE{f(X_1)-\inf_{x \in \cK}f(x)}+\frac{DL}{\alpha}  \right) %\nonumber
+\frac{Da}{\alpha} n^{r-1}+\dfrac{C_2 \delta^{-q}}{2a(1-r)}n^{-r}+ \left(2+\dfrac{2}{n}\right)C_1\delta^p \,.
\label{eq:ubToBeOpt}
 \end{align}
 Choosing $r = \frac{p+q}{2p+q}$,  
 $\delta = \left( \tfrac{C_2}{4aC_1}\tfrac{2p+q}{p}\tfrac{n}{n+1}\right)^{\frac{1}{p+q}}n^{-\frac{1}{2p+q}}$  and
 $a^{2p+q} =2^{q-p}\left( 2+\tfrac{q}{p} \right)^p\left(1+ \tfrac{1}{n} \right)^q \left( \tfrac{\alpha}{D} \right)^{p+q}C_1^q C_2^p $,
the last $3$ terms in \eqref{eq:ubToBeOpt} are optimized to
 \[
 K'_1 D^{p/(2p+q)} C_1^{q/(2p+q)} C_2^{p/(2p+q)} n ^{-p/(2p+q)} \,,
 \]
 where
% $K_1 = 2^{\dfrac{3q}{2(p+q)}} \left( \sqrt{\dfrac{D}{\alpha}} \right)^{\dfrac{p+q^2+2pq}{(p+q)(2p+q)}}+2^{\dfrac{3q}{2(2p+q)}} \left( \sqrt{\dfrac{D}{\alpha}} \right)^{2-\dfrac{1}{2p+q}}$.
 $(K'_1/3)^{2p+q} \le 2^{2q-p}\left(2+q/p \right)^p \left(1/\alpha\right)^{p}$. This implies \eqref{eq:MDbound1}.
% \begin{align*}
 
When $\tilde{f} \in \cF_{L,\mu, \cR}$ is $L$-smooth and $\mu$-strongly convex, for
$\eta_t = 2/(\mu t)$ and 
%\[
%\delta^{p+q} =  \sqrt{\dfrac{\alpha}{2D}}\dfrac{C_2 \log n}{\alpha \mu C_1 n} \,,
%\]
$
\delta^{p+q} =  \tfrac{C_2\left( \log n+1+\tfrac{\alpha \mu}{\alpha \mu -2L}\right)}{2\alpha \mu C_1 (n+1)} \,,
$
we similarly obtain 
 \begin{align*}
 \MoveEqLeft
\frac{1}{n} \EE{ \sum_{t=1}^n f( X_t) - \inf_{x \in \cK} \sum_{t=1}^n f(x)} -\frac{1}{n}\EE{f(X_1)-\inf_{x \in \cK}f(x)}\\
&\le (2+\dfrac{2}{n})C_1\delta^p+\dfrac{C_2 \delta^{-q}}{\alpha \mu n} \sum_{t=1}^{n-1}\dfrac{1}{t-\dfrac{2L}{\alpha \mu}}\\
&\le (2+\dfrac{2}{n})C_1\delta^p+\dfrac{C_2 }{\alpha \mu}\delta^{-q} \dfrac{\log n+1+\alpha \mu/(\alpha \mu-2L)}{n}\\
&\le K_2C_1^{\frac{q}{p+q}}C_2^{\frac{p}{p+q}} \left( \frac{\log n+1+\dfrac{\alpha \mu}{\alpha \mu -2L}}{n} \right)^{\frac{p}{p+q}}\,.
 \end{align*}
The last step optimizes the bound via the choice of $\delta$, and
%$\delta^{p+q} =  \sqrt{\dfrac{\alpha}{2D}}\dfrac{C_2 \log n}{\alpha \mu C_1 n}$,
%$\delta^{p+q} =  \dfrac{C_2\left( \log n+1+\dfrac{\alpha \mu}{\alpha \mu -2L}\right)}{2\alpha \mu C_1 (n+1)}$,
%where
%$K_2^{p+q}=\sqrt{2}^{2p+3q}D^{q/2}\alpha^{-p-q/2}\mu^{-p}$.
$K_2^{p+q}=2^{q}\alpha^{-p}\mu^{-p}$.



\subsection{Online optimization}
The proof in this section follows closely the derivation of \citet{saha2011improved}.
Assume that the gradients of all $\tf_t \in \cF$ are bounded by $M$, that is $\norm{\nabla \tf_t(x)}_* \le M$ for all $x \in \cK$ (here, the subindex of $\tf_t$ signifies that the oracle is non-uniform).

Let $\cS_t$ denote the $\sigma$-algebra of all random events up until and including the selection of $X_t$. Since the oracle is unbiased, that is,  $\EE{Y_t|\cS_t}=X_t$, we have
\begin{equation}
\EE{\tf_t(Y_t)-\tf_t(X_t)|\cS_t} \le \EE{\ip{\nabla \tf_t(X_t),Y_t-X_t}+\tfrac{L}{2}\|X_t-Y_t\|^2|\cS_t} \le L\delta^2/2~.
\label{eq:YX}
\end{equation}
Instead of Lemma~\ref{lem:mdlinregret} used in the optimization proof, we will employ the prox-lemma \citep[see, e.g.,][]{Beck2003mirror, NeJuLaSh09}:
\begin{equation}
\label{eq:proxlemma}
\ip{G_t,X_t-x} \le \frac{1}{\eta_t}(D_\cR(x,X_t)-D_\cR(x,X_{t+1})) + \eta_t \frac{\|G_t\|_*^2}{2\alpha}~.
\end{equation}
Using the linearization $\tf(X_t) - \tf(x) \le \ip{\nabla \tf_t(X_t),X_t-x} = \EE{\ip{ G_t,X_t-x}| \cS_t}$, summing up the bounds for all $t$, and applying \eqref{eq:div-telescope}, we get the standard mirror descent bound for any $x \in \cK$
\begin{align}
\EE{\sum_{t=1}^n \tf_t(X_t)- \sum_{t=1}^n \tf_t(x)} 
& \le \EE{\sum_{t=1}^n \ip{G_t,X_t-x}} 
\le \frac{D}{\eta_{n-1}} + \sum_{t=1}^n \eta_t \frac{\EE{\|G_t\|_*^2}}{2\alpha} 
\label{eq:md-bound}
\end{align}
Combining the latter with the bound
\[
\EE{\norm{G_t}_*^2 | \cS_t} \le 2\EE{\norm{G_t - \nabla \tf_t(X_t)}_*^2 + \norm{\nabla \tf_t(X_t)}_*^2 \Big| \cS_t}
\le 2(M^2 + C_2 \delta^{-q}),
\]
the bias condition of the oracle, and \eqref{eq:YX}, we obtain
\begin{align*}
\EE{\sum_{t=1}^n f_t(Y_t)} - \inf_{x \in \cK} \sum_{t=1}^n f_t(x)
& \le \EE{\sum_{t=1}^n \tf_t(Y_t) }- \inf_{x \in \cK} \sum_{t=1}^n \tf_t(x) + 2 n C_1 \delta^{p} \\
& \le \EE{\sum_{t=1}^n \tf_t(X_t)} - \inf_{x \in \cK} \sum_{t=1}^n \tf_t(x) + 2 n C_1 \delta^{p} + \frac{n L \delta^2}{2} \\
& \le \frac{D}{\eta_{n-1}}  + \sum_{t=1}^n \eta_t \frac{M^2 + C_2 \delta^{-q}}{\alpha} +  2 n C_1 \delta^{p} + \frac{n L \delta^2}{2}~.
\end{align*}
Setting the parameters 
$\delta=(\frac{q}{2p})^{\frac{2}{2p+q}} (\frac{C_2 D}{\alpha B^2})^{\frac{1}{2p+q}} n^{-\frac{1}{2p+q}}$, 
where $B=C_1$ if $p<2$ and $B=C_1+L/4$ if $p=2$, and 
$\eta_t=D^{\frac{p+q}{2p+q}} (\frac{q}{2p})^{\frac{q}{2p+q}} (\frac{C_2}{\alpha})^{-\frac{p}{2p+q}} B^{-\frac{q}{2p+q}} n^{-\frac{p+q}{2p+q}}$ gives the desired bound for $\F_{L,0}$ with bounded gradients. When $p>2$, $p$ should be replaced with $2$ in the bound and $B=L/4$ (in this case the $\delta^2$ term dominates the one with $\delta^p$).
%For $q=0$, $C_2$ has to be replaced with $C_2+M^2$.


When the set of functions is also strongly convex, instead of the linearization we use strong convexity to get
\[
\tf(X_t) - \tf(x) \le \ip{\nabla \tf_t,X_t-x} - \frac{\mu}{2} D_\cR(x,X_t) = \EE{\ip{ G_t,X_t-x}| \cS_t} - \frac{\mu}{2} D_\cR(x,X_t) ~.
\]
Combining this with \eqref{eq:proxlemma} gives the well-known variant of \eqref{eq:md-bound} for strongly convex loss functions \citep{BaHaRa07} for the choice $\eta_t=2/(t\mu)$:
\begin{align*}
\EE{\sum_{t=1}^n \tf_t(X_t)- \sum_{t=1}^n \tf_t(x)} 
& \le \sum_{t=1}^n  \frac{\EE{\|G_t\|_*^2}}{ t \alpha \mu} \le \frac{\max_t \EE{\|G_t\|_*^2}}{\alpha \mu} (1+\log n).
\end{align*}
Similarly to the non-strongly convex case, this implies
\begin{align}
\EE{\sum_{t=1}^n f_t(Y_t)} - \inf_{x \in \cK} \sum_{t=1}^n f_t(x) \le 
\frac{M^2 + C_2 \delta^{-q}}{2 \alpha \mu} (1+\log n)+  2 n C_1 \delta^{p} + \frac{n L \delta^2}{2}~.
\end{align}
Setting $\delta=(\frac{C_2 q (1+\log n)}{4 \alpha \mu C_1 p n})^{\frac{1}{p+q}}$, we obtain the desired bound for $p<2$. For $p \ge 2$, $C_1$ should be replaced with $C_1+L/4$, while for $p>2$,  $C_1$ has to be replaced with $L/4$ and $p$ with $2$ (in this case the $\delta^2$ term dominates the one with $\delta^p$).





\section{Reduction of Oracles}
%!TEX root =  bgo.tex
% please do not delete or change the first line (needed by Csaba's editor)
% \subsubsection{Reductions and Proofs}
% \label{sec:orrel}

\begin{theorem}\label{thm:typered}
\cref{def:oracle1} is a sufficient condition of \cref{def:oracle2}, given a bounded $\cK$. That is, for any $(c_1,c_2)$ Type-I oracle $\gamma$, there exist some constant $R$ such that  $\gamma$ is a $(c'_1,c_2)$ Type-II oracle, with $c'_1(\delta)=Rc_1(\delta)$.

Similarly, \cref{def:oracle2} is sufficient for \cref{def:oracle1} if the alternative condition \eqref{eq:oracle2alt} holds.
\end{theorem}
\begin{proof}
Given $\norm{ \EE{G}  - \nabla f(x)  }_* \le c_1(\delta) $ in \cref{def:oracle1}, we can always construct a function $\tilde{f}:\cK \to \mathbb{R}$, 
\[
\tilde{f}(y) =\EE{ f(y)+ \left( G-\nabla f(x) \right)^\top y} \,, 
\]
where the expection is over the randomness of $G$, satisfying
\begin{align*}
\MoveEqLeft
\norm{\tilde{f}(y)-f(y)}_{\infty} \\
= &
 \norm{ \EE{\left( G-\nabla f(x) \right)^\top y}}_{\infty}\\
 \le &\norm{\EE{G}-\nabla f(x)}_* \norm{y}
 \le  Rc_1(\delta)\,,
\end{align*} 
by choosing $R = \sup_{y \in \cK}\norm{y}$.

Also notice that
\[
 \nabla \tilde{f}(x) = \EE{\nabla f(x)+ \left( G-\nabla f(x)\right)}
 =\EE{G}\,.
\]
Therefore, $\gamma$ is also a $(c'_1,c_2)$ Type-II oracle.

Next, given \eqref{eq:oracle2alt} and $\EE{G}=\nabla \tilde{f}(x)$ in \cref{def:oracle2}, we can immediately get the bias condition of \cref{def:oracle1}. 
\end{proof}



\section{Gradient Estimation}
\label{sec:appendix-grad}
%!TEX root =  bgo-cam-ready.tex
% please do not delete or change the first line (needed by Csaba's editor)
In this section we present the proofs corresponding to the oracles introduced in \cref{sec:sbco}.

\subsection{Proof of Proposition \ref{prop:grad-onepoint}}



\if0
\begin{proposition}
\label{prop:grad-1spsa}
Given any $f$ that is three times continuously differentiable with bounded third derivative.
For any $x \in \cK$, and $\delta >0$, let oracle $\gamma$ return
\begin{align}
% Y = x+\delta U \,, \quad
G =  V \left(\dfrac{f(x+\delta U) + \xi}{\delta}\right),
 \label{eq:onesp}
\end{align}
where $V, U$ are random variables that satisfy $\E[V U\tr] = I_d$, $U_i, i=1,\ldots,d$ are i.i.d., $\E[ V U^2] = 0$, $\E[V_i]=0$, $|U_i|$ and $\E|V_i|$ have finite upper bounds.
Then, we have that $\gamma$ is a type-I oracle with $c_1(\delta) = C_1 \delta^2$ and $c_2(\delta) = C_2/\delta^2$.
\end{proposition}
\fi
%\begin{proof}\ \\
% \todoc[inline]{This proof must be updated to follow the proof of Prop 1. Don't write $\le O(\cdot)$,for example.}
\textbf{Case 1 ($f \in \C^3$): }\ \\
We use the proof technique of \cite{spall1997one}.
We start by bounding the bias.
Since by assumption $\EE{ \xi|V}=0$, we have
\begin{align*}
\E\left[  V\left(\dfrac{\xi}{\delta}\right) \right]= 0\,,
\end{align*}
implying that
\begin{align*}
\E[G] =  \E\left[ V \left(\dfrac{f(x+\delta U) }{\delta}\right)\right] \,.
\end{align*}
By Taylor's theorem, we obtain, a.s.,
\begin{align*}
f(x + \delta U) =
 f(x)
 +\delta\,  U\tr\,\nabla f(x)
  + \frac{\delta^2}{2}\, U\tr \nabla^2 f(x) U
  +  \frac{\delta^3}{2} \, R^{+}(x,\delta,U) \,(U, U, U),
\end{align*}
where
\begin{align}
 R^{+}(x,\delta,U)= \int_0^1  \nabla^3 f(  x + s \, \delta U ) (1-s)^2 ds. \label{eq:taylor-r}
\end{align}
In the above, $\nabla^3 f(\cdot)$ is considered as a rank-3 tensor.
Letting $B_3 = \sup_{x\in D} \norm{ \nabla^3 f(x) }$,%
\footnote{Here, $\norm{\cdot}$ is the implied norm: For a rank-3 tensor $T$, $\norm{T} = \sup_{x,y,z\ne 0}
\frac{|T (x,y,z)|}{\norm{x}\norm{y}\norm{z}}$.
}
we have $\norm{ R^{+}(x,\delta,U)} \le B_3/3$ a.s.
Now,
\begin{align*}
\EE{V\, \dfrac{f(x+\delta U)}{\delta}}
&= \EE{V \frac{f(x)}{\delta}} +  \EE{VU^{\tr}
\, \nabla f(x)}  + \EE{\frac{\delta}{2}\, V U\tr \nabla^2 f(x) U} \\
&\qquad+   \EE{\frac{\delta^2}{2}  V \,R^{+}(x,\delta,U)(U \otimes U \otimes U)}
\\
&= \, \nabla f(x)  + \EE{\frac{\delta^2}{2}  V \,R^{+}(x,\delta,U)(U \otimes U \otimes U)}\,.
\end{align*}
The final equality above follows from the facts that $\EE{V} = 0$, $\EE{V U\tr} = I$ and for any $i,j=1,\ldots,d$, $E[V_i U_j^2] = 0$ since $V$ is a deterministic odd function of $U$, with $U$ having a symmetric distribution.
Using the fact that $|R^{+}(x,\delta,U) (U \otimes U \otimes U)| \le
\norm{R^{+}(x,\delta,U)} \norm{U}^3$,
we obtain
\begin{align*}
\norm{ \EE{ G } - \nabla f(x) }_*
\le C_1\,\, \delta^2 \,,
\end{align*}
where $C_1 = \frac{B_3 \EE{ \norm{V}_* \norm{U}^3 }}{6}$.

Let us now bound the variance of $G$:
Using the identity $\E\left\|X -  E[X]\right\|^2 \le 4 \E \left\|X\right\|^2$, which holds for any random variable $X$,%
\footnote{When $\norm{\cdot}$ is defined from an inner product,
$\E\left\|X -  E[X]\right\|^2 = \EE{\norm{X}^2} - \norm{\EE{X}}^2 \le \EE{\norm{X}^2}$ also holds, shaving off a factor of four from the inequality below.}
we bound $\E\left\| G - \E G\right\|_*^2$ as follows:
\begin{align}
\E\left\| G - \E G\right\|_*^2
 &\le 4 \E \left\|G\right\|_*^2 \nonumber \\
& =  4\E\left( \left\| V \right\|_*^2 \left(\left(\dfrac{\xi}{\delta}\right)^2  + 2 \left(\dfrac{\xi}{\delta}\right) \left(\dfrac{f(x+\delta U)}{\delta}\right)
+ \left( \dfrac{f(x+\delta U) }{\delta} \right)^2 \right)\right) \nonumber \\
&=  4\E\left( \left\| V \right\|_*^2 \left(\dfrac{\xi}{\delta}\right)^2\right)
+ 4 \E \left(\left\| V \right\|_*^2 \right)\left( \dfrac{f(x+\delta U) }{\delta} \right)^2  \label{eq:h31} \\
& \le  \frac{C_2}{\delta^2}\,, \nonumber \label{eq:h41}
\end{align}
where $C_2 = 4 \EE{\norm{V}_*^2}\left( \sigma_\xi^2+B_0^2\right)$, where
$\sigma_\xi^2 = \essup \EE{\xi^2|V}$ and $B_0 = \sup_{x\in \D} f(x)$.
The equality in \eqref{eq:h31} follows from $\EE{ \xi \,|\, V } = 0$.

Therefore, for $f \in \C^3$, $\gamma$ defined by \eqref{eq:one-point} is a $(C_1\delta^2, C_2/\delta^2)$ type-I oracle.

\paragraph{Case 2 ($f$ is convex and $L$-smooth):}\ \\
Since $f$ is convex and $L$-smooth, for any $0<\delta <1$,
\begin{align*}
 0 \le \frac{f(x + \delta u)-f(x)}{\delta}-\<\nabla f(x), u\> \le&   \frac{L \delta \norm{ u}^2}{2}.
\end{align*}
Denoting
$\phi(x,\delta,u):=\frac{f(x + \delta u)-f(x)}{\delta}-\<\nabla f(x), u\>$, we have
$\left|\phi(x,\delta,u) \right| \le  \dfrac{L\delta}{2} \norm{u}^2$.
Then,  given $\EE{VU^\top}=I$, $\EE{V}=0$, we obtain
\begin{align}
\norm{ \EE{ G } - \nabla f(x) }_*
&= \norm{ \EE{\frac{f(x + \delta U)}{\delta}V}-\EE{VU^\top \nabla f(x)}  }_*\nonumber\\
&=\norm{ \EE{V\left( \frac{f(x + \delta U)}{\delta}- -U^\top \nabla f(x)\right)}  }_*\nonumber\\
&= \norm{ \EE{V\left( \phi(x,\delta,U)+\dfrac{f(x)}{\delta} \right)}  }_*\nonumber\\
&=\norm{ \EE{V\phi(x,\delta,U)} }_*\nonumber\\
&\le C_1 \,\, \delta\,, \label{eq:c1onepoint}
\end{align}
  where $C_1 = \dfrac{L}{2}\EE{\norm{V}_*\norm{U}^2}$.
The claim regarding the variance of $G$ follows in a similar manner as in Case 1, i.e., $f \in \C^3$.

Therefore, for $f$ convex and $L$-smooth, $\gamma$ defined by \eqref{eq:one-point} is a $(C_1\delta, C_2/\delta^2)$ type-I oracle, where $C_1$ is given by \eqref{eq:c1onepoint} and $C_2$ as defined in Case 1.
%\end{proof}

% By Taylor's series expansions, we obtain, a.s.,
% \begin{align*}
% f(x + \delta U) = f(x) \pm \delta U\tr \nabla f(x) + \frac{\delta^2}{2} U\tr \nabla^2 f(x) U +  \frac{\delta^3}{6} \nabla^3 f(\tilde x^{+})(U \otimes U \otimes U),
% \end{align*}
% where $\otimes$ and $\tilde x^+$ are as in the proof of Proposition \ref{prop:grad-spsa}.
% Using suitable Taylor's series expansions, we have %the following for any $i=1,\ldots,d$:
% \begin{align}
% &\E\left[V\left(\dfrac{f(x+\delta U)}{\delta}\right) \right] \nonumber\\
% = & \E\left[V \dfrac{f(x)}{\delta} \right] + \E\left[V U\tr \left.\nabla f(x)\right| \F_n\right]  +   \E\left[\frac{\delta}{2}  \nabla^2 f(\tilde  x^+)(V \otimes U \otimes U)\right] + \O\left( \delta^2\right). \label{eq:l1}\\
% \le & \nabla f(x) + \O\left( \delta^2\right).\label{eq:l2}
% \end{align}
% The last inequality follows from the facts that $E[V U\tr] = I$, and for any $i=1,\ldots,d$, $E[V_i U_j^2] = 0$, $|U_i|$ and $\E|V_i|$ have finite upper bounds.
% \paragraph{Case 2: $f$ is convex with a $L$-Lipschitz gradient:}\ \\
% Using the tangent bound for convex functions and $L$-Lipschitz property of the gradient of $f$, we obtain the following for any $\delta>0$:
% \begin{align*}
% \frac{f(x)}{\delta} + \frac{\<\nabla f(x), \delta u\>}{\delta} \le \frac{f(x + \delta u)}{\delta} \le& \frac{f(x) + \<\nabla f(x), \delta u\> + (L / 2) \norm{\delta u}^2}{\delta}.
% \end{align*}
% Letting
% $\phi(x,\delta,u):=\frac1{\delta}\left(\frac{f(x + \delta u)}{\delta} - \<\nabla f(x),  u\>\right)$, we have
% \begin{align*}
% \left|\phi(x,\delta,u) \right| \le&  \frac{B_4}{\delta^2} + \dfrac{L}{2} \norm{u}^2\,,
% % \frac{f(x - \delta u) -  f(x)}{\delta} \le& -\frac{\<\nabla f(x), \delta u\> + (L / 2) \norm{\delta u}^2}{\delta}
% \end{align*}
% where $B_4 = \sup_{x\in\D} \norm{f(x)}$.
% From the foregoing,
% \begin{align*}
%  \norm{\E[G] - \nabla f(x)}
%  = \scnorm{\E\left[V\,  \left(\frac{f(x+\delta U)  -f(x-\delta U)}{2\delta}\right)-V U^\top\nabla f(x) \right]}
%  \le \delta \norm{\E[ V \phi(x,\delta, U)]}
%  \le \frac{\delta L}{2} \E[ \dnorm{V} \norm{U}^2],
% \end{align*}
% and the claim for the bias follows by setting $c_1(\delta)= \frac{\delta L}{2} \E[ \dnorm{V} \norm{U}^2]$.

%%%%%%%%%%%%%%%%%%%%%%%%%%%%%%%%%%%%%%%%%%%%%%%%%%%%%%%%%%%%%%%%%%%%%%%%%%%%%%%%%%%%%%%%%%%%%%%%%%%%%%%%%%%%%%%%%%%%%%%%%%%%%%%%%%%%%%%%%%%%%%%%%%%%%%%%%%
%%%%%%%%%%%%%%%%%%%%%%%%%%%%%%%%%%%%%%%%%%%%%%%%%%%%%%%%%%%%%%%%%%%%%%%%%%%%%%%%%%%%%%%%%%%%%%%%%%%%%%%%%%%%%%%%%%%%%%%%%%%%%%%%%%%%%%%%%%%%%%%%%%%%%%%%%%
%%%%%%%%%%%%%%%%%%%%%%%%%%%%%%%%%%%%%%%%%%%%%%%%%%%%%%%%%%%%%%%%%%%%%%%%%%%%%%%%%%%%%%%%%%%%%%%%%%%%%%%%%%%%%%%%%%%%%%%%%%%%%%%%%%%%%%%%%%%%%%%%%%%%%%%%%%

\subsection{Proof of Proposition \ref{prop:flaxman}}
Before the proof, we introduce a fundamental theorem of vector calculus, which is commonly known as the Gauss-Ostrogradsky theorem or the divergence theorem . A special case of the theorem for real-valued functions in $\R^n$ can be stated as follows.
\begin{lemma}
\label{lem:gradientCalculus}
Suppose $W \subset \R^n$ is an open set with the boundary $\partial W$. At each point of $\partial W$ there is a normal vector $n_W$ such that $n_W$  (i) has unit norm, (ii) is orthogonal to $\partial W$, (iii) points outward from $W$. Suppose $f: \R^n\to \R$ is a function of class $C^1$ defined at least on the closure of $W$, then we have
\begin{align*}
\int_{ W} \nabla f\,d W = \int_{\partial W} f n_W \,d \partial W \,.
\end{align*}
\end{lemma}

\begin{proof}
Given that $\EE{\norm{V}_*^2}$ and $\EE{\xi^2}$ are bounded, the variance of $G$ remains the same as stated in \cref{prop:grad-onepoint}.

As to the bias, let $\tilde{f}$ be a smoothed version of $f$, i.e., $\forall x \in \cK$,
\begin{align*}
\tilde{f}(x) &= \EE{f(x+\delta V)} =\int_{v \in W} f(x+\delta v)\dfrac{\,d w}{\lvert W\rvert}\,,
\end{align*}
where the expectation is w.r.t. $V$, which is a random variable uniformly chosen from $W$. The second equality interprets the expectation as integral.
Now we want to prove that for any given $x\in \cK$, $G$ is an unbiased gradient estimate of $\tilde{f}$ at $x$.
Since $U$ is uniformly distributed over $\partial W$,  the expectation of $G$ can be written as
\begin{align*}
\EE{G} =\dfrac{\lvert \partial W\rvert}{\lvert W \rvert} \int_{\partial W} \dfrac{1}{\delta} f(x+\delta U)n_W(U)\dfrac{\,d U}{\lvert \partial W\rvert}
=  \int_W \nabla f(x+\delta U)\dfrac{\,d U}{\lvert W\rvert}\,,
\end{align*}
where the second equality follows from \cref{lem:gradientCalculus}, by replacing the gradient of $\hat{f}(u) =\dfrac{1}{\delta} f(x+\delta u) $ with $\nabla f(x+\delta u)$.
Then, the order of the gradient and the integral can be exchanged, because $\int_W f(x+\delta U)\,d U$ exists. Consequently, we obtain $\EE{G} = \nabla \tilde{f}(x)$.

Moreover, $\tilde{f}$ and $f$ are actually close. In particular, for any $x \in \cK$,
\begin{align}
\label{eq:f2tildef}
\tilde{f}(x)-f(x) =\int_{ W} f(x+\delta w)-f(x)\dfrac{\,d w}{\lvert W\rvert} \,.
\end{align}

When $f$ is $L_0$-Lipschitz, $\vert f(x+\delta w)-f(x)\vert \le L_0\delta \norm{w}$, which combined with \eqref{eq:f2tildef} gives that $\gamma$ is a type-II oracle with $c_1(\delta) = C_1 \delta$, where $C_1 = L_0 \sup_{w\in W}\norm{w}$.

When $f$ is convex and $L$-smooth,  $0\le f(x+\delta w)-f(x) - \ip{\nabla f(x), \delta w}\le\dfrac{L}{2}\delta^2 \norm{w}^2$. Given that $W$ is symmetric, $\int_{ W} \ip{\nabla f(x), \delta w} \,d w=0$. Hence, one can easily get that $\gamma$ is a type-II oracle with $c_1(\delta)=C'_1 \delta^2$, where $C'_1 = \dfrac{L}{2 |W|}\int_{ W}\norm{w}^2\,d w$.

Finally, if $f$ is $L$-smooth,
\begin{align*}
\norm{\nabla \tilde{f}(x)- \nabla f(x)}_*
&\le \int_W \norm{\nabla f(x+\delta w) - \nabla f(x)}_* \frac{dw}{|W|}
\le L\delta^2\int_W \norm{w}^2 \frac{dw}{|W|} =2 C'_1 \delta^2
\end{align*}
with the same value of $C'_1$ as before. So $\gamma$ is also a  type-I oracle with $c_1(\delta)=2C'_1 \delta^2$.
\end{proof}
%\begin{remark}
%Note that $f$ do not need to be differentiable here.
%$U$ and $V$ can be selected under other distributions, like Gaussian.
%For more examples about smoothing technique, see \cite{flaxman2005online}, \cite{nesterov2004introductory}, \cite{PoTsy90}, \cite{HaLe14:SOC}.
%\end{remark}


%
% \paragraph{One-point SPSA:}
%
%  Since $f$ is $3$-times continuously differentiable, using Taylor's expansion, we get for the the $i^{th}$ component of $G$,
% \begin{align*}
% &\EE{G_{\cdot i}}\\
% =&\EE{\dfrac{1}{\delta \Delta_{\cdot i}} \left( f(x+\delta \Delta)+\epsilon \right) }\\
% =& \EE{\dfrac{1}{\delta \Delta_{\cdot i}} \left( f(x)+\delta f'(x)^\top \Delta+\dfrac{1}{2}\delta^2 \Delta^\top f''(x)\Delta \right) } \numberthis \label{eq:spsaTaylorExp} \\
% &+\EE{\dfrac{1}{\delta \Delta_{\cdot i}} \left(O(\delta^3 \Delta\otimes\Delta\otimes\Delta) +\epsilon \right) }\\
% =& [f'(x)]_i +O(\delta^2) \,,
% \end{align*}
% where $[f'(x)]_i$ denotes the $i^{th}$ component of $f'(x)$. The last equality comes from the properties of symmetry, and bounded moment for $\Delta$.
% Hence, $G$ is a estimate of $f'(x)$ with bias $O(\delta^2)$.
%
% \paragraph{Two-point SPSA:}
%
% Under this situation, using Taylor expansion again, the $f(x)$ and $f''(x)$ terms in \eqref{eq:spsaTaylorExp} can be canceled and one can conclude that $G$ is only an order $O(\delta^2)$ term away from $f'(x)$. Note that the second-order term in one-point SPSA is zero-mean, while in the two-point SPSA it is zero.
% As a result, we only need $\Delta_{\cdot i}$ to be zero-mean instead of symmetry.

\subsection{Proof of Proposition \ref{prop:grad-spsa}}
%\begin{proof}\ \\
\textbf{Case 1 ($f \in \C^3$):}\ \\
We use the proof technique of \cite{spall1992multivariate}
  (in particular, Lemma 1 there).
We start by bounding the bias.
Since by assumption $\EE{ \xi^+-\xi^-|V}=0$, we have
\begin{align*}
\E\left[  V\left(\dfrac{\xi_n^+ - \xi_n^-}{2\delta}\right) \right]= 0\,,
\end{align*}
implying that
\begin{align*}
\E[G] =  \E\left[V\,  \dfrac{f(X^+)  -f(X^-)}{2\delta} \right]\,.
\end{align*}

By Taylor's theorem, using that $f\in C^3$, we obtain, a.s.,
% \todoc{I am using the integral form, because otherwise you would need to argue that the point picked on the line segment between $x$ and $x\pm \delta U$ is measurable.
% For the notation etc see \url{http://www.gold-saucer.org/math/taylor/taylor.pdf}}
\begin{align*}
f(x \pm \delta U) =
 f(x)
 \pm\delta\,  U\tr\,\nabla f(x)
  + \frac{\delta^2}{2}\, U\tr \nabla^2 f(x) U
  \pm  \frac{\delta^3}{2} \, R^{\pm}(x,\delta,U) \,(U, U, U),
\end{align*}
where, as in the proof of Proposition \ref{prop:grad-onepoint}, $R^{\pm}(x,\delta,U)$ is defined as follows:
\begin{align}
 R^{\pm}(x,\delta,U)= \int_0^1  \nabla^3 f(  x \pm s \, \delta U ) (1-s)^2 ds. \label{eq:taylor-r-1p}
\end{align}
%In the above, $\nabla^3 f(\cdot)$ is considered as a rank-3 tensor.
Letting $B_3 = \sup_{x\in D} \norm{ \nabla^3 f(x) }$,%
%\footnote{Here, $\norm{\cdot}$ is the implied norm: For a rank-3 tensor $T$, $\norm{T} = \sup_{x,y,z\ne 0}
%\frac{|T (x,y,z)|}{\norm{x}\norm{y}\norm{z}}$.
%}
we have $\norm{ R^{\pm}(x,\delta,U)} \le B_3/3$ a.s.
Now,
\begin{align}
\begin{split}
\MoveEqLeft       V\, \dfrac{f(X^+)-f(X^-)}{2\delta}
  = V\, \dfrac{f(x+\delta U) - f(x-\delta U)}{2\delta} \\
&= VU^{\tr}
\, \nabla f(x)   +   \frac{\delta^2}{4}  V \,(R^{+}(x,\delta,U)+R^{-}(x,\delta,U))(U \otimes U \otimes U)\,.
\end{split}
\label{eq:l1}
\end{align}
and therefore,
by taking expectations of both sides,
using $\EE{V U\tr} = I$ and then $|R^{\pm}(x,\delta,U) (U \otimes U \otimes U)| \le
\norm{R^{\pm}(x,\delta,U)} \norm{U}^3$,
we get that
\begin{align*}
\norm{ \EE{ G } - \nabla f(x) }_*
\le C_1\,\, \delta^2 \,,
\end{align*}
where $C_1 = \frac{B_3 \EE{ \norm{V}_* \norm{U}^3 }}{6}$.

Using arguments similar to that in the proof of Proposition \ref{prop:grad-onepoint}, the variance of $G$ is bounded as follows:
\begin{align}
\MoveEqLeft \E\left\| G - \E G\right\|_*^2
 \le 4 \E \left\|G\right\|_*^2 \nonumber \\
& =  4\E\left( \left\| V \right\|_*^2 \left(\left(\dfrac{\xi^+ - \xi^-}{2\delta}\right)^2  + 2 \left(\dfrac{\xi^+ - \xi^-}{2\delta}\right) \left(\dfrac{f(X^+) - f(X^-)}{2\delta}\right)
+ \left( \dfrac{f(X^+) - f(X^-)}{2\delta} \right)^2 \right)\right) \nonumber \\
&=  4\E\left( \left\| V \right\|_*^2 \left(\dfrac{\xi^+ - \xi^-}{2\delta}\right)^2\right)
+ 4 \E \left(\left\| V \right\|_*^2 \right)\left( \dfrac{f(X^+) - f(X^-)}{2\delta} \right)^2  \label{eq:h3} \\
& \le  \frac{C_2}{\delta^2}\,, \nonumber \label{eq:h4}
\end{align}
where $C_2 = 4 \EE{\norm{V}_*^2}\left( \sigma_\xi^2+\fspan(f)\right)$
and $\fspan(f) = \sup_{x\in \D} f(x) - \inf_{x\in \D} f(x)$.
The equality in \eqref{eq:h3} follows from $\EE{ \xi^+-\xi^- \,|\, U,V } = 0$.

Therefore, for $f \in \C^3$, $\gamma$ defined by \eqref{eq:twosp} is a $(C_1\delta^2, C_2/\delta^2)$ type-I oracle.

\paragraph{Case 2 (Controlled noise and $F$ is convex and $L_{\psi}$-smooth):}\ \\
%\subsubsection*{Proof of Proposition \ref{prop:grad-spsa} for convex and smooth $f$}
The proof follows by parallel arguments to that used in the proof of Lemma 1 in \cite{duchi2015optimal} and we give it here for the sake of completeness.

For any convex function $f$ with an $L$-Lipschitz gradient, for any $\delta>0$ it holds that
\begin{align*}
\frac{\<\nabla f(x), \delta u\>}{2\delta} \le \frac{f(x + \delta u) -  f(x)}{2\delta} \le& \frac{\<\nabla f(x), \delta u\> + (L / 2) \norm{\delta u}^2}{2\delta}.
\end{align*}
Using similar inequalities for $f(x-\delta u)$, we obtain
\begin{align*}
\<\nabla f(x), u\> - \frac{L \delta \norm{ u}^2}{2} \le \frac{f(x + \delta u) -  f(x-\delta u)}{2\delta} \le& \<\nabla f(x), u\> + \frac{L \delta \norm{ u}^2}{2}.
\end{align*}
Letting
$\phi(x,\delta,u):=\frac1{\delta}\left(\frac{f(x + \delta u) -  f(x-\delta u)}{2\delta} - \<\nabla f(x),  u\>\right)$, we get
\begin{align*}
\left|\phi(x,\delta,u) \right| \le&  \dfrac{L}{2} \norm{u}^2\,.
% \frac{f(x - \delta u) -  f(x)}{\delta} \le& -\frac{\<\nabla f(x), \delta u\> + (L / 2) \norm{\delta u}^2}{\delta}
\end{align*}

Using $\EE{V U^\top}=I$, we obtain
\begin{align*}
\E\left[V\,  \left(\frac{f(x+\delta U)  -f(x-\delta U)}{2\delta}\right)\right]=&
\E\left[ V U^\top\nabla f(x) +  \delta\phi(x,\delta,U) V \right]\\
= &  \nabla f(x) + \delta \widehat\phi(x,\delta),
\end{align*}
where $\widehat\phi(x,\delta)$ satisfies $\scnorm{\widehat\phi(x,\delta)}_*  \le \, \dfrac{L}{2}\E[ \dnorm{V} \norm{U}^2]$.

Applying the above expression to $F(\cdot, \Psi)$ and recalling that $G=V\,  \left(\tfrac{F(X^+,\psi)  -F(X^-,\psi)}{2\delta}\right)$, we have, for $P$-almost every $\psi$, 
$$\E[G] = \nabla F(x,\psi) + \delta \widehat\phi(x,\delta),$$
where, as before, $\widehat\phi(x,\delta)$ satisfies $\scnorm{\widehat\phi(x,\delta)}_*  \le \, \dfrac{L_{\psi}}{2}\E[ \dnorm{V} \norm{U}^2]$.

Using the fact that $E[\nabla F(x,\Psi)] = \nabla f(x)$, we obtain
\begin{align*}
 \norm{\E[G] - \nabla f(x)}_*
 &= \scnorm{\E\left[V\,  \left(\frac{f(x+\delta U)  -f(x-\delta U)}{2\delta}\right)-V U^\top\nabla f(x) \right]}_*
 \le \, \delta \norm{\E[ V \phi(x,\delta, U)]}_*\\
 &\le \,\frac{\delta \overline{L}_{\Psi}}{2} \E[ \dnorm{V} \norm{U}^2],
\end{align*}
and the claim for the bias follows by setting $C_1= \frac{\overline{L}_{\Psi}}{2} \E[ \dnorm{V} \norm{U}^2]$.

We now bound $\EE{ \norm{G}_*^2}$ as follows:
%In the case of controlled noise, i.e., $\xi^+ = \xi^-$,
\begin{align*}
 \E \norm{G}^2
& = \mathbb{E}\norm{V\left(\delta \phi(x,\delta, U)+ U^\top\nabla f(x) \right)}^2
 \le  \E\left[ \left( \dnorm{ V U \tr \nabla f(x)} + \frac{\delta L}{2} \dnorm{V} \norm{U}^2 \right)^2\right]\\
& \le  2 \E\left[  \dnorm{ V U \tr \nabla f(x)}^2\right]  + \frac{\delta^2 \overline{L}_{\Psi}^2}{2}\E\left[ \dnorm{V}^2 \norm{U}^4 \right],
\end{align*}
and the claim for the variance follows by setting $C_2 =  2 B_1^2  + \frac{ \overline{L}_{\Psi}^2}{2}\E\left[ \dnorm{V}^2 \norm{U}^4 \right]$ with $B_1 = \sup_{x\in \K} \scnorm{\nabla f(x)}_*$.

Therefore, for the case of controlled noise with a convex and $L_{\psi}$-smooth $F$, we have that $\gamma$ defined by \eqref{eq:twosp} is a $(C_1\delta, C_2)$ type-I oracle.  
%For the case of uncontrolled noise, the claim for the variance follows in the same manner as that in the proof of Proposition \ref{prop:grad-spsa} and we obtain a $(C_1\delta, C_2/\delta^2)$ type-I oracle.
%\end{proof}




\section{Proofs for the Lower Bound}
\label{sec:appendix-lb-proof}
%!TEX root =  bgo.tex
% please do not delete or change the first line (needed by Csaba's editor)
\subsection{Proofs for the Lower Bound}
\label{sec:lb-proof}
%%%%%%%%%%%%%%%%%%%%%%%%%%%%%%%%%%%%%%%%%%%%%%%%%%%%%%%%%%%%%%%%%%%%%%%%%%%%%%% 

For brevity, let $\Delta_n^*$ denote $\Delta_n^*(\F, c_1,c_2)$.
The proof uses only $(c_1,c_2)$ Type-I oracles $\gamma$ that map $\cK$ to $\R$ (i.e., the oracles do not have memory).
Fix a  $(c_1,c_2)$ Type-I oracle $\gamma$ and an algorithm $\A$.
We restrict the class of oracles to those that return a random gradient estimate $G = m(x) + \xi$ with some map $m: \cK \to \R$,
where $\xi$ is standard normal with variance $\sigma^2 = C_2 \delta^q$, satisfying the variance requirement. 
The map $m$, which, by slightly abusing notation, we will also denote by $\gamma$ in what follows, will be chosen based on $f$ to satisfy the requirement on the bias. The $Y$ value returned by the oracles is made equal to $x$.

Let $\delta$ denote the tolerance parameter chosen by $\A$.
Define the probability space $(\Omega, \B, P_{\A,\gamma})$, 
where $\Omega = \R^n\times \{-1,1\}$, $\B$ is the associated Borel sigma algebra. 
Further, the probability measure $P_{\A,\gamma} := p_{\A,\gamma} d(\lambda \times m)$, 
	$\lambda$ is the Lebesgue measure on $\B$, 
	$m$ is the counting measure on $\{-1,1\}$ and 
	$p_{\A,\gamma}$ is the density function defined as
\begin{align*}
&p_{\A,\gamma}(g_{1:n}, v) = \dfrac{1}{2} \bigg(p_{\A,\gamma}(g_n \mid g_{1:n-1})\times \dots \\
&\times p_{\A,\gamma }(g_{n-1} \mid g_{1:n-2}) \ldots p_{\A,\gamma}(g_1)\bigg)\\
 = & \dfrac{1}{2} \bigg( p_{\N}(g_n - \gamma(\A_n(g_{1:n-1}))) \cdot
%  &\times p_{\normal(0,\sigma^2)}(g_{n-1} - \gamma(\A_n(g_{1:n-2}))) \times \ldots \\
 \ldots \cdot  p_{\N}(g_1 - \gamma(\A_1))\bigg),
\end{align*}
where $p_{\N}$ is the density of a $\normal(0,\sigma^2)$ random variable,
and $\cA_t$ denotes the map from the algorithm's past observations
that picks the point that is sent to the oracle in round $t$.
\todoc{Enough to consider deterministic algorithms}

%%%%%%%%%%%%%%%%%%%%%%%%%%%%%%%%%%%%%%%%%%%%%%%%%%%%%%%%%%%%%%%%%%%%%%%%%%%%%%% 

By assumption,  $\{f_+, f_-\}\subset \F$, with 
\begin{align*}
  f_+(x) := \dfrac{\epsilon}{2} (x - 1)^2 \text{ and } f_-(x) := \dfrac{\epsilon}{2} (x + 1)^2, \,\, x \in \cK
\end{align*}
(we will slightly abuse notation by using $f_+$ ($f_-$) in place of $f_{+1}$ (resp., $f_{-1}$).
%%%%%%%%%%%%%%%%%%%%%%%%%%%%%%%%%%%%%%%%%%%%%%%%%%%%%%%%%%%%%%%%%%%%%%%%%%%%%%% 

Clearly, $f_+$ (resp, $f_-$) is minimized at $x^*_+ = 1$ (resp. $x^*_- = -1$) with the minimum value being zero.
Hence, \todoc{No need for strong convexity..}
%Using the fact that $f_+$ and $f_-$ are strongly convex with associated constant $\left(\dfrac{\epsilon}{2}\right)$, we obtain
\begin{align}
  f_+(x) - f_+(x^*_+)
  = &  \dfrac{\epsilon}{2} (x - 1)^2 \ge  \dfrac{\epsilon}{2}  \indic{x  < 0}. \label{eq:fv-lb}
\end{align}
Similarly,   $f_-(x) - f_-(x^*_-) \ge  \dfrac{\epsilon}{2}  \indic{x  >0}$.
We will consider the oracles $\gamma_v$ defined using 
by $\gamma_v(x) = \epsilon(x-v) + v\, C_1 \delta^p$ (as with $f_v$, we will also use $\gamma_{+}$ ($\gamma_-$) 
to denote $\gamma_{+1}$ (resp., $\gamma_{-1}$).
The oracle is indeed a $(c_1,c_2)$ Type-I oracle.
%%%%%%%%%%%%%%%%%%%%%%%%%%%%%%%%%%%%%%%%%%%%%%%%%%%%%%%%%%%%%%%%%%%%%%%%%%%%%%% 

The minimax error \eqref{eq:minimax-err} is lower bounded by
%\footnote{$f_{+1} \equiv f_+$ and $f_{-1}\equiv f_-$.}:
\begin{align}
\MoveEqLeft 
\Delta_n^* %\nonumber\\
  \ge  \inf_{\A} \,  \E[f_V(\hat X_n) - \inf_{x \in X}
  f_V(x)],\label{eq:avg-bd}
  \end{align}
where the expectation is w.r.t. the distribution $\P:= \dfrac{1}{2} \left(P_{\A, \gamma_+} \indic{v=+1} + P_{\A, \gamma_-}\indic{v=-1}\right)$ and $V: \Omega \to \{\pm 1 \}$ is defined by $V(g_{1:n},v) = v$.%
\footnote{Here, we are slightly abusing the notation as $\P$ depends on $\A$, but the dependence is suppressed.
In what follows, we will define several other distributions derived from $\P$, which will all depend on $\A$, but
for brevity this dependence will be also suppressed.}
%Here $\gamma_v$  is the oracle for $f_v$, for $v=+1,-1$ and is defined by $\gamma_v(x) = \epsilon(x-v) + v C_1 \delta^p$. 
Using \eqref{eq:fv-lb}, we obtain
\begin{align}
\Delta_n^*  \ge & \inf_{\A} \dfrac{\epsilon}{2}\,  \P(\hat X_n V < 0), \label{eq:strong-convex-bd}\\
  = & \inf_{\A} \dfrac{\epsilon }{2} \, \left(\P_{+}(\hat X_n < 0) + \P_{-}(\hat X_n > 0)\right), \label{eq:Pplus}\\
  \ge &\inf_{\A} \dfrac{\epsilon }{2} \,\left(1 - \tvnorm{\P_{+}- \P_{-}}\right), \label{eq:lecam}\\
  \ge &\inf_{\A} \dfrac{\epsilon }{2}  \,\left( 1 - \left(\dkl{P_{+}}{P_{-}}\right)^{\frac{1}{2}}\right), \label{eq:pinsker}
\end{align}
where 
the equality in \eqref{eq:Pplus} uses the definitions $\P_{+}(\cdot) := \P(\cdot\mid V=1)$, $\P_{-}(\cdot) := \P(\cdot\mid V=-1)$. Further, the inequality in \eqref{eq:lecam} follows from the definition of total variation distance, while \eqref{eq:pinsker} follows from Pinsker's inequality. \todoc{You lost a factor of two here!}


%%%%%%%%%%%%%%%%%%%%%%%%%%%%%%%%%%%%%%%%%%%%%%%%%%%%%%%%%%%%%%%%%%%%%%%%%%%%%%% 

% \paragraph{Upper-bounding the difference in gradient estimates:}



%%%%%%%%%%%%%%%%%%%%%%%%%%%%%%%%%%%%%%%%%%%%%%%%%%%%%%%%%%%%%%%%%%%%%%%%%%%%%%% 
Define $G_t$ to be the $t$th observation of $\A$: $G_t:\Omega \to \R$, with $G_t( g_{1:n}, v) = g_t$.
Let $P_+^t(g_1,\dots,g_t)$ denote the joint distribution of $G_1,\dots,G_t$ conditioned on $V=+1$.
Let $P_{+}^t(\cdot\mid g_1,\ldots,g_{t-1})$ denote the distribution of $G_t$ conditional on $V=+1$ and $G_1=g_1,\ldots,G_{t-1}=g_{t-1}$. Define  $P_{-j}^t(\cdot\mid g_1,\ldots,g_{t-1})$ in a similar fashion.
Then, by the chain rule for KL-divergences, we have
\begin{scriptsize}
\begin{align}
&\dkl{P_{+}}{P_{-}}\label{eq:dklchain}\\ 
&= \sum_{t=1}^n \int_{\R^{t-1}} \dkl{P_{+}^t(\cdot\mid g_{1:t-1})}{P_{-}^t(\cdot\mid g_{1:t-1})} d P_{+}^t( g_{1:t-1}).\nonumber
\end{align}
\end{scriptsize}
By the oracle's definition, $G_t \sim  \normal(\gamma_{+}(\cA_t(G_{1:t-1})),\sigma^2)$. Hence, 
$P_{+}^t(\cdot\mid g_{1:t-1})$ is the normal distribution with the mean $\cA_t(g_{1:t-1})$ and variance $\sigma^2$.
%, where $A(g_{1:t-1})$ denotes the point chosen by the algorithm given observations $g_1,\ldots, g_{t-1}$ and $\gamma_{+}$ is the gradient oracle defined earlier. 
Observe that, for any $x\in \R$, $f_+'(x) - f_-'(x) = 2\epsilon$ and hence
\begin{align}
 |\gamma_+(x) - \gamma_-(x)| 
& = | f'_+(x) + C_1 \delta^p - (f'_-(x)-C_1 \delta^p | \nonumber \\
& = 2| \epsilon - C_1 \delta^p |\,.
 \label{eq:gdiff-ub}
\end{align}
From the foregoing, 
\begin{align}
 \MoveEqLeft \dkl{P_{+}^t(\cdot\mid g_{1:t-1})}{P_{-}^t(\cdot\mid g_{1:t-1})}\nonumber\\
 =&\dfrac{(g_{+}(A(g_{1:t-1})) - g_{-}(A(g_{1:t-1})))^2}{2 \sigma^2}\label{eq:dkgauss1}\\
 =& \dfrac{4(\epsilon-C_1\delta^p)^2\delta^q}{C_2}.\label{eq:dkgauss}
\end{align}
The equality \eqref{eq:dkgauss1} follows from the fact that the KL-divergence between normal distributions $\normal(\mu_1,\sigma^2)$ and $\normal(\mu_2,\sigma^2)$ is equal to $\dfrac{(\mu_1 - \mu_2)^2}{2 \sigma^2}$, while the equality \eqref{eq:dkgauss} follows from \eqref{eq:gdiff-ub} and the choice $\sigma^2 = C_2 \delta^{-q}$.

Plugging \eqref{eq:dkgauss} into \eqref{eq:dklchain}, we obtain
\begin{align}
\dkl{P_{+}}{P_{-}} \le \dfrac{4n(\epsilon-C_1\delta^p)^2 \delta^q}{C_2}.\label{eq:dkbd}
\end{align}
Note that the above bound holds uniformly over all algorithms $\A$. 
Substituting the above bound into \eqref{eq:pinsker}, we obtain 
\todoc[inline]{This seems to assume that $\sqrt{ (\epsilon - C_1 \delta^p )^2 } = \epsilon - C_1 \delta^p$, which will only hold if $\epsilon$ is big compared to $C_1 \delta^p$. We will need to do a case analysis.}
\begin{align}
 \Delta_n^*
  \ge & \dfrac{\epsilon}{2} \left(1 - 2\sqrt{
    n}  \dfrac{|\epsilon-C_1\delta^p|\delta^{q/2}}{C_2}
  \right)\label{eq:final-lower-bd}
\end{align}

%%%%%%%%%%%%%%%%%%%%%%%%%%%%%%%%%%%%%%%%%%%%%%%%%%%%%%%%%%%%%%%%%%%%%%%%%%%%%%%
\paragraph{Derivations of rates:}
 Optimizing over $\epsilon$ in \eqref{eq:final-lower-bd}, 
\todoc{As said, this only works for $\epsilon$ large enough.} we get
\[
\epsilon^* = \left(\dfrac{\delta^{-q/2}C_2}{4\sqrt{n}} + \dfrac{C_1\delta^p}{2}\right).
\]

 Plugging in $\epsilon^*$, we obtain
 \[
 \Delta_n^*
 \ge \left(\dfrac{\delta^{-q/2}C_2}{8\sqrt{n}} + \dfrac{C_1\delta^p}{4}\right) \left( \dfrac{1}{2} + \dfrac{\sqrt{n}C_1 \delta^{p + q/2}}{C_2} \right).
 \]

Substituting $p=1$ and $q=2$, we obtain
\[
\delta^*= \sqrt{\dfrac{C_2}{2C_1}}\dfrac{1}{n^{\frac{1}{4}}} \text{ and } \Delta_n^* \ge \dfrac{\sqrt{C_1C_2}}{2n^{\frac{1}{4}}}.
\]

On the other hand, substituting $p=q=2$, we obtain
\[
\delta^*= \left(\dfrac{C_2}{2C_1}\right)^{\frac{1}{3}}\dfrac{1}{n^{\frac{1}{6}}} \text{ and } \Delta_n^* \ge \dfrac{C_1^{\frac{1}{3}}C_2^{\frac{2}{3}}}{2^{\frac{5}{3}}n^{\frac{1}{3}}}.
\]






\end{document}
