\documentclass[twoside, 11pt]{article}
%\usepackage[accepted]{aistats2016}
%!TEX root =  bgo.tex
% please do not delete or change the first line (needed by Csaba's editor)

\usepackage{etex}
\usepackage[utf8]{inputenc}
\usepackage{amsmath}
\usepackage{amsfonts}
\usepackage{amssymb}
%\usepackage{amsthm}
\usepackage{mathtools}
\usepackage{graphicx}
\usepackage{algorithm}
\usepackage{natbib}
\usepackage[noend]{algpseudocode}
\usepackage{url}
\usepackage{bm}
\usepackage[margin=2.5cm]{geometry}
\setlength{\marginparwidth}{20mm}

\usepackage[textsize=tiny]{todonotes}
%\usepackage[disable]{todonotes}
\newcommand{\todoc}[2][]{\todo[color=red!20!white,size=\tiny,#1]{Cs: #2}}
\newcommand{\todox}[2][]{\todo[color=orange!20!white,size=\tiny,#1]{X: #2}}
\newcommand{\todoa}[2][]{\todo[color=blue!20!white,size=\tiny,#1]{A: #2}}
\newcommand{\todop}[2][]{\todo[color=green!20!white,size=\tiny,#1]{P: #2}}
\usepackage{hyperref}

\makeatletter
\def\BState{\State\hskip-\ALG@thistlm}
\makeatother

%\newtheorem{theorem}{Theorem}
%\newtheorem{corollary}{Corollary}[theorem]
%\newtheorem{lemma}[theorem]{Lemma}
\usepackage[capitalize]{cleveref}
\usepackage[amsmath,standard,thmmarks]{ntheorem} % ntheorem makes cleveref work properly 

\newcommand{\cE}{\mathcal{E}}
\newcommand{\cF}{\mathcal{F}}
\newcommand{\cA}{\mathcal{A}}
\newcommand{\cK}{\mathcal{K}}
\newcommand{\cB}{\mathcal{B}}
\newcommand{\cY}{\mathcal{Y}}
\newcommand{\R}{\mathbb{R}}
\newcommand{\EE}[1]{\mathbb{E}\left[#1\right]}
\newcommand{\ip}[1]{\langle #1 \rangle}
\newcommand{\norm}[1]{\|#1\|}
\newcommand{\scnorm}[1]{\left\|#1\right\|}
\DeclareMathOperator\erf{erf}
\DeclareMathOperator{\argmin}{argmin}
\DeclareMathOperator{\argmax}{argmax}
\DeclareMathOperator{\interior}{int}
\DeclareMathOperator{\dom}{dom}
\newcommand\numberthis{\addtocounter{equation}{1}\tag{\theequation}}

%%% Prashanth: some of my macros..need to remove duplicates
\newcommand{\A}{\mathcal{A}}
\newcommand{\I}{\mathcal{I}}
\newcommand{\B}{\mathcal{B}}
\newcommand{\V}{\mathcal{V}}
\newcommand{\F}{\mathcal{F}}
\newcommand{\N}{\mathcal{N}}
\newcommand{\E}{\mathbb{E}}
\renewcommand{\P}{\mathbb{P}}
\newcommand{\G}{\mathcal{G}}
\renewcommand{\O}{\mathcal{O}}



\newcommand{\sign}{\mathop{\rm sign}}
% \newcommand{\norm}[1]{\left\|{#1}\right\|} % A norm with 1 argument
\newcommand{\tvnorm}[1]{\norm{#1}_{\rm TV}}
\newcommand{\dkl}[2]{D_{\rm kl}\left({#1} |\!| {#2} \right)}
\newcommand{\normal}{\mathsf{N}}  % Normal distribution
\newcommand{\openleft}[2]{\left({#1},{#2}\right]} % Interval open on left
\def\mbf#1{\mathbf{#1}}
\def\indic#1{\mbf{1}\left\{#1\right\}} % Indicator function
% Compilation note:
% Don't forget to add --shell-escape (or -enable-write18 if you are using MikTeX) to the pdflatex commandline.
% see http://www.howtotex.com/tips-tricks/faster-latex-part-ii-external-tikz-library/

%%%%%%%%%%%%%%%%%%%%%%%%%%%%%%%%%%%%%%%%%%%%%%%%%%%%%%%%%%%%%%
%%%%%%%%%%%%%%%%%%%%%%%%%%%%%%%%%%%%%%%%%%%%%%%%%%%%%%%%%%%%%%
\ShortHeadings{(Bandit) Convex Optimization with Biased Noisy Gradient Oracles}{}

\title{(Bandit) Convex Optimization with Biased Noisy Gradient Oracles\thanks{An earlier version of this paper was published in AISTATS 2016 \citep{HuPrGySz16}.}}
\author{\name Xiaowei  Hu \email xhu3@ualberta.ca \\
  \addr Department of Computing Science,\\
  University of Alberta,\\
	Edmonton T6G 2E8, Canada
  \AND
  \name Prashanth L.A. \email prashla@isr.umd.edu\\
  \addr Institute for Systems Research, \\
  University of Maryland, \\
	College Park, MD 20742, US
  \AND
  \name Andr\'as Gy\"orgy \email  a.gyorgy@imperial.ac.uk\\
  \addr Department of Electrical and Electronic Engineering,\\
	Imperial College London,\\
  South Kensington Campus, London SW7 2BT, UK 
  \AND
  \name Csaba Szepesv\'ari \email szepesva@cs.ualberta.ca \\
  \addr Department of Computing Science,\\
  University of Alberta,\\
	Edmonton T6G 2E8, Canada
}

 \editor{}

\begin{document}

\maketitle

%%%%%%%%%%%%%%%%%%%%%%%%%%%%%%%%%%%%%%%%%%%%%%%%%%%%%%%%%%%%%%
%%%%%%%%%%%%%%%%%%%%%%%%%%%%%%%%%%%%%%%%%%%%%%%%%%%%%%%%%%%%%%
%%%%%%%%%%%%%%%%%%%%%%%%%%%%%%%%%%%%%%%%%%%%%%%%%%%%%%%%%%%%%%
%%%%%%%%%%%%%%%%%%%%%%%%%%%%%%%%%%%%%%%%%%%%%%%%%%%%%%%%%%%%%%
%%%%%%%%%%%%%%%%%%%%%%%%%%%%%%%%%%%%%%%%%%%%%%%%%%%%%%%%%%%%%%




\begin{abstract}
Algorithms for bandit convex optimization and online learning often 
rely on constructing noisy gradient estimates, which are then used
 in appropriately adjusted first-order algorithms, replacing
 actual gradients.
Depending on the properties of the function to be optimized and the nature of ``noise'' in the bandit feedback,
the bias and variance of gradient estimates exhibit various tradeoffs.
% and, correspondingly,
%the algorithms were proven to achieve various convergence rates.
 In this paper we propose a novel framework that replaces the specific gradient estimation
 methods with an abstract oracle.
 % capturing the controlleable bias-variance tradeoff
 %of existing gradient estimators. 
 With the help of the new framework we unify previous works,
 reproducing  their results in a clean and concise fashion, 
 while, perhaps more importantly, the framework also allows us to formally show that to achieve the optimal 
  root-$n$ rate %in online/stochastic bandit convex optimization
  either the algorithms that use existing gradient estimators,
  or the proof techniques used to analyze them 
  have to go beyond what exists today.
\if0
We present a novel noisy gradient oracle model for convex optimization. The model allows to explicitly address the bias-variance tradeoff of gradient estimation methods typical in the literature, as well as to prove upper and lower bounds for the minimax error. As such the oracle model allows a clean way of addressing the limits of achievable performance
when biased, noisy gradient estimators with controllable bias-variance tradeoffs are employed.
When considering the model for online/stochastic convex optimization,
one consequence of our results is that the currently used gradient estimates and the proof techniques used to analyze them cannot 
achieve the optimal square-root regret rate for online/stochastic bandit convex optimization.
\fi
% when the function cannot be queried outside the domain.
\end{abstract}

%\vspace{-0.3cm}

\section{Introduction}
\label{sec:intro}
%!TEX root =  bgo-opt-ml-nips.tex
% please do not delete or change the first line (needed by Csaba's editor)

In stochastic bandit convex optimization (also known as  convex optimization with stochastic zeroth order oracles)
an algorithm submits queries to an oracle in a sequential manner in $n$ rounds.
The oracle returns noisy values of the convex objective function at the submitted points.
At the end, the algorithm also produces a guess of the objective's minimizer; the algorithm's performance
is measured in terms of the suboptimality of this guess, measured using the objective function.
In their seminal work \citet{NeYu83} consider two approaches to this problem: Methods that try to construct
gradient information and methods that avoid gradient estimation and rather use 
geometric principles (the ellipsoid method, essentially).
While methods in the second class make the error decay at the $O(1/\sqrt{n})$ rate, 
the error scales extremely poorly with the number of optimization variables $d$.
For example, \citet{liang2014zeroth} proves a bound of the form $\sqrt{d^{14}/n}$, improving the $\sqrt{d^{33}/n}$ bound
of \citet{AgFoHsuKaRa13:SIAM}.
A lower bound due to \citep{shamir2012complexity} however scales only with $\sqrt{d^2/n}$.
Methods that first construct gradient estimates, which are fed to algorithms that expect gradient-information
have a long history, though in early years the focus was either asymptotic convergence,
or asymptotic rates. \todoc{Prashanth: Can you please add some citations}
The story with these methods is that they get the optimal rate and dimension dependence for ``nice'' problems, such
as when the objective function is strongly convex and smooth 
\citep{HaLe14:SOC}, but their performance, mostly in terms of the rate degrades as one removes constraints
from the objective function. For example, for smooth convex problems, the best published results with this technique 
is $O(n^{1/3})$ \citep{saha2011improved}.%
%\footnote{
%A day before the submission to the workshop we became aware of the recent NIPS submission of
%\citet{DeKo15:BSCO}, which claims to improve this result. 
%}
 \todoc{This \citet{jamieson2012query} paper is weird. Also, it does not construct gradient estimates, so I took out from here.}
Our motivation in this paper is to formally study the possible limitations of the gradient-based approach.
We do this by precisely defining a new oracle model. The new oracles can be consulted to obtain gradient estimates.
They have a tuneable parameter, which controls the bias-variance tradeoff that 
exists for all known gradient construction methods. We then prove lower bounds and (sometimes) matching upper bounds
for algorithms that use these oracles. Our lower bound for stochastic smooth bandit convex optimization gives the rate 
$\Omega(n^{1/3})$, assuming that the currently available gradient-estimation methods are unimprovable.


%The authors provided some algorithms and upper bounds, but as they themselves emphasize (cf. pg. 359), the attainable complexity is far from clear. Quite recently, Jamieson et al. (2012) provided an �(??d/T) lower bound for strongly-convex functions, which demonstrates that the �fast� O(1/T ) rate in terms of T , that one enjoys with gradient information, is not possible here. In contrast, the current best-known upper bounds are O(??4 d2/T),O(??3 d2/T),O(??d2/T) for convex, strongly-convex, and strongly-convex-and-smooth functions respectively (Flaxman et al. (2005); Agarwal et al. (2010)); And a O(??d32/T ) bound for convex functions (Agarwal et al. (2011)), which is better in terms of dependence on T but very bad in terms of the dimension d.






\section{Problem Setup}
\label{sec:problem}
%!TEX root =  bgo.tex
% please do not delete or change the first line (needed by Csaba's editor)

$\cK \subset \R^d$: domain, convex, closed, non-emtpy.

Interaction with the oracle + diagrams

Definitions of oracles

\begin{definition}
Input is $x\in \cK,\delta>0$,
$f\in \cF$ ($f$ coming from the environment, others from the algorithm).
Output is $Y\in \cK$,$G\in \R^d$, $G$ is random.

$\norm{x-Y}\le \delta$, 
$\norm{ \EE{G}  - \nabla f(x)  }_* \le C_1 \delta^p$.
$\EE{\norm{ G -  \EE{G} }_*^2} \le C_2 \delta^{-q}$,

Meaning of $Y$: the function is evaluated at $Y$. This will matter only for the online case, when we consider the cumulative regret.
\end{definition}

Next one:

\begin{definition}
Input is $x\in \cK,\delta>0$,
$f\in \cF$ ($f$ coming from the environment, others from the algorithm).
Output is $Y\in \cK$,$G\in \R^d$, $G$ is random.

$\norm{x-Y}\le \delta$, 

Also: there exists $\tilde{f} \in \cF$ such that 
$\norm{\tilde{f}- f}_\infty \le C_1 \delta^p$ (bias)
$\EE{G}  = \nabla \tilde{f}(x)$,
$\EE{\norm{ G -  \EE{G} }_*^2} \le C_2 \delta^{-q}$,

\end{definition}

alternative to bias condition for second definition:
\begin{align}
\norm{\nabla \tilde{f}- \nabla f}_\infty \le C_1 \delta^p
\end{align}


\section{Main Results}
\label{sec:results}
%!TEX root =  bgo.tex
% please do not delete or change the first line (needed by Csaba's editor)

Unconstrained case:

Upper bound: 

Lower bound:

Constrained case:

Upper bound: 

Reductions:
Between oracles;
Optimization and cumulative regret minimization;




\section{Applications to Stochastic BCO}
\label{sec:sbco}
%!TEX root =  bgo_camera_ready.tex
% please do not delete or change the first line (needed by Csaba's editor)
The main application of the biased noisy gradient oracle based convex optimization of the previous section
is to bandit convex optimization, which we briefly introduce now. Readers familiar with these problems and the associated
gradient estimation techniques, may skip this description to jump directly to \cref{thm:aaa},
and come back to it only to clarify our notation and terminology in case some confusion arises later.

\if0
For example, \cite{AgDeXi10} used an algorithm that queries at two points per round, and defined the incurred loss as the average of losses at the two observation points. In our oracle, it can be stated as in round $t$, the same oracle $\gamma_t$ responds to the same inputs $(X_t, \delta_t, f_t)$ with two different pairs $(G_{t,1}, Y_{t,1})$ and $(G_{t,2}, Y_{t,2})$. The accumulated regret can be written as
$% \[
R_n = \sum_{t=1}^n \dfrac{1}{2}\left( f_t(Y_{t,1})+f_t(Y_{t,2}) \right) -\inf_{x \in \cK}\sum_{t=1}^n f_t(x) \,.
$ %\]
Recalling that $Y_{t,1}$, $Y_{t,2}$ are in the $\delta$-vicinity of $X_t$, the relationship between $f_t(X_t)$ and $\dfrac{1}{2}\left( f_t(Y_{t,1})+f_t(Y_{t,2})\right)$ is then determined by the environment (i.e. the property of $f_t$). It is straightforward to bound $| \dfrac{1}{2}\left( f_t(Y_{t,1})+f_t(Y_{t,2})\right)- f_t(X_t)|$ as a function of $\delta$. \todoc{How? When $f_t$ is Lipschitz, smooth, etc? So you mean, when $f_t$ is a smooth function?}
The common assumption for this setting is: The oracle is a stochastic mapping from $(X, \delta, f)$ to $(G, Y)$; The algorithm selects the point $X_t$ depending on $\left( X_1, G_{1,1}, G_{1,2}, \cdots, X_{t-1},G_{t-1,1}, G_{t-1,2}  \right)$. This two-point feedback can be easily extended to multi-point feedback, too.
\fi

In the \emph{stochastic BCO} setting,
the algorithm sequentially chooses the points $X_1,\dots,X_n\in \cK$ while observing the loss function at these points in noise.
In particular, in round $t$, the algorithm chooses $X_t$ based on the earlier observations $Z_1,\dots,Z_{t-1}\in \R$ and $X_1,\dots,X_{t-1}$, after which it observes $Z_t$, where $Z_t$ is the value of $f(X_t)$ corrupted by ``noise''.
Previous research considered several possible constraints connecting $Z_t$ and $f(X_t)$.
One simple assumption is that $Z_t-f(X_t)$ is an $\cF_t = \sigma(X_{1:t},Z_{1:t-1})$-adapted martingale difference sequence (with favourable tail properties).
% \todoc{Some readers might be put off by martingales..}
A specific case is when $Z_t - f(X_t) = \xi_t$, where $(\xi_t)$ is an i.i.d. sequence.
A stronger assumption, which is most appropriate in stochastic programming,
is that $Z_t = F(X_t,U_t)$, where $U_t\in \R$, $\int F(X_t,u) dP_U(u) = f(X_t)$ with some distribution function $P_U$ over the reals and the algorithm has access to an oracle that can produce independent samples from $F$ (in which case, $(U_t)$ may be an i.i.d. sequence sampled from $P_U$).
This assumption is stronger because the algorithm controls the
``noise''.
For instance, the algorithm may obtain samples of $F$ at $X^+$ and $X^-$, with the same noise levels, i.e., $\xi^+=\xi^-$.
Controlling the noise this way helps in improving the accuracy of the estimates and is common in the field of \textit{simulation optimization},.
Again, it is also possible to consider multi-point feedback as in the online case.

However, what differs in the optimization variant is that the value of the loss at the points sent to the oracle does not matter.
Hence, in this setting the distinction between one-point and multi-point feedback (as long as the number observations is fixed, independently of the dimension) is irrelevant:
By grouping $K$ multiple consecutive observations, one can turn an $n$ rounds one-point feedback setup into an $n/K$-round $K$-point feedback setup. The reduction in the number of rounds, being a fixed constant factor, is negligible, as far as the convergence rates are considered.


%A major tool in bandit convex optimization is to design gradient estimators, which are then used in conjunction with variants of gradient descent. \todoc{zillions of references.}
A common popular idea in bandit convex optimization is to use the bandit feedback to construct noisy (and biased) estimates of the gradient.
In the following, we provide a few examples for oracles that construct gradient estimates for function classes that are increasingly general - from smooth, convex to non-differentiable functions.

\paragraph{One-point feedback}
Given $x\in \cK$, $0<\delta\le 1$, common gradient estimates that are
based on a single query to the function evaluation oracle (the so-called
``one-point feedback'') take the form
\begin{align}
  \label{eq:one-point}
G = \frac{Z}{\delta}V, \textrm{ where } Z = f(x+\delta U) + \xi\,,
\end{align}
where $(U,V)\in \R^d\times \R^d$ are jointly distributed random variables
and $\xi$ is the function evaluation noise (the distribution of $\xi$ may depend on $x+\delta U$) and $G$ is the estimate of $\nabla f(x)$ ($f:\cK\to \R$).

In all oracle constructions we will use the following assumption:
\begin{ass}
  \label{ass:gradbasic}
  $\K \subset \D^\circ \subset \R^d$, where $f:\D \to \R$.
  \footnote{Here, $\D^\circ$ denotes the interior of $\D$.}
  The joint distribution for $(U,V)$ is such that
  for any $x\in \K$, $x+\delta U \in \D$ a.s.,
  $\E[V U\tr] = I$, while
  $\EE{\norm{V}_*^2}$, $\EE{ \norm{U}^3 }<+\infty$.
\end{ass}

For uncontrolled noise, this can be shown to be a $(c_1,c_2)$ Type-I oracle with slightly stronger assumptions on $(U,V)$ (e.g., $U$ symmetrically distributed, $V = h(U)$, with $h$ odd in addition to the previous assumptions)
with $(c_1,c_2)$ essentially the same as before, except that now instead of the span of $f$, the magnitude of $f$ appears in the bound. For the controlled noise case, the oracle has the same variance as in the case of uncontrolled noise, because this oracle does not ``cancel out'' the noise.%
This is exactly the property that is exploited by \cite{duchi2015optimal}.

\begin{proposition}
\label{prop:grad-onepoint}
Let $f:\cD \to \R$ and let $\gamma$ be the one-point feedback oracle defined in \eqref{eq:one-point}.
Let \cref{ass:gradbasic} hold.
Assume further that
  $U$ is symmetrically distributed,
  $V = h(U)$, where $h:\R^d \to \R^d$ is an odd function,
  and
  $\EE{V}=0$,
Then, $\gamma$ is a type-I oracle with $c_1(\delta)$ and $c_2(\delta)$ same as the uncontrolled noise case in \cref{tab:oracles}.
% \todoc{Add the convex+smooth case}
\end{proposition}
\begin{proof}
See Section \ref{sec:appendix-grad}.
\end{proof}

\paragraph{One-point feedback with smoothing}
Another possibility is to use the so-called smoothing technique
\citep{PoTsy90,flaxman2005online,HaLe14:SOC}\todoa{ Cs: Is this correct?}
to obtain type-II oracles, see \cref{prop:flaxman}.
While the one-point estimators are intriguing, as discussed beforehand,
in the optimization setting one can also always group two consecutive observations and obtain similar smoothing-type estimates (see, e.g., the above discussion on the choice of $U$ and $V$)
%\citep[see][]{katkul,kushcla,spall1992multivariate,spall1997one,bhatnagar-book,duchi2015optimal}
at the price of reducing the number of rounds by a factor of two only, which has a negligible effect on the rate of convergence.

Type-II oracles can be obtained using the same construction as above, but with a different analysis:
\begin{proposition}
\label{prop:flaxman}
For the one-point feedback oracle stated in \cref{prop:grad-onepoint}, let
\begin{align*}
V = n_W(U)\dfrac{\lvert \partial W\rvert}{\lvert W \rvert}\,,
\end{align*}
where $W \subset \R^n$ is a convex body with the boundary $\partial W$, $U \in \partial W$ is uniformly distributed. $n_W(U)$ denotes the normal vector of $\partial W$ at $U$, and $\lvert \cdot \rvert$ denotes the volume.
Then, if $f$ is Lipschitz, $\gamma$ is a type-II oracle with $c_1(\delta)=C_1 \delta$, $c_2(\delta) = C_2/\delta^2$.
If $f$ is smooth, further assuming $W$ is symmetric w.r.t. the origin, $\gamma$ is a type-II, and also a type-IIb oracle with $c_1(\delta) = C_1\delta^2$, $c_2(\delta) = C_2/\delta^2$. Hence, in this case $\gamma$ is also a type-I oracle with the same $c_1$ and $c_2$.
\end{proposition}
\begin{proof}
See Section \ref{sec:appendix-grad}.
\end{proof}



\paragraph{Two-point feedback}
Here we present an oracle that uses two function evaluations to obtain a gradient estimate.
As will be discussed later, this oracle
 encompasses several simultaneous perturbation methods (see \cite{bhatnagar-book}):
Given the inputs $x\in \K$,  $0<\delta\le 1$,
% \todoc{We need an upper bound I believe. Add it earlier.}
the gradient estimate is
\begin{align}
G &=  \dfrac{Z^+ - Z^-}{2\delta}\, V \,, 
 \label{eq:twosp}
\end{align}
where $Z^{\pm} = f(X^{\pm}) + \xi^{\pm}$, $X^{\pm} = x \pm \delta U$, $U,V\in \R^d$, $\xi^{\pm}\in \R$ are random, jointly distributed random variables, $U,V$ chosen by the oracle
from some fixed distribution characterizing the oracle and $\xi^{\pm}$ being the noise of the returned feedback $Z^{\pm}$ at points $X^{\pm}$.
For the following proposition we consider $4=2\times 2$ cases.
First, the function is either assumed to be $L$-smooth and convex (i.e., the derivative of $f$ is $L$-Lipschitz w.r.t. $\norm{\cdot}_*$), or it is assumed to be three times continuously differentiable (in notation: $f\in C^3$).
The other two options are that either $\xi^+=\xi^-$, which we call the \emph{controlled noise} setting, or we make the alternate assumptions
\begin{align}
\E[\xi^+-\xi^- |\, U,V] = 0 \text{~~ and ~~}\nonumber\\
\E [ (\xi^{+} - \xi^-)^{2} \mid V] \le \sigma_\xi^2 <\infty\,.
\label{eq:noiseass}
\end{align}

%%%%%%%%%%%%%%%%%%%%%%%%%%%%%%%%%%%%%%%%%%%%%%%%%%%%%%%%%%%%%%%%%%%%%%%%%%%%%%%%%%%%%%%%%%%%%%%%%%%%%%%%%%%%%%%%%%%%%%%%%%%%%%%%%%%%%%%%%%%%%%%
\begin{table*}
\small
\centering
\begin{tabular}{|c|c|c|}
\toprule
\textbf{Noise }$\bm{ \rightarrow}$ & \multirow{2}{*}{\textbf{Controlled }($\bm{\xi^+ = \xi^-}$)} & \multirow{2}{*}{\textbf{Uncontrolled }(see~\eqref{eq:noiseass})} \\
\textbf{Function } &&\\
$\bm{\downarrow}$ &&\\\midrule
\multirow{2}{*}{\textbf{Convex + Smooth}} & \multirow{2}{*}{$(C_1 \delta, C_2)$} & \multirow{2}{*}{$(C_1\delta, C_2/\delta^2)$}\\
 &&\\\midrule
\multirow{2}{*}{$\bm{f \in \C^3}$} & \multirow{2}{*}{$(C_1 \delta^2, C_2/\delta^2)$} & \multirow{2}{*}{$(C_1 \delta^2, C_2/\delta^2)$} \\
 &&\\\bottomrule
\end{tabular}
\caption{Gradient oracles for different function classes and noise categories. Each table entry specifies the pair $(c_1(\delta), c_2(\delta))$.
For the first row, $C_1 =
\frac{L}{2} \E[ \dnorm{V} \norm{U}^2]$ and
$C_2 =   L^2 (2 + \frac{1}{2}\E\left[ \dnorm{V}^2 \norm{U}^4 \right])$
for the controlled noise and
 $C_2 =  C_{2}^{(u)} \doteq 4 \EE{\norm{V}_*^2}\left( \sigma_\xi^2+\fspan(f)\right)$ for the uncontrolled noise.
For the second row, $C_1 = \frac{B_3 \EE{ \norm{V}_* \norm{U}^3 }}{6}$ and $C_2 =  C_{2}^{(u)}$,
with $B_3 = \sup_{x\in \K} \norm{\nabla^3 f(x)}$, where $\norm{\cdot}$ is the implied norm for rank-3 tensors.
}
\label{tab:oracles}
\end{table*}
%%%%%%%%%%%%%%%%%%%%%%%%%%%%%%%%%%%%%%%%%%%%%%%%%%%%%%%%%%%%%%%%%%%%%%%%%%%%%%%%%%%%%%%%%%%%%%%%%%%%%%%%%%%%%%%%%%%%%%%%%%%%%%%%%%%%%%%%%%%%%%%%
The following proposition provides conditions under which the bias-variance parameters $(c_1,c_2)$ can be bounded as shown in \cref{tab:oracles}:
\begin{proposition}
\label{prop:grad-spsa}
Consider a function $f:\D \to \R$, with $\K \subset \D^\circ \subset \R^d$.%
\footnote{Here, $\D^\circ$ denotes the interior of $\D$.}
For any $x \in \cK$, and $0< \delta \le 1$ let the oracle $\gamma$ return $G$ as specified in~\eqref{eq:twosp},
where $(U,V)$ are such that $x+\delta U \in \D$ a.s.,
$\E[V U\tr] = I$, while
$\EE{\norm{V}_*^2}$ and $\EE{ \norm{U}^3 }<+\infty$.
Then, we have that $\gamma$ is a type-I oracle with $c_1(\delta)$ and $c_2(\delta)$ given by \cref{tab:oracles}.
\end{proposition}
\begin{proof}
See Section \ref{sec:appendix-grad}.
\end{proof}
\todoc[inline]{I am pretty sure that the bias could be brought down
for two-point feedback, too, using the technique of \cref{prop:flaxman}.
}

\paragraph{Popular choices for $V$ and $U$:}
\begin{inparaenum}[$\bullet$]
 \item If we set $U_i$ to be independent, symmetric $\pm 1$-valued r.v.s and $V_i = 1/U_i$, then we recover the popular SPSA scheme proposed by \cite{spall1992multivariate}.
It is easy to see that $\EE{  V U\tr } = I$ holds in this case.
 When the norm $\norm{\cdot}$ is the $2$-norm, $C_1 = O(d^2)$ and $C_2 = O(d)$. If we set $\norm{\cdot}$ to be the max-norm, $C_1 = O(\sqrt{d})$ and $C_2 = O(d)$.
 \item If we set $V=U$, with $U$ chosen uniform at random on the surface of a sphere with radius $\sqrt{d}$,
 then we recover the RDSA scheme proposed by  \citeauthor{kushcla} \citep[cf. pp.~58--60][]{kushcla}.
 In particular, $(U_i)_i$ are identically distributed with $\EE{ U_i U_j } = 0$ if $i\ne j$ and $\EE{ U\tr U } = d$, hence $\EE{U_i^2} = 1$. Thus, if we choose $\norm{\cdot}$ to be the $2$-norm, $C_1 = O( d^2 )$ and $C_2 = O(d)$.
 \item If we set $V=U$, with $U$ the standard $d$-dimensional Gaussian with unit covariance matrix, we recover the smoothed functional (SF) scheme proposed by \cite{katkul}.
Indeed, in this case, by definition, $\EE{VU\tr} = \EE{U U\tr } = I$.
When $\norm{\cdot}$ is the $2$-norm, $C_1 = O(d^2)$
%$\norm{U}^4 = (\sum_{i=1}^d U_i^2)^{2} = \sum_i U_i^4 + 2 \sum_{i<j} U_i^2 U_j^2$,
%hence $\EE{ \norm{U}^4} = O(d^2)$
 and $C_2 = O( d)$.
 This scheme can also be interpreted as a smoothing operation that  convolves the gradient of the function $f$ with a Gaussian density.
%  , followed by an integration by parts and the resulting integral can be estimated using samples (without access to the gradient of $f$).
\end{inparaenum}


% If the function $f$ is assumed to be convex and smooth, then gradient estimates similar to \eqref{eq:twosp} can be constructed and this does not require higher order smoothness conditions as in Proposition \ref{prop:grad-spsa}.
% \begin{proposition}
% \label{prop:grad-convex}
% Consider a function $f:\D \to \R$, with $\K \subset \D \subset \R^d$,
% that is convex and has a $L$-Lipschitz continuous gradient. .
% For any $x \in \cK$, and $\delta >0$ such that $\B(x,\delta) \in \D$, let the oracle $\gamma$ return
% \begin{align}
% % Y = x+\delta U \,, \quad
% G =  V \left(\dfrac{f(x+\delta U) + \xi^+ - (f(x-\delta U) + \xi^-)}{2\delta}\right),
%  \label{eq:twosp}
% \end{align}
% where $\xi^+, \xi^-$, $V$, $U$ are as in Proposition \ref{prop:grad-spsa}.
% Then, we have that $\gamma$ is a type-I oracle with $c_1(\delta) = C_1 \delta$ and $c_2(\delta) = C_2 d/\delta^2$.
% \end{proposition}
% \begin{proof}
% See Appendix \ref{sec:appendix-grad}.
% \end{proof}


% A popular idea for estimating gradient using one-point feedback is the smoothed function approach, which was originally proposed in \citep{katkul}. The idea is to convolve the gradient of the objective function with a suitable density function and then, via an integration by parts arguments show that the resulting integral is an estimate of the gradient of the smoothed objective function.
% This approach has been adopted in a stochastic convex optimization setup in \cite{duchi2015optimal}.

%% \paragraph{Simultaneous perturbation methods:}
%The idea of simultaneous perturbation can work even for functions that are not differentiable, as shown in \cite{flaxman2005online}.
%% follow this approach in the context of bandit convex optimization.
%Formally,
%% a $(c_1,c_2)$ Type-II oracle when $\cF$ is a general class of functions (this does not even require differentiability).
%given any $f \in \cF$, $x \in \cK$, and $\delta >0$, the oracle returns\footnote{See also \cite[pp.~58-60]{kushcla} for an old reference that proposed a gradient estimate similar to \eqref{eq:flaxman}.}
%\begin{align}
% Y = x+\delta u \in \cK', \quad
% G = \dfrac{d}{\delta}f(x+\delta u)u \in \R^d, \label{eq:flaxman}
%\end{align}
%where $u\in \R^d$ is a random unit vector, so the first condition of \cref{def:oracle2} immediately follows.
%The variance of $G$ is bounded by $d^2C^2 \delta^{-2}$ for some constant $C = \sup_{y\in \cK'}f(y)$.
%\todox[inline]{Fix this proposition statement to say Flaxman scheme is a type II oracle}
%\begin{proposition}
%Let $\tilde{f}(x) = \EE{f(x+\delta v)}$ denote the smoothed version of $f$. Then, we have
%$\EE{G} = \nabla \tilde{f}(x)$.
%\end{proposition}
%\begin{proof}
% See Lemma 1 in \citep{flaxman2005online}.
%\end{proof}
% As to the bias condition, it was proved that $\EE{G} = \nabla \tilde{f}(x)$, where $\tilde{f}$ is a smoothed version of $f$, i.e.,
%$\tilde{f}(x) = \EE{f(x+\delta v)}$,
%$v$ is a random vector in a unit ball. There are different ways to bound the bias depending on the property of $f$.
%If $f$ is $L_{lip}$-Lipschitz over $\cK'$, then we have
%\begin{align*}
%\MoveEqLeft
%\norm{\tilde{f}(x)-f(x)}_\infty \\
%=&\norm{\EE{f(x+\delta v)-f(x)}}_\infty
%\le L_{lip} \delta \,.
%\end{align*}
%If $f$ is convex, and $L_{smo}$-smooth,
%\begin{align*}
%\MoveEqLeft
%\norm{\tilde{f}(x)-f(x)}_\infty \\
%\le& \norm{\EE{\ip{\nabla f(x), \delta v}+\dfrac{L_{smo}}{2}\delta^2\norm{v}^2}}_\infty\\
%\le &\dfrac{L_{smo}}{2}\delta^2 \,.
%\end{align*}
%Therefore, the estimator \eqref{eq:flaxman} can always fit the oracle setting by choosing $c_1(\delta) = C_1 \delta$ (or $C_1\delta^2$), $c_2(\delta) = C_2 \delta^{-2}$, for some constant $C_1$, $C_2$.

% \begin{remark}
%  Instead of picking $u$ randomly on the surface of a unit sphere, one can employ an random variable $u$ that satisfies $E[u u\tr] = I_d$, where $I_d$ is the $d$-dimensional identity matrix. A popular choice for $u$ that satisfies the aforementioned constraint is the $d$-dimensional standard Gaussian - a choice that has been explored in the context of zeroth order optimization in \citep{duchi2015optimal}. See \citep{bhatnagar-book} for an overview of gradient and Hessian estimation techniques using random perturbations.
% \end{remark}

It may be noticed that the function domain $\D$ can be larger than or equal to the set $\K$, where the algorithm chooses $x$. This is to ensure that the oracle will not receive invalid inputs, i.e., queries where $f$ is not defined.
When the functions are defined over $\K$ only and $\K$ is bounded, the above constructions only work for $\delta$ small enough.
In this case, the best approach perhaps is to use Dikin ellipsoids to construct the oracles, as done by \citet{HaLe14:SOC}.

\paragraph{Results for stochastic BCO}
We know consider stochastic BCO with $L$-smooth functions over a convex, closed non-empty domain $\K$. \todoc{Check that $K$ is always assumed to be close. Otherwise the optimum may not belong to $K$.}
Let $\cF$ denote the set of these functions.
\cite{duchi2015optimal} proves that minimax expected optimization error
for the functions $\cF$ with uncontrolled noise is lower bounded by $\Omega(n^{-1/2})$. \todoc{We should sort out dimension dependences later.}
They also give an algorithm which uses two-point gradient estimates which matches this lower bound for the case of \emph{controlled noise}.
For controlled noise, the previous section's construction give that for two-point estimators $c_1(\delta) = C_1 \delta^p$ and $c_2(\delta) = C_2\delta^{-q}$ with $p=1$ and $q=0$. Plugging this into
\cref{thm:ub} we get the rate $O(n^{-1/2})$ (which is unsurprising
given that the algorithms and the upper bound proof techniques are essentially the same as that of \cite{duchi2015optimal}).
However, when the noise is uncontrolled, the best that we get is $p=1$ and $q=2$ (if, in addition, $f\in \C^3$, then we get $p=2$, $q=2$).
From \cref{thm:lb-convex} we get that with such oracles, no algorithm can get better rate than $\Omega(n^{-1/4})$ (resp., $\Omega(n^{-1/3})$), while from
\cref{thm:ub} we get that these rates are matched by mirror descent.
We can summarize these findings as follows:
\todoc{Can we show that no better oracles exist?}
\begin{theorem}\label{thm:aaab}
Let $\cF$ be the space of convex, $L$-smooth functions over a convex, closed non-empty domain $\K$.
Then, we have that\\
\textit{\textbf{Uncontrolled noise}}:
Take any $(\delta^1,\delta^{-2})$ type-I oracle $\gamma$.
There exists an algorithm that uses $\gamma$
and achieves the rate $O(n^{1/4})$.
Furthermore, no algorithm using $\gamma$
 can achieve better error than $\Omega(n^{-1/4})$.\\
\textit{\textbf{Uncontrolled noise}, $f\in \C^3$}:
Take any $(\delta^2,\delta^{-2})$ type-I oracle $\gamma$.
There exists an algorithm that uses $\gamma$
and achieves the rate $O(n^{1/3})$.
Furthermore, no algorithm using $\gamma$
 can achieve better error than $\Omega(n^{-1/3})$.\\
\textit{\textbf{Controlled noise}}:
Take any $(\delta,1)$ type-I oracle $\gamma$.
There exists an algorithm that uses $\gamma$ an
achieves the rate $O(n^{-1/2})$.
Furthermore, no algorithm using $\gamma$
 can achieve better error than $\Omega(n^{-1/2})$.
\end{theorem}

For stochastic BCO with uncontrolled noise, \cite{AgFoHsuKaRa13:SIAM} analyse a variant of the well-known ellipsoid method and provide regret bounds for the case of convex, $1$-Lipschitz and bounded functions on $[0,1]$. Their regret bound implies a minimax error \eqref{eq:minimaxerrdef} bound of the order  $O\left(\sqrt{d^{32}/n}\right)$.
%\todoc{Not for this setting! This paper should be mentioned in the previous section actually. For this setting, we  do not have \emph{any} algorithms that would achieve $O(n^{1/2})$.}
\cite{liang2014zeroth} provide an algorithm based on random walks (and not using gradient estimates) for the setting of convex, bounded functions whose domain is contained in the unit cube and their algorithm results in a bound of the order $\O\left((d^{14}/n)^{1/2}\right)$ for the minimax error.
%\todoc{What are the conditions for their theorem? Boundedness? Lipschitzness? Or nothing?}
%This algorithm achieves an upper bound of $\O\left((d^{14}/n)^{1/2}\right)$,
These bounds decrease faster in $n$ than the bound available in Theorem \ref{thm:aaab}, while showing a much worse dependence on the dimension.\todoc{We should show the dimension dependence in the result to make this point. This should be shown for the same setting as the one considered by \cite{liang2014zeroth}.}
However, what is more interesting is that our results also shows that an $O(n^{-1/2})$ upper bound \emph{cannot} be achieved using the gradient oracles available for this setting, no matter what algorithm would be used used to interact with these oracles.

The above result also shows that the gradient oracle based algorithms are optimal for smooth problems, under a controlled noise setting.
While \cite{duchi2015optimal} suggests that it is the power of two-point gradient estimators that help to achieve this, we need to add that a critical condition to achieve the optimal rate is that the noise must be controlled. For the case of uncontrolled noise, the best known upper bound for convex+smooth functions is $O(n^{-1/3}$ and this is due to \cite{saha2011improved}.

Finally, let us make some remarks on the early literature on this problem.
A finite time lower bound for stochastic, smooth BCO is presented by  \citet{Chen88:LB-AoS} for
convex functions on the real line.
While the lower bound is stated for $r$-smooth functions with $r$ odd (these are functions $f$ with $\norm{f^{(r)}}_{\infty}\le L$), and $r$ greater than one, careful checking of the results show that nor $r>1$, neither that $r$ must be odd is ever used in the proof.
Hence, the result holds for $r\ge 2$.
The lower bound presented for a given value of $r$ takes the form $\Omega( (1/n)^{ (r-1)/(2r)} )$. For $r=2$ (which approximately corresponds to the smooth case), we get $\Omega((1/n)^{1/4})$, while for $r=3$ (which is ``almost'' the same as $f\in \C^3$), one gets $\Omega((1/n)^{1/3})$.
While these match the bounds in \cref{thm:aaab}, they are larger than the error achieved by
the algorithm of \cite{liang2014zeroth}. The resolution of the apparent contradiction is that the lower bounds of \citet{Chen88:LB-AoS} concern distance to the optimum (i.e., $\EE{ \norm{\hat{X}_n - x^* }}$), while the optimization error is defined in terms of the objective function gap $\EE{ f(\hat{X}_n) -f( x^*)}$.
Similar results are obtained by \citet{PoTsy90}, who also consider distance to the optimum and proves that mirror descent with gradient estimation achieves asymptotic optimal rates for these settings.

\todoc{Cite all the other papers, summarizing what they achieve. }

%%% Local Variables:
%%% mode: latex
%%% TeX-master: "bgo"
%%% End:


\section{Applications to Online BCO}
\label{sec:obco}
%!TEX root =  bgo-cam-ready.tex
% please do not delete or change the first line (needed by Csaba's editor)
In the \emph{online BCO} setting a learner sequentially chooses the points $X_1,\dots,X_n\in \cK$ while observing the losses $f_1(X_1),\dots,f_n(X_n)$. More specifically, in round $t$, having observed $f_1(X_1),\dots,f_{t-1}(X_{t-1})$ of the previous rounds, the learner chooses $X_t\in \cK$, after which it observes $f_t(X_t)$. The learner's goal is to minimize its expected regret $\EE{ \sum_{t=1}^n f_t(X_t) - \inf_{x\in \cK} \sum_{t=1}^n f_t(x) }$. 
This problem is also called online convex optimization with one-point feedback.
A slightly different problem is obtained if we allow the learner to choose multiple points in every round, at which points the function $f_t$ is observed. The loss is suffered at $X_t$. The points where the function is observed (``observation points'' for short) may or may not be tied to $X_t$. One possibility is that $X_t$ is one of the observation points.  
Another possibility is that $X_t$ is the average of the observation points (e.g., \citet{AgDeXi10}). Yet another possibility is that there is no relationship between them. 

The oracle constructions from the previous section also apply to the online BCO setting
where the algorithm is evaluated at $Y_t$, though in this case 
one cannot employ two-point feedback as the functions change between rounds. 
This also rules out the controlled noise case. 
Thus, for the online BCO setting, one should consider type-I (and II) oracles with $c_1(\delta) = C_1 \delta^p$ and $c_2(\delta) = C_2\delta^{-q}$ with $p=q=2$.
For these type of oracles, the results from \cref{thm:lb-convex} give the following result: 
\begin{theorem}\label{thm:aaa}
Let $\cF_{L,0}$ be the space of convex, $L$-smooth functions over a convex body $\K$.
No algorithm that relies on 
 $(\delta^2,\delta^{-2})$ type-I oracles
 can achieve better regret than $\Omega(n^{2/3})$.
\end{theorem}
%\vspace{-0.2cm}
With a noisy gradient oracle of \cref{prop:flaxman}, \cref{thm:aaa} implies that this regret rate is achievable, essentially recovering, and in some sense proving optimality of the result of \citet{saha2011improved}:
\begin{theorem}
For zeroth order noisy optimization with smooth convex functions, the gradient estimator of \cref{prop:flaxman} together with mirror descent (see \cref{alg}) achieve $\O(n^{2/3})$ regret.
\end{theorem}
%\vspace{-0.2cm}
This ``optimality result'' shows that with the usual analysis of the current gradient estimation techniques, no gradient method can achieve the optimal regret $O(n^{1/2})$ for online bandit convex optimization, established by \citet{BubeckDKP15,BuEl15}. Note that \cref{thm:aaa} contradicts the recent result of \citet{DeElKo15}, who claimed to achieve $\tilde{O}(n^{5/8})$ regret with the same $(\delta^2,\delta^{-2})$ type-II gradient oracle as \citet{saha2011improved}, but their proof only used the $(\delta^2,\delta^{-2})$ tradeoff in the bias and variance properties of the oracle.\todoa{Actually, all these proofs use Type-IIb oracles....}





%%% Local Variables:
%%% mode: latex
%%% TeX-master: "bgo"
%%% End:


\section{Related Work}
\label{sec:related}
%!TEX root =  bgo.tex
% please do not delete or change the first line (needed by Csaba's editor)
Gradient oracle models have been studied in a number of previous papers 
\citep{dAsp08,Baes09,SchRoBa11,DeGliNe14}.
%assume that set $\K$ is bounded and that the oracle provides at each point $x\in \K$ an approximate gradient $g$ satisfying condition
%$|\ip{g-\nabla f(x), v-w}| \le \delta$ for all $v,w,x\in \K$.
A full comparison between these oracle models is given by \cite{DeGliNe14}.
For illustration, here we only review the model of this latter paper as a typical example of these previous works.
The model of \cite{DeGliNe14} assumes a first-order approximation to the function
with parameters $(\delta,L)$. In particular, 
given $(x,\delta,L)$ and the convex function $f$, 
the oracle gives a pair $(t,g)\in \R \times \R^d$
such that $t + \ip{g,\cdot-x}$ is a linear lower approximation to $f(\cdot)$ in the sense that 
$0\le f(y) - \left\{ t+ \ip{g,y-x}\right\} \le \frac{L}{2} \norm{y-x}^2 + \delta$.
\cite{DeGliNe14} argue that this notion appears naturally in several optimization problems and study whether the so-called accelerated gradient techniques are still superior to their non-accelerated counterparts (and find a negative answer).
The authors study both lower and upper rates of convergence, similarly to our paper.

%, except that in the lower bounds the algorithm is not allowed to control the oracle parameters.

A major difference between the previous and our settings is that we allow stochastic noise (and bias), which the algorithms can control, while the oracle in these previous paper must guarantee that the accuracy requirements hold in each time step
with probability one.
This is a much stronger requirement, which may be impossible to satisfy in some problems, such as when 
the only information available about the functions is noise contaminated.
Some works, such as \citet{SchRoBa11} allow arbitrary sequences of errors and show error bounds as a function
of the accumulated errors. 
Our proof technique is actually essentially the same (as can be expected).
However, the noisy case requires special care. For example, Proposition~3 of
\citet{SchRoBa11}  bounds the optimization error for the smooth, convex case by 
$O(1/n^2 ( \norm{x_1-x^*}^2 + A_n^2 )$ where $A_n = O( \sum_{t=1}^n t \norm{e_t})$, $e_t$ being the error of the approximate gradient. This expression is upper and lower bounded, up to a constant factor by
 $\frac{1}{n^2} \sum_{t=1}^n t^2 \approx n$,
assuming that errors' noise level is a positive constant (in all our result, this holds).
This clearly shows the differences between the analysis and that the noisy case requires (somewhat) special treatment.

Nevertheless, we do not claim to invent methods for proving upper bounds for the stochastic case
as these methods have been developed for special cases for a long time by now. \todoc{Add references.}
Our main contribution here lies in abstracting away the properties of gradient estimation procedures 
to create our new oracle model, in which we are able to study not only upper rates, but also lower rates of convergence.
A similar, but simpler oracle model building on the inexact oracle model of  \cite{DeGliNe14}, 
but one which lacks the bias-variance tradeoff central to this paper (i.e., they assume the variance and bias can be controlled independently of each other) appeared recently in a conference program \citep{DvoGa15}. The results in this paper are upper bounds on the error of certain gradient methods and they correspond to the bounds we obtained with $q=0$.
\todoc[inline]{The paper of \citet{SchRoBa11} has 100 citations on google scholar.
The paper of \cite{DeGliNe14} has similarly many.
There are probably more relevant works. Someone could take a quick look and perhaps add a few more relevant ones.
}





%\section{PROOFS}
%\label{sec:proofs}
%%%!TEX root =  bgo.tex
% please do not delete or change the first line (needed by Csaba's editor)
\subsection{Proofs for Upper Bounds}
\label{sec:ub-proof}

Before giving the proof, we introduce a useful lemma, which is essentially Theorem~C.4 of \cite{MahdaviPhd:2014}.
\begin{lemma}
\label{lem:ub}
Let $(\cF_t)_{t}$ be a filtration such that $X_t$ is $\cF_t$-measurable.
Let $\overline G_t = \EE{G_t|\cF_t}$ 
and assume that the nonnegative real-valued deterministic sequence $(\beta_t)_{1\le t\le n}$ is such that 
$\norm{\overline G_t - \nabla f(X_t)}_* \le \beta_t$ holds almost surely. 
Further, assume that $D=\sup_{x,y\in \cK} D_{\mathcal{R}}(x,y)$ and let $\eta_t = \frac{\alpha}{a_t+L}$ for some increasing 
sequence $(a_t)_{t=1}^{n-1}$ of numbers. Then, 
\begin{align*}
\MoveEqLeft \EE{ \sum_{t=1}^n f(X_t) - f(x) }  \\
\le& 	 \EE{f(X_1)-f(x)}+ \\
 & \sqrt{\tfrac{2D}{\alpha}} \sum_{t=1}^{n-1} \beta_t 
 +\frac{D(a_{n-1}+L)}{\alpha} +
	  \sum_{t=1}^{n-1}\frac{\sigma_t^2}{2a_t}\,,
\end{align*}
where $\sigma_t^2 = \EE{ \norm{G_t-\overline G_t}_*^2}$ is the ``variance'' of $G_t$.
\end{lemma}
This is also identical to Theorem~6.3 of \cite{Bu:Convex14}, who cites \cite{Dekel:minibatch12} as the source. These results assume that $\beta_t=0$ above.
The proof of this lemma can be found in \cref{sec:appendix}.

Proof of Theorem \ref{thm:ub}:

Since $f$ is convex, by Jensen's inequality,
\[
 \E[f(\hat X_n) - \inf_{x \in \cK} f(x)] \le  \dfrac{1}{n}\EE{ \sum_{t=1}^n f(X_t) - \inf_{x \in \cK}f(x) } \,.
\]
Then the result immediately follows from Lemma \ref{lem:ub}, by replacing
 $\beta_t = C_1\delta^p$, $\sigma^2_t = C_2 \delta^{-q}$ and $a_t$, respectively.
%%!TEX root =  bgo.tex
% please do not delete or change the first line (needed by Csaba's editor)
\subsection{Proof of Theorem  \ref{thm:lb-convex}}
\label{sec:lb-proof}
We provide a sketch of the proof below for the case of convex and smooth function class $\F_{L,0}(\K)$. The detailed proof for $\F_{L,0}(\K)$ as well as $\F_{L,1}(\K)$ is available in Appendix \ref{sec:appendix-lb-proof}.
%%%%%%%%%%%%%%%%%%%%%%%%%%%%%%%%%%%%%%%%%%%%%%%%%%%%%%%%%%%%%%%%%%%%%%%%%%%%%%% 
\paragraph{Proof in one dimension}
% For brevity, let $\Delta_n^{(1)*}$ denote the minimax error $\Delta_n^*(\F, c_1,c_2)$ for the case of $1$-dimensional function family $\F$.
% The proof uses only $(c_1,c_2)$ Type-I oracles $\gamma$ that map $\cK$ to $\R$ (i.e., the oracles do not have memory).
% Fix a  $(c_1,c_2)$ Type-I oracle $\gamma$ and an algorithm $\A$.
% We restrict the class of oracles to those that return a random gradient estimate $G = m(x) + \xi$ with some map $m: \cK \to \R$,
% where $\xi$ is standard normal with variance $\sigma^2 = C_2 \delta^{-q}$, satisfying the variance requirement. 
% The map $m$, which, by slightly abusing notation, we will also denote by $\gamma$ in what follows, will be chosen based on $f$ to satisfy the requirement on the bias. The $Y$ value returned by the oracles is made equal to $x$.

% Let $\delta$ denote the tolerance parameter chosen by $\A$.
% Define the probability space $(\Omega, \B, P_{\A,\gamma})$, 
% where $\Omega = \R^n\times \{-1,1\}$, $\B$ is the associated Borel sigma algebra. 
% Further, the probability measure $P_{\A,\gamma} := p_{\A,\gamma} d(\lambda \times m)$, 
% 	$\lambda$ is the Lebesgue measure on $\R^n$, 
% 	$m$ is the counting measure on $\{-1,1\}$ and 
% 	$p_{\A,\gamma}$ is the density function defined as
% \begin{align*}
% &p_{\A,\gamma}(g_{1:n}, v) = \dfrac{1}{2} \bigg(p_{\A,\gamma}(g_n \mid g_{1:n-1})\times \dots \\
% &\times p_{\A,\gamma }(g_{n-1} \mid g_{1:n-2}) \ldots p_{\A,\gamma}(g_1)\bigg)\\
%  = & \dfrac{1}{2} \bigg( p_{\N}(g_n - \gamma(\A_n(g_{1:n-1}))) \cdot
% %  &\times p_{\normal(0,\sigma^2)}(g_{n-1} - \gamma(\A_n(g_{1:n-2}))) \times \ldots \\
%  \ldots \cdot  p_{\N}(g_1 - \gamma(\A_1))\bigg),
% \end{align*}
% where $v\in\{0,1\}$, $p_{\N}$ is the density of a $\normal(0,\sigma^2)$ random variable,
% and $\cA_t$ denotes the map from the algorithm's past observations
% that picks the point that is sent to the oracle in round $t$.
% \todoc{Enough to consider deterministic algorithms}

%%%%%%%%%%%%%%%%%%%%%%%%%%%%%%%%%%%%%%%%%%%%%%%%%%%%%%%%%%%%%%%%%%%%%%%%%%%%%%% 
% We first prove the theorem in a one-dimensional setting and generalize it later.
Define the function class $\F_{L,0}$ to be comprised of $f_v$ for $v \in \{-1,+1\}$, where
% By assumption,  $\{f_+, f_-\}\subset \F$, with 
\begin{align*}
  f_v(x) := \dfrac{\epsilon}{2} (x - v)^2, \,\, x \in \cK \subset \R\,.
\end{align*}
% we will slightly abuse notation by using $f_+$ ($f_-$) in place of $f_{+1}$ (resp., $f_{-1}$).
%%%%%%%%%%%%%%%%%%%%%%%%%%%%%%%%%%%%%%%%%%%%%%%%%%%%%%%%%%%%%%%%%%%%%%%%%%%%%%% 
Clearly, $f_v$ is minimized at $x^*_v = v$ with the minimum value being zero.
Hence,
%Using the fact that $f_+$ and $f_-$ are strongly convex with associated constant $\left(\dfrac{\epsilon}{2}\right)$, we obtain
\begin{align}
  f_v(x) - f_v(x^*_v)
  = &  \dfrac{\epsilon}{2} (x - v)^2 \ge  \dfrac{\epsilon}{2}  \indic{x v  < 0}. \label{eq:main:fv-lb-main}
\end{align}
% Similarly,   $f_-(x) - f_-(x^*_-) \ge  \dfrac{\epsilon}{2}  \indic{x  >0}$.
We will consider the oracles $\gamma_v$ defined as 
\begin{align}
 \gamma_v(x) = \epsilon(x-v) - v\, \min(\epsilon,C_1 \delta^p) + \xi, \label{eq:main:oracle-1d}
\end{align}
where $\xi \sim \normal(0,\frac{C_2}{\delta^q})$.
% ; as with $f_v$, we will also use $\gamma_{+}$ ($\gamma_-$)  to denote $\gamma_{+1}$ (resp., $\gamma_{-1}$).
The oracle is indeed a $(c_1,c_2)$ Type-I oracle, with $c_1(\delta)=C_1\delta^p$ and $c_2(\delta)=\frac{C_2}{\delta^q}$.
%%%%%%%%%%%%%%%%%%%%%%%%%%%%%%%%%%%%%%%%%%%%%%%%%%%%%%%%%%%%%%%%%%%%%%%%%%%%%%% 

Using \eqref{eq:main:fv-lb-main}, the minimax error \eqref{eq:minimaxerrdef} can be lower bounded as follows
%\footnote{$f_{+1} \equiv f_+$ and $f_{-1}\equiv f_-$.}:
\begin{align*}
\MoveEqLeft 
\Delta_n^*(\F_{L,0}, C_1 \delta^p ,C_2 \delta^{-q}) %\nonumber\\
  \ge  \inf_{\A} \,  \E[f_V(\hat X_n) - \inf_{x \in X}
  f_V(x)],
  \end{align*}
where the expectation is w.r.t. $\P:= \dfrac{1}{2} \left(P_{\A, \gamma_+} \indic{v=+1} + P_{\A, \gamma_-}\indic{v=-1}\right)$, with $P_{\A,\gamma_v}$ denoting the joint distribution of the $n$ observations $g_{1:n}$ of the algorithm $\A$, for $v \in \{-1,+1\}$
%Here $\gamma_v$  is the oracle for $f_v$, for $v=+1,-1$ and is defined by $\gamma_v(x) = \epsilon(x-v) + v C_1 \delta^p$. 
Using \eqref{eq:main:fv-lb-main}, we obtain
\begin{align}
&\Delta_n^*(\F_{L,0}, C_1 \delta^p ,C_2 \delta^{-q})  \ge  \inf_{\A} \dfrac{\epsilon}{2}\,  \P(\hat X_n V < 0), \label{eq:main:strong-convex-bd}\\
  = & \inf_{\A} \dfrac{\epsilon }{2} \, \left(\P_{+}(\hat X_n < 0) + \P_{-}(\hat X_n > 0)\right), \label{eq:main:Pplus}\\
  \ge &\inf_{\A} \dfrac{\epsilon }{2} \,\left(1 - \tvnorm{\P_{+}- \P_{-}}\right), \label{eq:main:lecam}\\
  \ge &\inf_{\A} \dfrac{\epsilon }{2}  \,\left( 1 - \left(\frac12\dkl{P_{+}}{P_{-}}\right)^{\frac{1}{2}}\right), \label{eq:main:pinsker-main}
\end{align}
where $\P_{+}$ (resp. $P_-$) is the distribution $\P$ conditioned on $v=+1$ (resp. $v=-1$).

Observe that, for any $x\in \R$, $f_+'(x) - f_-'(x) = 2\epsilon$ and hence
\begin{align}
& |\gamma_+(x) - \gamma_-(x)| \nonumber\\
& = | f'_+(x) - \min(\epsilon,C_1 \delta^p) - (f'_-(x)+\min(\epsilon,C_1 \delta^p)) | \nonumber \\
& = 2 (\epsilon - C_1 \delta^p)_+\,,
 \label{eq:main:gdiff-ub}
\end{align}
where $(\epsilon - C_1 \delta^p)_+ = \max(\epsilon - C_1 \delta^p,0)$. 
From the foregoing, it can be shown that 
\begin{align}
\dkl{P_{+}}{P_{-}} \le \dfrac{2n(\epsilon-C_1\delta^p)_+^2 \delta^q}{C_2}.\label{eq:main:dkbd-main}
\end{align}
Note that the above bound holds uniformly over all algorithms $\A$. 
Substituting the above bound into \eqref{eq:main:pinsker-main}, we obtain 
\begin{align*}
 \Delta_n^*(\F_{L,0}, C_1 \delta^p ,\frac{C_2}{\delta^{q}})
  \ge & \frac{\epsilon}{2} \left(1 - \sqrt{
    n}  \frac{(\epsilon-C_1\delta^p)_+\delta^{q/2}}{C_2}
  \right).\label{eq:main:final-lower-bd}
\end{align*}


%%%%%%%%%%%%%%%%%%%%%%%%%%%%%%%%%%%%%%%%%%%%%%%%%%%%%%%%%%%%%%%%%%%%%%%%%%%%%%%
%%%%%%%%%%%%%%%%%%%%%%%%%%%%%%%%%%%%%%%%%%%%%%%%%%%%%%%%%%%%%%%%%%%%%%%%%%%%%%%
%%%%%%%%%%%%%%%%%%%%%%%%%%%%%%%%%%%%%%%%%%%%%%%%%%%%%%%%%%%%%%%%%%%%%%%%%%%%%%%
\paragraph{Generalization to $d$ dimensions:}
Let $\F_{L,0}$ be comprised of $f_v$ defined by: For any $v\in \{-1,+1\}^d$
\begin{align*}
  f_v(x) :=& \sum_{i=1}^d f^{(i)}_{v_i}(x_i), \text{ where } f^{(i)}_{v_i}(x_i) := \dfrac{\epsilon}{2} (x_i - v_i)^2,
\end{align*}
for $i=1,\ldots,d$.
Define
$\gamma_v(x) = \sum_{i=1}^d \gamma_{v_i}^{(i)}e_i$, where 
$e_i$ is the $i$th unit vector in the standard Euclidean base,
$\gamma_{v_i}^{(i)} = \epsilon(x_i-v_i) - v_i\, \frac{C_1 \delta^p}{\sqrt{d}} + \xi_i$, where $\xi_i \sim \normal(0,\frac{C_2}{d\delta^q})$. 
% It is evident that the $\gamma$ is a $(c_1,c_2)$ Type-I oracle, with $c_1(\delta)=C_1\delta^p$ and $c_2(\delta)=\frac{C_2}{\delta^q}$.

Since $f_v$ is additively separable, we bound the $d$-dimensional minimax error using the $1$-dimensional ones as follows
% Let $\Delta_n^{(d)*}$ denote the minimax error $\Delta_n^*\left(\F_d, C_1\delta^p,\frac{C_2}{\delta^q}\right)$ for the $d$-dimensional family of functions $\F_d$, which is composed of $f_v$s, as defined above. We have
% \begin{align*}
%  \Delta_n^{(d)*} \ge& \inf_\A  \sup_{v\in \{\pm 1 \}^d}\sum_{i=1}^d L_i(v), \text{ where, for } i=1,\ldots,d,\\
%  L_i(v) := & \int f^{(i)}_{v_i}(\hat X_{ni}(g_{1:n}) )\, dP_{\A,\gamma_v,\sigma^2}(g_{1:n}),  
% \end{align*}
% where $\hat{X}_{ni}$ denotes the $i$th coordinate of the vector $\hat{X}_n \in \R^d$ chosen by $\A$, for $i=1,\ldots,d$
% (again, to simplify the notation the dependence of $\hat{X}_n$ on $\A$ is suppressed).
% 
% 
% % Fix an $i$ and consider the loss $L_i(v)$.
% % We now argue that, since $f_v$ is separable, any algorithm $\A$ that uses information in dimensions other than $i$ in the gradient samples of the oracle, can be replaced by $d$ algorithms $\A_1^*$, $\dots$, $\A_d^*$ such that $\A_i^*$ is only using information for dimension $i$ and is only predicting the $i$th coordinate at the end, and the sum of losses they incur is utmost that of $\A$ in the worst-case.
% 
% \begin{lemma}
% For any algorithm $\A$, there exist  $d$ ``$1$-dimensional'' algorithms, $\A_i^*$, $1\le i \le d$,
% the $i$th algorithm interacting with only a one dimensional oracle, supplying gradient estimates about $f_{v_i}^{(i)}$ 
% such that if $L^{\A}(v)$ is the loss of $\A$ on environment with index $v$ and the following holds
% the 
% \begin{align}
% \label{eq:main:onedimlb}
% % \begin{split}
% % \MoveEqLeft 
% &\hspace{-1em}\max_{v\in \{\pm 1\}^d} L^{\A}(v) 
% \ge \nonumber\\
% &  \max_{v_1\in \{\pm 1\}} L_1^{\A_1^*}(v_1) + \dots + \max_{v_d\in \{\pm 1 \}} L_d^{\A^*_d}(v_d)\,.
% % \end{split}
% \end{align}
% \end{lemma}
% Using the above lemma, the minimax error $\Delta_n^{(d)*}$ can be lower bounded as follows:
\begin{align*}
\Delta_n^*(\F_{L,0}, C_1 \delta^p ,C_2 \delta^{-q}) \ge d \Delta_n^{*}\left(\F_1,\frac{C_1\delta^p}{\sqrt d}, \frac{C_2}{d\delta^q}\right)\,. 
\end{align*}
See Lemma \ref{lemma:sep} in Appendix \ref{sec:appendix-lb-proof} for a rigorous justification of the above inequality. 
The final claim follows by plugging the minimax error bounds in one dimension and the reader is referred to Appendix \ref{sec:appendix-lb-proof} for details.

% where in inequality \eqref{eq:main:splitA}, algorithms $\A_i$ are interacting with the respective one-dimensional oracles $\gamma^{(i)}_{v_i}$ only 
% and the inequality follows by~\eqref{eq:main:onedimlb},
% while the inequality \eqref{eq:main:dlb} follows from the fact that each infimum term in \eqref{eq:main:splitA} is solving a $1$-dimensional problem and hence the bound in \eqref{eq:main:final-lower-bd} applies with $C_1$ replaced by $C_1/\sqrt{d}$ and $C_2$ replaced by $C_2/d$. The resulting bound is denoted by $\Delta_n^{*}\left(\F_1,\frac{C_1\delta^p}{\sqrt d}, \frac{C_2}{d\delta^q}\right)$. 
% is the one-dimensional minimax error, which is lower-bounded in \eqref{eq:main:final-lower-bd}. \todoc{However, I would have written just the $1$d final bound there, or $\Delta_n^*( \dots )$ with some notation, because we don't want to redo this whole optimization business..}

% Using the one-dimensional minimax error $\Delta_n^{(1)*}$, we obtain
% \[\Delta_n^* \ge d \Delta_n^{(1)*}.\]

%%% Local Variables: 
%%% mode: latex
%%% TeX-master: "bgo"
%%% End: 

% \paragraph{Proof of Theorem \ref{thm:lb-strongly-convex}:}
% 
% \todoc{This proof sketch can be removed if we are short of space}
% % This follows in the same manner as the proof of Theorem \ref{thm:lb-convex}.
% First, lower bound \\$f_v:=\dfrac{1}{2} x^2 - v x,$ as follows: \\$f_v(x) - f_v(x^*_v)
% \ge  \dfrac{\nu^2}{2}  \indic{x v  < 0}.$\\
% Next, use the following oracles:
% \begin{align}
%  \gamma_v(x) = \epsilon(x-v) - \sign(v)\, \min(v,C_1 \delta^p) + \xi, \label{eq:main:oracle-1d}
% \end{align}
% where $\xi \sim \normal(0,\frac{C_2}{\delta^q})$.
% %%%%%%%%%%%%%%%%%%%%%%%%%%%%%%%%%%%%%%%%%%%%%%%%%%%%%%%%%%%%%%%%%%%%%%%%%%%%%%% 
% The KL-divergence bound (corresponding to \eqref{eq:main:dkbd-main}) turns out to be 
% \begin{align}
% \dkl{P_{+}}{P_{-}} \le \dfrac{2n(\nu-C_1\delta^p)_+^2 \delta^q}{C_2},\label{eq:main:dkbd-sc-main}
% \end{align}
% which implies the following bound:
% \begin{align}
%  \Delta_n^{(1)*}
%   \ge & \dfrac{\nu^2}{2} \left(1 - \sqrt{
%     n}  \dfrac{(\nu-C_1\delta^p)_+\delta^{q/2}}{C_2}
%   \right)\label{eq:main:final-lower-bd-sc}
% \end{align}
% The generalization to $d$-dimensions follows using a separability argument as before.


%%%%%%%%%%%%%%%%%%%%%%%%%%%%%%%%%%%%%%%%%%%%%%%%%%%%%%%%%%%%%%%%%%%%%%%%%%%%%%%
%%%%%%%%%%%%%%%%%%%%%%%%%%%%%%%%%%%%%%%%%%%%%%%%%%%%%%%%%%%%%%%%%%%%%%%%%%%%%%%


%%%!TEX root =  bgo.tex
% please do not delete or change the first line (needed by Csaba's editor)
\subsection{Reductions and Proofs}
\label{sec:orrel}

\section{Proof of the Upper Bounds}
\label{sec:appendix-md}
%!TEX root =  bgo.tex
% please do not delete or change the first line (needed by Csaba's editor)

Before the proof, we introduce a well known bound on the instantaneous linearized regret of Mirror Descent. 
\begin{lemma}
\label{lem:mdlinregret}
For any $x \in \cK$ and any $t \ge 1$,
\begin{align*}
\MoveEqLeft
\ip{G_t, X_{t+1}-x} 
\le \dfrac{1}{\eta_t} \left( \DR(x,X_t)-\DR(x,X_{t+1})-\DR(X_{t+1},X_t) \right)\,,
\end{align*}
where $X_{t+1}$ is selected as in \cref{alg}.
\end{lemma}
\begin{proof}
The point $X_{t+1}$ is the minimizer of 
$\Psi_{t+1}(x)=\eta_t \ip{G_t,x}+\DR(x,X_t)$ over $\cK$. The gradient of $\Psi_{t+1}(x)$ is
\[
\nabla \Psi_{t+1}(x) = \eta_t G_t + \nabla\mathcal{R}(x)-\nabla\mathcal{R}(X_t)\,.
\]
By the optimality condition, for any $x \in \cK$,
\[
\ip{\eta_t G_t + \nabla\mathcal{R}(x)-\nabla\mathcal{R}(X_t), x-X_{t+1}}\ge 0 \,,
\]
which is equivalent to the result by substituting the definition of Bregman divergence $\DR$.
\end{proof}

Proof of Lemma \ref{lem:ub}:

Denote $\hat{G}_t = \nabla f(X_t)$. 
From the smoothness and convexity of $f$ and using that $\mathcal{R}$ is strongly convex, we have
\begin{align}
\MoveEqLeft
f(X_{t+1}) - f(x) \nonumber\\
 \le& f(X_t) + \ip{ \hat{G}_t, X_{t+1}-X_t } + \frac{L}{2} \norm{X_{t+1}-X_t}^2 - \left\{ f(X_t) + \ip{ \hat{G}_t, x-X_t} \right\} \nonumber \\
 =& \ip{\hat{G}_t, X_{t+1} - x } +  \frac{L}{2} \norm{X_{t+1}-X_t}^2 \nonumber\\
 \le& \ip{\hat{G}_t,X_{t+1}-x} + \frac{L}{\alpha} D_{\mathcal{R}}(X_{t+1},X_t)\,. \label{eq:mdsmoothnoisy1}
\end{align}
Now write $\hat{G}_t = (\hat{G}_t-\overline G_t)  + \xi_t + G_t$ where $\xi_t = \overline G_t - G_t$ is the ``noise''.
Hence, 
\begin{align*}
\MoveEqLeft
\ip{\hat{G}_t,X_{t+1}-x} \\
\le& \beta_t \norm{X_{t+1}-x} + \ip{\xi_t,X_{t+1}-x} + \ip{ G_t, X_{t+1} - x }\\
\le &\beta_t \sqrt{ \frac{2D}{\alpha} } + \ip{\xi_t,X_{t+1}-x} + \ip{ G_t, X_{t+1} - x } \,.
\end{align*}
After we plug this into~\eqref{eq:mdsmoothnoisy1},
the plan is to take the conditional expectation of both sides w.r.t. $\F_{t}$.
As $X_t$ is $\F_{t}$-measurable, and by the definition of $\xi_t$ and $\overline G_t$, $\EE{ \xi_t|\F_{t}} = 0$, 
we have 
\begin{align*}
\MoveEqLeft
\EE{\ip{\xi_t, X_{t+1}-x}|\F_{t}} = \underbrace{\EE{\ip{\xi_t,X_{t}-x}|\F_{t}}}_{=0} + \EE{\ip{\xi_t,X_{t+1}-X_t}|\F_{t}} \,.
\end{align*}
We bound the second term inside the expectation from the Fenchel-Young inequality: 
\begin{align*}
\ip{\xi_t,X_{t+1}-X_t} \le \frac12 \left(\frac{\norm{\xi_t}_*^2}{a_t} + a_t \norm{X_{t+1}-X_t}^2\right)\,.
\end{align*}
Using again that $\mathcal{R}$ is $\alpha$-strongly convex, we get 
\[
\ip{\xi_t,X_{t+1}-X_t} \le \frac12 \left(\frac{\norm{\xi_t}_*^2}{a_t} + \frac{2a_t}{\alpha} D_{\mathcal{R}}(X_{t+1},X_t) \right)\,.
\]
Applying 
\cref{lem:mdlinregret} 
to bound $\ip{ G_t, X_{t+1} - x }$, and putting everything together gives
\begin{align*}
 \EE{f(X_{t+1}) - f(x) |\F_{t} }
\le & \quad
 \beta_t \sqrt{ \frac{2D}{\alpha} }+
\frac{1}{2a_t}  \EE{\norm{\xi_t}_*^2|\F_{t}}+
\frac{1}{\eta_t} \left(D_{\mathcal{R}}(x,X_t)-D_{\mathcal{R}}(x,X_{t+1})\right) \\&+
\underbrace{\left(\frac{a_t+L}{\alpha}-\frac{1}{\eta_t}\right) D_{\mathcal{R}}(X_{t+1},X_t)}_{=0}\numberthis \label{eq:ubProofWithNoise} \,.
\end{align*}
Given $\sigma_t^2 = \EE{ \norm{\xi_t}_*^2}$,
we will sum up these inequalities for $t=1,\dots,n-1$.
To bound the right-hand side of the result, we use
\begin{align*}
&\sum_{t=1}^{n-1} \frac{1}{\eta_t} \left(\DR(x,X_t)-\DR(x,X_{t+1})\right)
 \\
=& \DR(x,X_1) \frac{1}{\eta_1} + \DR(x,X_2) \left(\frac{1}{\eta_2}-\frac{1}{\eta_1}\right)
+ \ldots+\DR(x,X_{n-1}) \left(\frac{1}{\eta_{n-1}} -\frac{1}{\eta_{n-2}}\right)- \frac{1}{\eta_{n-1}} \DR(x,X_n)\\
 \le& D \frac{1}{\eta_1} + D \sum_{t=2}^{n-1} \left(\frac1{\eta_{t}}-\frac1{\eta_{t-1}}\right)\\
 = &\frac{D}{\eta_{n-1}}\,.
\end{align*}
The inequality results from the fact that $\{\eta_t\}$ is non-increasing.
Hence, by the tower rule,
\begin{align*}
 &\EE{ \sum_{t=1}^n f(X_t) - f(x) }  \\
\le& 
  \EE{f(X_1)-f(x)} + \sqrt{\tfrac{2D}{\alpha}} \sum_{t=1}^{n-1} \beta_t +
	   \frac{D}{\eta_{n-1}} +
	  \sum_{t=1}^{n-1}\frac{\sigma_t^2}{2a_t}\\
=& 	  
  \EE{f(X_1)-f(x)} + \sqrt{\tfrac{2D}{\alpha}} \sum_{t=1}^{n-1} \beta_t +
	   \frac{D(a_{n-1}+L)}{\alpha} +
	  \sum_{t=1}^{n-1}\frac{\sigma_t^2}{2a_t}\,.
\end{align*}

When $f$ is $L$-smooth and $\mu$-strongly convex,  we can rewrite \eqref{eq:mdsmoothnoisy1} as
\begin{align*}
\MoveEqLeft
f(X_{t+1}) - f(x) \nonumber\\
 \le& f(X_t) + \ip{ \hat{G}_t, X_{t+1}-X_t } + \frac{L}{2} \norm{X_{t+1}-X_t}^2 - \left\{ f(X_t) + \ip{ \hat{G}_t, x-X_t}+\dfrac{\mu}{2}\DR(x, X_t) \right\} \nonumber \\
 =& \ip{\hat{G}_t, X_{t+1} - x } +  \frac{L}{2} \norm{X_{t+1}-X_t}^2-\dfrac{\mu}{2}\DR(x, X_t) \nonumber\\
 \le& \ip{\hat{G}_t,X_{t+1}-x} + \frac{L}{\alpha} D_{\mathcal{R}}(X_{t+1},X_t)-\dfrac{\mu}{2}\DR(x, X_t)\,.
\end{align*}
Again, \eqref{eq:ubProofWithNoise} can be written as
\begin{align*}
 \EE{f(X_{t+1}) - f(x) |\F_{t} }
\le & \quad
 \beta_t \sqrt{ \frac{2D}{\alpha} }+
\frac{1}{2a_t}  \EE{\norm{\xi_t}_*^2|\F_{t}}\\
&+\left(\dfrac{1}{\eta_t}-\dfrac{\mu}{2}  \right)\DR(x,X_t)-\dfrac{1}{\eta_t}\DR(x,X_{t+1})
+\left( \dfrac{L+a_t}{\alpha}-\dfrac{1}{\eta_t} \right)\DR(X_{t+1},X_t) \,.
\end{align*}
Given $\dfrac{1}{\eta_t}=\dfrac{\mu t}{2}=\dfrac{L+a_t}{\alpha}$, summing up theses inequalities for $t=1,2,\cdots, n-1$, we get
\begin{align*}
 \EE{ \sum_{t=1}^n f(X_t) - f(x) }  
\le& 
  \EE{f(X_1)-f(x)} + \sqrt{\tfrac{2D}{\alpha}} \sum_{t=1}^{n-1} \beta_t +
	  \sum_{t=1}^{n-1}\frac{\sigma_t^2}{2a_t}-\dfrac{1}{\eta_{n-1}}\DR(x,X_{n})\\
\le& 	  
  \EE{f(X_1)-f(x)} + \sqrt{\tfrac{2D}{\alpha}} \sum_{t=1}^{n-1} \beta_t +
	  \sum_{t=1}^{n-1}\frac{\sigma_t^2}{2a_t}\,.
\end{align*}
Proof is done.

\section{Proof of the Lower Bounds}
\label{sec:appendix-lb-proof}
%!TEX root =  bgo.tex
% please do not delete or change the first line (needed by Csaba's editor)
\subsection{Proofs for the Lower Bound}
\label{sec:lb-proof}
%%%%%%%%%%%%%%%%%%%%%%%%%%%%%%%%%%%%%%%%%%%%%%%%%%%%%%%%%%%%%%%%%%%%%%%%%%%%%%% 

Define the probability space $(\Omega, \B, P_{\A,\gamma,\sigma^2})$, where $\Omega = (\R\times \{-1,1\})^n$, $\B$ is the associated Borel sigma algebra. Further, the probability measure $P_{\A,\gamma,\sigma^2} := p_{\A,\gamma,\sigma^2} d(\lambda \times m)$, where $\gamma$ is the oracle in the family of oracles $\Gamma(f)$ for some given $f$, $\lambda$ is the Lebesgue measure on $\B$, $m$ is the counting measure on $\{-1,1\}$ and $p_{\A,\gamma,\sigma^2}$ is the density function defined as\footnote{We use the shorthand $g_{1:n}$ to denote $g_1,\ldots,g_n$.}
\begin{align*}
&p_{\A,\gamma,\sigma^2}(g_{1:n}, v) = \dfrac{1}{2} \bigg(p_{\A,\gamma,\sigma^2}(g_n \mid g_{1:n-1}) \\
&\times p_{\A,\gamma,\sigma^2}(g_{n-1} \mid g_{1:n-2}) \ldots p_{\A,\gamma,\sigma^2}(g_1)\bigg)\\
 = & \dfrac{1}{2} \bigg( p_{\N}(g_n - \gamma(\A_n(g_{1:n-1})))
%  &\times p_{\normal(0,\sigma^2)}(g_{n-1} - \gamma(\A_n(g_{1:n-2}))) \times \ldots \\
 \ldots \times  p_{\N}(g_1 - \gamma(\A_1))\bigg),
\end{align*}
where $p_{\N}$ is the density of a $\normal(0,\sigma^2)$ random variable.  

%%%%%%%%%%%%%%%%%%%%%%%%%%%%%%%%%%%%%%%%%%%%%%%%%%%%%%%%%%%%%%%%%%%%%%%%%%%%%%% 

We consider a class of functions $\F = \{f_+, f_-\}$, with 
\begin{align*}
  f_+(x) := \dfrac{\epsilon}{2} (x - 1)^2 \text{ and } f_-(x) := \dfrac{\epsilon}{2} (x + 1)^2, \forall x \in \R.
\end{align*}
%%%%%%%%%%%%%%%%%%%%%%%%%%%%%%%%%%%%%%%%%%%%%%%%%%%%%%%%%%%%%%%%%%%%%%%%%%%%%%% 

It is easy to see that $f_+$ (resp, $f_-$) is minimized at $x^*_+ = 1$ (resp. $x^*_- = -1$). 
Using the fact that $f_+$ and $f_-$ are strongly convex with associated constant $\left(\dfrac{\epsilon}{2}\right)$, we obtain
\begin{align}
  f_+(x) - f_+(x^*_+)
  \ge &  \dfrac{\epsilon}{2} (x - 1)^2 \ge  \dfrac{\epsilon}{2}  \indic{x  < 0}. \label{eq:fv-lb}
\end{align}
Similarly,   $f_-(x) - f_-(x^*_-) \ge  \dfrac{\epsilon}{2}  \indic{x  >0}$.

%%%%%%%%%%%%%%%%%%%%%%%%%%%%%%%%%%%%%%%%%%%%%%%%%%%%%%%%%%%%%%%%%%%%%%%%%%%%%%% 

The minimax error \eqref{eq:minimax-err} can be lower bounded as follows\footnote{$f_{+1} \equiv f_+$ and $f_{-1}\equiv f_-$.}:
\begin{align}
 &\Delta_n(\F, \sigma^2)\nonumber\\
  \ge & \inf_{\A} \sup_{v \in \{+1,-1\}} \sup_{\gamma \in \Gamma(f_v)} \E[f_V(\hat x_n) - \inf_{x \in X}
  f_V(x)],\label{eq:avg-bd}
  \end{align}
where the expectation is w.r.t. the distribution $\P:= \dfrac{1}{2} \left(P_{\A, \gamma_+, \sigma^2} \indic{v=+1} + P_{\A, \gamma_-, \sigma^2}\indic{v=-1}\right)$. Here $\gamma_v$  is the oracle for $f_v$, for $v=+1,-1$ and is defined by $\gamma_v(x) = \epsilon(x-v) + v C_1 \delta^p$. 
Using \eqref{eq:fv-lb}, we obtain
\begin{align}
\Delta_n(\F, \sigma^2)  \ge & \inf_{\A} \dfrac{\epsilon}{2}  \P(\hat x V < 0), \label{eq:strong-convex-bd}\\
  = & \inf_{\A} \dfrac{\epsilon }{2}  \left(\P_{+}(\hat x_n < 0) + \P_{-}(\hat x_n > 0)\right), \label{eq:Pplus}\\
  \ge &\inf_{\A} \dfrac{\epsilon }{2} \left(1 - \tvnorm{\P_{+}- \P_{-}}\right), \label{eq:lecam}\\
  \ge &\inf_{\A} \dfrac{\epsilon }{2}  \left( 1 - \left(\dkl{P_{+}}{P_{-}}\right)^{\frac{1}{2}}\right), \label{eq:pinsker}
\end{align}
where 
the equality in \eqref{eq:Pplus} uses the definitions $\P_{+}(\cdot) := \P(\cdot\mid V=1)$, $\P_{-}(\cdot) := \P(\cdot\mid V=-1)$. Further, the inequality in \eqref{eq:lecam} follows from the definition of total variation distance, while \eqref{eq:pinsker} follows from Pinksker's inequality.


%%%%%%%%%%%%%%%%%%%%%%%%%%%%%%%%%%%%%%%%%%%%%%%%%%%%%%%%%%%%%%%%%%%%%%%%%%%%%%% 

% \paragraph{Upper-bounding the difference in gradient estimates:}



%%%%%%%%%%%%%%%%%%%%%%%%%%%%%%%%%%%%%%%%%%%%%%%%%%%%%%%%%%%%%%%%%%%%%%%%%%%%%%% 

Let $P_{+}^t(\cdot\mid g_1,\ldots,g_{t-1})$ denote the distribution of the $t$th observation $G_t$ conditional on $V=+1$ and $G_1=g_1,\ldots,G_{t-1}=g_{t-1}$. Define  $P_{-j}^t(\cdot\mid g_1,\ldots,g_{t-1})$ in a similar fashion.
Then, by the chain rule for KL-divergences, we have
\begin{scriptsize}
\begin{align}
&\dkl{P_{+}}{P_{-}}\label{eq:dklchain}\\ 
&= \sum_{t=1}^n \int_{\R^{t-1}} \dkl{P_{+}^t(\cdot\mid g_{1:t-1})}{P_{-}^t(\cdot\mid g_{1:t-1})} d P_{+}^t(\cdot\mid g_{1:t-1}).\nonumber
\end{align}
\end{scriptsize}
Since the noise $\xi \sim \normal(0,\sigma^2)$, it is easy to see that $P_{+}^t(\cdot\mid g_{1:t-1}) \sim \normal(\gamma_{+}(A(g_{1:t-1})),\sigma^2)$, where $A(g_{1:t-1})$ denotes the point chosen by the algorithm given observations $g_1,\ldots, g_{t-1}$ and $\gamma_{+}$ is the gradient oracle defined earlier. 
Observe that, for any $x\in \R$, $
 |f'_+(x) - f'_-(x)| \le 2 \epsilon.$ This implies 
 \begin{align}
 |\gamma_+(x) - \gamma_-(x)| \le 2(\epsilon-C_1\delta^p).  \label{eq:gdiff-ub}
 \end{align}
From the foregoing, 
\begin{align}
 &\dkl{P_{+}^t(\cdot\mid g_{1:t-1})}{P_{-}^t(\cdot\mid g_{1:t-1})} d P_{+}^t(\cdot\mid g_{1:t-1}) \nonumber\\
 \le&\dfrac{(g_{+}(A(g_{1:t-1})) - g_{-}(A(g_{1:t-1})))^2}{2 \sigma^2}\label{eq:dkgauss1}\\
 \le& \dfrac{4(\epsilon-C_1\delta^p)^2\delta^q}{C_2}.\label{eq:dkgauss}
\end{align}
The inequality \eqref{eq:dkgauss1} follows from the fact that the KL-divergence between normal distributions $\normal(\mu_1,\sigma^2)$ and $\normal(\mu_2,\sigma^2)$ is upper-bounded by $\dfrac{(\mu_1 - \mu_2)^2}{2 \sigma^2}$, while the inequality \eqref{eq:dkgauss} follows from \eqref{eq:gdiff-ub} and the variance constraint that governs the observations output by the oracles $\gamma_+$ and $\gamma_-$.

Plugging \eqref{eq:dkgauss} into \eqref{eq:dklchain}, we obtain
\begin{align}
\dkl{P_{+}}{P_{-}} \le \dfrac{4n(\epsilon-C_1\delta^p)^2 \delta^q}{C_2}.\label{eq:dkbd}
\end{align}
Note that the above bound holds uniformly over all algorithms $\A$. 
Substituting the above bound into \eqref{eq:pinsker}, we obtain
\begin{align}
 \Delta_n(\F, \sigma^2) 
  \ge & \dfrac{\epsilon}{2} \left(1 - 2\sqrt{
    n}  \dfrac{(\epsilon-C_1\delta^p)\delta^{q/2}}{C_2}
  \right)\label{eq:final-lower-bd}
\end{align}

%%%%%%%%%%%%%%%%%%%%%%%%%%%%%%%%%%%%%%%%%%%%%%%%%%%%%%%%%%%%%%%%%%%%%%%%%%%%%%%
\paragraph{Derivations of rates:}
 Optimizing over $\epsilon$ in \eqref{eq:final-lower-bd}, we get
 $$\epsilon^* = \left(\dfrac{\delta^{-q/2}C_2}{4\sqrt{n}} + \dfrac{C_1\delta^p}{2}\right).$$

 Plugging in $\epsilon^*$, we obtain
 $$\Delta_n(\F, \sigma^2) 
 \ge \left(\dfrac{\delta^{-q/2}C_2}{8\sqrt{n}} + \dfrac{C_1\delta^p}{4}\right) \left( \dfrac{1}{2} + \dfrac{\sqrt{n}C_1 \delta^{p + q/2}}{C_2} \right).$$

Substituting $p=1$ and $q=2$  and optimizing over $\delta$, we obtain
$$\Delta_n(\F, \sigma^2) \ge \dfrac{C_3}{ n^{1/4}} \text{ for some } C_3>0.$$

On the other hand, substituting $p=q=2$, we obtain
$$\Delta_n(\F, \sigma^2) \ge \dfrac{C_4}{ n^{1/3}} \text{ for some } C_4>0.$$


\section{Gradient Estimation Proofs}
\label{sec:appendix-grad}
%!TEX root =  bgo.tex
% please do not delete or change the first line (needed by Csaba's editor)
\subsubsection*{Proof of Proposition \ref{prop:grad-spsa}}

\paragraph{One-point SPSA:}

 Since $f$ is $3$-times continuously differentiable, using Taylor's expansion, we get for the the $i^{th}$ component of $G$,
\begin{align*}
&\EE{G_{\cdot i}}\\
=&\EE{\dfrac{1}{\delta \Delta_{\cdot i}} \left( f(x+\delta \Delta)+\epsilon \right) }\\
=& \EE{\dfrac{1}{\delta \Delta_{\cdot i}} \left( f(x)+\delta f'(x)^\top \Delta+\dfrac{1}{2}\delta^2 \Delta^\top f''(x)\Delta \right) } \numberthis \label{eq:spsaTaylorExp} \\
&+\EE{\dfrac{1}{\delta \Delta_{\cdot i}} \left(O(\delta^3 \Delta\otimes\Delta\otimes\Delta) +\epsilon \right) }\\
=& [f'(x)]_i +O(\delta^2) \,,
\end{align*}
where $[f'(x)]_i$ denotes the $i^{th}$ component of $f'(x)$. The last equality comes from the properties of symmetry, and bounded moment for $\Delta$.
Hence, $G$ is a estimate of $f'(x)$ with bias $O(\delta^2)$.

\paragraph{Two-point SPSA:}

Under this situation, using Taylor expansion again, the $f(x)$ and $f''(x)$ terms in \eqref{eq:spsaTaylorExp} can be canceled and one can conclude that $G$ is only an order $O(\delta^2)$ term away from $f'(x)$. Note that the second-order term in one-point SPSA is zero-mean, while in the two-point SPSA it is zero. 
% As a result, we only need $\Delta_{\cdot i}$ to be zero-mean instead of symmetry.


\section{Conclusions}
\label{sec:conc}
%!TEX root =  bgo_camera_ready.tex
% please do not delete or change the first line (needed by Csaba's editor)
We presented a general noisy  gradient oracle for model for convex optimization. The oracle model covers several gradient estimation methods in the literature designed for algorithms that can observe only noisy function values, while allowing to handle explicitly the bias-variance tradeoff of these estimators. The framework allows to derive sharp upper and lower bounds on the minimax optimization error, reducing the optimization problem to study properties of gradient estimators.


\appendix


%!TEX root =  bgo.tex
% please do not delete or change the first line (needed by Csaba's editor)
\subsection{Reductions and Proofs}
\label{sec:orrel}



\subsubsection*{Acknowledgements}
This work was supported by the Alberta Innovates Technology Futures through the Alberta Ingenuity Centre for Machine Learning and RLAI, NSERC, the National Science Foundation (NSF) under Grants CMMI-1434419, CNS-1446665, and CMMI-1362303, and by the Air Force Office of Scientific Research (AFOSR) under Grant FA9550-15-10050.

%\clearpage\newpage
%\begin{small}
%\bibliographystyle{apalike}
\bibliography{main}
%\end{small}

%\end{document}

\end{document}
