%!TEX root =  bgo.tex
% please do not delete or change the first line (needed by Csaba's editor)
Gradient oracle models have been studied in a number of previous papers 
\citep{dAsp08,Baes09,SchRoBa11,DeGliNe14}.
%assume that set $\K$ is bounded and that the oracle provides at each point $x\in \K$ an approximate gradient $g$ satisfying condition
%$|\ip{g-\nabla f(x), v-w}| \le \delta$ for all $v,w,x\in \K$.
A full comparison between these oracle models is given by \cite{DeGliNe14}.
For illustration, here we only review the model of this latter paper as a typical example of these previous works.
The model of \cite{DeGliNe14} assumes a first-order approximation to the function
with parameters $(\delta,L)$. In particular, 
given $(x,\delta,L)$ and the convex function $f$, 
the oracle gives a pair $(t,g)\in \R \times \R^d$
such that $t + \ip{g,\cdot-x}$ is a linear lower approximation to $f(\cdot)$ in the sense that 
$0\le f(y) - \left\{ t+ \ip{g,y-x}\right\} \le \frac{L}{2} \norm{y-x}^2 + \delta$.
\cite{DeGliNe14} argue that this notion appears naturally in several optimization problems and study whether the so-called accelerated gradient techniques are still superior to their non-accelerated counterparts (and find a negative answer).
The authors study both lower and upper rates of convergence, similarly to our paper.

%, except that in the lower bounds the algorithm is not allowed to control the oracle parameters.

A major difference between the previous and our settings is that we allow stochastic noise (and bias), which the algorithms can control, while the oracle in these previous paper must guarantee that the accuracy requirements hold in each time step
with probability one.
This is a much stronger requirement, which may be impossible to satisfy in some problems, such as when 
the only information available about the functions is noise contaminated.
Some works, such as \citet{SchRoBa11} allow arbitrary sequences of errors and show error bounds as a function
of the accumulated errors. 
Our proof technique is actually essentially the same (as can be expected).
However, the noisy case requires special care. For example, Proposition~3 of
\citet{SchRoBa11}  bounds the optimization error for the smooth, convex case by 
$O(1/n^2 ( \norm{x_1-x^*}^2 + A_n^2 )$ where $A_n = O( \sum_{t=1}^n t \norm{e_t})$, $e_t$ being the error of the approximate gradient. This expression is upper and lower bounded, up to a constant factor by
 $\frac{1}{n^2} \sum_{t=1}^n t^2 \approx n$,
assuming that errors' noise level is a positive constant (in all our result, this holds).
This clearly shows the differences between the analysis and that the noisy case requires (somewhat) special treatment.

Nevertheless, we do not claim to invent methods for proving upper bounds for the stochastic case
as these methods have been developed for special cases for a long time by now. \todoc{Add references.}
Our main contribution here lies in abstracting away the properties of gradient estimation procedures 
to create our new oracle model, in which we are able to study not only upper rates, but also lower rates of convergence.
A similar, but simpler oracle model building on the inexact oracle model of  \cite{DeGliNe14}, 
but one which lacks the bias-variance tradeoff central to this paper (i.e., they assume the variance and bias can be controlled independently of each other) appeared recently in a conference program \citet{DvoGa15}. The results in this paper are upper bounds on the error of certain gradient methods and they correspond to the bounds we obtained with $q=0$.
\todoc[inline]{The paper of \citet{SchRoBa11} has 100 citations on google scholar.
The paper of \cite{DeGliNe14} has similarly many.
There are probably more relevant works. Someone could take a quick look and perhaps add a few more relevant ones.
}


\todoc[inline]{The stuff below is junk, mostly just listing references.}
Bandit convex optimization:

Online adversarial setting. Optimal rate is $\Theta(\sqrt{T})$.
No one knows general practical algorithms achieving this rate.
Strongly convex + smooth: \cite{hazan2014bandit}

Stochastic optimization setting: ??? Shamir's paper \cite{shamir2012complexity}??
Lower bounds, upper bounds for the general case?
\cite{hazan2014bandit} for upper bound.

All kind of papers about how to estimate gradients going back to maybe 70s in Russia.

One-point: \cite{flaxman2005online}

Two-point: \cite{AgDeXi10}
\cite{duchi2013optimal}

Lower bds: \cite{raginsky2011information} \cite{Chen88:LB-AoS}

Ellipsoid: \cite{AgFoHsuKaRa13:SIAM}

