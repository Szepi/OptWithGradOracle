\documentclass[11pt,letterpaper,english]{article}
%\usepackage{bbold}
\usepackage[round]{natbib}
%!TEX root =  bgo-cam-ready.tex
% please do not delete or change the first line (needed by Csaba's editor)

%%%%%%%%%%%%%%%%%%%% Added for arxiv report %%%%%%%%%%%%%%%%%%%%%
\usepackage[T1]{fontenc}
\usepackage[utf8]{inputenc}
\usepackage[english]{babel}
\usepackage{authblk}
\usepackage{times}
\usepackage[margin=1in]{geometry}
%%%%%%%%%%%%%%%%%%%%%%%%%%%%%%%%%%%%%%%%%%%%%%%%%%%%%%%%%%%%%%%%%


\usepackage{etex}
\usepackage[utf8]{inputenc}
\usepackage{amsmath}
\usepackage{amsfonts}
\usepackage{amssymb}
%\usepackage{amsthm}
\usepackage{mathtools}
\usepackage{graphicx}
\usepackage{algorithm}
\usepackage[numbers]{natbib}
\usepackage[noend]{algpseudocode}
\usepackage{url}
\usepackage{bm}
% \usepackage[margin=2.5cm]{geometry}
\setlength{\marginparwidth}{20mm}
\usepackage{booktabs}
\usepackage{multirow}
\usepackage{colortbl}
\usepackage{hhline}
\usepackage{paralist}
\usepackage{xspace}
\usepackage{sidecap}

\usepackage{caption}
\usepackage{subcaption}
\usepackage{tikz}
\usepackage{pgfplots}
\usepackage{makecell}
\usetikzlibrary{positioning,arrows.meta,shapes,calc,patterns,fit,decorations}
\usepgfplotslibrary{fillbetween}
\pgfplotsset{compat=1.10}

%%% Added by Prashanth: the following two lines are used to make compilation faster with tikz. Each tikzfigure will become a seprate tex job and the figures get stored in a sub-folder named tikz. 
%%%% Dont forget to add --shell-escape(or -enable-write18 if you are using MikTeX) to the pdflatex commandline
\usetikzlibrary{external}
\tikzexternalize[prefix=tikz/]
%%%%%%%%%%

\usepackage[disable]{todonotes}
%\usepackage{todonotes}
%%% Added by Prashanth: The tikz externalize business to speed up compilation is incompatible with todonotes, unless we add the following workaround that disables tikz-externalize in todo definition
\makeatletter
\renewcommand{\todo}[2][]{\tikzexternaldisable\@todo[#1]{#2}\tikzexternalenable}
\makeatother
%%%%%%%%%%%

\newcommand{\todoc}[2][]{\xspace\todo[color=red!20!white,size=\tiny,#1]{Cs: #2}}
\newcommand{\todox}[2][]{\xspace\todo[color=orange!20!white,size=\tiny,#1]{X: #2}}
\newcommand{\todoa}[2][]{\xspace\todo[color=blue!20!white,size=\tiny,#1]{A: #2}}
\newcommand{\todop}[2][]{\xspace\todo[color=green!20!white,size=\tiny,#1]{P: #2}}

\makeatletter
\def\BState{\State\hskip-\ALG@thistlm}
\makeatother


%\usepackage{amsthm}
%\newtheorem{remark}{Remark}
%\newtheorem{proposition}{Proposition}
%\newtheorem{definition}{Definition}
% \newtheorem{theorem}{Theorem}
\usepackage[amsmath,standard,thmmarks]{ntheorem} % ntheorem makes cleveref work properly
% \newtheorem{corollary}[theorem]{Corollary}
% \newtheorem{lemma}[theorem]{Lemma}

% Hyperlinks make it very easy to navigate an electronic document.
% In addition, this is where you should specify the thesis title
% and author as they appear in the properties of the PDF document.
% Use the "hyperref" package
% N.B. HYPERREF MUST BE THE SECOND TO LAST PACKAGE LOADED; ADD ADDITIONAL PKGS ABOVE
\usepackage[pdftex,letterpaper=true,pagebackref=false]{hyperref} % with basic options
		% N.B. pagebackref=true provides links back from the References to the body text. This can cause trouble for printing.
\hypersetup{
    plainpages=false,       % needed if Roman numbers in frontpages
    pdfpagelabels=true,     % adds page number as label in Acrobat's page count
    bookmarks=true,         % show bookmarks bar?
    unicode=false,          % non-Latin characters in Acrobat’s bookmarks
    pdftoolbar=true,        % show Acrobat’s toolbar?
    pdfmenubar=true,        % show Acrobat’s menu?
    pdffitwindow=false,     % window fit to page when opened
    pdfstartview={FitH},    % fits the width of the page to the window
    pdftitle={(Bandit) Convex Optimization with Biased Noisy Gradient Oracles},    % set
    pdfauthor={},    % set
%     pdfauthor={D\'avid Szepesv\'ari},    % set
%    pdfsubject={Subject},  % subject: CHANGE THIS TEXT! and uncomment this line
%    pdfkeywords={keyword1} {key2} {key3}, % list of keywords, and uncomment this line if desired
    pdfnewwindow=true,      % links in new window
    colorlinks=true,        % false: boxed links; true: colored links
    linkcolor=blue,         % color of internal links
    citecolor=blue,        % color of links to bibliography
    filecolor=magenta,      % color of file links
    urlcolor=cyan           % color of external links
}
% N.B. CLEVEREF MUST BE LOADED; AFTER HYPERREF
\usepackage[capitalize]{cleveref}
\newtheorem{ass}{Assumption}
\crefname{ass}{Assumption}{Assumptions}

%%%%%%%%%%%%%%%%%%%%%%%%%%%%%%%%%%%%%%%%%%%%%%%%%%%%%%%%
\newcommand{\cE}{\mathcal{E}}
\newcommand{\cF}{\mathcal{F}}
\newcommand{\cA}{\mathcal{A}}
\newcommand{\cK}{\mathcal{K}}
\newcommand{\cB}{\mathcal{B}}
\newcommand{\cY}{\mathcal{Y}}
\newcommand{\R}{\mathbb{R}}
\newcommand{\EE}[1]{\mathbb{E}\left[#1\right]}
\newcommand{\ip}[1]{\langle #1 \rangle}
\newcommand{\norm}[1]{\|#1\|}
\newcommand{\scnorm}[1]{\left\|#1\right\|}
\DeclareMathOperator{\fspan}{span}
\DeclareMathOperator\erf{erf}
\DeclareMathOperator{\argmin}{argmin}
\DeclareMathOperator{\argmax}{argmax}
\DeclareMathOperator{\interior}{int}
\DeclareMathOperator{\dom}{dom}
\DeclareMathOperator{\esssup}{ess\,sup}
\DeclareMathOperator{\essup}{ess\,sup}
\newcommand\numberthis{\addtocounter{equation}{1}\tag{\theequation}}
\newcommand{\DR}{D_{\mathcal{R}}}
\newcommand{\g}{\gamma}
\newcommand{\og}{\overline{\gamma}}


%%% Prashanth: some of my macros..need to remove duplicates
\newcommand{\C}{\mathcal{C}}
\newcommand{\A}{\mathcal{A}}
\newcommand{\I}{\mathcal{I}}
\newcommand{\B}{\mathcal{B}}
\newcommand{\V}{\mathcal{V}}
\newcommand{\F}{\mathcal{F}}
\newcommand{\N}{\mathcal{N}}
\newcommand{\E}{\mathbb{E}}
\renewcommand{\P}{\mathbb{P}}
\newcommand{\G}{\mathcal{G}}
\renewcommand{\O}{\mathcal{O}}
\newcommand{\D}{\mathcal{D}}
\newcommand{\cD}{\mathcal{D}}
\newcommand{\K}{\mathcal{K}}
\newcommand{\cR}{\mathcal{R}}
\newcommand{\tf}{\tilde{f}}

\newcommand{\noise}{\psi}
\newcommand{\noiseCap}{\Psi}

\newcommand{\dnorm}[1]{\norm{#1}_*} % Dual norm
\def\<{\left\langle} % Angle brackets
\def\>{\right\rangle}




\newcommand{\sign}{\mathop{\rm sign}}
% \newcommand{\norm}[1]{\left\|{#1}\right\|} % A norm with 1 argument
\newcommand{\tvnorm}[1]{\norm{#1}_{\rm TV}}
\newcommand{\dkl}[2]{D_{\rm kl}\left({#1} |\!| {#2} \right)}
\newcommand{\normal}{\mathsf{N}}  % Normal distribution
\newcommand{\openleft}[2]{\left({#1},{#2}\right]} % Interval open on left
\def\mbf#1{\mathbf{#1}}
\def\indic#1{\mbf{1}\left\{#1\right\}} % Indicator function
\newcommand{\tr}{^\mathsf{\scriptscriptstyle T}} % transpose


\author{Xiaowei Hu, L. A. Prashanth, Andr\'as Gy\"orgy, and Csaba Szepesv\'ari}
\title{A lower bound for bandit smooth convex optimization for the algorithm of \citet{DekelEK15}}
\begin{document}
\maketitle

\section{Expression of the estimate}


We consider the problem of iterative optimization of a convex function $f:\R \to [0,\infty)$ using a gradient oracle.
In every round, the optimizer can query the gradient oracle $g_t$ at some point $x_t$, and the goal of the algorithm is to find a point $x^*_T$ after $T$ steps such that $x^*_T$ is a function of $x_1,g_t(x_1),\ldots, x_{T},g_T$ and $\EE{f(x^*_T)-f(x^*)}$ is small where $x^*$ is the minimizer of $f$, that is, $f(x^*)=\min_x f(x)$. Assume $f(x) = \dfrac{\epsilon}{2} (x-1)^2$ and 
$g_t(x) = \epsilon (x-1) + C_1 \delta^2 + \xi_t$ where $\xi_t$ are zero-mean iid random variables with variance $C_2 /\delta^2$.
Note that 
$g_t(x)-f'(x)=C_1 \delta^2 + \xi_t$, therefore, the bias of the gradient oracle is 
\[
\EE{g_t(x)-f'(x)}=C_1 \delta^2,
\] 
while the variance of the oracle is
\[
\EE{g_t(x)-f'(x)-C_1 \delta^2} = \E{\xi^2} = C_2 /\delta^2.
\]
Note that the above bias-variance bounds (with inequalities instead of equalities) are used in bandit convex optimization in papers that consider gradient methods using estimated gradients, including \citep{DekelEK15}, and they do not use anything else about the gradient estimates.

Next we consider two algorithms, SGD with gradient estimate $g_t$ and the method of \citet{DekelEK15}. With a slight abuse of notation, we will write $g_t=g_t(x_t)$, and we define $k^+=\max\{k,1\}$.
\begin{description}
\item[Algorithm 1:] $ x_{t+1} = x_t - \eta  g_t$;

\item[Algorithm 2:] $\bar{g}_t = \dfrac{1}{K+1}  \sum_{s=(t-K)^+}^t g_s$ and
$x_{t+1} = x_t - \eta  \bar{g}_t$.
\end{description}

\begin{proposition}
\label{prop1}
Assume Algorithm 1 or 2 is run to produce $x_t$ for $t=1,2,\ldots$. Then
\begin{align}
\label{eq:expression_x_t}
x_{t} = w^{(t)}_0 x_1 + 1-w^{(t)}_0 -\frac{C_1 \delta^2}{\epsilon}(1-w^{(t)}_0) -w^{(t)}_1 \xi_1 - w^{(t)}_2 \xi_2 - \cdots - w^{(t)}_{t-1} \xi_{t-1} \,,
\end{align}
for some weights $w^{(t)}_0,\ldots,w^{(t)}_{t-1}$ satisfying
$w^{(t)}_0 = 1-\epsilon (w^{(t)}_1+\cdots+w^{(t)}_{t-1})$.
\end{proposition}
\begin{proof}
Assume Algorithm~1 is used. Then
\begin{align*}
x_{t+1} =& (1-\eta \epsilon) x_{t} + \eta \epsilon - \eta C_1 \delta^2-\eta \xi_{t}\\
=&  \left(1-\eta \epsilon\right)^t x_1+ \eta \epsilon+\eta \epsilon\left(1-\eta \epsilon\right)+\cdots + \eta \epsilon\left(1-\eta \epsilon\right)^{t-1}\\
&-C_1 \delta^2\left( \eta + \eta\left( 1-\eta \epsilon \right)+\cdots + \eta \left(1-\eta \epsilon\right)^{t-1} \right) \\
&-\eta \xi_t -\eta \left(1-\eta \epsilon\right)\xi_{t-1}- \cdots -\eta \left(1-\eta \epsilon\right)^{t-1}\xi_{1}\\
=& \left(1-\eta \epsilon\right)^t x_1+ 1-\left(1-\eta \epsilon \right)^t
-C_1 \delta^2 \dfrac{1}{\epsilon}\left( 1-\left(1-\eta \epsilon \right)^t \right)\\
&- \eta \xi_t -\eta \left(1-\eta \epsilon\right)\xi_{t-1}- \cdots -\eta \left(1-\eta \epsilon\right)^{t-1}\xi_{1},
\end{align*}
showing that the proposition holds with $w^{(t+1)}_0=(1-\eta \epsilon)^t$ and $w^{(t+1)}_s= \eta (1-\eta\epsilon)^{t-s}$ for $s=1,\ldots,t$.

\medskip

To prove the proposition for Algorithm~2, we use induction.
It is obvious that \eqref{eq:expression_x_t} holds for $t=1$ with $w^{(1)}_0=1$.
Assume that \eqref{eq:expression_x_t} holds for $t=1,2,\ldots,n$, we will prove that it is also true for $t=n+1$.
Given the expression of $x_t$, we have
\begin{align*}
g_t &= \epsilon (x_t-1) + C_1 \delta^2 + \xi_t \\
&= \epsilon w^{(t)}_0 x_1 - \epsilon w^{(t)}_0 + C_1 \delta^2w^{(t)}_0 -\epsilon w^{(t)}_1 \xi_1-\cdots-\epsilon w^{(t)}_{t-1} \xi_{t-1}+\xi_t \,.
\end{align*}
Therefore, 
\begin{align*}
\bar{g}_n &= \dfrac{1}{K+1}\sum_{i=(n-K)^+}^n g_i \\
& = \frac{\epsilon x_1-\epsilon +C_1\delta^2}{K+1} \sum_{j=(n-K)^+}^n w^{(j)}_0 
% - \frac{\epsilon}{K+1} \sum_{j=(n-K)^+}^n w^{(j)}_0 +\frac{C_1 \delta^2}{K+1} \sum_{j=(n-K)^+}^n w^{(j)}_0 \\
 - \frac{\epsilon}{K+1} \sum_{i=1}^{n-1} \sum_{j=\max\left\{i+1,(n-K)^+\right\}}^n w^{(j)}_{i} \xi_{i} 
%&\quad -\epsilon \dfrac{\sum_{j=\max\left\{2,(n-K)^+\right\}}^n w^{(j)}_1}{K+1} \xi_1
%-\epsilon \dfrac{\sum_{j=\max\left\{3,(n-K)^+\right\}}^n w^{(j)}_2}{K+1} \xi_2-\cdots
%-\epsilon \dfrac{w^{(n)}_{n-1}}{K+1} \xi_{n-1}  
+\frac{1}{K+1}\sum_{i=(n-K)^+}^n \xi_i \,.
\end{align*}
Then, following the algorithm, by the induction hypothesis we have
\begin{align*}
x_{n+1} &= x_n -\eta \bar{g}_n \\
&= \left( w^{(n)}_0-\dfrac{\eta \epsilon}{K+1} \sum_{j=(n-K)^+}^n w^{(j)}_0  \right)x_1
+1- w^{(n)}_0+\dfrac{\eta \epsilon}{K+1} \sum_{j=(n-K)^+}^n w^{(j)}_0 \\
&\quad - \frac{C_1 \delta^2}{\epsilon} \left(1- w^{(n)}_0+\dfrac{\eta \epsilon}{K+1} \sum_{j=(n-K)^+}^n w^{(j)}_0    \right) \\
&\quad -  \sum_{i=1}^{n-1} \left( w^{(n)}_i- \dfrac{\eta \epsilon}{K+1} \sum_{j=\max \left\{i+1,(n-K)^+\right\}}^n w^{(j)}_i +\dfrac{\eta}{K+1} \mathbb{I}\left\lbrace i \geq n-K \right\rbrace \right) \xi_i  -\dfrac{\eta}{K+1}\xi_n.
\end{align*}
Letting
\begin{align*}
w^{(n+1)}_0 &= w^{(n)}_0-\dfrac{\eta \epsilon}{K+1} \sum_{j=(n-K)^+}^n w^{(j)}_0  \,,\\
w^{(n+1)}_i &= w^{(n)}_i- \dfrac{\eta \epsilon}{K+1} \sum_{j=\max \left\{i+1,(n-K)^+\right\}}^n w^{(j)}_i +\dfrac{\eta}{K+1} \mathbb{I}\left\lbrace i \geq n-K \right\rbrace, \quad \,i=1,2,\cdots,n-1 \,, \\
w^{(n+1)}_n &=\dfrac{\eta}{K+1} \,,
\end{align*}
we get 
\begin{align*}
x_{n+1} = w^{(n+1)}_0 x_1 + 1-w^{(n+1)}_0 -C_1 \delta^2 \frac{1}{\epsilon}(1-w^{(n+1)}_0) -w^{(n+1)}_1 \xi_1 - w^{(n+1)}_2 \xi_2 - \cdots - w^{(n+1)}_{n} \xi_{n} \,.
\end{align*}
Now we only need to prove that $w^{(n+1)}_0=1-\epsilon \sum_{i=1}^n w^{(n+1)}_i$. Given that $w^{(j)}_0=1-\epsilon \sum_{i=1}^{j-1} w^{(j)}_i$ for $j=1,2,\cdots,n$, we have
\begin{align*}
 \sum_{i=1}^n w^{(n+1)}_i
&=  \sum_{i=1}^{n-1} w^{(n)}_i +\dfrac{\eta}{K+1} \left( n+1-(n-K)^+ -\epsilon  \sum_{i=1}^{n-1} \sum_{j=\max \left\{i+1,(n-K)^+\right\}}^n w^{(j)}_i \right) \\
&= \dfrac{1}{\epsilon} \left( 1-w^{(n)}_0 \right) + \dfrac{\eta}{K+1}  \sum_{j=(n-K)^+}^n w^{(j)}_0  \\
&=  \dfrac{1}{\epsilon} \left( 1-w^{(n+1)}_0 \right) .
\end{align*}
Thereby, \eqref{eq:expression_x_t} also holds for $t=n+1$, finishing the proof.
\end{proof}



\section{Regret lower bounds}

Now we assume that the sequence of estimates $x_t$ satisfies Proposition~\ref{prop1}. Then, letting $w_i=w^{(T+1)}_i$, the final estimate $x_{T+1}$ has the form
\begin{align*}
x_{T+1} = w_0 x_1 + 1-w_0 -\frac{C_1 \delta^2 }{\epsilon}(1-w_0) -w_1 \xi_1 - w_2 \xi_2 - \cdots - w_T \xi_T
\end{align*}
where
\[w_0 = 1-\epsilon (w_1+\cdots+w_T).\]
%We should also have 
%\[w_0 \to 0 \text{ as } T \to +\infty.\]
%For Algorithm 1, $w_0 =  \left(1-\eta \epsilon\right)^T = (1-T^{-\frac{23}{24}})^T = (\frac{1}{e})^{T^{1/24}} \to 0$.
Since $\{\xi_t\}$ is independent, $\EE{\xi_t}=0$, $\EE{\xi_t^2}=\dfrac{C_2}{\delta^2}$, 
the regret is
\begin{align}
\EE{R} &=\EE{ \frac{\epsilon}{2} (x_{T+1}-1)^2} \nonumber \\
&= \EE{\frac{\epsilon}{2} \left(w_0 x_1 -w_0 -\frac{C_1 \delta^2 }{\epsilon}(1-w_0) -w_1 \xi_1 - w_2 \xi_2 - \cdots - w_T \xi_T  \right)^2} \nonumber \\
&= \frac{\epsilon}{2} \left( (x_1-1)-(\epsilon x_1 -\epsilon +C_1\delta^2)(w_1+\cdots+w_T) \right)^2 + \dfrac{\epsilon}{2} \dfrac{C_2}{\delta^2} \left(w_1^2+\cdots+w_T^2  \right) \label{eq:lb2}\\
&\geq \dfrac{\epsilon}{2} \left( (x_1-1)-(\epsilon x_1 -\epsilon +C_1\delta^2)(w_1+\cdots+w_T) \right)^2 + \dfrac{\epsilon}{2} \dfrac{C_2}{\delta^2}\dfrac{1}{T} \left(w_1+\cdots+w_T  \right)^2 \nonumber \\
&= \frac{\epsilon}{2} \Bigl( (x_1-1)^2-2(x_1-1)(\epsilon x_1-\epsilon +C_1 \delta^2)W + [(\epsilon x_1-\epsilon +C_1 \delta^2)^2+C_2 \delta^{-2}T^{-1}]W^2  \Bigr)  \nonumber
\end{align}
where we introduced the shorthand notation $W = w_1+\cdots+w_T$.
%\begin{align*}
%\EE{R} \geq  \dfrac{\epsilon}{2} \Bigl( (x_1-1)^2-2(x_1-1)(\epsilon x_1-\epsilon +C_1 \delta^2)W + [(\epsilon x_1-\epsilon +C_1 \delta^2)^2+C_2 \delta^{-2}T^{-1}]W^2  \Bigr) \,.
%\end{align*}
Using that $a W^2 + bW +c \ge c-b^2/(4a)$, we get 
\begin{align*}
\EE{R}  &\ge \frac{\epsilon (x_1-1)^2}{2} \frac{C_2}{C_2 + \delta^2 T(\epsilon x_1 - \epsilon +C_1 \delta^2)^2} \\
& \ge \frac{\epsilon (x_1-1)^2}{2} \frac{C_2}{C_2 + 2(x_1-1)^2 T \delta^2 \epsilon^2+ 2 C_1^2 T \delta^6}.
\end{align*}


%Now, for Algorithm~2, the parameter choices are $\epsilon = T^{-\frac{1}{3}}$, $\eta = T^{-\frac{5}{8}}$, $K = T^{\frac{1}{8}}$, $\delta=T^{-\frac{3}{16}}$, leading to the lower bound $\EE{R} = \Omega(T^{-1/3})$, contradicting the upper bound $O(T^{-3/8})$ of \citet{DekelEK15}.

Now, for Algorithm~2, using the parameter choice $\delta=T^{-\frac{3}{16}}$ of \citet{DekelEK15}, we can choose $\epsilon=T^{-\frac{5}{16}}$, leading to the lower bound $\EE{R} = \Omega(T^{-5/16})$, contradicting the upper bound $O(T^{-3/8})$ of \citet{DekelEK15}.


\bibliographystyle{plainnat}
\bibliography{../main}


\end{document}



--------------------------------

 
The lower bound is minimized when $W=\dfrac{(x_1-1)(\epsilon x_1-\epsilon +C_1 \delta^2)}{\left(\epsilon x_1-\epsilon +C_1 \delta^2+C_2 \delta^{-2}T^{-1}\right)^2} $. Therefore, 
\begin{align*}
\EE{R} = O(\dfrac{\epsilon}{2}) = O(T^{-\frac{1}{3}})\,.
\end{align*}

Specially, for Algorithm 1, 
\begin{align*}
w_1+\cdots+w_T&= \eta \left( 1+(1-\eta \epsilon)+\cdots+(1-\eta \epsilon)^{T-1} \right)\to \dfrac{1}{\epsilon} =T^{\frac{1}{3}}\,,\\
w_1^2+\cdots+w_T^2&=\eta^2 \left( 1+(1-\eta \epsilon)^2+\cdots+(1-\eta \epsilon)^{2(T-1)} \right) \to \dfrac{\eta^2}{1-(1-\eta \epsilon)^2}= T^{-\frac{7}{24}}\,.
\end{align*}
Therefore, $\EE{R} = O(T^{-\frac{5}{12}})+O(T^{-\frac{1}{4}})>O(T^{-\frac{1}{3}})$.


\bibliographystyle{plainnat}
\bibliography{../main}


\end{document}
