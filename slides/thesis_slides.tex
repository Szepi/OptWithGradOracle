\documentclass[12pt]{beamer}


%\usepackage{palatino}
%\usepackage[mathcal]{euscript}
%\usepackage[mathbf,mathcal]{euler}
%\usepackage{mathbbol}
\usepackage{multicol}
\usepackage[disable]{todonotes}

%\usefonttheme{serif}
%\usefonttheme[onlymath]{serif}

%!TEX root = BanditConvex.tex

\mode<presentation>
{
  %\definecolor{mygreen}{cmyk}{0.82,0.11,1,0.25}
  \definecolor{mygreen}{cmyk}{0.53,0,0.32,0.64}
  \colorlet{darkgreen}{green!50!black}
  \usetheme{Madrid}
  \usecolortheme{beaver}
  \setbeamercolor*{palette tertiary}{bg=mygreen}
\setbeamercolor{frametitle}{fg=mygreen,bg=mygreen!20}
\setbeamercolor*{title}{bg=white,fg=blue}
 % \setbeamercovered{transparent}
  \setbeamertemplate{items}[circle]
 \setbeamertemplate{itemize items}[default]
 \setbeamertemplate{enumerate items}[default]


%\setbeamertemplate{itemize item}{\scriptsize\raise1.25pt\hbox{\donotcoloroutermaths$\blacktriangleright$}}
\setbeamertemplate{itemize subitem}{\tiny\raise1.5pt\hbox{\donotcoloroutermaths$\blacktriangleright$}}
%\setbeamertemplate{itemize subsubitem}{\tiny\raise1.5pt\hbox{\donotcoloroutermaths$*$}}
%\setbeamertemplate{enumerate item}{\insertenumlabel.}
%\setbeamertemplate{enumerate subitem}{\insertenumlabel.\insertsubenumlabel}
%\setbeamertemplate{enumerate subsubitem}{\insertenumlabel.\insertsubenumlabel.\insertsubsubenumlabel}
%\setbeamertemplate{enumerate mini template}{\insertenumlabel}
}

\setbeamercolor{block title}{bg=mygreen,fg=white}%bg=background, fg= foreground
%\setbeamercolor{block body}{bg=yellow,fg=green}%bg=background, fg= foreground


\setbeamertemplate{blocks}[rounded][shadow=false]
\addtobeamertemplate{block begin}{\pgfsetfillopacity{0.8}}{\pgfsetfillopacity{1}}
\setbeamercolor{structure}{fg=mygreen}
%\setbeamercolor*{block title idea}{fg=blue!50,bg= blue!10}
%\setbeamercolor*{block body idea}{fg= blue,bg= blue!5}


\newenvironment<>{idea}[1]{
  \setbeamercolor{block title}{fg=white,bg=blue!75!black!80!white}
  \begin{block}#2{#1}}{\end{block}}

\newenvironment<>{brownblock}[1]{
  \setbeamercolor{block title}{fg=white,bg=yellow!50!black!80!white}
  \begin{block}#2{#1}}{\end{block}}

%\usepackage{natbib}
\setbeamertemplate{navigation symbols}{}
\setbeamertemplate{footline}[frame number]
\setbeamercolor{math text}{fg=blue!50!normal text.fg}

\usepackage{multimedia}

\makeatletter
\newcommand{\widthofbold}[1]{%
  \settowidth{\dimen0}{\textbf{#1}}%
  \makebox[\dimen0]{#1}}
\newcommand<>{\animbf}[1]{%
  \alt#2{\textbf{#1}}{\widthofbold{#1}}}
\makeatother

\usepackage{etex}
\usepackage[absolute,overlay]{textpos}

\usepackage{wasysym} % for \smiley \frownie
%
% \usepackage{wrapfig}
% \begin{wrapfigure}{POS}{WIDTH}{LINES TO RESERVE}
% ..
% \end{wrapfigure}
% The last argument is optional

% Figures need exact positioning in columns
% Use the following command to place them
\newcommand{\figincol}[1]{
\pgfputat{\pgfxy(0,0)}{\pgfbox[left,top]{#1}}
}

% ALSO:
% \pgfputat {\pgfxy(XX,YY)}{\pgfbox[left,base]{#1}}
% 
% LL = (0cm,-7cm) 
% UR = (11cm,1cm)
%
% pgfdeclareimage
% pgfuseimage

\newcommand{\Ra}{\Rightarrow}
%% BEAMER SPECIFIC COMMANDS

\newcommand{\bi}{\begin{itemize}[<+->]}
\newcommand{\ei}{\end{itemize}}
\newcommand{\bif}{\begin{itemize}}
\newcommand{\bc}{\begin{center}}
\newcommand{\ec}{\end{center}}


\setbeamercolor{math text}{fg=blue!50!normal text.fg}
\newcommand{\animframe}[2]{\begin{frame}[<+->]{#1}#2\end{frame}}
\newcommand{\animframesq}[2]{\begin{frame}[<+->][shrink,squeeze]{#1}#2\end{frame}}
\newcommand{\animframejsq}[2]{\begin{frame}[<+->][squeeze]{#1}#2\end{frame}}

\newcommand{\animframen}[2]{\begin{frame}[<+->]{#1}#2\emptynote\end{frame}}
\newcommand{\animframesqn}[2]{\begin{frame}[<+->][shrink,squeeze]{#1}#2\emptynote\end{frame}}
\newcommand{\animframejsqn}[2]{\begin{frame}[<+->][squeeze]{#1}#2\emptynote\end{frame}}

\newcommand{\myframe}[2]{\begin{frame}{#1}#2\end{frame}}
\newcommand{\myframesq}[2]{\begin{frame}[shrink,squeeze]{#1}#2\end{frame}}
\newcommand{\myframejsq}[2]{\begin{frame}[squeeze]{#1}#2\end{frame}}

\newcommand{\myframen}[2]{\begin{frame}{#1}#2\emptynote\end{frame}}
\newcommand{\myframesqn}[2]{\begin{frame}[shrink,squeeze]{#1}#2\emptynote\end{frame}}
\newcommand{\myframejsqn}[2]{\begin{frame}[squeeze]{#1}#2\emptynote\end{frame}}


%\setbeamertemplate{footline}[frame number]
\newtheorem{Solution}[theorem]{Solution}
\newtheorem{Comm}[theorem]{Comment}
\newtheorem{Note}[theorem]{Note}

\newcommand{\bcol}[1][t]{\begin{columns}[#1]} % optional argument: alignment (t,b,c)
\newcommand{\ecol}{\end{columns}}
\newcommand{\col}[1][0.5\textwidth]{\column{#1}} % argument: width of the column
%\parindent = 10pt
\newcommand{\scaletext}[3]{ % scale-factor original-width TEXT
\scalebox{#1}{\begin{minipage}[h]{#2\textwidth} #3 \end{minipage}
}}

% note: beamer slides are 128mm by 96 mm
\newcommand{\putatUL}[4]{ % width xpos ypos WHAT; upper left corner is put at the said pos
\begin{textblock*}{#1}[0,0](#2,#3)
#4
 \end{textblock*}
}
\newcommand{\putatBR}[4]{ % width xpos ypos WHAT; bottom right corner is put at the said pos
\begin{textblock*}{#1}[1,1](#2,#3)
#4
 \end{textblock*}
}
\newcommand{\putatBL}[4]{ % width xpos ypos WHAT; bottom left corner is put at the said pos
\begin{textblock*}{#1}[0,1](#2,#3)
#4
 \end{textblock*}
}
\newcommand{\putatUR}[4]{ % width xpos ypos WHAT; upper right corner is put at the said pos
\begin{textblock*}{#1}[1,0](#2,#3)
#4
 \end{textblock*}
 }
\newcommand{\putatMID}[4]{ % width xpos ypos WHAT; middle of pic is put at the said pos
\begin{textblock*}{#1}[0.5,0.5](#2,#3)
#4
 \end{textblock*}
 }
\newcommand{\putatUM}[4]{ % width xpos ypos WHAT; upper-middle of pic is put at the said pos
\begin{textblock*}{#1}[0.5,0](#2,#3)
#4
 \end{textblock*}
 }
\newcommand{\putatLM}[4]{ % width xpos ypos WHAT; lower-middle of pic is put at the said pos
\begin{textblock*}{#1}[0.5,1.0](#2,#3)
#4
 \end{textblock*}
 }

\newcommand{\putat}[3]{\begin{picture}(0,0)(0,0)\put(#1,#2){#3}\end{picture}} % xrelpos yrelpos WHAT

\makeatletter
\newcommand{\insertprevframe}[1]{
	\def\beamer@origlmargin{\Gm@lmargin}
%    \vbox{\hfill\insertslideintonotes{0.125}\hskip-\Gm@rmargin\hskip0pt%
%      \vskip-0.125\paperheight\nointerlineskip}%
	\insertslideintonotes{#1}
}

%\newcommand{\insertslideintonotes}[1]{{%
%  \begin{pgfpicture}{0cm}{0cm}{#1\paperwidth}{#1\paperheight}
%    \begin{pgflowlevelscope}{\pgftransformscale{#1}}%
%      \color[gray]{0.8}
%      \pgfpathrectangle{\pgfpointorigin}{\pgfpoint{\paperwidth}{\paperheight}}
%      \pgfusepath{fill}
%      \color{black}
%      {\pgftransformshift{\pgfpoint{\beamer@origlmargin}{\footheight}}\pgftext[left,bottom]{\copy\beamer@frameboxcopy}}
%    \end{pgflowlevelscope}
%  \end{pgfpicture}%
%  }}
\makeatother
% For adding items to the notes pages (we do not want animation there)
\newcommand{\bin}{\bi[<1->]}


\usetikzlibrary{positioning} 

\newcommand{\todoc}[2][]{\todo[color=blue!10,size=\tiny,#1]{Cs: #2}}


%!TEX root = BanditConvex.tex

\usepackage[normalem]{ulem}

%\usepackage{stmaryrd}
%\usepackage{dsfont}

%\usepackage{booktabs}
%\usepackage{multirow}
%\usepackage{colortbl}
%\usepackage{hhline}
%\usepackage{paralist}

\usepackage{minted} % syntax highlighting
\definecolor{mintedbg}{rgb}{0.95,0.95,0.95} %$ for bgcolor

\usepackage{pdfsync}
\usepackage{graphicx} % more modern
%\usepackage{epsfig} % less modern

% Hyperlinks make it very easy to navigate an electronic document.
% In addition, this is where you should specify the thesis title
% and author as they appear in the properties of the PDF document.
% Use the "hyperref" package
% N.B. HYPERREF MUST BE THE SECOND TO LAST PACKAGE LOADED; ADD ADDITIONAL PKGS ABOVE
%\usepackage[pdftex,letterpaper=true,pagebackref=false]{hyperref} % with basic options
		% N.B. pagebackref=true provides links back from the References to the body text. This can cause trouble for printing.
\hypersetup{
    plainpages=false,       % needed if Roman numbers in frontpages
    pdfpagelabels=true,     % adds page number as label in Acrobat's page count
    bookmarks=true,         % show bookmarks bar?
    unicode=false,          % non-Latin characters in Acrobat’s bookmarks
    pdftoolbar=true,        % show Acrobat’s toolbar?
    pdfmenubar=true,        % show Acrobat’s menu?
    pdffitwindow=false,     % window fit to page when opened
    pdfstartview={FitH},    % fits the width of the page to the window
    pdftitle={(Bandit) Convex Optimization with Biased Noisy Gradient Oracles},    % set
    pdfauthor={},    % set
%     pdfauthor={D\'avid Szepesv\'ari},    % set
%    pdfsubject={Subject},  % subject: CHANGE THIS TEXT! and uncomment this line
%    pdfkeywords={keyword1} {key2} {key3}, % list of keywords, and uncomment this line if desired
    pdfnewwindow=true,      % links in new window
    colorlinks=true,        % false: boxed links; true: colored links
    linkcolor=blue,         % color of internal links
    citecolor=blue,        % color of links to bibliography
    filecolor=magenta,      % color of file links
    urlcolor=cyan           % color of external links
}


%\newcommand{\todog}{\todo[color=green]}

%\usepackage[disable]{todonotes}
\usepackage{todonotes}
% todo by Csaba
%\newcommand{\todoc}[2][]{\todo[color=blue!20!white,#1]{Csaba: #2}}
% todo by Yasin
\newcommand{\todoy}[2][]{\todo[color=red!20!white,#1]{Yasin: #2}}
% todo by David
\newcommand{\todod}[2][]{\todo[color=green!20!white,#1]{Yasin: #2}}



% For citations
%\usepackage[round,sectionbib]{natbib}
% See:
% http://merkel.zoneo.net/Latex/natbib.php
%\bibpunct{[}{]}{;}{n}{}{,}

\usepackage{algorithm}
\usepackage{algpseudocode}
%\usepackage{algorithmic}

\usepackage{amssymb}
%\usepackage[dvips]{graphics}

\usepackage{amsmath} % Learn about the AMS package, again very useful!
\usepackage{amsthm}
\usepackage{mathtools} % MoveEqLeft
\usepackage{color}
\usepackage{graphicx}
\usepackage{epstopdf}
%\usepackage{stmaryrd}
%\usepackage{algorithm,algorithmic}
\usepackage{verbatim} 

%\usepackage{mathabx}

\usepackage{nicefrac}

\usepackage{enumerate}
%\usepackage{dsfont}
%\usepackage{subfig}
%\usepackage{small-headings}

%\usepackage{cleveref}

% this is not needed at the end..
% \usepackage{etoolbox} \ifstrempty{STR}{IF EMPTY}{IF NOT EMPTY}

% Turn on/off notes and descriptions of research problems
\newif\ifcomm
%\commfalse % also turns off internal todo comments
\commtrue

\newif\iflong
\longtrue
%\longfalse


% THEOREMS -------------------------------------------------------
\newtheorem{thm}{Theorem}%[chapter]
\newtheorem{cor}[thm]{Corollary}
\newtheorem{lem}[thm]{Lemma}
\newtheorem{prop}[thm]{Proposition}
\newtheorem{conj}[thm]{Conjecture}
\newtheorem{defn}[thm]{Definition}
\newtheorem{rem}[thm]{Remark}

%\newtheorem{name}{Printed output}[numberby]

%\newtheorem{lemma}{Lemma}
%\newtheorem{definition}[lemma]{Definition}
%\newtheorem{theorem}[lemma]{Theorem}
%\newtheorem{proposition}[lemma]{Proposition}
%\newtheorem{corollary}[lemma]{Corollary}

\DeclareMathOperator{\Exp}{\mathbf{E}}
\newcommand{\V}{\overline{V}}
\newcommand{\X}{\mathbf{X}}
\newcommand{\Y}{\mathbf{Y}}
\newcommand{\ETA}{\mathbf{\eta}}
\newcommand{\bPhi}{\mathbf{\Phi}}

%\renewtheorem{definition}[thm]{Definition}
%\theoremstyle{definition}
%\newtheorem{remark}[thm]{Remark}

\newcounter{assumption}%[section]
\renewcommand{\theassumption}{A\arabic{assumption}}
\newenvironment{ass}[1][]{\begin{trivlist}\item[] \refstepcounter{assumption}%
 {\bf Assumption\ \theassumption\ #1} }{%\par\nobreak\noindent\sl\ignorespaces}{%
 \ifvmode\smallskip\fi\end{trivlist}}

\def\ddefloop#1{\ifx\ddefloop#1\else\ddef{#1}\expandafter\ddefloop\fi}

% \bA, \bB, ...
\def\ddef#1{\expandafter\def\csname b#1\endcsname{\ensuremath{\mathbf{#1}}}}
\ddefloop ABCDEFGHIJKLMNOPQRSTUVWXYZ\ddefloop

% \bbA, \bbB, ...
\def\ddef#1{\expandafter\def\csname bb#1\endcsname{\ensuremath{\mathbb{#1}}}}
\ddefloop ABCDEFGHIJKLMNOPQRSTUVWXYZ\ddefloop

% \cA, \cB, ...
\def\ddef#1{\expandafter\def\csname c#1\endcsname{\ensuremath{\mathcal{#1}}}}
\ddefloop ABCDEFGHIJKLMNOPQRSTUVWXYZ\ddefloop

% \vecA, \vecB, ..., \veca, \vecb, ...
\def\ddef#1{\expandafter\def\csname vec#1\endcsname{\ensuremath{\boldsymbol{#1}}}}
\ddefloop ABCDEFGHIJKLMNOPQRSTUVWXYZabcdefghijklmnopqrstuvwxyz\ddefloop

% \valpha, \vbeta, ...,  \vGamma, \vDelta, ...,
\def\ddef#1{\expandafter\def\csname v#1\endcsname{\ensuremath{\boldsymbol{\csname #1\endcsname}}}}
\ddefloop {alpha}{beta}{gamma}{delta}{epsilon}{varepsilon}{zeta}{eta}{theta}{vartheta}{iota}{kappa}{lambda}{mu}{nu}{xi}{pi}{varpi}{rho}{varrho}{sigma}{varsigma}{tau}{upsilon}{phi}{varphi}{chi}{psi}{omega}{Gamma}{Delta}{Theta}{Lambda}{Xi}{Pi}{Sigma}{varSigma}{Upsilon}{Phi}{Psi}{Omega}\ddefloop

\newcommand\Sig{\ensuremath{\varSigma}}
\newcommand\veps{\ensuremath{\varepsilon}}
\newcommand\eps{\ensuremath{\epsilon}}

\renewcommand\t{{\ensuremath{\scriptscriptstyle{\top}}}}

%\newenvironment{remark}

%\newenvironment{proof}{{\bf Proof.}}{\hfill\rule{2mm}{2mm}\\}

% Keep whatever you need from here

\newcommand{\bd}[1]{\mathbf{#1}}

\newcommand{\norm}[1]{\left\Vert#1\right\Vert}
\newcommand{\smallnorm}[1]{\|#1\|}
\newcommand{\abs}[1]{\left\vert#1\right\vert}
\newcommand{\supnorm}[1]{\norm{#1}_\infty}
\newcommand{\trace}{\mathop{\rm trace}}
\newcommand{\maxEig}{\lambda_{\mbox{max}}}

%\newcommand{\set}[1]{\left\{#1\right\}}
\newcommand{\cset}[2]{\left\{\,#1\,:\,#2\,\right\}}

\renewcommand{\natural}{\mathbb N}                   % Natural numbers
\newcommand{\Real}{\mathbb R}                        % Real numbers
\newcommand{\real}{\mathbb R}                        % again..
\newcommand{\R}{{\mathbb{R}}}                        % again..

\newcommand{\vv}{\vspace*{-2mm}}
\newcommand{\hh}{\hspace*{4mm}}
\newcommand{\hfig}{\hrule\vspace*{-5mm}}


\newcommand{\Prob}[1]{{\mathbb P}\left(#1\right)}    % Probabilities; example: \Prob{X>\eps}<1-\delta
\renewcommand{\P}{{\mathbb P}}                         % Probabilities when we want to control the parenthesis
\newcommand{\EE}[1]{{\mathbb E}\left[#1\right]}      % Expectations
\newcommand{\E}{{\mathbb E}}                         % Expectations  when we want to control the parenthesis
\newcommand{\Var}[1]{{\mathrm{Var}}\left[#1\right]}  % Variances
%\newcommand{\one}{\mathbb I}
\newcommand{\one}[1]{{\mathbb I}_{\{#1\}}}           % Characteristic function

\newcommand{\MB}{\mathcal{B}}
\newcommand{\MA}{\mathcal{A}}
\newcommand{\MS}{\mathcal{S}}
\newcommand{\MF}{\mathcal{F}}
\newcommand{\MG}{\mathcal{G}}
\newcommand{\MH}{\mathcal{H}}
\newcommand{\MC}{\mathcal{C}}
\newcommand{\MRR}{\mathcal{R}}
\newcommand{\MD}{\mathcal{D}}
\newcommand{\MP}{\mathcal{P}}
\newcommand{\MU}{\mathcal{U}}
\newcommand{\MO}{\mathcal{O}}
\newcommand{\MX}{\mathcal{X}}
\newcommand{\MZ}{\mathcal{Z}}
\newcommand{\MN}{\mathcal{N}}
\newcommand{\MM}{\mathcal{M}}
\newcommand{\MR}{\mathcal{R}}
\newcommand{\ME}{\mathcal{E}}
\newcommand{\MT}{\mathcal{T}}

\newcommand{\GG}{\mathcal{G}}


%\newcommand{\eps}{\varepsilon}                       % Nice epsilon
\newcommand{\ep}{\varepsilon}                        % Shorthand for nice epsilon
\newcommand{\de}{\delta}                             % Shorthand for delta
\newcommand{\To}{\longrightarrow}
\newcommand{\ra}{\rightarrow}

\newcommand{\argmin}{\mathop{\rm argmin}}
\newcommand{\argmax}{\mathop{\rm argmax}}
\newcommand{\diag}{\mathop{\rm diag}}
\newcommand{\inlinemin}{\wedge}
\newcommand{\inlinemax}{\vee}

\newcommand{\ip}[1]{\langle #1\rangle}
\newcommand{\eqdef}{\doteq} %\stackrel{\mbox{\rm\tiny def}}{=}}
\newcommand{\aP}{{\cal P}}

% Shorthands I use for math environments
\newcommand{\beq}{\begin{equation}}
\newcommand{\eeq}{\end{equation}}
\newcommand{\beqa}{\begin{eqnarray}}
\newcommand{\eeqa}{\end{eqnarray}}
\newcommand{\beqan}{\begin{eqnarray*}}
\newcommand{\eeqan}{\end{eqnarray*}}
\newcommand{\ben}{\begin{eqnarray*}}
\newcommand{\een}{\end{eqnarray*}}
\newcommand{\bea}{\begin{align*}}
\newcommand{\eea}{\end{align*}}


\newcommand{\RA}{$\Rightarrow$}
\ifcomm
   \newcommand\comm[1]{\textcolor{blue}{ #1}}
   \newcommand{\mtodo}[2]{\todo{{\bf #1}: #2}} % To add comments into the text; the first argument is "who", the second is "what"
   \def\here#1{{\bf $\langle\langle$#1$\rangle\rangle$}}

\else
   \newcommand\comm[1]{}
   \newcommand{\mtodo}[2]{}
   \def\here#1{}
\fi
\newcommand{\remove}[1]{\textcolor{blue}{\sout{#1}}}
\newcommand{\tO}{\tilde{O}}

\newcommand{\sfrac}{\nicefrac}
\newcommand{\pai}{{(i)}}
\newcommand{\integer}{\mathbb{Z}}
\renewcommand{\phi}{\varphi}
\newcommand{\ewithin}{\eps_{\text{W}}}
\newcommand{\ebetween}{\eps_{\text{B}}}
\newcommand{\emean}{\eps_{\text{M}}}
%\DeclareMathOperator{\trace}{trace}
\newcommand{\e}{\mathbf{e}}
\renewcommand{\eps}{\varepsilon}
\newcommand{\tM}{\tilde{M}}
\newcommand{\MRP}{{\cal M}}
\newcommand{\kfun}{\mathbb{K}}

\newcommand{\aparam}{\omega}
\newcommand{\dimaction}{d_{\Actions}}
\newcommand{\dimaparam}{d_{\aparam}}
\newcommand{\traj}{\xi}
\newcommand{\Trajset}{\Xi}
\newcommand{\Traj}{X}
\newcommand{\perf}{\rho}
\newcommand{\dstat}{\mu} % stationary distribution underlying a Markov chain
\newcommand{\Regret}{{\bf R}}
\renewcommand{\AA}{{\cal A}}
%\newcommand{\SA}{\States\times\Actions}
\newcommand{\Sample}{{\cal D}}
\newcommand{\hA}{\hat{A}}
\newcommand{\hb}{\hat{b}}
\newcommand{\ZZ}{{\cal Z}}
\newcommand{\ttop}{^\top}
\newcommand{\Stexpra}{\sum_{k=1}^{t} \eta_k m_{k-1}}
\newcommand{\Stexpraz}{\sum_{k=1}^{t} \eta_k z_{k-1}}
\newcommand{\Vinit}{V}
\newcommand{\Us}{U_s}
\newcommand{\Vtexpra}{\Vinit +\sum_{k=1}^t m_{k-1} m_{k-1}\ttop }
\newcommand{\CPE}[2]{\EE{ #1 \,| #2 }}
\newcommand{\PP}[1]{\mathbb{P}\left( #1\right)}
\newcommand{\normm}[2]{\norm{#1}_{#2}}
\newcommand{\Vta}[1]{\overline{V}_{#1}}
\newcommand{\Vt}{\Vta{t}}
\newcommand{\Vtta}[1]{V_{#1}}
\newcommand{\Vtt}{\Vtta{t}}
\newcommand{\beps}{\mathbb{\varepsilon}}
\newcommand{\bw}{\mathbf{w}}

\newcommand{\FF}{{\cal F}}
\newcommand{\CrossRef}[1]{#1}

\newcommand{\oZ}{\overline{Z}}
\newcommand{\oA}{\overline{A}}
\newcommand{\eqf}[1]{\exp({#1})-1-#1}
\newcommand{\orho}{\overline{\rho}}
\newcommand{\QQ}{{\cal Q}}

\newcommand{\mfrac}[2]{{}^{#1}/_{#2}}


\newcommand{\hth}{\widehat{\theta}}
\newcommand{\tth}{\widetilde{\theta}}
\newcommand{\hTh}{\widehat\Theta}
\newcommand{\TTh}{\widetilde\Theta}
\newcommand{\Xmax}{X_{\max}}
\newcommand{\Zmax}{Z_{\max}}
\newcommand{\bVparam}[1]{\mathbf{V}_{#1}}
\newcommand{\bVn}{\bVparam{n}}
\newcommand{\bVt}{\bVparam{t}}
\newcommand{\BVt}{\overline{\bV}_{t}}
\newcommand{\BVta}{\overline{\bV}_{\tau}}

% for Hilbert spaces
\newcommand{\keybound}[3]{
2 R^2 \log\left( \frac{\det\left(I + #2_{1:#3}#1^{-1}#2_{1:#3}^*\right)^{\sfrac12}}{\delta} \right)
%2 R^2 \log\left( \frac{\det (\Vta{\tau})^{\nicefrac12}\det(\Us)^{-\nicefrac{\kern-2pt-\kern-2pt 1}{2}}}{\delta} \right)
}
\newcommand{\keyboundappliedp}[2]{
 R \sqrt{2 \log\left( \frac{\det (I + #2_{1:#1}#2_{1:#1}^*/\lambda)^{\sfrac12}}{\delta} \right)}
}
\newcommand{\keyboundapplied}[1]{\keyboundappliedp{t}{#1}}
\newcommand{\keyboundappliedworstp}[1]{
 R \sqrt{\frac{#1 L^2}{\lambda} + 2\log\left( \frac{1}{\delta}\right) }
}
\newcommand{\keyboundappliedworst}{\keyboundappliedworstp{t}}

% for Euclidean spaces
\newcommand{\keyboundEuclidean}[1]{
2 R^2 \log\left( \frac{\det (\Vta{#1})^{\nicefrac12}\det(\Vinit)^{\nicefrac{\kern-2pt-\kern-2pt 1}{2}}}{\delta} \right)
}

\newcommand{\keyboundappliedpEuclidean}[1]{
 R \sqrt{2 \log\left( \frac{\det (\V_{#1})^{1/2}\det(\lambda I)^{-1/2}}{\delta} \right)}
}
\newcommand{\keyboundappliedEuclidean}{\keyboundappliedpEuclidean{t}}
\newcommand{\keyboundappliedworstpEuclidean}[1]{
 R \sqrt{d \log\left( \frac{1+\frac{#1 L}{\lambda}}{\delta} \right)}
}
\newcommand{\keyboundappliedworstEuclidean}{\keyboundappliedworstpEuclidean{t}}

% for LQR
\newcommand{\keyboundappliedplqr}[1]{
 n L \sqrt{2 \log\left( \frac{\det (\V_{#1})^{\sfrac12}\det(\lambda I)^{\sfrac{\kern-2pt-\kern-2pt 1}{2}}}{\delta} \right)}
}
\newcommand{\keyboundappliedlqr}{\keyboundappliedplqr{t}}
 \newcommand{\keyboundappliedlqrworstp}[1]{
 n L \sqrt{(n+d) \log\left( \frac{1+\frac{#1 c_m}{\lambda}}{\delta} \right)}
}
 \newcommand{\keyboundappliedlqrworst}{\keyboundappliedlqrworstp{t}}


\newcommand{\DR}{D_{\mathcal{R}}}


% tikz includes
\usepackage{tikz}
\usetikzlibrary{positioning,arrows,shapes,calc,patterns,snakes}
% For every picture that defines or uses external nodes, you'll have to
% apply the 'remember picture' style. To avoid some typing, we'll apply
% the style to all pictures.
\tikzstyle{every picture}+=[remember picture]
\tikzstyle{na} = [baseline=-.5ex]
\everymath{\displaystyle}
% end of tikz includes %%%%%%

\title{Bandit Convex Optimization with\\
 Biased Noisy Gradient Oracles}
\institute{
 University of Alberta \\ 
 \href{mailto:xhu3@ualberta.ca}{\texttt{xhu3@ualberta.ca}}
\\ 
\vspace{0.5cm}
%\includegraphics[scale=0.9]{google-logo} 
%\hspace{2cm} \vspace{0.5cm} 
\includegraphics[scale=0.3]{figs/u-of-alberta-logo}
 }
\author{Xiaowei Hu}
\date{December 20, 2016}

\usepackage{natbib}
%\bibliographystyle{apalike}
\bibliographystyle{apalike}
%\bibliographystyle{alpha}

\DeclareMathOperator{\supp}{supp}
% number sets
\newcommand{\N}{{\mathbb N}}
%\newcommand{\R}{{\mathbb R}}
%\newcommand{\real}{{\mathbb R}}

% spaces, vectors, matrices, distributions
%\newcommand{\eps}{\varepsilon}
\renewcommand{\H}{{\mathcal H}}
%\newcommand{\X}{\mathbf{X}}
%\newcommand{\Y}{\mathbf{Y}}
%\newcommand{\E}{{\mathcal E}}
\newcommand{\F}{{\mathcal F}}
\newcommand{\G}{{\mathcal G}}
\newcommand{\Z}{{\mathcal Z}}
\newcommand{\w}{{\mathbf w}}
\newcommand{\x}{{\mathbf x}}
\newcommand{\z}{{\mathbf z}}
\newcommand{\h}{{\mathbf h}}
\renewcommand{\P}{{\mathbf P}}
\renewcommand{\c}{{\mathbf c}}
\newcommand{\M}{{\mathbf M}}
\newcommand{\y}{{\mathbf y}}
\newcommand{\D}{{\mathcal D}}
%\newcommand*{\newblock}{}
%\newcommand{\ip}[1]{\langle #1\rangle}
%\newcommand{\norm}[1]{\left\|#1\right\|}
\newcommand{\tnorm}[1]{\|#1\|}
\newcommand{\A}{\mathcal{A}}
%\newcommand{\one}[1]{\mathbb{I}_{\{#1\}}}
%\newcommand{\EE}[1]{\mathbb{E}[#1]}
%\newcommand{\EEgrow}[1]{\mathbb{E}\left[#1\right]}

% operators
%\DeclareMathOperator*{\Exp}{\mathbf{E}}   % expectation
%\DeclareMathOperator*{\Prob}{\mathbf{Pr}}   % expectation
%\DeclareMathOperator*{\Var}{\mathbf{Var}}   % expectation
\DeclareMathOperator*{\Err}{Err}   
%\DeclareMathOperator*{\argmin}{argmin}   
\DeclareMathOperator*{\VC}{VC-dim}   
\DeclareMathOperator*{\Ldim}{L-dim}   
\DeclareMathOperator{\Rademacher}{\mathcal{R}}   
\DeclareMathOperator{\SRademacher}{\mathcal{SR}}   
\DeclareMathOperator*{\sign}{sign}   
\DeclareMathOperator*{\indicator}{\mathbb{1}}   
\DeclareMathOperator*{\minimize}{minimize}   
\DeclareMathOperator*{\conv}{conv}   
%\DeclareMathOperator{\Regret}{Regret}   
\DeclareMathOperator{\rowspace}{rowspace}   
%\DeclareMathOperator*{\argmax}{argmax}   
\DeclareMathOperator*{\polylog}{polylog}

\DeclareMathOperator*{\rows}{\#rows}   
\DeclareMathOperator*{\shattered}{\#shattered}   

\newtheorem{conjecture}{Conjecture}

\begin{document}
%%%%%%%%%%%%%%%%%%%%%%%%%%%%%%%%%%%%%%%%%%%%%%%%
\begin{frame}
	\maketitle
\end{frame}
%%%%%%%%%%%%%%%%%%%%%%%%%%%%%%%%%%%%%%%%%%%%%%%%%%%%%%%%%%%%%%%

%\section{Introduction}
\begin{frame}{Outline}
\tableofcontents %[currentsection]
\end{frame}


%%%%%%%%%%%%%%%%%%%%%%%%%%%%%%%%%%%%%%%%%%%%%%%%
\section{Introduction: Bandit Convex Optimization}
%%%%%%%%%%%%%%%%%%%%%%%%%%%%%%%%%%%%%%%%%%%%%%%%

\frame{
	\frametitle{Convex Optimization with Noisy Bandit Feedback}
	\bcol[c]
	\col[0.5\textwidth]
	\bc
	\includegraphics[width=\textwidth]{figs/Interaction}
	\ec
	\col[0.45\textwidth]
%	\begin{block}<+->{Goal}
    Assume $f$ convex (and smooth etc).\\
	\alert{Goal}\\
	Find a near-minimizer of $f$ using $n>0$ queries!
%	\end{block}

	\ecol
	\bigskip
	\uncover<+->{
		Noisy Bandit Feedback $\equiv$ Noisy zeroth-order information\\
		Derivative-free Methods
	}
}
%%%%%%%%%%%%%%%%%%%%%%%%%%%%%%%%%%%%%%%%%%%%%%%%
\frame{
	\frametitle{Problem Statement}
	\bi
	\item \alert{Optimization:} a single fixed objective function $f: \cK \to \R$, where $\cK \subset \R^d$
		\bi
		\item Round $t$: query at $X_t \subset \cK$
		\item The algorithm outputs $\hat{X}_n$ after $n$ rounds
		\item Optimization error: $\Delta_n = \EE{f(\hat{X}_n) }- \inf_{x\in \cK} f(x)$
		\ei
	\item \alert{Online learning:} a sequence of functions $f_1, \cdots, f_n$ chosen by the environment
		\bi
		\item Round $t$: query at $Y_t$ in a small vicinity of $\cK$
		\item The algorithm suffers the loss $f_t(Y_t)$
		\item expected regret: $R_n = \EE{\sum_{t=1}^n f_t(Y_t) }- \inf_{x\in \cK} \sum_{t=1}^n f_t(x)$
		\ei
	\ei
	\begin{block}<+->{Main Question}
	\bc
		How fast can/will/should the optimization error or learning regret
		 decrease with $n$?
	\ec		 
	\end{block}
}

%%%%%%%%%%%%%%%%%%%%%%%%%%%%%%%%%%%%%%%%%%%%%%%%

\frame{
	\frametitle{Motivation: Why Bandit Feedback?}
	\setlength\itemsep{1em}
	\bi
	\item Explicit gradient is unavailable:
		\bi
		\setlength\itemsep{0.5em}
		\item Black-box problem: Controlling an unknown system
		\item Simulation-based optimization: 
			\bi
			\setlength\itemsep{0.5em}
			\item
				$f(x) = \EE{ F(x,\xi) }$ --- $\xi$: simulation noise
			\item $F(x,\xi)$ is the output of a simulation;
			\item \citep{spall2005introduction} Introduction to stochastic search and optimization 
			\ei
		\ei
	\ei
	\bigskip
	\bi
			\setlength\itemsep{0.5em}
	\item Gradient calculation is too expensive/complicated:
		\bi
			\setlength\itemsep{0.5em}
		\item Graphical model inference
			\bi
			\item Objective functions defined in a variational way
			\item \citep{wainwright2008graphical} Graphical models, exponential families, and variational inference
			\ei
		\ei
	\ei
}

%%%%%%%%%%%%%%%%%%%%%%%%%%%%%%%%%%%%%%%%%%%%%%%%
\frame{
	\frametitle{Example: Resource Allocation}
	\bcol[c]
	\col[0.45\textwidth]
	\bc
	\includegraphics[width=\textwidth]{figs/memory}
	\ec
	\col[0.55\textwidth]
	\bi
	\item Distribute memory between jobs
	\item Maximize success: $\max_{x\in \cK} f(x)$
	\item Linear constraints: $\cK = \cset{ x\in [0,1)^d}{\sum_i x_i = 1}$
	\item Concave objective $f$: 
	Find the best configuration quickly.
	\ei
	\ecol
	\bigskip
	\uncover<+->{The only way to learn about $f$ is to try different configurations.}
	\vspace{0.5cm}
	\uncover<+->{Feedback: $f(x) + \mathrm{noise}$ for each $x$ selected. \alert{``Bandit'' feedback}}\\
}


%%%%%%%%%%%%%%%%%%%%%%%%%%%%%%%%%%%%%%%%%%%%%%%%
\frame{
	\frametitle{Subclasses of Problems: Curviness}
	\bcol[c]
	\col
	\bc
	\uncover<+->{
	\begin{minipage}[t]{0.9\textwidth}
		\alert{Curviness} of functions:\\
		$\cK \subset \R^d$ \\
		convex, closed, non-empty.\\
	    Bregman divergence:
		\begin{align*}
		& D_f(w||v)\doteq\\
		& f(w)- \left\{ f(v) + \ip{\nabla f(v),w-v} \right\}.
		\end{align*}
	\end{minipage}}
	\ec
	\col
	\includegraphics[width=\textwidth]{figs/entropy-16-06338f2-1024}
	\ecol
	
	\bcol[t]
	\col[0.6\textwidth]
	\bi
	\item
	Smoothness: 
	$\qquad
	D_f(x||y) \le \frac{L}{2} \norm{x-y}^2\,.
	$
	\item
	Strong convexity:
	$\qquad
	D_f(x||y) \ge \frac{\mu}{2} \norm{x-y}^2\,.
	$
	\ei
	\col[0.4\textwidth]
	\figincol{
	\includegraphics[width=\textwidth]<4->{figs/Problems}}
	\ecol

}

%%%%%%%%%%%%%%%%%%%%%%%%%%%%%%%%%%%%%%%%%%%%%%%%
\frame{
	\frametitle{Subclasses of Problems: Noise}
	\alert{Noisy} bandit feedback:
	Recall $f(X) = \EE{ F(X,\xi) }$.

	\bigskip
	\bigskip

	\begin{block}{Assumptions}
	\medskip
	\bcol
	\col[0.015\textwidth]
	\mbox{}
	\col[0.995\textwidth]
	\bi
	\item[(A1)] \emph{``Uncontrolled noise''}: $F(X,\xi)$ can be obtained at any point.
	\item[(A2)] \emph{``Controlled noise''}: $\xi$  can be kept fixed between queries. 
	\ei
	\ecol
	\bigskip
	\end{block}
}

%%%%%%%%%%%%%%%%%%%%%%%%%%%%%%%%%%%%%%%%%%%%%%%%


\section{The State-of-the-Art for Noisy Bandit Convex Optimization}
%%%%%%%%%%%%%%%%%%%%%%%%%%%%%%%%%%%%%%%%%%%%%%%%

\begin{frame}{Outline}
\tableofcontents[currentsection]
\end{frame}

%%%%%%%%%%%%%%%%%%%%%%%%%%%%%%%%%%%%%%%%%%%%%%%%
\frame{
	\frametitle{State of the Art}
	\bi
	\item For general convex functions
	\begin{align*}
	        \tikz[baseline]{
            \node[fill=red!20,anchor=base] (t1)
            {Optimal rate: $\Delta_n = \tilde{O}(\sqrt{1/n})$.};}	
	\end{align*}	
	\item But it is \textbf{non-constructive}!   \citep{BubeckDKP15}
	\item Two approaches to this problem:
		\bi
		\item \alert{``Gradient'' Method}: try to construct gradient information with noisy bandit feedback
		\item \alert{Ellipsoid Method}: avoid gradient estimation and rather use  geometric principles
		\ei
	\ei
}
%%%%%%%%%%%%%%%%%%%%%%%%%%%%%%%%%%%%%%%%%%%%%%%%

\frame{
	\frametitle{Controlled Noise: Simple Story}

	\bcol
	\col[0.1\textwidth]
	\col[0.9\textwidth]

	\uncover<+->{``Gradient'' method}
	\bigskip
	\bigskip
	
	\uncover<+->{$\Delta_n \le C \sqrt{d^2/n}$.}

	\bigskip
	\bigskip
	\uncover<+->{\citet{Ne11:TR}	,
	\citet{duchi2015optimal} }

	
	\bigskip
	\bigskip
	
	\uncover<+->{Optimal!}
	\ecol
}


%%%%%%%%%%%%%%%%%%%%%%%%%%%%%%%%%%%%%%%%%%%%%%%%

\frame{
	\frametitle{Uncontrolled Noise: Big Gaps}
			
%In their seminal work \citet{NeYu83} consider two approaches to this problem: Methods that try to construct gradient information and methods that avoid gradient estimation and rather use  geometric principles (the ellipsoid method, essentially). While methods in the second class make the error decay at the $O(1/\sqrt{n})$ rate,  the error scales extremely poorly with the number of optimization variables $d$.

\uncover<+->{Ellipsoid method {(\footnotesize \citet{NeYu83}, Section 9.4)}}
\bi
\item  \citet{AgFoHsuKaRa13:SIAM} --  $\sqrt{d^{33}/n}$
\item \citet{liang2014zeroth} -- $\sqrt{d^{14}/n}$
\ei 

\bigskip
 
\uncover<+->{``Gradient'' methods: {(\footnotesize \citet{NeYu83}, Section 9.3)}}
\bi
\item
	\textbf{Convex}: -- $\qquad (d^2/n)^{1/4}$\\
	{\footnotesize  \citep{NeYu83,flaxman2005online}}
\item
	\textbf{Smooth}:  -- $\qquad  (d^2/n)^{1/3}$ \\
	{\footnotesize \citep{NeYu83,saha2011improved}}
\item
	Smooth + SOC:  $\qquad \alert{\sqrt{d^2/n}}$\\
	{\footnotesize \citep{HaLe14:SOC} }
\ei

\bigskip
\uncover<+->{
Lower bound: $\alert{\sqrt{d^2/n}}$  \citep{shamir2012complexity}.
}
}

%%%%%%%%%%%%%%%%%%%%%%%%%%%%%%%%%%%%%%%%%%%%%%%%
\frame{
	\frametitle{Can We Do Better?}
	\bi
	\item Ellipsoid method: actual scaling with the dimension $d$ is extremely poor
	\bigskip
	\bigskip
	\item ``Gradient'' method: 
		\bi
		\item Better gradient estimates?
		\bigskip
		\item {\color{darkgreen}\textbf{A ``clever'' gradient method?}} Make better use of gradient estimates?
		\ei
	\bigskip
	\bigskip
	\item $\cdots$
	\ei
}

%%%%%%%%%%%%%%%%%%%%%%%%%%%%%%%%%%%%%%%%%%%%%%%%

\section{Gradient Estimation Methods}
%%%%%%%%%%%%%%%%%%%%%%%%%%%%%%%%%%%%%%%%%%%%%%%%

\begin{frame}{Outline}
\tableofcontents[currentsection]
\end{frame}

%%%%%%%%%%%%%%%%%%%%%%%%%%%%%%%%%%%%%%%%%%%%%%%%


\frame{
	\frametitle{Gradient Estimation: Two-sided Differences}	
			Explore in the direction $e_i$:
			\begin{align*}
			        \tikz[baseline]{
            \node[fill=green!20,anchor=base] (t1)
            {$g_i = \frac{1}{2\delta} \left\{ f(x+\delta e_i) - f(x-\delta e_i) \right\},\quad i=1,\dots,d.$};}	
						\end{align*}		

	Taylor-series expansions:
	\begin{align*}
	f(x+ \delta e_i)  &= f(x) + \delta\, \nabla f(x) e_i + \delta^2\, e_i^\top \nabla^2 f(x) e_i +  O(\delta^3).\\
	f(x\alert{-} \delta e_i)  &= f(x) \alert{-} \delta\, \nabla f(x) e_i + \delta^2\, e_i^\top \nabla^2 f(x) e_i +  O(\delta^3).
	\end{align*}

	\bigskip

		\begin{align*}
			        \tikz[baseline]{
            \node[fill=green!20,anchor=base] (t1)
            {Accuracy: $\norm{ g - \nabla f(x) }_2 = O(\sqrt{d} \delta^2 )$.};}	
						\end{align*}		
}



%%%%%%%%%%%%%%%%%%%%%%%%%%%%%%%%%%%%%%%%%%%%%%%%

\frame{
	\frametitle{Gradient Estimation: Noisy Bandit Feedback}	
\begin{small}
	\uncover<+->{
	
			\begin{align*}
			        \tikz[baseline]{
            \node[fill=magenta!20,anchor=base] (t1)
            {$\tilde{g}_i = \frac{1}{2\delta} \left\{ \left(f(x+\delta e_i) + \xi^+_i\right)- \left(f(x-\delta e_i)+\xi^-_i\right) \right\},\quad i=1,\dots,d.$};}	
						\end{align*}		
}

	\uncover<+->{ 
	Assumption: $\EE{ \xi^{\pm} } = 0$, $\EE{ (\xi^{\pm})^2 } \le \sigma^2 <+\infty$.}
		
	\uncover<+->{ Note: $\tilde{g}_i = g_i + \dfrac{\xi_i^+ - \xi_i^-}{2\delta}$, $\EE{ \tilde{g}_i } = g_i$.}
	
	\uncover<+->{ \alert{Bias}:
				\begin{align*}
			        \tikz[baseline]{
            \node[fill=green!20,anchor=base] (t1)
            {$\norm{ \EE{\tilde{g}} - \nabla f(x) }_2=\norm{ g - \nabla f(x) }_2 = O(\sqrt{d} \alert{\delta^2} )\,.$ };}	
						\end{align*}}		
	
	\uncover<+->{ \alert{Second moment}: $\EE{\tilde{g}_i^2} = g_i^2 + \frac{2\sigma^2}{4 \delta^2} $}
	
	\bi
	\item \emph{Controlled:} $	\EE{ \norm{\tilde{g}}_2^2 } = \norm{g}_2^2\,.$
	\item \emph{Uncontrolled:} $	\EE{ \norm{\tilde{g}}_2^2 } = \norm{g}_2^2 + O\left( \frac{d}{\alert{\delta^2}} \right)\,.$
	\ei
\end{small}	
}

%%%%%%%%%%%%%%%%%%%%%%%%%%%%%%%%%%%%%%%%%%%%%%%%

\frame{
	\frametitle{Random Exploration Direction}
	\uncover<+->{We need $2d$ queries to get $g$.}	
	
	\uncover<+->{Can we reduce the number of queries?}
	
	\uncover<+->{Idea: \alert{ Randomly choose direction! }
	
	$I \sim p(\cdot)$ a positive \emph{probability mass function} over $\{1,\dots,d\}$.}
	
	\uncover<+->{Choose
	\begin{align*}
	G_i &= \frac{1}{p(i)} \, \frac{f(x+\delta e_I)-f(x-\delta e_I)}{2\delta} \,e_{I,i}\\
	&=
	\begin{cases}
		\frac{1}{p(I)} \, \frac{f(x+\delta e_I)-f(x-\delta e_I)}{2\delta},  & I= i\,;\\
		0, & \text{otherwise}\,.
	\end{cases}
	\end{align*}}
	
	\uncover<+->{
	\alert{Only 2 queries, regardless of $d$!}}
	
	\uncover<+->{$\EE{G_i} = g_i$! }
	\uncover<+->{Hence, bias: $\norm{\EE{ G } - \nabla f(x)}_2 = O( \sqrt{d} \delta^2 )$.}
}
%%%%%%%%%%%%%%%%%%%%%%%%%%%%%%%%%%%%%%%%%%%%%%%%

\frame{
	\frametitle{Second Moment from Randomization}
	\uncover<+->{
	\begin{align*}
	G_i &= \frac{1}{p(i)} \, \frac{f(x+\delta e_I)-f(x-\delta e_I)}{2\delta} \,e_{I,i}
	\end{align*}}
	
	\uncover<+->{
	Taylor-series expansion:	
	\begin{align*}
	f(x+ \delta e_i)  &= f(x) + \delta\, \nabla f(x) e_i + \delta^2\, e_i^\top \nabla^2 f(x) e_i +  O(\delta^3),\\
	f(x- \delta e_i)  &= f(x) - \delta\, \nabla f(x) e_i + \delta^2\, e_i^\top \nabla^2 f(x) e_i +  O(\delta^3).
	\end{align*}	
	}
	\uncover<+->{
	\begin{align*}
	 G_i^2 & = \frac{\one{I=i}}{p^2(i)} 
		\frac{(2\delta f'_i(x) + O(\delta^3))^2}{4 \delta^2} 
		= %%GYA
	 \frac{\one{I=i}}{p^2(i)} 
		\frac{(\delta^2 (f_i'(x) )^2+ O(\delta^4))}{ \delta^2} 		\\
		&=
	 \frac{\one{I=i}}{p^2(i)} 
		\left\{ ( f'_i(x) )^2+ O(\delta^2) \right\}
	\end{align*}}
	
	\uncover<+->{
	Hence, $\EE{ G_i^2} = O(1/p(i))$, so at best $\EE{ \norm{G}_2^2 } = O( d^2 )$. 
	}
}
%%%%%%%%%%%%%%%%%%%%%%%%%%%%%%%%%%%%%%%%%%%%%%%%

\frame{
	\frametitle{Noisy Bandit Feedback}

	\uncover<+->{
	\begin{align*}
	\tilde{G}_i &= \frac{1}{p(i)} \, \frac{(f(x+\delta e_I)+\xi^+)-(f(x-\delta e_I)+\xi^-)}{2\delta} \,e_{I,i}
	\end{align*}}
	
	\uncover<+->{
	Hence,
	$\tilde{G}_i= \frac{\one{I=i}}{p(i)} \frac{ \xi^+-\xi^- }{2 \delta} + G_i$}
	\uncover<+->{
	and
	\begin{align*}
	\EE{\tilde{G}_i} &= \EE{ G_i}\,,\\
	 \EE{ \tilde{G}_i^2 } &= \frac{1}{p(i)} \frac{ \EE{ (\xi^+-\xi^-)^2} }{ 4 \delta^2 } + \EE{ G_i^2 }\,,
	\end{align*}	}
	\uncover<+->{
	so
	\begin{align*}
	\norm{\EE{ \tilde{G} }_2 - \nabla f(x) }_2 &= O( \alert{\sqrt{d} \delta^2} )\,,\qquad \text{ \alert{remains the same}}\\
	\EE{ \norm{\tilde{G} }_2^2 } & = O( d^2(1 + 1/\alert{\delta^{2}} ) )\,.  \qquad \text{ \alert{increase by $\times d$}}
	\end{align*}}

}
%%%%%%%%%%%%%%%%%%%%%%%%%%%%%%%%%%%%%%%%%%%%%%%
\frame{
	\frametitle{Gradient Estimation: One-Point}
	\textbf{Intuition:}
	\bigskip
	\bigskip
	\bi
	\item For one-dimension: let $u=\pm 1$ with equal probability
		\bi
		\item $\EE{\dfrac{f(x+\delta u)}{\delta} u} = \dfrac{f(x+\delta )-f(x-\delta )}{2\delta}$
		\ei
	\bigskip
	\bigskip
	\item For $d$-dimensions: $\nabla f(x) = \dfrac{df}{dx_1}(x), \cdots,  \dfrac{df}{dx_d}(x)$
	\bigskip
	\bigskip
	\item Vector calculus: Gauss-Ostrogradsky theorem or the divergence theorem
	\ei
}

%%%%%%%%%%%%%%%%%%%%%%%%%%%%%%%%%%%%%%%%%%%%%%%%
\frame{
	\frametitle{General Two-Point \& One-Point Estimates}
	\uncover<+->{Two-point estimate:
	
			\begin{align*}
			        \tikz[baseline]{
            \node[fill=green!20,anchor=base] (t1)
            {$G = \frac{ (f(x+U)+\xi^+) - (f(x-U)+\xi^-)}{2\delta} V\,.$};}	
						\end{align*}		
	}

	\uncover<+->{
	One-point estimate!
			\begin{align*}
			        \tikz[baseline]{
            \node[fill=green!20,anchor=base] (t1)
            {$G = \frac{ (f(x+U)+\xi^+) }{\delta} V\,.$};}	
						\end{align*}		
}
	\uncover<+->{
	Choose $U,V$ such that $\EE{ V U^\top  } = I$, $\EE{ V } = 0$.}\\
	\uncover<+->{How does this work??}\\
	\uncover<+->{$\EE{G} = \EE{ G - \frac{f(x)}{\delta} V } = \EE{\frac{ (f(x+U)+\xi^+)-f(x) }{\delta} V}$.}
	
}
%%%%%%%%%%%%%%%%%%%%%%%%%%%%%%%%%%%%%%%%%%%%%%%%
\frame{
	\frametitle{Examples of Gradient Estimation Methods}
	\bi
	\item $U \sim \delta\, \cN(0,I)$, 
	         $V = \delta^{-1}\, U$
	         \bi
	         \item Smoothed functional scheme by \cite{katkul};
	         \item Refined by   \citet{PoTsy90};
	         \item Further studied by \cite{Dip03:AoS,Ne11:TR}.
	         \ei
	\item $U \sim \delta\, \mathrm{Unif}(\mathbb{S}_d)$, 
			 $V =d \delta^{-1}\,  U$
			 \bi
			 \item RDSA by \cite{kushcla};
			 \item Rediscovered by \cite{flaxman2005online}
			 \ei
	\item $U_i \sim \delta\, \mathrm{Rademacher}(\pm 1)$, 
			 $V = \delta^{-1} \,U$
			 \bi
			 \item
			  SPSA by \cite{spall1992multivariate}.
			 \ei		 
	\ei
	
	\uncover<+->{
	\bc
	\textbf{Does it matter which of these we select? Not really:}\\
	when $ f\in \cC^3$,\\ 
	Bias: $O(\delta^2)$, second moment: $O(1)$ or $O(\delta^{-2})$.
	\ec
	}

}



%%%%%%%%%%%%%%%%%%%%%%%%%%%%%%%%%%%%%%%%%%%%%%%%
\section{Oracle Model: Biased, Noisy Gradient Oracles}
%%%%%%%%%%%%%%%%%%%%%%%%%%%%%%%%%%%%%%%%%%%%%%%%

\begin{frame}{Outline}
\tableofcontents[currentsection]
\end{frame}

%%%%%%%%%%%%%%%%%%%%%%%%%%%%%%%%%%%%%%%%%%%%%%%%

\frame{
	\frametitle{Gradient Estimation Oracles}
	
	\bc
	\includegraphics[width=\textwidth]{figs/oracle0}

	\ec	
	

\begin{enumerate}[<+->]
\item Bias: $\norm{ \EE{G}  - \nabla f(x)  }_* \le c_1(\delta) $; and
\item Second moment: $\EE{\norm{ G  }_*^2} \le c_2(\delta)$.
\end{enumerate}


\bc
	\uncover<+->{
Polynomial oracle: $c_1(\delta) = C_1 \delta^p$, $c_2(\delta) = C_2 \delta^{-q}$, $p,q\ge 0$.}
	\bigskip

	\uncover<+->{
Controlled noise: $c_1(\delta) = C_1\delta^2$, $c_2(\delta) = C_2$.}

	\uncover<+->{
Uncontrolled noise:  $c_1(\delta) = C_1\delta^2$, $c_2(\delta) = C_2 \delta^{-2}$.}

%%%GYA: mention what matters is p/q

\ec



}

%%%%%%%%%%%%%%%%%%%%%%%%%%%%%%%%%%%%%%%%%%%%%%%%

\section{Main Results}
%%%%%%%%%%%%%%%%%%%%%%%%%%%%%%%%%%%%%%%%%%%%%%%%
\begin{frame}{Outline}
\tableofcontents[currentsection]
\end{frame}

%%%%%%%%%%%%%%%%%%%%%%%%%%%%%%%%%%%%%%%%%%%%%%%%
\frame{
	\frametitle{Upper Bound: Algorithm}

\begin{block}{Mirror descent \citep{NeYu83}}
\begin{algorithmic}
    \State {\bf Input:}  Closed convex set $\cK$, regularization function $\mathcal{R}:\mathbb{R}^d\to \mathbb{R}$, tolerance parameter $\delta$, learning rates $\{\eta_t\}_{t=1}^{n-1}$.
%     In round $t=1, 2, \cdots, n-1$:
\State Initialize $X_1\in \cK$ arbitrarily.
\For{$t=1, 2, \cdots, n-1$}	
	\State Query the oracle at $X_t$.
	\State Receive $G_t$.
	\State Update
	$$
	X_{t+1}=\argmin_{x\in \mathcal{K}}\left[ \eta_{t} \ip{G_t,x}+D_{\mathcal{R}}(x,X_t) \right] \,.
	$$
\EndFor
\State {\bf Return:} $\hat{X}_n = \frac{1}{n}\sum_{t=1}^n X_t \,.$
\end{algorithmic}
\end{block}

}

%%%%%%%%%%%%%%%%%%%%%%%%%%%%%%%%%%%%%%%%%%%%%%%%

\frame{
\frametitle{Upper Bound}

\begin{theorem}[\textit{\textbf{Upper bound}}]
Consider the Mirror Descent algorithm with a $(c_1, c_2)$, $(p,q)$ polynomial oracle, $\alpha$-strongly convex regularizer $\cR$.
Then:
\begin{align*}
\Delta_n(\F_{L,0},\mathrm{MD},c_1,c_2 ) &
= O( n^{- \frac{p}{2p+q} } )\\
\Delta_n(\F_{L,\mu},\mathrm{MD},c_1,c_2 ) &
= O( n^{- \frac{p}{p+q} } )\,.
\end{align*}
\end{theorem}
}

%%%%%%%%%%%%%%%%%%%%%%%%%%%%%%%%%%%%%%%%%%%%%%%%

\frame{
	\frametitle{Can We Get an Optimal Rate?}
\uncover<+->{
Recall:\vspace*{-0.2in}
\begin{align*}
\Delta_n(\F_{L,0},\mathrm{MD},c_1,c_2 ) &
= O( n^{- \frac{p}{2p+q} } )\\
% \le K_1\left(\dfrac{D {C_1^{\frac{q}{p}} C_2}}{ n}\right)^{\frac{p}{2p+q}},\\
\Delta_n(\F_{L,\mu},\mathrm{MD},c_1,c_2 ) &
= O( n^{- \frac{p}{p+q} } )\,.
%\le K_2\left(\dfrac{C_1^{\frac{q}{p}} C_2}{ n}\right)^{\frac{p}{p+q}},
\end{align*}
}

\bigskip
\uncover<+->{
\center{\alert{\textbf{Yes!}}}\\ 
\bigskip
if $p/(2p+q)\ge 1/2$ for smooth, $p/(p+q)\ge 1/2$ for smooth+SOC.}\\
\bigskip
\uncover<+->{
First holds iff $q=0$. Second holds iff $p\ge q$.\\
}
}

\frame{
	\frametitle{Recoverinf State-of-the-art Results}
\uncover<+->{
Recall:\vspace*{-0.2in}
\begin{align*}
\Delta_n(\F_{L,0},\mathrm{MD},c_1,c_2 ) &
= O( n^{- \frac{p}{2p+q} } )\\
% \le K_1\left(\dfrac{D {C_1^{\frac{q}{p}} C_2}}{ n}\right)^{\frac{p}{2p+q}},\\
\Delta_n(\F_{L,\mu},\mathrm{MD},c_1,c_2 ) &
= O( n^{- \frac{p}{p+q} } )\,.
%\le K_2\left(\dfrac{C_1^{\frac{q}{p}} C_2}{ n}\right)^{\frac{p}{p+q}},
\end{align*}
}

\bigskip

\uncover<+->{
\alert{Uncontrolled noise}; under smoothness, $p=q=2$.} \\
\uncover<+->{
\tikz[baseline]{
            \node[fill=green!20,anchor=base] (t1)
            {For $\F_{L,\mu}$ we get $O(\alert{n^{-1/2}})$ as \citet{HaLe14:SOC}.};}	
}\\[1ex]
\uncover<+->{
\tikz[baseline]{
            \node[fill=red!20,anchor=base] (t1)
            {For $\F_{L,0}$ we get $O(n^{-1/3})$ as \citet{saha2011improved}.};}	
}

\bigskip

\uncover<+->{
\alert{Controlled noise}: under smoothness, $p=2$, $q=0$.}\\
\uncover<+->{
\tikz[baseline]{
            \node[fill=blue!20,anchor=base] (t1)
            {For $\F_{L,\mu}$ we get $O(\alert{n^{-1}})$ as 
	\citet{Ne11:TR}.};}	
}\\
\uncover<+->{
\tikz[baseline]{
            \node[fill=brown!20,anchor=base] (t1)
            {For $\F_{L,0}$ we get $O(\alert{n^{-1/2}})$ as	
	\citet{duchi2015optimal}.};}	
}

}
%%%%%%%%%%%%%%%%%%%%%%%%%%%%%%%%%%%%%%%%%%%%%%%%
\frame{
	\frametitle{Does the ``Clever'' Algorithm Exist?}

\uncover<+->{
\begin{theorem}[\textit{\textbf{Lower bound}}]
\label{thm:lb-convex}
$\cK\subset \R^d$ convex, closed, with  $\{+1,-1\}^d\subset \cK$.
For any algorithm $\mathrm{A}$ that observes $n$ random elements from a $(c_1, c_2)$,  $(p,q)$ polynomial oracle, we have
 \vspace*{-0.1in}
\begin{align*}
\Delta_n(\F_{L,0},\mathrm{A},c_1,c_2 ) &= \Omega( n^{-\frac{p}{2p+q}}),\\
\Delta_n(\F_{L,1},\mathrm{A},c_1,c_2 ) & = \Omega(  n^{-\frac{\alert{2}p}{2p+q}})\,.
\end{align*}
\vspace*{-0.2in}
\end{theorem}}

\uncover<+->{
Compare with $O( n^{- \frac{p}{2p+q} } )$ and $O(n^{-\frac{2p}{2p+\alert{2}q}})$
%\begin{align*}
%\Delta_n(\F_{L,0},\mathrm{MD},c_1,c_2 ) &
%= O( n^{- \frac{p}{2p+q} } )\\
%% \le K_1\left(\dfrac{D {C_1^{\frac{q}{p}} C_2}}{ n}\right)^{\frac{p}{2p+q}},\\
%\Delta_n(\F_{L,1},\mathrm{MD},c_1,c_2 ) &
%= O( n^{- \frac{p}{p+q} } )  = O(n^{-\frac{2p}{2p+\alert{2}q}})\,.
%%\le K_2\left(\dfrac{C_1^{\frac{q}{p}} C_2}{ n}\right)^{\frac{p}{p+q}},
%\end{align*}
%\vspace*{-2cm}
\bi
\item The lower bound for $\F_{L,0}$ is tight, for $\F_{L,1}$ it is weak.
\item The result rules out the improvement of ``gradient'' methods as in \citep{DeKo15:BSCO} and \citep{yang2016optimistic}.
\ei
}

}


%%%%%%%%%%%%%%%%%%%%%%%%%%%%%%%%%%%%%%%%%%%%%%%%
\frame{
	\frametitle{Lower Bound Idea}
	\bcol[c]
	\col[0.4\textwidth]
	\bc
	\includegraphics[width=\textwidth]{figs/lb_proof_graph}
	\ec
	\col[0.55\textwidth]
	\bi
	\item	Construct two loss functions: noisy gradient estimates are \alert{so close} that they can \textbf{NOT} be distinguished.

 \item Choose $\epsilon \left| x-v \right|$, where $v \in \{-1,+1\}$.

 \item Smooth approximation:
\begin{align*}
f_v(x) :=\epsilon\left( x-v\right)+2\epsilon^2 \ln\left(1+e^{-\frac{x-v}{\epsilon}}  \right)
\end{align*}
 is $0.5$-smooth.
 
 \item Oracle {\color{darkgreen}(dashed line)} shifted with a bias $\min\left(\epsilon,C_1\delta^p\right)$.
	\ei
	\ecol
}

%%%%%%%%%%%%%%%%%%%%%%%%%%%%%%%%%%%%%%%%%%%%%%%%
\frame{
	\frametitle{Proof Sketch}
	Denote $b=\min\left(\epsilon,C_1\delta^p\right)$.\\
	Define
\begin{align*}
\gamma_+(x,\delta)&=
			\begin{cases}
               f'_+(x)+b &x\leq 0\\
               \min \left( f'_+(x)+b, f'_-(x)-b \right) &x>0 
            \end{cases}\\            
\gamma_-(x,\delta)&=-\gamma_+(-x,\delta) \,.
\end{align*}
We get
$
\left| \gamma_+(x,\delta)-\gamma_-(x,\delta)\right|<2(\epsilon-C_1\delta^p)_+ 
$. 
\begin{align*}
\Delta_n^{*}  \ge & \inf_{\A} \dfrac{\epsilon}{2}\,  \P(\hat X_n V < 0), \\
  \ge &\dfrac{\epsilon}{2} \left(1 - \sqrt{
    n}  \dfrac{ \sup_{\delta>0} (\epsilon-C_1\delta^p)_+\delta^{q/2}}{C_2}
  \right)
\end{align*}
Choosing $\epsilon$ that maximizes the bound gives the result.
}
%%%%%%%%%%%%%%%%%%%%%%%%%%%%%%%%%%%%%%%%%%%%%%%%

\section{Conclusion}
%%%%%%%%%%%%%%%%%%%%%%%%%%%%%%%%%%%%%%%%%%%%%%%%
\begin{frame}{Outline}
\tableofcontents[currentsection]
\end{frame}
%%%%%%%%%%%%%%%%%%%%%%%%%%%%%%%%%%%%%%%%%%%%%%%%
\frame{
	\frametitle{Conclusion}
\begin{Corollary}	
To get the optimal $O(n^{-1/2})$ rate for $\F_{L,0}$  with uncontrolled noise,
with a low-complexity algorithm,
one of the following must be done:
\begin{enumerate}
\item An oracle with $q=0$ (constant second moment bound) must be designed.
\item 
\sout{An algorithm that makes better use of the gradient estimates must be designed.}
\item Some extra properties of gradient estimates must be exploited beyond bias/variance.
\item Design a non-gradient algorithm. 
\end{enumerate}
\end{Corollary}	
}
%%%%%%%%%%%%%%%%%%%%%%%%%%%%%%%%%%%%%%%%%%%%%%%%
\frame{
	\frametitle{Big Picture: Noisy Bandit Optimization}
	\bi
	\item Linear case: Pretty well understood
	\bigskip
	\item Controlled noise or noise-free: Pretty well understood
	\bigskip
	\item Uncontrolled noise: Not much is known about low complexity, optimal algorithms!
		\bi
		\item Theoretical progress to bridge the gap
		\bigskip
		\item Current ``gradient'' methods are essentially sub-optimal
		\bigskip
		\item Go beyond bias-variance trade-off
		\ei
	\ei
}



%%%%%%%%%%%%%%%%%%%%%%%%%%%%%%%%%%%%%%%%%%%%%%%%



\begin{frame}
\begin{center}
\LARGE{Thanks! Questions?}
\end{center}
\end{frame}


%%%%%%%%%%%%%%%%%%%%%%%%%%%%%%%%%%%%%%%%%%%%%%%%%%%%%%%%%%%%%%%

%\begin{frame}
%\frametitle{}
%\end{frame}

%%%%%%%%%%%%%%%%%%%%%%%%%%%%%%%%%%%%%%%%%%%%%%%%%%%%%%%%%%%%%%%

\begin{frame}[allowframebreaks]
	\frametitle{References}
%        \begin{center}
%          Thanks!
%        \end{center}
	%\bibliographystyle{acm}
%	\begin{multicols}{2}
	\scriptsize 
	%\scriptsize\tiny
%	\bibliography{biblio,allbib,shortconfs,../thebib}
%	\bibliography{allbib,shortconfs,../thebib}
%\bibliography{allbib,shortconfs}
\bibliography{allbib,main}
%	\bibliography{../thebib}
%	\end{multicols}
\end{frame}


\end{document}
