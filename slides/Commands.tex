%!TEX root = BanditConvex.tex

\usepackage[normalem]{ulem}

%\usepackage{stmaryrd}
%\usepackage{dsfont}

%\usepackage{booktabs}
%\usepackage{multirow}
%\usepackage{colortbl}
%\usepackage{hhline}
%\usepackage{paralist}

%\usepackage{minted} % syntax highlighting
%\definecolor{mintedbg}{rgb}{0.95,0.95,0.95} %$ for bgcolor

\usepackage{pdfsync}
\usepackage{graphicx} % more modern
%\usepackage{epsfig} % less modern

% Hyperlinks make it very easy to navigate an electronic document.
% In addition, this is where you should specify the thesis title
% and author as they appear in the properties of the PDF document.
% Use the "hyperref" package
% N.B. HYPERREF MUST BE THE SECOND TO LAST PACKAGE LOADED; ADD ADDITIONAL PKGS ABOVE
%\usepackage[pdftex,letterpaper=true,pagebackref=false]{hyperref} % with basic options
		% N.B. pagebackref=true provides links back from the References to the body text. This can cause trouble for printing.
\hypersetup{
    plainpages=false,       % needed if Roman numbers in frontpages
    pdfpagelabels=true,     % adds page number as label in Acrobat's page count
    bookmarks=true,         % show bookmarks bar?
    unicode=false,          % non-Latin characters in Acrobat’s bookmarks
    pdftoolbar=true,        % show Acrobat’s toolbar?
    pdfmenubar=true,        % show Acrobat’s menu?
    pdffitwindow=false,     % window fit to page when opened
    pdfstartview={FitH},    % fits the width of the page to the window
    pdftitle={(Bandit) Convex Optimization with Biased Noisy Gradient Oracles},    % set
    pdfauthor={},    % set
%     pdfauthor={D\'avid Szepesv\'ari},    % set
%    pdfsubject={Subject},  % subject: CHANGE THIS TEXT! and uncomment this line
%    pdfkeywords={keyword1} {key2} {key3}, % list of keywords, and uncomment this line if desired
    pdfnewwindow=true,      % links in new window
    colorlinks=true,        % false: boxed links; true: colored links
    linkcolor=blue,         % color of internal links
    citecolor=blue,        % color of links to bibliography
    filecolor=magenta,      % color of file links
    urlcolor=cyan           % color of external links
}


%\newcommand{\todog}{\todo[color=green]}

%\usepackage[disable]{todonotes}
\usepackage{todonotes}
% todo by Csaba
%\newcommand{\todoc}[2][]{\todo[color=blue!20!white,#1]{Csaba: #2}}
% todo by Yasin
\newcommand{\todoy}[2][]{\todo[color=red!20!white,#1]{Yasin: #2}}
% todo by David
\newcommand{\todod}[2][]{\todo[color=green!20!white,#1]{Yasin: #2}}



% For citations
%\usepackage[round,sectionbib]{natbib}
% See:
% http://merkel.zoneo.net/Latex/natbib.php
%\bibpunct{[}{]}{;}{n}{}{,}

\usepackage{algorithm}
\usepackage{algpseudocode}
%\usepackage{algorithmic}

\usepackage{amssymb}
%\usepackage[dvips]{graphics}

\usepackage{amsmath} % Learn about the AMS package, again very useful!
\usepackage{amsthm}
\usepackage{mathtools} % MoveEqLeft
\usepackage{color}
\usepackage{graphicx}
\usepackage{epstopdf}
%\usepackage{stmaryrd}
%\usepackage{algorithm,algorithmic}
\usepackage{verbatim} 

%\usepackage{mathabx}

\usepackage{nicefrac}

\usepackage{enumerate}
%\usepackage{dsfont}
%\usepackage{subfig}
%\usepackage{small-headings}

%\usepackage{cleveref}

% this is not needed at the end..
% \usepackage{etoolbox} \ifstrempty{STR}{IF EMPTY}{IF NOT EMPTY}

% Turn on/off notes and descriptions of research problems
\newif\ifcomm
%\commfalse % also turns off internal todo comments
\commtrue

\newif\iflong
\longtrue
%\longfalse


% THEOREMS -------------------------------------------------------
\newtheorem{thm}{Theorem}%[chapter]
\newtheorem{cor}[thm]{Corollary}
\newtheorem{lem}[thm]{Lemma}
\newtheorem{prop}[thm]{Proposition}
\newtheorem{conj}[thm]{Conjecture}
\newtheorem{defn}[thm]{Definition}
\newtheorem{rem}[thm]{Remark}

%\newtheorem{name}{Printed output}[numberby]

%\newtheorem{lemma}{Lemma}
%\newtheorem{definition}[lemma]{Definition}
%\newtheorem{theorem}[lemma]{Theorem}
%\newtheorem{proposition}[lemma]{Proposition}
%\newtheorem{corollary}[lemma]{Corollary}

\DeclareMathOperator{\Exp}{\mathbf{E}}
\newcommand{\V}{\overline{V}}
\newcommand{\X}{\mathbf{X}}
\newcommand{\Y}{\mathbf{Y}}
\newcommand{\ETA}{\mathbf{\eta}}
\newcommand{\bPhi}{\mathbf{\Phi}}

%\renewtheorem{definition}[thm]{Definition}
%\theoremstyle{definition}
%\newtheorem{remark}[thm]{Remark}

\newcounter{assumption}%[section]
\renewcommand{\theassumption}{A\arabic{assumption}}
\newenvironment{ass}[1][]{\begin{trivlist}\item[] \refstepcounter{assumption}%
 {\bf Assumption\ \theassumption\ #1} }{%\par\nobreak\noindent\sl\ignorespaces}{%
 \ifvmode\smallskip\fi\end{trivlist}}

\def\ddefloop#1{\ifx\ddefloop#1\else\ddef{#1}\expandafter\ddefloop\fi}

% \bA, \bB, ...
\def\ddef#1{\expandafter\def\csname b#1\endcsname{\ensuremath{\mathbf{#1}}}}
\ddefloop ABCDEFGHIJKLMNOPQRSTUVWXYZ\ddefloop

% \bbA, \bbB, ...
\def\ddef#1{\expandafter\def\csname bb#1\endcsname{\ensuremath{\mathbb{#1}}}}
\ddefloop ABCDEFGHIJKLMNOPQRSTUVWXYZ\ddefloop

% \cA, \cB, ...
\def\ddef#1{\expandafter\def\csname c#1\endcsname{\ensuremath{\mathcal{#1}}}}
\ddefloop ABCDEFGHIJKLMNOPQRSTUVWXYZ\ddefloop

% \vecA, \vecB, ..., \veca, \vecb, ...
\def\ddef#1{\expandafter\def\csname vec#1\endcsname{\ensuremath{\boldsymbol{#1}}}}
\ddefloop ABCDEFGHIJKLMNOPQRSTUVWXYZabcdefghijklmnopqrstuvwxyz\ddefloop

% \valpha, \vbeta, ...,  \vGamma, \vDelta, ...,
\def\ddef#1{\expandafter\def\csname v#1\endcsname{\ensuremath{\boldsymbol{\csname #1\endcsname}}}}
\ddefloop {alpha}{beta}{gamma}{delta}{epsilon}{varepsilon}{zeta}{eta}{theta}{vartheta}{iota}{kappa}{lambda}{mu}{nu}{xi}{pi}{varpi}{rho}{varrho}{sigma}{varsigma}{tau}{upsilon}{phi}{varphi}{chi}{psi}{omega}{Gamma}{Delta}{Theta}{Lambda}{Xi}{Pi}{Sigma}{varSigma}{Upsilon}{Phi}{Psi}{Omega}\ddefloop

\newcommand\Sig{\ensuremath{\varSigma}}
\newcommand\veps{\ensuremath{\varepsilon}}
\newcommand\eps{\ensuremath{\epsilon}}

\renewcommand\t{{\ensuremath{\scriptscriptstyle{\top}}}}

%\newenvironment{remark}

%\newenvironment{proof}{{\bf Proof.}}{\hfill\rule{2mm}{2mm}\\}

% Keep whatever you need from here

\newcommand{\bd}[1]{\mathbf{#1}}

\newcommand{\norm}[1]{\left\Vert#1\right\Vert}
\newcommand{\smallnorm}[1]{\|#1\|}
\newcommand{\abs}[1]{\left\vert#1\right\vert}
\newcommand{\supnorm}[1]{\norm{#1}_\infty}
\newcommand{\trace}{\mathop{\rm trace}}
\newcommand{\maxEig}{\lambda_{\mbox{max}}}

%\newcommand{\set}[1]{\left\{#1\right\}}
\newcommand{\cset}[2]{\left\{\,#1\,:\,#2\,\right\}}

\renewcommand{\natural}{\mathbb N}                   % Natural numbers
\newcommand{\Real}{\mathbb R}                        % Real numbers
\newcommand{\real}{\mathbb R}                        % again..
\newcommand{\R}{{\mathbb{R}}}                        % again..

\newcommand{\vv}{\vspace*{-2mm}}
\newcommand{\hh}{\hspace*{4mm}}
\newcommand{\hfig}{\hrule\vspace*{-5mm}}


\newcommand{\Prob}[1]{{\mathbb P}\left(#1\right)}    % Probabilities; example: \Prob{X>\eps}<1-\delta
\renewcommand{\P}{{\mathbb P}}                         % Probabilities when we want to control the parenthesis
\newcommand{\EE}[1]{{\mathbb E}\left[#1\right]}      % Expectations
\newcommand{\E}{{\mathbb E}}                         % Expectations  when we want to control the parenthesis
\newcommand{\Var}[1]{{\mathrm{Var}}\left[#1\right]}  % Variances
%\newcommand{\one}{\mathbb I}
\newcommand{\one}[1]{{\mathbb I}_{\{#1\}}}           % Characteristic function

\newcommand{\MB}{\mathcal{B}}
\newcommand{\MA}{\mathcal{A}}
\newcommand{\MS}{\mathcal{S}}
\newcommand{\MF}{\mathcal{F}}
\newcommand{\MG}{\mathcal{G}}
\newcommand{\MH}{\mathcal{H}}
\newcommand{\MC}{\mathcal{C}}
\newcommand{\MRR}{\mathcal{R}}
\newcommand{\MD}{\mathcal{D}}
\newcommand{\MP}{\mathcal{P}}
\newcommand{\MU}{\mathcal{U}}
\newcommand{\MO}{\mathcal{O}}
\newcommand{\MX}{\mathcal{X}}
\newcommand{\MZ}{\mathcal{Z}}
\newcommand{\MN}{\mathcal{N}}
\newcommand{\MM}{\mathcal{M}}
\newcommand{\MR}{\mathcal{R}}
\newcommand{\ME}{\mathcal{E}}
\newcommand{\MT}{\mathcal{T}}

\newcommand{\GG}{\mathcal{G}}


%\newcommand{\eps}{\varepsilon}                       % Nice epsilon
\newcommand{\ep}{\varepsilon}                        % Shorthand for nice epsilon
\newcommand{\de}{\delta}                             % Shorthand for delta
\newcommand{\To}{\longrightarrow}
\newcommand{\ra}{\rightarrow}

\newcommand{\argmin}{\mathop{\rm argmin}}
\newcommand{\argmax}{\mathop{\rm argmax}}
\newcommand{\diag}{\mathop{\rm diag}}
\newcommand{\inlinemin}{\wedge}
\newcommand{\inlinemax}{\vee}

\newcommand{\ip}[1]{\langle #1\rangle}
\newcommand{\eqdef}{\doteq} %\stackrel{\mbox{\rm\tiny def}}{=}}
\newcommand{\aP}{{\cal P}}

% Shorthands I use for math environments
\newcommand{\beq}{\begin{equation}}
\newcommand{\eeq}{\end{equation}}
\newcommand{\beqa}{\begin{eqnarray}}
\newcommand{\eeqa}{\end{eqnarray}}
\newcommand{\beqan}{\begin{eqnarray*}}
\newcommand{\eeqan}{\end{eqnarray*}}
\newcommand{\ben}{\begin{eqnarray*}}
\newcommand{\een}{\end{eqnarray*}}
\newcommand{\bea}{\begin{align*}}
\newcommand{\eea}{\end{align*}}


\newcommand{\RA}{$\Rightarrow$}
\ifcomm
   \newcommand\comm[1]{\textcolor{blue}{ #1}}
   \newcommand{\mtodo}[2]{\todo{{\bf #1}: #2}} % To add comments into the text; the first argument is "who", the second is "what"
   \def\here#1{{\bf $\langle\langle$#1$\rangle\rangle$}}

\else
   \newcommand\comm[1]{}
   \newcommand{\mtodo}[2]{}
   \def\here#1{}
\fi
\newcommand{\remove}[1]{\textcolor{blue}{\sout{#1}}}
\newcommand{\tO}{\tilde{O}}

\newcommand{\sfrac}{\nicefrac}
\newcommand{\pai}{{(i)}}
\newcommand{\integer}{\mathbb{Z}}
\renewcommand{\phi}{\varphi}
\newcommand{\ewithin}{\eps_{\text{W}}}
\newcommand{\ebetween}{\eps_{\text{B}}}
\newcommand{\emean}{\eps_{\text{M}}}
%\DeclareMathOperator{\trace}{trace}
\newcommand{\e}{\mathbf{e}}
\renewcommand{\eps}{\varepsilon}
\newcommand{\tM}{\tilde{M}}
\newcommand{\MRP}{{\cal M}}
\newcommand{\kfun}{\mathbb{K}}

\newcommand{\aparam}{\omega}
\newcommand{\dimaction}{d_{\Actions}}
\newcommand{\dimaparam}{d_{\aparam}}
\newcommand{\traj}{\xi}
\newcommand{\Trajset}{\Xi}
\newcommand{\Traj}{X}
\newcommand{\perf}{\rho}
\newcommand{\dstat}{\mu} % stationary distribution underlying a Markov chain
\newcommand{\Regret}{{\bf R}}
\renewcommand{\AA}{{\cal A}}
%\newcommand{\SA}{\States\times\Actions}
\newcommand{\Sample}{{\cal D}}
\newcommand{\hA}{\hat{A}}
\newcommand{\hb}{\hat{b}}
\newcommand{\ZZ}{{\cal Z}}
\newcommand{\ttop}{^\top}
\newcommand{\Stexpra}{\sum_{k=1}^{t} \eta_k m_{k-1}}
\newcommand{\Stexpraz}{\sum_{k=1}^{t} \eta_k z_{k-1}}
\newcommand{\Vinit}{V}
\newcommand{\Us}{U_s}
\newcommand{\Vtexpra}{\Vinit +\sum_{k=1}^t m_{k-1} m_{k-1}\ttop }
\newcommand{\CPE}[2]{\EE{ #1 \,| #2 }}
\newcommand{\PP}[1]{\mathbb{P}\left( #1\right)}
\newcommand{\normm}[2]{\norm{#1}_{#2}}
\newcommand{\Vta}[1]{\overline{V}_{#1}}
\newcommand{\Vt}{\Vta{t}}
\newcommand{\Vtta}[1]{V_{#1}}
\newcommand{\Vtt}{\Vtta{t}}
\newcommand{\beps}{\mathbb{\varepsilon}}
\newcommand{\bw}{\mathbf{w}}

\newcommand{\FF}{{\cal F}}
\newcommand{\CrossRef}[1]{#1}

\newcommand{\oZ}{\overline{Z}}
\newcommand{\oA}{\overline{A}}
\newcommand{\eqf}[1]{\exp({#1})-1-#1}
\newcommand{\orho}{\overline{\rho}}
\newcommand{\QQ}{{\cal Q}}

\newcommand{\mfrac}[2]{{}^{#1}/_{#2}}


\newcommand{\hth}{\widehat{\theta}}
\newcommand{\tth}{\widetilde{\theta}}
\newcommand{\hTh}{\widehat\Theta}
\newcommand{\TTh}{\widetilde\Theta}
\newcommand{\Xmax}{X_{\max}}
\newcommand{\Zmax}{Z_{\max}}
\newcommand{\bVparam}[1]{\mathbf{V}_{#1}}
\newcommand{\bVn}{\bVparam{n}}
\newcommand{\bVt}{\bVparam{t}}
\newcommand{\BVt}{\overline{\bV}_{t}}
\newcommand{\BVta}{\overline{\bV}_{\tau}}

% for Hilbert spaces
\newcommand{\keybound}[3]{
2 R^2 \log\left( \frac{\det\left(I + #2_{1:#3}#1^{-1}#2_{1:#3}^*\right)^{\sfrac12}}{\delta} \right)
%2 R^2 \log\left( \frac{\det (\Vta{\tau})^{\nicefrac12}\det(\Us)^{-\nicefrac{\kern-2pt-\kern-2pt 1}{2}}}{\delta} \right)
}
\newcommand{\keyboundappliedp}[2]{
 R \sqrt{2 \log\left( \frac{\det (I + #2_{1:#1}#2_{1:#1}^*/\lambda)^{\sfrac12}}{\delta} \right)}
}
\newcommand{\keyboundapplied}[1]{\keyboundappliedp{t}{#1}}
\newcommand{\keyboundappliedworstp}[1]{
 R \sqrt{\frac{#1 L^2}{\lambda} + 2\log\left( \frac{1}{\delta}\right) }
}
\newcommand{\keyboundappliedworst}{\keyboundappliedworstp{t}}

% for Euclidean spaces
\newcommand{\keyboundEuclidean}[1]{
2 R^2 \log\left( \frac{\det (\Vta{#1})^{\nicefrac12}\det(\Vinit)^{\nicefrac{\kern-2pt-\kern-2pt 1}{2}}}{\delta} \right)
}

\newcommand{\keyboundappliedpEuclidean}[1]{
 R \sqrt{2 \log\left( \frac{\det (\V_{#1})^{1/2}\det(\lambda I)^{-1/2}}{\delta} \right)}
}
\newcommand{\keyboundappliedEuclidean}{\keyboundappliedpEuclidean{t}}
\newcommand{\keyboundappliedworstpEuclidean}[1]{
 R \sqrt{d \log\left( \frac{1+\frac{#1 L}{\lambda}}{\delta} \right)}
}
\newcommand{\keyboundappliedworstEuclidean}{\keyboundappliedworstpEuclidean{t}}

% for LQR
\newcommand{\keyboundappliedplqr}[1]{
 n L \sqrt{2 \log\left( \frac{\det (\V_{#1})^{\sfrac12}\det(\lambda I)^{\sfrac{\kern-2pt-\kern-2pt 1}{2}}}{\delta} \right)}
}
\newcommand{\keyboundappliedlqr}{\keyboundappliedplqr{t}}
 \newcommand{\keyboundappliedlqrworstp}[1]{
 n L \sqrt{(n+d) \log\left( \frac{1+\frac{#1 c_m}{\lambda}}{\delta} \right)}
}
 \newcommand{\keyboundappliedlqrworst}{\keyboundappliedlqrworstp{t}}


\newcommand{\DR}{D_{\mathcal{R}}}
