%%%%%%%%%%%%%%%%%%%%%%%%%%%%%%%%%%%%%%%%%
% baposter Landscape Poster
% LaTeX Template
% Version 1.0 (11/06/13)
%
% baposter Class Created by:
% Brian Amberg (baposter@brian-amberg.de)
%
% This template has been downloaded from:
% http://www.LaTeXTemplates.com
%
% License:
% CC BY-NC-SA 3.0 (http://creativecommons.org/licenses/by-nc-sa/3.0/)
%
%%%%%%%%%%%%%%%%%%%%%%%%%%%%%%%%%%%%%%%%%

%----------------------------------------------------------------------------------------
%	PACKAGES AND OTHER DOCUMENT CONFIGURATIONS
%----------------------------------------------------------------------------------------

\documentclass[portrait,a0paper,fontscale=0.285]{baposter} % Adjust the font scale/size here

\usepackage{graphicx} % Required for including images
\usepackage[normalem]{ulem}
\usepackage{macros}
\graphicspath{{figures/}} % Directory in which figures are stored

\usepackage{amsmath} % For typesetting math
\usepackage{amssymb} % Adds new symbols to be used in math mode

\usepackage{booktabs} % Top and bottom rules for tables
\usepackage{enumitem} % Used to reduce itemize/enumerate spacing
\usepackage{palatino} % Use the Palatino font
\usepackage[font=small,labelfont=bf]{caption} % Required for specifying captions to tables and figures

\usepackage{multicol} % Required for multiple columns
\setlength{\columnsep}{1.5em} % Slightly increase the space between columns
\setlength{\columnseprule}{0mm} % No horizontal rule between columns

\usepackage{tikz} % Required for flow chart
\usetikzlibrary{shapes,arrows} % Tikz libraries required for the flow chart in the template



\begin{document}

\begin{poster}
{
headerborder=closed, % Adds a border around the header of content boxes
colspacing=1em, % Column spacing
bgColorOne=white, % Background color for the gradient on the left side of the poster
bgColorTwo=white, % Background color for the gradient on the right side of the poster
borderColor=green, % Border color
headerColorOne=olive, % Background color for the header in the content boxes (left side)
headerColorTwo=green, % Background color for the header in the content boxes (right side)
headerFontColor=white, % Text color for the header text in the content boxes
boxColorOne=white, % Background color of the content boxes
textborder=roundedleft, % Format of the border around content boxes, can be: none, bars, coils, triangles, rectangle, rounded, roundedsmall, roundedright or faded
eyecatcher=true, % Set to false for ignoring the left logo in the title and move the title left
headerheight=0.1\textheight, % Height of the header
headershape=roundedright, % Specify the rounded corner in the content box headers, can be: rectangle, small-rounded, roundedright, roundedleft or rounded
headerfont=\Large\bf\textsc, % Large, bold and sans serif font in the headers of content boxes
%textfont={\setlength{\parindent}{1.5em}}, % Uncomment for paragraph indentation
linewidth=2pt % Width of the border lines around content boxes
}
%----------------------------------------------------------------------------------------
%	TITLE SECTION 
%----------------------------------------------------------------------------------------
%
%{\includegraphics[height=4em]{u-of-alberta-logo.pdf}} % First university/lab logo on the left
{}
{\bf\textsc{(Bandit) Convex Optimization with\\ Biased Noisy Gradient Oracles}\vspace{0.2em}} % Poster title
{ Xiaowei  Hu, Prashanth L A, Andr\'as Gy\"orgy, Csaba Szepesv\'ari } % Author names and institution
{\includegraphics[height=2em]{u-of-alberta-logo.pdf}
\hspace{12pt}
\includegraphics[height=2em]{clark.png}
\hspace{12pt}
\includegraphics[height=2em]{icl.png}} % Second university/lab logo on the right



%----------------------------------------------------------------------------------------
%	PROBLEMS
%----------------------------------------------------------------------------------------

\headerbox{Convex Optimization}
{name=problems,column=0,row=0}{

\begin{tabular}[b]{cc}
\begin{minipage}{0.5\textwidth}
\includegraphics[width=\linewidth]{Interaction} 
\end{minipage} &
\begin{minipage}{0.4\textwidth}
Assume $f$ convex (and smooth etc)\\
{\color{red!80!black} Goal:} Find a near-minimizer of $f$ using $n>0$ queries!
\\

\end{minipage} 
\end{tabular}
 \textbf{How fast can the optimization error}\\[1.5ex]
		$\bm{\Delta_n = \EE{f(X_n) }- \inf_{x\in \cK} f(x)}$\\[1.5ex]
\textbf{decrease with $n$?}
%\vspace{-5ex}
}
%----------------------------------------------------------------------------------------


%----------------------------------------------------------------------------------------
%	SUBCLASSES
%----------------------------------------------------------------------------------------

\headerbox{Subclasses of Problems}{name=subclasses,column=0, below = problems}{
\begin{center}
\includegraphics[width=0.9\textwidth]{entropy-16-06338f2-1024}
\end{center}

\textbf{Curviness} (deviation from linearity): 
	\begin{align*}
	D_f(x,y)\doteq f(x)- \left\{ f(y) + \ip{\nabla f(y),x-y} \right\}.
	\end{align*}
	
	Smoothness: 
	$
	D_f(x,y) \le \frac{L}{2} \norm{x-y}^2\,.
	$
	
	Strong convexity:
	$
	D_f(x,y) \ge \frac{\mu}{2} \norm{x-y}^2\,.
	$
	\\[1ex]
	
	\textbf{Noisy} Bandit Feedback:
	
	Recall: $y = f(x) + \xi$, $\EE{y} = \EE{F(x,\xi)}$.
	\begin{itemize}
	\item[(A1)] \textbf{\textit{``Uncontrolled noise''}}: $F(x,\xi)$ can be obtained at any point. 
	\item[(A2)] \textbf{\textit{``Controlled noise''}}: $\xi$  can be kept fixed between queries. 
	\end{itemize}
	
}
%----------------------------------------------------------------------------------------

%----------------------------------------------------------------------------------------
%	STATE-OF-THE-ART
%----------------------------------------------------------------------------------------

\headerbox{State of the Art}{name=stateoftheart,column=0,below=subclasses}{
Controlled Noise: $\Delta_n \leq C\sqrt{d^2/n}$.

Optimal! Let's talk about the uncontrolled case.
\\[1ex]
Ellipsoid method and relatives:
\begin{itemize}
\item  Agarwal et al. (2013) --  $\color{blue}\sqrt{d^{33}/n}$.
\item Liang et al. (2014) -- $\color{blue}\sqrt{d^{14}/n}$.
\end{itemize}

Gradient methods:
\begin{itemize}
\item Convex: $\color{blue}\left( d^2/n \right)^{1/4}$.
\item Smooth: $\color{blue}\left( d^2/n \right)^{1/3}$.
\item Smooth + Strongly Convex: $\color{red}\sqrt{d^2/n}$
\end{itemize}

Lower Bound: $\color{red}\sqrt{d^2/n}$\\[0.5ex]

 \large \textbf{Big Gaps: Can we do better using a ``clever" gradient method?}
}
%----------------------------------------------------------------------------------------

%----------------------------------------------------------------------------------------
%	ORACLES
%----------------------------------------------------------------------------------------

\headerbox{Gradient Estimation Oracles}{name=oracles,column=1,span = 2}{
\begin{center}
\includegraphics[width=0.8\textwidth]{oracle0}


1. \textbf{Bias}: $\norm{ \EE{G}  - \nabla f(x)  }_* \le c_1(\delta) $; \quad and  \quad
2. \textbf{Second moment}: $\EE{\norm{ G  }_*^2} \le c_2(\delta)$.\\[1ex]

Polynomial oracle: $c_1(\delta) = C_1 \delta^p$, $c_2(\delta) = C_2 \delta^{-q}$, $p,q>0$.

Controlled noise: $c_1(\delta) = C_1\delta^2$, $c_2(\delta) = C_2$.
\quad
Uncontrolled noise:  $c_1(\delta) = C_1\delta^2$, $c_2(\delta) = C_2 \delta^{-2}$.
\end{center}
}
%----------------------------------------------------------------------------------------

%----------------------------------------------------------------------------------------
%	UPPER BOUND
%----------------------------------------------------------------------------------------

\headerbox{Upper Bound}{name=upper_bound,column=1, below=oracles}{

\textbf{Theorem 1}: 
Consider the Mirror Descent algorithm with a $(c_1,c_2)$, $(p,q)$ polynomial oracle, $\alpha$-strongly convex regularizer $\cR$.
Then:

\begin{center}
\tikz[baseline]{
            \node[trapezium,trapezium left angle=90,
  trapezium right angle=90, fill=green!20,anchor=base] (t1)
            {\makecell{
            $
\Delta_n(\F_{L,0},\mathrm{MD},c_1,c_2 ) 
= O( n^{- \frac{p}{2p+q} } )
$\\[0.7ex]
$
\Delta_n(\F_{L,\mu},\mathrm{MD},c_1,c_2 ) 
= O( n^{- \frac{p}{p+q} } )
$
}};}	
\end{center}


\textbf{Can we get an \color{red}{$n^{-1/2}$} \color{black}rate?}\\
\textbf{Yes}, if $p/(2p+q)\ge 1/2$ vs. $p/(p+q)\ge 1/2$.\\
First holds iff \color{red}$q=0$\color{black}. Second holds iff \color{red}$p\ge q$\color{black}.\\

\textit{Uncontrolled noise}: \\
under smoothness $p=q=2$. \\
For $\F_{L,\mu}$ we get $O(n^{-1/2})$ as Hazan and Levy (2014).\\
For $\F_{L,0}$ we get $O(n^{-1/3})$ as Saha and Tewari (2011).

\textit{Controlled noise}: \\
under smoothness $p=2$, $q=0$.\\
For $\F_{L,\mu}$ we get $O(1/n)$ as Nesterov (2011).\\
For $\F_{L,0}$ we get $O(n^{-1/2})$ as Duchi et al (2015).

}
%----------------------------------------------------------------------------------------

%----------------------------------------------------------------------------------------
%	LOWER BOUND
%----------------------------------------------------------------------------------------

\headerbox{Lower Bound}{name=lower_bound,column=2, below=oracles}{

\textbf{Theorem 2}: 
$\cK\subset \R^d$ convex, closed, with  $\{+1,-1\}^d\subset \cK$, $n$ large enough.
For any algorithm $\mathrm{A}$ that observes $n$ random elements from a  $(p,q)$ polynomial oracle, we have
\begin{center}
\tikz[baseline]{
            \node[trapezium,trapezium left angle=90,
  trapezium right angle=90, fill=green!20,anchor=base] (t1)
            {\makecell{
            $
\Delta_n(\F_{L,0},\mathrm{A},c_1,c_2 ) = \Omega( n^{-\frac{p}{2p+q}})
$\\[0.7ex]
$
\Delta_n(\F_{L,1},\mathrm{A},c_1,c_2 )  = \Omega(  n^{-\frac{2p}{2p+q}})
$
}};}	
\end{center}

The lower bound for $\F_{L,0}$ is tight, for $\F_{L,1}$ it is weak.

}
%----------------------------------------------------------------------------------------

%----------------------------------------------------------------------------------------
%	CONCLUSION
%----------------------------------------------------------------------------------------

\headerbox{Conclusions}{name=conclusions,column=2, below=lower_bound}{
To get the optimal $O(n^{-1/2})$ rate for $\F_{L,0}$  with uncontrolled noise,
one of the following must be done:
\begin{enumerate}
\item An oracle with $q=0$ (constant second moment bound) must be designed.
\item 
\sout{An algorithm that makes better use of the gradient estimates must be designed.}
\item Some extra properties of gradient estimates must be exploited beyond bias/variance.
\end{enumerate}

}
%----------------------------------------------------------------------------------------

\end{poster}

\end{document}